%TODO: once the text is "finished": [see sublime note]

\documentclass[11pt,
  paper=a4, 
%   twoside,
%   openright,
  hidelinks,
  bibliography=totocnumbered,
	captions=tableheading,
	BCOR=10mm
]{scrreprt}

% ====================== /parts from MastersDoctoralThesis.cls ======================
%	HEADERS AND FOOTERS


\usepackage[automark,headsepline]{scrlayer-scrpage} %https://esc-now.de/_/latex-individuelle-kopf--und-fusszeilen-update/?lang=de
\pagestyle{scrheadings}
\clearpairofpagestyles
% \automark*[section]{chapter} %[] = right side, {} = left side
\automark*[chapter]{section} %[] = right side, {} = left side
% TODO: maybe have a SEPARATE automark for ONE-sided, wo's section ist?!
\ihead{\headmark}% Inner header
\ohead[\pagemark]{\pagemark}% Outer header
\ifoot{}% Inner footer
\ofoot{}% Outer footer
%TODO: check vor dem Drucken, ich bin gerade etwas verwirrt was wo wann. 
% Ich dachte bei Okular: View Mode -> "Facing Pages, first centered" ist wie ich es gedruckt auch hab, aber scheinbar krieg ich genau die non-facing pages da als facing angezeigt? Und for some reason geht nur \automark[section]{chapter} und nicht \automark[chapter]{section}, was falsch rum aussieht? wie dumm?! Wenn ich das mit Okular wie gesagt anzeige sind halt auch inner und outer head vertauscht... Maybe I need the doc-class-option "openright"?? 
%TODO: create final PDF at least twice, once with twoside and once without
%TODO: on the two-sided version, addtitionally have \automark*[section]{chapter}

% ====================== /parts from MastersDoctoralThesis.cls ======================

\usepackage[utf8]{inputenc}
 

\usepackage{makecell} % linebreaks in tables, see https://tex.stackexchange.com/a/176780/108199
\renewcommand{\cellalign}{vh}
\usepackage{lscape} % landscape tables
\usepackage{footnote}
\makesavenoteenv{tabular} % this line and line above see https://tex.stackexchange.com/a/109471/108199
\newcommand{\specialcell}[2][l]{%
  \begin{tabular}[#1]{@{}l@{}}#2\end{tabular}} %https://tex.stackexchange.com/a/19678/108199
\newcommand{\tabitem}{\textbullet~~}


%\usepackage[text={7in,10in},centering]{geometry}  %such that appendices etc can define new margins etc
\usepackage{caption}  % https://tex.stackexchange.com/a/176175/108199
\captionsetup[table]{position=below} 

\usepackage[anythingbreaks]{breakurl}
\usepackage[onehalfspacing]{setspace}
\usepackage{amsmath} % Standard math.
\usepackage{amsthm} % Math theorems.
\usepackage{amssymb} % More math symbols.
\usepackage{dsfont} % Render |R and the like
\usepackage[british]{babel} %was [english] (see note of underscore), is now [british] 
\usepackage{underscore} % I need underscore to not have to write "\_" for underscores, but that would break labels with unterscores in it unless I also include babel, see https://tex.stackexchange.com/a/121438/108199
\theoremstyle{definition}
\newtheorem{definition}{Definition}[chapter]
 
% for https://pandas.pydata.org/docs/reference/api/pandas.DataFrame.to_latex.html, https://pandas.pydata.org/docs/reference/api/pandas.io.formats.style.Styler.to_latex.html:
\usepackage{booktabs} 
\usepackage{multirow} 
\usepackage[table,dvipsnames]{xcolor} %dvipsnames for yaml-code-listings
\usepackage{siunitx}
\colorlet{lightgreen}{green!40!white}
\usepackage{etoolbox}
\robustify\bfseries
\robustify\itshape
% end 

\usepackage{pdflscape} % if this is used, those pages within {landscape} are turned when viewed digitally (https://tex.stackexchange.com/a/141444/108199)


\usepackage{url}
\usepackage[section]{placeins} % Keep floats in the section they were defined in.
\usepackage{tabularx}
\usepackage{booktabs} % Scientific table styling.
\usepackage{floatrow} % Option for keeping floats in the place they were defined in the code.
\floatsetup[table]{style=plaintop}
% \usepackage[breaklinks=true]{hyperref} % Hyperlinks.
\usepackage[pdftex,
	   breaklinks=true,%
       bookmarks=true,%
       bookmarksopen=true,
       bookmarksopenlevel=0,
       bookmarksnumbered=false,
       plainpages=false,
       hyperindex=true,
       pdfstartview=%
 ]{hyperref} % why so many options? https://golatex.de/viewtopic.php?p=22119&sid=ebe25f27fce1765ab7d8b1d2e91ee979#p22119
\usepackage{bookmark} %also https://golatex.de/viewtopic.php?p=22119&sid=ebe25f27fce1765ab7d8b1d2e91ee979#p22119

\usepackage[all]{nowidow} % Prevent widows and orphans.
\usepackage{xstring} % logic string operations
\usepackage{bbm} % \mathbb on numerals.
\usepackage{mathtools}
\usepackage[ruled,vlined]{algorithm2e} % Pseudocode
\usepackage{scrhack} % Make warning go away.
\usepackage{graphicx}
\usepackage{subcaption} % Subfigures with subcaptions.
\usepackage{authoraftertitle} % Make author, etc., available after \maketitle
\usepackage{listofitems}
\usepackage{blindtext} % Placeholder text.
\usepackage[automake, nopostdot, nonumberlist]{glossaries} % glossary for definitions and acronyms, without dot after entry and page reference 
\makeglossaries % Generate the glossary


% ================== biblatex stuff ==================
\newcommand\posscite[1]{\citeauthor{#1}'s \cite{#1}}
\newcommand\gencite[1]{\citeauthor{#1}'s \cite{#1}} %https://tex.stackexchange.com/a/22279/108199

% \PassOptionsToPackage{obeyspaces}{url}%
\usepackage[
	backend=bibtex,% 
	style=nature,% 
	doi=true,
	isbn=false,
	url=false, 
	eprint=false
	]{biblatex}
% \renewbibmacro*{url}{\printfield{urlraw}}

\addbibresource{mendeley_bibs/Masterarbeit.bib}

\DeclareStyleSourcemap{
  \maps[datatype=bibtex, overwrite=true]{
    \map{
      \step[fieldsource=url, final]
      \step[typesource=misc, typetarget=online]
    }
    \map{
      \step[typesource=misc, typetarget=patent, final]
      \step[fieldsource=institution, final]
      \step[fieldset=holder, origfieldval]
    }
  }
}

% ================
% https://tex.stackexchange.com/a/468286/108199 
\DeclareCiteCommand{\fancyquotecite}
  {\usebibmacro{prenote}}
  {\usebibmacro{citeindex}%
   \usebibmacro{fancyquotecite}}
  {\multicitedelim}
  {\usebibmacro{postnote}}

\newbibmacro{fancyquotecite}{%
  \printnames[given-family]{labelname}%
  \setunit{\addcomma\space}%
  \printfield{maintitle}%
  \setunit{\addcomma\space}%
  \printfield{booktitle}%
  \setunit{\addcomma\space}%
  \printfield{title}%
}
% ================

\usepackage{dirtytalk} %https://de.overleaf.com/learn/latex/Typesetting_quotations
% \usepackage{csquotes} % Context sensitive quotation.
\usepackage[autostyle=false, style=english]{csquotes}
\newcommand{\q}[1]{\enquote{#1}}
% \MakeOuterQuote{"} %https://tex.stackexchange.com/a/216166/108199 to auto-replace " with `` '' (parity must be given)
% See also regarding quotation:
% IEEE Standard: https://libraryguides.vu.edu.au/ieeereferencing/gettingstarted
% https://de.overleaf.com/learn/latex/Typesetting_quotations, https://www.andy-roberts.net/writing/latex/formatting, https://wiki.carleton.edu/download/attachments/20155418/textguide.pdf?version=1&modificationDate=1387231254000&api=v2
% ================== END biblatex stuff ==================

%\linespread{1.5} % set line spacing

\DeclareFontFamily{U}{mathx}{\hyphenchar\font45}
\DeclareFontShape{U}{mathx}{m}{n}{
      <5> <6> <7> <8> <9> <10>
      <10.95> <12> <14.4> <17.28> <20.74> <24.88>
      mathx10
      }{}
\DeclareSymbolFont{mathx}{U}{mathx}{m}{n}
\DeclareFontSubstitution{U}{mathx}{m}{n}
\DeclareMathSymbol{\bigtimes}{1}{mathx}{"91}


 

%%% Custom definitions %%%
% Shorthands
\newcommand{\ie}{i.\,e.~}
\newcommand{\eg}{e.\,g.~}
\newcommand{\wrt}{w.\,r.\,t.~}
\newcommand{\ind}{\mathbbm{1}}
\DeclarePairedDelimiter{\norm}{\lVert}{\rVert} 
% Functions
\newcommand{\tpow}[1]{\cdot 10^{#1}}
\newcommand{\fref}[1]{Figure~\ref{#1}}
\newcommand{\figref}[1]{Figure~\ref{#1}}
\newcommand{\figureref}[1]{Figure~\ref{#1}}
\newcommand{\tref}[1]{Table~\ref{#1}}
\newcommand{\aref}[1]{Appendix~\ref{#1}}
\newcommand{\tabref}[1]{(Table~\ref{#1})}
\newcommand{\tableref}[1]{Table~\ref{#1}}
\newcommand{\secref}[1]{%
	\IfBeginWith{#1}{chap:}{%
		(cf. Chapter \ref{#1})}%
		{(cf. Section \ref{#1})}%
		}
\newcommand{\sectionref}[1]{%
	\IfBeginWith{#1}{chap:}{%
		Chapter \ref{#1}}%
		{\IfBeginWith{#1}{s}{%
			Section \ref{#1}}%
			{[\PackageError{sectionref}{Undefined option to sectionref: #1}{}]}}}
\newcommand{\chapref}[1]{(see chapter \ref{#1})}
% \newcommand{\unit}[1]{\,\mathrm{#1}}
\newcommand{\unitfrac}[2]{\,\mathrm{\frac{#1}{#2}}}
\newcommand{\codeil}[1]{\lstinline{#1}}{} % wrapper for preventing syntax highlight error
\newcommand{\techil}[1]{\texttt{#1}}
\newcommand{\Set}[2]{%
  \{\, #1 \mid #2 \, \}%
}
% Line for signature.
\newcommand{\namesigdate}[1][5cm]{%
	\vspace{5cm}
	{\setlength{\parindent}{0cm}
	\begin{minipage}{0.3\textwidth}
		\hrule 
		\vspace{0.5cm}
		{\small city, date}
	\end{minipage}
	 \hfill
	\begin{minipage}{0.3\textwidth}
		\hrule
		\vspace{0.5cm}
	    {\small signature}
	\end{minipage}
	}
}
% Automatically use the first sentence in a caption as the short caption.
\newcommand\slcaption[1]{\setsepchar{.}\readlist*\pdots{#1}\caption[{\pdots[1].}]{#1}}

% Variables. 
% Adapt if necessary, use to refer to figures and graphics.
\def \figwidth {0.9\linewidth}
\graphicspath{ {./graphics/figures/}{./graphics/figures/} } % Path to figures and images.

% Pandoc creates tightlists (https://tex.stackexchange.com/a/258486/108199)
\providecommand{\tightlist}{%
  \setlength{\itemsep}{0pt}\setlength{\parskip}{0pt}}

% Customizations of existing commands.

% vec-command to be used in text and mathmode. If called with \vec[m]{a} it's math-mode, default text.
\renewcommand{\vec}[2][t]{%
	\IfEqCase{#1}{%
		{m}{\mathbf{#2}}%
		{t}{\textbf{#2}}%
	}[\PackageError{tree}{Undefined option to vec: #1}{}]%
}%



% Capitalized \autoref names.
\renewcommand*{\chapterautorefname}{Chapter}
\renewcommand*{\sectionautorefname}{Section}

%have multiple references to the same footnote, see https://tex.stackexchange.com/a/35044/108199
\usepackage{cleveref}
\crefformat{footnote}{#2\footnotemark[#1]#3} %https://tex.stackexchange.com/a/10116/108199
\makeatletter 
\newcommand\footnoteref[1]{\protected@xdef\@thefnmark{\ref{#1}}\@footnotemark}
\makeatother
%....but the above doesn't work for tables, so we need something else as well, see https://tex.stackexchange.com/a/95905/108199
\usepackage{scrextend}


\title{Data-Driven Embedding of Educational Resources in a Vector Space with Interpretable Dimensions for Explainable Recommendation}
\author{Christoph Stenkamp}


% Scale images if necessary, so that they will not overflow the page
% margins by default, and it is still possible to overwrite the defaults
% using explicit options in \includegraphics[width, height, ...]{}
\makeatletter
\def\maxwidth{\ifdim\Gin@nat@width>\linewidth\linewidth\else\Gin@nat@width\fi}
\def\maxheight{\ifdim\Gin@nat@height>\textheight\textheight\else\Gin@nat@height\fi}
\makeatother
% Scale images if necessary, so that they will not overflow the page
% margins by default, and it is still possible to overwrite the defaults
% using explicit options in \includegraphics[width, height, ...]{}
\setkeys{Gin}{width=\maxwidth,height=\maxheight,keepaspectratio}
% see https://github.com/jgm/pandoc/issues/4941#issuecomment-425975499, https://github.com/jgm/pandoc/issues/4384#issuecomment-367585913


% ##################################################################################
% === CODE LISTINGS ===

\usepackage{listings} % rendering program code

\lstset{% general command to set parameter(s)
	basicstyle=\ttfamily\color{grey},          % print whole listing small
	keywordstyle=\color{black}\bfseries\underbar,
	% underlined bold black keywords
	identifierstyle=,           % nothing happens
	commentstyle=\color{white}, % white comments
	stringstyle=\ttfamily,      % typewriter type for strings
	showstringspaces=false}     % no special string spaces


% === Taken from my BA ===
\usepackage{pifont}% http://ctan.org/pkg/pifont for \ding (used in listing)
\definecolor{verylightgray}{gray}{0.92}
\definecolor{evenmorelightgray}{gray}{0.95}

\newlength\lsthorizontalpadding
\setlength\lsthorizontalpadding{3pt}
\newcommand*\lstnumberstyle{\ttfamily\scriptsize}
\newlength\lstnumbersep
\setlength\lstnumbersep{8pt}
\newlength\lstnumberwidth
\setlength\lstnumberwidth{\widthof{\lstnumberstyle00}+\lstnumbersep+\lsthorizontalpadding}

\lstset{
	numberblanklines=false
	,basicstyle=\ttfamily%
	,breaklines=true%
	,tabsize=1%
	,showstringspaces=false%
	,numbers=left%   
	,numbersep=\lstnumbersep%
	,numberstyle=\lstnumberstyle%
	,framesep=0pt% 
	,xleftmargin=\lstnumberwidth%
	,framexleftmargin=\lsthorizontalpadding%
	,xrightmargin=\lsthorizontalpadding%
	,framexrightmargin=\lsthorizontalpadding%
	,backgroundcolor=\color{verylightgray}%
	,postbreak=\ding{229}\space%
	,escapeinside={*(}{*)}
	\linespread{1.0}
}

% === YAML code listing === 
% comes from https://tex.stackexchange.com/a/152856/108199)


\newcommand\YAMLcolonstyle{\color{red}\mdseries}
\newcommand\YAMLkeystyle{\color{black}\bfseries}
\newcommand\YAMLvaluestyle{\color{blue}\mdseries}

\makeatletter

% here is a macro expanding to the name of the language
% (handy if you decide to change it further down the road)
\newcommand\language@yaml{yaml}

\expandafter\expandafter\expandafter\lstdefinelanguage
\expandafter{\language@yaml}
{
  keywords={true,false,null,y,n},
  keywordstyle=\color{darkgray}\bfseries,
  basicstyle=\YAMLkeystyle,                                 % assuming a key comes first
  sensitive=false,
  comment=[l]{\#},
  morecomment=[s]{/*}{*/},
  commentstyle=\color{purple}\ttfamily,
  stringstyle=\YAMLvaluestyle\ttfamily,
  moredelim=[l][\color{orange}]{\&},
  moredelim=[l][\color{magenta}]{*},
  moredelim=**[il][\YAMLcolonstyle{:}\YAMLvaluestyle]{:},   % switch to value style at :
  morestring=[b]',
  morestring=[b]",
  literate =    {---}{{\ProcessThreeDashes}}3
                {>}{{\textcolor{red}\textgreater}}1     
                {|}{{\textcolor{red}\textbar}}1 
                {\ -\ }{{\mdseries\ -\ }}3,
}

% switch to key style at EOL
\lst@AddToHook{EveryLine}{\ifx\lst@language\language@yaml\YAMLkeystyle\fi}
\makeatother

\newcommand\ProcessThreeDashes{\llap{\color{cyan}\mdseries-{-}-}}

% ################################################
% fonts for code-objects etc (taken from BA)

\newcommand{\inlinecode}[1]{\colorbox{verylightgray}{\lstinline[basicstyle=\ttfamily\color{white}]{#1}}}
% \newcommand{\inlinecode}[1]{\colorbox{red}{\lstinline[basicstyle=\ttfamily\color{black}]{#1}}}


\newcommand{\term}[1] {{\spaceskip=.95\fontdimen2\font minus \fontdimen4\font
	\xspaceskip=0pt\relax \large\texttt{#1}}}

\newcommand{\codefunc}[1]{\colorbox{evenmorelightgray}{\lstinline[basicstyle=\ttfamily\color{black},keywordstyle=\ttfamily]{#1}}}

\newcommand{\codeobj}[1]{\colorbox{evenmorelightgray}{{\spaceskip=.95\fontdimen2\font minus \fontdimen4\font	\xspaceskip=0pt\relax \large\texttt{#1}}}}

\newcommand{\codeobjFN}[1]{\colorbox{evenmorelightgray}{{\spaceskip=.95\fontdimen2\font minus \fontdimen4\font	\xspaceskip=0pt\relax \texttt{#1}}}}

\newcommand{\codeother}[1]{\colorbox{evenmorelightgray}{\lstinline[basicstyle=\ttfamily\color{black},keywordstyle=\ttfamily]{#1}}}



% ##################################################################################
% specialstuff mentioned later
\newcommand{\mainalgos}{\cite{Derrac2015,Ager2018,Alshaikh2020} }


\input{special_highlight.tex}
\newcommand{\todoparagraph}[1]{\highlight[red]{#1}}


% ################################################################################## 
% ################################################################################## 
% ################################################################################## 

\makeatletter

\begin{document}

\cleardoublepage
\pagestyle{plain} %for the headers & footers


\begin{titlepage}
	\begin{flushleft}
		Universität Osnabrück\\
		Fachbereich Humanwissenschaften\\
		Institute of Cognitive Science
	\end{flushleft}

	\vspace{2cm}
	\centering{
		Master's thesis\vspace{1cm}\\
		\textbf{\Large{\MyTitle}}
		\vspace{1cm}\\
		\begin{tabular}{c}
			\MyAuthor                          \\
			955004                             \\
			Master's Program Cognitive Science \\
			April 2017 - April 2022
		\end{tabular}}
	\vspace{1cm}

	\begin{tabular}{ll}
		First supervisor:  & Dr. Tobias Thelen          \\
		                   & Institute of Cognitive Science \\
		                   & University of Osnabrück  \\\\
		Second supervisor: & Johannes Schrumpf, M.Sc.         \\
		                   & Institute of Cognitive Science \\
		                   & Osnabrück
	\end{tabular}

\end{titlepage}


\chapter*{Declaration of Authorship}
I hereby certify that the work presented here is, to the best of my knowledge and belief, original and the result of my own investigations, except as acknowledged, and has not been submitted, either in part or whole, for a degree at this or any other university.

\namesigdate
\pagenumbering{gobble}
\pagebreak

\begin{abstract}
	\textbf{\LARGE{Abstract}}\\\\
	%TODO summarize the main objectives and outcomes of your work. The abstract should fit on one page.
	In this thesis, I want to generate a conceptual space for the domain of educational reasources such as university courses, automatically created in data-driven way from their descriptions.

	Conceptual Spaces are seen as something that may be able to link sub-symbolic and symbolic approaches by standing in between them: In Conceptual Spaces, Concepts are represented as convex regions in high-dimensional spaces. Optimally, these spaces are cartesian, and the axes correspond to human-interpretable dimensions. If that is the case, you could for example classify the concept of "Apple" as a region that is in the color-dimension somwhere between green and red, and in the form-dimension roughly at "round".
	Creating these concpetual spaces is a very cumbersome task, which is why an automated method may lead to reasonable results. Unfortunately, this is still computationally very complex.
	The method of [DESC15] uses MDS, blablabla, then a Support-Vector-Machine separating concepts, and the orthogonal of the separating hyperplane is then an axis
\end{abstract}




\tableofcontents
\listoffigures
\listoftables
\listofalgorithms
\lstlistoflistings %TODO: figure out which of these two is correct

\pagestyle{scrheadings}

\cleardoublepage
\pagenumbering{arabic} % https://tex.stackexchange.com/a/313306/108199 https://tex.stackexchange.com/a/154666/108199

% #########################################################################################################################################################################################################################################################################################################################################################################################################################################################################################################################################################################################################################################################################################################################################################################################################################################################


%----------------------------------------------------------------------------------------
%	THESIS CONTENT - MAIN PART
%----------------------------------------------------------------------------------------
	

% INTRO
	% Broad - "Initialkontextualisierung" - warum mach ich das, aus was für ner domain kommen die daten, was will ich damit machen (use rrecommendation, ich bau AI part, ..)
	\chapter{Introduction}

In this thesis, I want to generate a conceptual space for the domain of university courses, automatically created in data-driven way from their descriptions.

\section{Reading Instructions}

\paragraph*{Document Structure}
\todoparagraph{
	Chapter 1 is Intro, with motivations etc. 
	Chapter 2 is Background, explaining the the required concepts - what are conceptual spaces generally, how does the base algorithm work, what kinds of algorithms occur in it. I am explaining the rquired algorithm before the main algo such that I can rely on definitions there.
	Chapter 3 is then methods. Dataset, algorithm, architecture. 
	4 results, 5 conclusion, that's it.
}

\paragraph*{Regarding Terminology}

Throughout this thesis, many abbreviations, symbols and technical terms will be used. \todoparagraph{I hope that all of that cannot be exptected to be known by the reader are defined. At the end of this thesis there is a} \nameref{sec:glossary} \todoparagraph{with the subsections yadda yadda. If you are reading this document digitally, all occurrences of the terms described there should be a hyperlink (as are all section, table, figure, etc references). If you don't have the version with colored hyperlinks, you can download it here:} \url{https://nightly.link/cstenkamp/MastersThesisText/workflows/create_pdf_artifact/master/Thesis.zip}

\section{Motivation}

\includeMD{pandoc_generated_latex/1_1_motivation}

\section{Research Questions \& Thesis Goals}

\includeMD{pandoc_generated_latex/1_3_whatamidoing}


% BACKGROUND
	% (zuspitzung von generell auf spezifisch, sowohl technisch (conceputalspaces -> was macht das paper konkret), (und bei dem anderen teil was sind educational resources, was sind die schwierigkeiten dabei, warum möchte man überhaupt empfehlen))
	\chapter{Background}

% SIDDATA (mit Educational Resources dabei, muss kein sub-chapter sein)
% Conceptual Spaces (What is this?)
% Required techniques and algorithms

\section{Replication and Software Quality}

Having established the goals of replicating an algorithm for a new domain, let us look at how such a replication should be performed and how software quality can be measured.

\subsection{Replication and Reproducibility}
\label{sec:howtoreplicate}

\todoparagraph{this field of work seems constrained to a small community, without any alternative implementations or substantial improvements from outside of it.}

The workflow of data science generally follows the same pattern: A paper states there is some problem \textit{X}, claims that their algorithm \textit{Y} may be good at problem \textit{X}, creates datasets \textit{Z} for \textit{X}, and then tests the code on these datasets. This test generally compares \textit{X} to alternative approaches from the literature and explores if any regularities in the algorithm \textit{Y} can be found. This may yield future research opportunities, showing what other domains the algorithm may work for as well.

Replication fills this role by applying an existing algorithm to another domain. Results for this are important, as it helps to see first if the claimed results are valid, and if they work on datasets that are not artificial and specifically created for the sole purpose of testing the algorithm. Furthermore, the details of experiments in publsihed work are often opaque and omit important information to reproduce the algorithm. These issues are mitigated trough repetition: The robustness of the algorithm to changes in parameters or dataset is investigated. If changes in either of these have a major impact on the results, there is reason to doubt generalization of the algorithm, showing that it may not be good to solve problem \textit{X} after all.

It is absolutely crucial in science to ensure that all claims that are made are reproducible and testable, ensuring ease of replication. Reproducibility is the pinnacle of \textit{Open Science}.\footnote{There is no single definition of open science, however reproducibility appears in most tries, such as \eg \url{https://www.talyarkoni.org/blog/2019/07/13/i-hate-open-science/}, accessed at \date{2022}{03}{25}} And \q{Open Science is just science done right}.\footnote{Quote from John Tennant, see \eg \url{https://soundcloud.com/tidningen-curie/jon-tennant-open-science-is-just-science-done-right}, accessed at \date{2022}{03}{25}} Reproducibility refers to the Ability to reproduce - computationally or experimentally - the methods used to produce a given result, by virtue of being accessible and understandable. Being a hot topic in psychology since the reproducibility crisis,\footnote{Baker M: 1,500 scientists lift the lid on reproducibility. Nature. 2016; 533(7604): 452-4. \url{https://www.nature.com/articles/533452a}} the topic is just es relevant in computer science research.\footnote{Mesirov JP: Computer science. Accessible reproducible research. Science. 2010; 327(5964): 415-6. \url{https://www.ncbi.nlm.nih.gov/pmc/articles/PMC3878063/}} In that realm, Reproducibility may be seen as sub-goal of (the more fundamental) Sustainability, as \eg by \textcite{Molder2021a}, who claim that \q{reproducibility alone is not enough to sustain the hours of work that scientists invest in crafting data analyses}.

To ensure that the analysis performed in this thesis is sustainable and adheres to best scientific and software quality standards, let us look at at ways to formally define such.

\subsection{Software Quality}
\label{sec:reproducibility}

The International Organization for Standardization (ISO) provides an official international standard for the evaluation of software quality as \textbf{ISO/IEC 25010:2011} \cite{2013ISOI}. The full title of the norm is \textit{ISO/IEC 25010:2011 Systems and software engineering - Systems and software Quality Requirements and Evaluation (SQuaRE)}. It has the objective to ensure the quality of software by providing objective and clearly defined standards for definitions of success regarding of software products. It classifies software quality in eight characteristics, which each consists of several sub-characteristics. The main characteristics are \textit{Functional Suitability, Performance Efficiency, Compatibility, Usability, Reliability, Security, Maintainability} and \textit{Portability}. Not all of these are relevant to projects like this, such as \textit{Security}, which mostly measures the ability to track actions and identity of users. 

What is needed here are those goals relevant for sustainable data analysis. In this realm, the stated metrics align much with the hierarchy of aspects to consider for sustainable data analysis as published by \textcite{Molder2021a}, which is is reprinted as \autoref{fig:snakemake_aspect_hierachy}.

\begin{figure}[H]
	\centering
	\includegraphics[width=0.7\textwidth]{graphics/stolenfigures/snakemake_aspect_hierachy.png}
	\slcaption{
		Hierarchy of aspects to consider for sustainable data analysis. Reproduced from {\cite[Fig.1]{Molder2021a}} (Creative Commons Attribution License) \label{fig:snakemake_aspect_hierachy}
	}
\end{figure}

Some important aspects to conduct proper computer science and data analysis that allows for \textbf{Sustainability} - allowing the analysis to be of lasting impact - thus include \cite{Molder2021a, 2013ISOI}:

\begin{description}
    \item[Functional Suitability] which means complete, correct and appropriate functionality.
	\item[Reproducibility] \ie allowing validation and regeneration of results on the original or even new data. This understandable and well documented code (also \textit{Changeability and Stability})
	\item[Maintainability and Adaptability] the ability to modify the analysis to answer extended or slightly different research questions by allowing modifications.
	\item[Transparency] \ie the ability for others to understand it well enough to judge if it's technically as well as methodologically valid - also ensuring Understandability, Appropriateness and Accessibility, Analyzability and Testability.
	\item[Scalability] \ie enabling the scalable execution of the algorithm and each involved step, including deployment on complex compute clusters, grids or clouds. This includes Performance Efficiency and efficient Resource Utilization.
	\item[Modularity] \ie changes in one component have a minimal impact on others allowing for easy exchange and extension.
\end{description}


% \begin{description}
%     \item[Functional Suitability], assessing the existance of a set of functions that satisfy the stated needs. It includes \textit{Functional Completeness, Functional Appropriateness (Suitability)} \textit{and Functional Correctness (Accuracy)}
%     \item[Performance Efficiency], assessing the relationship between the performance and the amount of resources used, specifially \textit{Time behaviour, Resource Utilization} and \textit{Capacity}
%     \item[Compatibility], consisting of \textit{Co-Existence and Interoperability}
%     \item[Usability], assessing the effort needed for use, measured by \textit{Appropriateness Recognizability (Understandability), Learnability, Operability, User Error Protection, Accessibility} and \textit{Interface Aesthetics (Attractiveness)}
%     \item[Reliability], assessing the capacity of software to maintain its level of performance, comprising \textit{Maturity, Fault tolerance, Recoverability} and \textit{Availability}
%     \item[Security], among others allowing to track actions and identity: \textit{confidentiality, integrity, non-repudiation, accountability} and authenticity
%     \item[Maintainability], assessing the effort needed to make modifications, which includes \textit{Analyzability, Testability, Modularity} (meaning changes in one component have a minimal impact on others), \textit{reusability} and \textit{modifiability (Changeability+Stability)}
%     \item[Portability], measruing the ability to be transferred to other environments: \textit{Adaptability, Installability} and \textit{Replaceability}
% \end{description}

% Some important aspects to conduct proper (computer) science and data analysis that allows for \textbf{Sustainability} (such that the analysis is of lasting impact), may thus be \cite{Molder2021}:
% \begin{description}
% 	\item[Reproducibility] \ie allowing validation and regeneration of results on the original or even new data. Requiring understandable and well documented code.
% 	\item[Adaptability] \ie the ability to modify the analysis to answer extended or slightly different research questions.
% 	\item[Transparency] \ie the ability for others to understand it well enough to judge if it's technically as well as methodologically valid.
% 	\item[Scalability] \ie enabling the scalable execution of the algorithm and each involved step, including deployment on complex compute clusters, grids or clouds. 
% \end{description}

% Principles of open science are very important to me, so I want to ensure that the claims I am making in this thesis are backed by code that is scalable, reproducible, modular, easily-understood, easily set up and run, well documented, ... . To support this, I will as often as necessary refer to the actual code in this thesis, to allow to understand and reproduce the claims and results, and also highly encourage to critically read everything here and check the respective code (...and let me know if you spot any errors! Just open a Github Issue!)



\section{SIDDATA and Educational Resources}
\label{sec:siddata}

\todoparagraph{Educational resources has no definition, aber siddata ist ein beispiel projekt was darum revolviert}

"Frage war ja, Lässt sich der Algo auf die Domäne von Ed.Res. beziehen, und dafür muss ich sie erklären"

\todoparagraph{auf die kategorien in} \ref{tab:siddata_metadata} hinaus: in principle könnnen educaitonal resources können auch transkribierte videos sein, PDFs als vorlesungsmaterial, ganze paper oder bücher, oder multimedia-data such as in a mooc). \todoparagraph{Hier ider in dataset-section}


% \includeMD{pandoc_generated_latex/2_0_siddata}

To get a better understanding of the domain, this section elaborates on the specific use case of recommendation of education resources that shall be handled, and introduces the SIDDATA project and platform under which this thesis was developed.\footnote{As \textsc{Siddata} signifies both the project and the developed digital assistant, the all-upper 'SIDDATA' henceforth refers to the project, while the specific developed software will be denoted 'Siddata' or 'DSA'.}

This thesis was started while working at the SIDDATA project, with the idea to add a recommender to the platform that can generate course recommendations with the user \textit{in the loop}. SIDDATA is a joint interdisciplinary project for \q{\emph{Individualization of Studies through Digital, Data-Driven Assistants}}\footnote{\url{https://www.siddata.de/en/}} of the universities Osnabrück, Bremen and Leibniz Universität Hannover, funded by the German \emph{Federal Ministry of Education and Research}.\footnote{BMBF. Funding number: 16DHB2124} 

The project adresses the same problem as stated in the Introduction (\ref{sec:many_resources}), namely that e-learning and the amount of avilable resources increase, making the choice of right resources an increasingly relevant problem for the learning success of students. Its deliverable is a flexible data-driven \gls{dsa}, that supports students in higher education in their invidual learning and achievement of personal study goals by giving hints, reminders and recommendation for their individual study paths \cite{Schurz2021}, helping students in setting and achiving individual and self-regulated personal educational goals. This is in line with the increasing importance of skills such as self-organized knowledge acquisition and self-regulatory competencies due increasing importance of individualization in educational paths of the globalized learning environment \cite{Ehlers2019,Schurz2021}.

For that, the collaborative project combines heterogenous data and information in a digital study assistant. Data is collected from multiple sources, such as the \gls{lms}, offers and resources of other universities and institutions, and data collected from its users. To allow for this heterogeniety and also future extensions, for example new data sources  such as \glspl{mooc} through several apis APIs, different front-ends, or different recommendation methods, the system relies on a highly modular and extensible architecture. The Frontend is realized as plugin for the  the university's \gls{lms} Stud.IP \cite{stockmann2005}. This not only allows for easy user access, but also to get data about courses from the LMS using cronjobs. The Frontend is connected over a RESTful API to the Backend, which is written in Python on basis of the Web Framework \textit{Django} and relies on a relational \textit{PostgreSQL} database to store information.

The Backend consists of seperate encapsulated recommender modules in a loosely coupled architecture and a common ontology, allowing to easily add new subsystems. The modules generate recommendations towards personal educational goals on basis of the collected data, which are displayed to the user in the Frontend. What comprises a recommender is grouped from a user perspective, such that each each recommender focuses on a topic. The currently implemented recommenders include for example one to find peers with similar interests, get information about scientific careers, personality-based learning behaviour- and study tips, or information regarding local and remote courses and \glspl{oer}. Another module recommends courses using a combination of rule-based and modern \gls{ml} techniques that relate natural language queries with the courses known to the system (picked up in \autoref{sec:sidbert}) \cite{Schurz2021}.

The system is currently in its third prototype, and preliminary evaluation has shown that modules that provide personal recommendation are most well received \cite{Schurz2021}. This and the ease of use to add new recommenders indicate a high likelyhood of success for adding a new module that recommends courses in the way as described above.

The dataset used here was collected through the Siddata platform as well, which collected courses and events from the three universities currently connected to it, as well as other sources for \glspl{mooc} and other \glspl{oer} through respective APIs. More details in \autoref{sec:dataset_siddata}. It should be noted that the dataset is not artificially generated (unlike \mainalgos) but collected from current courses and their descriptions - making an algorithm for this domain incorporated as recommender to the platform a contribution with practical application.



\section{Conceptual Spaces}
\label{sec:cs}

This section will introduce conceptual spaces as tool of choice as well as how to generate them and how reasoning on them works, as well as some other related work to what's done in this thesis.

\subsection*{Theory of Conceptual Spaces}

The theory of Conceptual Spaces was first introduced by Peter Gärdenfors in his 2000 book \citetitle{Gardenfors2000a} \cite{Gardenfors2000a} both as a theoretical model of human concept formation, but also as format for knowledge representation in artificial systems \cite{Gardenfors2004}. 

On the one hand, \glspl{cs} should serve as bridge between symbolistic and connectionistic approaches to knowledge representation. By having CS as layer of reasoning and representation in between both, classical symbols would be grounded in noisy high-dimensional data, allowing for high-level syllogistic reasoning from real-world data. 

If a computer has a knowledge base that says $\exists x.Red(x) \& Apple(x)$, does it know what "red" and "apple" mean? We need to ground symbols, to express meaning!

According to Gärdenfors, concept-representation in humans is represented by three levels of accounting for observations: The symbolic level, the conceptual level and the subconceptual level \cite[204]{Gardenfors2000a}:
\begin{description}
    \item[Subconceptual] Observations are the firing of the neurons of our sensory receptors, without any conceptualization.  (connectionism, \glspl{ann})
    \item[Conceptual] Observations are defined not as token of a symbol, but as vector in a conceptual space of some quality  (prototype theory, linear algebra)
    \item[Symbolic] represents observations by describing them in some specified language (formal logic, syllogisms, symbolism, classical AI, logical positivism)
\end{description}

These levels are not in conflict, but different models of the same phenonemon, each covering distinct important aspects and each allowing a set of algorithms. The process of inducing a general rule from few samples for example is represented as pattern-matching on the firing patterns in the subconceptual level, which translates to the conceptual level as geometric reasoning through regions and direction. As another example, semantic relations such as hyponyms from the symbolic level are modelled as geometric sub-regions on the conceptual level. So on the one hand, automatically generated conceptual spaces could allow for high-level syllogistic reasoning on real-world data without the need to manually add countless facts. However it also provides a new way to model reasoning and inference for both other levels through geometric relations, providing explanations for the noisy subconceptual level and computationally less complex algorithms for the symbolic level. The validity of the statement \textit{a robin is a bird} is given because robins are gemmetrically a subregion of the region of birds.

Summarized, rregardless of the theory's aspiration to accurately model human conceptualization and reasoning, it provides a useful knowledge representation method and tool that allows to model kinds of human reasoning with novel algorithms that cannot be done with both other well-researched methods \cite[Sec.~6.7]{Gardenfors2000a}. Furthermore it can serve as representational format to express semantic relations for the semantic web \cite{Gardenfors2004} with a richer structure than classical ontologies (\eg RDS, OWL or WordNet) and thus allowing more than deductive reasoning and based on strict is-a relationships and explicit, unambiguous, universal truths.

\subsection*{Definition}

In conceptual spaces, concepts are represented as convex regions in domain-specific, human-interpretable spaces. \autoref{fig:apple_cs} represents a sample space for the concept of \textit{apple}, such that every instance of an apple is thus a vector that lies inside the region of the concept. This allows for high-level reasoning: The question \textit{\q{Will an apple fit into my bag?}} can be answered by checking if the \textit{size} dimension of the region is smaller than the dimension of the bag.

\begin{figure}[H]
	\centering
	\includegraphics[width=\textwidth]{graphics/stolenfigures/apple_space.png}
	\slcaption{
		Inner form of a Conceptual Space for an apple, displayed as product of different properties, which are convex regions in different quality domain spaces. Reprinted from \cite{Hernandez-Conde2017}, who adapted it from \cite{Fiorini2013}.
	}
    \label{fig:apple_cs}
\end{figure}



\fbox{\begin{minipage}{40em}
    \newtheorem*{theorem*}{Conceptual Space}
    \begin{theorem*}
        A conceptual space is a geometric structure used to encode the meaning of natural language terms, properties and concepts. The metric space is spanned by \emph{quality dimensions} denoting basic domain-specific properties based on perception or sub-symbolic processing. Natural language categories (\emph{concepts}) correspond to convex regions, whereas points denote individual objects (instances/\emph{entities}, allowing for geometric solutions to commonsense reasoning tasks such as \emph{betweeness} or \emph{induction}.
        % \cite{Derrac2015}: "Conceptual spaces \cite{Gardenfors2000a} are metric spaces which are used to encode the meaning of natural language concepts and properties."
    \end{theorem*}
\end{minipage}}
\label{sec:csdefinition}


Formally defined, a conceputal spaces needs the following definitions:
\begin{description}
    \item[Quality Dimensions] are atomic units of perception. Some of these are necessarily linked (such as hue and saturation), making them \textit{integral}, whereas others (\eg. temperature and weight) are \textit{seperable}. Typically each dimension corresponds to a primitive cognitive feature.
    \item[Domain] A set of integral dimensions that are seperable from others, like the \textit{color} domain made up from hue, saturation and value. Conceptual spaces are grouped into several low-dimensional subspaces according to these domains.
    \item[Similarity] is defined as inverse distance, which requries a metric. A distinction can be made for the aggregation of integral and separable dimensions. 
    \item[Betweenness] An object Y is between two other objects X and Z iff d(x,y) + d(y,z) = d(x,z).
    \item[Natural Properties (\textit{criterion P} \cite{Gardenfors2000a})] are defined as convex regions of a domain in a conceptual space. A convex region has the property that an interpolation between any two points in this region is necessarily also in this region. 
    \item[Concepts (\textit{criterion C} \cite{Gardenfors2000a})] are combinations of (potentially correlated) properties. \q{A concept is represented as a set of convex regions in a number of
    domains together with a prominence assignment to the domains and information about how the regions in different domains are correlated} \cite[8]{Gardenfors2004}
    \item[Entities] are specific instances (tokens) of a concept, encoded as points. 
    \item[Context] can be modelled in a CS by weighting certain dimensions higher than others, influencing distance and how concepts are fromed from properties.
\end{description}

Some corollaries: 

\begin{itemize}
    \item Each conceptual space contains only items for which the space's dimensions make sense, so you wouldn't find kings in a conceptual space of cabbages.
    \item Concepts roughly correspond to (non-proper) nouns, adjectives to properties and proper nouns (the name of a particular person, place, organization, or thing to points.)
    \item From the criterion of convexity for natural properties and the definition of betweenness, it follows that if an object Y is between X and Z, and both X and Z have a property, Y must also have this property.
    \item Relative properties can be defined as regions on a relative scale - the property "\textit{tall}" acccordingly can be defined to be true iff the entity is in the top 33\% \wrt the size-property of all relevant objects.
\end{itemize}

% \includeMD{pandoc_generated_latex/2_2_conceptualspaces}
\todoparagraph{look if I want to take a paragraph from 22conceptualspacesMD}

\subsection{Data-Driven Generation of Conceptual Spaces}
\label{sec:generate_cs}

So far, the area of application for Conceptual Spaces has been small. Most of the current works that rely on conceptual spaces create custom \textit{phenonemal} spaces for semantic domains, where quality dimensions are chosen by the researchers (eg \cite{Schockaert2011}). 

The previous section has shown that \glspl{cs} can serve as a framework allowing to allow for interpretable classifiers that allow for explainable recommendation based on geometric reasoning, replacing the need to manually create knowledge bases. If however now one needs to manually create these spaces, not much was gained. The work of \cite{Derrac2015} is a technique that alllows to automatically generate spaces from pairwise dissimilarites of a corpus of texts in a data-driven fashion. In his book, Gärdenfors provided several suggestions how one could build spaces from high-dimensional input neurons. One of these was to use the \gls{mds} algorithm for that creates a euclidian space from pairwise distance matrices, such as an individual's assessement of similarities. The work of \cite{Derrac2015} basically follows this suggestion using classical AI algorithms. % and \cite{Ager2018} and \cite{Alshaikh2020} provided some useful additions for it without changing the main logic. So, we'll work with their algorithm, also only making small improvemenents. So the two main areas of work are implementing the original algorithm, and changing small details of it where most appropriate such that it works well for the domain we're interested in.

To create the space, the authors unsupervisedly extract words of the text descriptions of corpus of \glspl{entity} of a specified domain. These words serve as candidates for semantic feature directions of a conceptual space. To find out which candidates are useful as features, they first embed all entities in a vector space. To identify which of the candidates constitute meaningful features, they create create a linear classifier for each of the candidates that splits embeddings of descriptions that contain the word from those that do not. Those words for which the classification performance performs well enough are considered meaningful features.  

In their paper, the authors create domain-specific conceptual spaces for three domains that allowed easy collection of text corpora, which were movies and their \gls{imdb}-reviews, place-types and tags of photos at these places and wines and their reviews on a respective platform. Each movie, place-type or wine will henceforth be termed \gls{entity}. A representation of a movie is then generated from the \gls{bow} of the descriptions of the individual movies, leading to a very high-dimensional and sparse representation for all movies. To make the representations less sparse and more meaningful, the words in the \gls{bow} are subsequently \gls{ppmi}-weighted, which (similar to \gls{tf-idf}) weights words that appear often in the description of a particular movie while being infrequent in the corpus overall high, ensuring that discriminative words are more relevant in the embedding.This weighted \gls{bow} is however no Euclidian space, which is why the authors subsequently use \gls{mds}, a dimensionality reduction technique that creates a Euclidian space while ensuring that original distances are preserved as well as possible. This space already allows for interpretable geometric reasoning such as betweeness, but its directions are not interpretable human concepts. To find these, the authors assume that words that describe relevant features of the respective entites appear among their descriptions and that words describing meaningful features correlate with good classifier performance in separating entities based on them. To classify, the authors then use linear classifiers such as \glspl{svm}. 

Consider the domain of movies, and the word "scary" as candidate feature direction. The movie-embeddings are grouped into those that contain the words and those that do no, and finds a hyperplane that divides both groups. The advantage of linear classifiers is that they create a linear hyperplane that best separates positive from negative entities. The orthogonal of that hyperplane is a vector, which can serve as feature axis: The distance of orthogonally projecting an entity onto this vector induces a ranking of entities. The further away an entities' embedding is from the decision surface on the positive side, the more this feature entity. To assess the performance, \cite{Derrac2015} use Cohen's kappa measure to compare the ranking induced by the plane with the number of occurences of the word for the entity. The better these rankings match, the higher the likelyhood that a good feature direction was found. To reduce the number of resulting features, they are subsequently clustered based on the similarity of their orthogonals before the mean direction of this feature-cluster is calculated. 

The final \textbf{feature-based representation} is generated by representing each entity as vector whose individual components corespond to the ranking of this entity compared to all others for each of the most salient feature directions. Thus, in the final feature-based representation only the relation of each entity in relation to all others with respect to the salient directions is relevant. Given that respective directions are not orthogonal \cite[22]{Derrac2015} and that rankings are only ordinal (where distances are not quantifiable), the final embedding loses some geometric properties such as the Euclidean distance metric, but gains interpretable directions.

The algorithm is optimized to \textit{look good to humans}, meaning there are no staright-forward metrics or obvious evaluations. To evaluate its performance quantitatively, we will test if it is possible to use the detected semantic directions to classify human concepts among the data, such as the genre assigned to a movie.

\includeMD{pandoc_generated_latex/2_3_datadrivengeneration}


\subsection{Explainable Reasoning with Conceptual Spaces}
% was "computational reasoning"
\label{sec:reasoning}

\todoparagraph{So the commonsense reasoning classifier fromd errac are betweeness-based, relational-similarity-based, and classification based on salient properties, a fortiori reasoning "if X is scarier than the shining it is likely a horror film". FOr the last one they then consider the salient directions that they extracted. For the others the directions are irrelevant}


\todoparagraph{Ager bringt viele gute Punkte regarding warum der bums sinnvoll fur explainable recommenders ist}

\todoparagraph{important for us is that we} don't have ONE SINGLE SIMILARTIY; BUT THAT IT's context-dependent!!!!

\todoparagraph{However a short paragraph about reasoning-based classifiers and the respective intutitive explanations for known classifiers}

The goal of this thesis is to provide explainable recommendation for educational resources. This section elaborates how the framework of \glspl{cs} allows to computationally model commonsense reasoning through analytic geometry and algebra.

\todoparagraph{We are looking how symbolistic stuff relates to CS}

According to \cite{Gardenfors2000a}, Representations don't need to be similiar to the objects they represent, but the *similarity relations of the representations* should correspond to those of the objects they represent


Unlike many ther NLP approaches that rely on embedding (see \autoref{sec:embeddings}), in a Conceptual Space, natural language terms are not modelled as points or vectors, but as convex regions. \todoparagraph{advantages: Stuff below!}
% * It allows \q{to distinguish borderline instances of a concept from more prototypical instances, by taking the view that instances which are closer to the center of a region are more typical} \cite{Derrac2015} (they cite \cite{Gardenfors2000a})
% * There are really good Region-based models, eg \cite{Erk2009}
% * stuff liek subsumption, mutual exclusiveness etc are obivous (see blow)

\subsubsection*{Categories and Ontologies}
\label{sec:ontology_rcc}

% CS and with it ontologies "automatically" (BIG question mark) arise from prototypes + metric domain + voronoi tesselation

\todoparagraph{Vor allem sollte hier ruberkommen warum das gut fur mich und meinen recommender fur educational resources ist argh}

Logic-Based reasoning/inference can do many things already, however it requires the knowledge to be encoded in logic (a lot of manual work) and doesn't allow for fuzzyness.
In formal logic/ontologies/lexical databases, semantic relations of concepts are explicitly modelled. The \gls{rcc} \cite{Cohn1997a} links these relations to their geometric interpretation, providing a bridge between this and Conceptual Spaces \cite{Gardenfors2001}. Once you have created the structure, the following emerges automatically:

\vspace{2ex}

\begin{tabularx}{1.05\textwidth}{P{0.16\textwidth}|P{0.25\textwidth}|P{0.25\textwidth}|X}
    Ontology Relation & Other Names        & RCC5 \cite{Cohn1997a} analog / \textit{Geometric equivalent} & Example \\ \midrule

    Type Identity     & {\scriptsize Equality of Concepts, Synonymy } & Identical Regions (EQ)      & Animals with a liver \& Animals with a heart \\ 

    Subsumption       & {\scriptsize Hyponyms/ Hypernyms, \textit{is-a-relationship}, Concept Hierachies, Taxonomies }
                                           & Proper Parts(PP, PP\textsuperscript{-1}), \textit{Subregions}
                                                                          & \textit{Every pizzaria is a restaurant} \\  
    
    Mutual 
    Exclusiveness     &                    & Discrete Regions (DR), \textit{unconnected/disjoint regions}
                                                                          & \textit{No Restaurant can be a beach} \\  

    Overlapping 
    Concepts          &                    & Partial Overlap (PO)         & \textit{Some bars serve wine, but not all} \\  

    Opposites         &                    & \textit{Set inverse}         & Humans \& Non-human animals \\ 

    Token Identity    & {\scriptsize Equality of Names, Synonymy } & \textit{Equal coordinates}   & Morning star \& Venus \\ 
    Meronymy / Holonymy & {\scriptsize \textit{part-whole realationship} }& \textit{Product spaces} (see \cite{Fiorini2013}        & Trees \& Leaves
\end{tabularx}

regions and betweeness:
* convex hull as the set of all convex combinations of a point
* can be done by linear programming with polynomial cost


\vspace{4ex}

% Also: characteristics of properties (transitivity, symmetry) (from intrinsic features of dimensions (time is linear because the dimension is linear))

So for example the validity of "a robin is a bird" is encoded by it being a subregion. So, no reason for a symbolic inference when using the richer structure of CS instead of ontologies.

So a CS can model all semantic relations of classical ontologies/formal logic/symbolistic approaches/knowledge bases. However as it encodes more than just a taxonomy of concepts, it for higher/more forms of inference and (common-sense) reasoning, among others by enabling for interpolation and extrapolation of knowledge. These are:

\subsubsection*{Similarity-based reasoning}

\label{sec:similaritybasedreasoning}

As discussed already in \autoref{sec:amazonalgo}, modern recommendation algorithms often rely on similarity-based reasoning by suggesting that users that liked and item may also like similar items (collaborative filtering). In terms of classification, this corresponds to the 1-Neirest-Neighbor approach, where an object is assigned the class of the most similar item:\footnote{1-NN: \textit{Y is of the same class as X because X closest to Y}}

\noindent
\begin{minipage}{.6\textwidth}
  \syllogism{Alice likes \textit{the Lord of the Rings} \\
           \textit{The Hobbit} and \textit{Lord of the Rings} are similar}{Alice will probably like \textit{The Hobbit}}
\end{minipage}% 
\begin{minipage}{.4\textwidth}
  \syllogism{\textbf{A} has property \textbf{x} \\
             \textbf{B} is  similar to \textbf{A}}
             {\textbf{B} likely also has property \textbf{x}}
\end{minipage}% 

\vspace{2ex}

This is something that cannot be done by classical logic, which can not model degrees of something but only universal full truths. Another advantage is that it's easy to train, however a disadvantage is that it requires enough similar concepts which may not be given This algorithm however lacks \textit{explainability}: The employed distance function does not encode \textit{in what respect} two items are similar. For human reasoning however, there is no \textbf{overall Similarity} - instead similarity is relative to a domain and only meaningful in context \cite{Goodman1972-GOOPAP-3}: \q{any measurement of similarity is based on assumptions concerning the properties of a similarity relation} \cite[110]{Gardenfors2000a}. In a conceptual space, this can be modelled: The distance function can give weight for certain dimensions depending on context or objects of different concepts can be considered similar if they share enough properties. Most importantly however, in a CS a system for recommendation can ask users what dimensions of a given entity the user liked to suggest items that are similar in that regard.
% a system serving me a similar wine to the one I want that is not available can ask me what dimensions of that wine were important before suggesting new ones.

\subsubsection*{Induction}

Another very important tool of human common-sense reasoning is abstraction/generalization/ \textbf{induction}, going from single observations to general rules. For that, we need to decide which properties of the respective observation are relevant, distilling sensible information from the receptors from unimportant information to make inferences from limited information about an object? The connectionistic answer to that would be "pattern matching", and ML algorithms work extremely well for that, however this lacks explainability, and also we want to model the underlying algorithm and the underlying patterns. All three levels can model some kinds of induction and generalization, remember they are not in conflict. On the Conceptual Level you can eg model \textit{If two different entities of the same class have a property, maybe all entities of that class have a property}, which would be \textit{categroical inductive inverences}

\syllogism{Grizzly bears love onions \\
Polar bears love onions}
{All bears love onions \cite[226]{Gardenfors2000a}}

Other kinds of inductive reasoning we can model:

\textbf{Interpolative Reasoning}

\cite{Schockaert2011}: \q{intermediary conditions lead to intermediary conclusions} 

\noindent
\begin{minipage}{.6\textwidth}
  \syllogism{Cars have tires \\
             Bikes have tires \\
             Motorcycles are between cars and bikes
             }{Motorcycles likely also have tires}
\end{minipage}% 
\begin{minipage}{.4\textwidth}
    \syllogism{\textbf{A} has property \textbf{x} \\
               \textbf{C} has property \textbf{x} \\
               \textbf{B} is conceptually between \textbf{A} and \textbf{C}}
               {\textbf{B} likely also has property \textbf{x}}
\end{minipage}% 

% Source looking into it ([28] of [DESC15]): M. Abraham, D. Gabbay, U. Schild, Analysis of the talmudic argumen- tum a fortiori inference rule (kal vachomer) using matrix abduction, Studia Logica 92 (2009) 281–364.

\vspace{2ex}
\textbf{A fortiori reasoning}

\noindent
\begin{minipage}{.6\textwidth}
    \syllogism{\textit{The shining} is a horror film \\
            \textit{The shining} is scary \\
            \textit{It} is more scary than \textit{The shining}}
            {\phantom{penis} \\ 
            \textit{It} is likely also a horror film}
\end{minipage}% 
\begin{minipage}{.4\textwidth}
    \syllogism{\textbf{A} has property \textbf{x} \\
               \textbf{B} is \textit{more severe} than \textbf{A} \\ \phantom{penis}}
               {\textbf{B} likely also has property \textbf{x} \\ 
               \vspace{-0.5em} or a \textit{more severe} property}
\end{minipage}% 

\vspace{2ex}

\textbf{Extrapolative reasoning (Analogical Reasoning)} \cite{Schockaert2011}: \q{Analogous changes in the conditions should lead to analogous changes in the conclusion}, or \q{Concepts which differ in analogous ways have properties which differ in analaogous ways}. This is \textit{relational similarity}: If the analogical proportion a : b :: c : d holds, the pairs (a, b) and (c, d) are called relationally similar. Simply maps to geometric parallelism. 

I wrote down the syllogisms here, however that's not the only way to apply it, I can also "look for things that". In the case of analogies, I can ask "what is the thing that behaves to waterski like snowboarding to skiing"?

\textbf{Metaphors / Metonymy} (saying "the Pentagon" when referring to the US military) follow from a fortiori/analogical reasoning


\todoparagraph{I am referring to this figure in the discussion!}
\begin{figure}[H]
	\centering
	\includegraphics[width=\textwidth]{graphics/stolenfigures/reasoning_samples.png}
	\slcaption{
		Different forms of reasoning represented graphically. \todoparagraph{Holy shit remove me}
	}
    \label{fig:graphic_reasoning}
\end{figure}

To automate these forms of inference, we need a richer form of knowledge than what is available in calssical logic: We need a notion of / Information about \textbf{Betweenness} and \textbf{Directionality}.

To model the stuff in \autocite{fig:graphic_reasoning}, we need a good metric, and what we see there definitely sounds euclidian!

\textbf{Summarized, CS are better than symbolism because we save the inference engine (plus we can extend and simulate knowledge bases with commonsense reasoning, just like humans deal with incomplete knowledge!) and better than connectionism because they are explainable. Klar soweit?!}

\todoparagraph{The problems of polysemy und synonymy for classical stuff, see fundamental information retrieval problem}


% \includeMD{pandoc_generated_latex/2_4_typesofreasoning}

\section{Other Related Work}
\label{sec:otherwork}

This thesis focuses on the aforementioned algorithm, primarily considering \cite{Derrac2015} and on top of that only two follow-up works: \cite{Ager2018, Alshaikh2020}, which have shown to provide useful extensions for it without changing its core logic. This shall by no means mean that these are the only ones that could be considered.

\paragraph{Tag Genome} 

By far the closest to what we do is the algorithm of \textcite{VISR12}, who generate a so-called tag genome for the domain of movies supervisedly based on keywords that users have assigned manually. Their algorithm takes these binary assignments and creates a dense representation that encodes a degree of relevance for each combination of movie and tag. Furthermore, they create a dedicated movie recommendation system on basis of this (interface reprinted in \autoref{fig:movietuner}). This system provides explainable recommendation based on these tags, allowing users to request recommendations for movies such as \textit{\q{I‘d like something less violent than Reservoir Dogs}} \cite[3]{VISR12}. Not only is their application exactly what is being demanded here, but the algorithm itself also performed significantly better than the one of \textcite{Derrac2015} in a human study of the latter, in which they directly compared the techniques by asking subjects which of the respectively extracted keywords better describes the difference between two movies \cite[44]{Derrac2015}. Considering however that \gencite{VISR12} algorithm is supervised and requires data which does not exist for our domain, it cannot be applied in our case. On the contrary, the final results of their algorithm are preferred by users over the ones generated with the algorithm considered in this work, but structurally exactly equal. This provides clear evidence that the desired application of this work is possible, albeit of lower quality than what their work achieved.

Generally, what is being done here corresponds to \textbf{Representation Learning}, whose aim is to discover the inherent semantic structure of a representation unsupervisedly \cite{Dayan1995}. More specifically \textbf{Disentangled Representation Learning}, where only salient attributes relevant to the task at hand should be extracted, which means finding latent embeddings whose dimensions are meaningful interpretable features. Generative Adversarial Networks \cite{Goodfellow2014} or Variational Autoencoders \cite{Kingma2013} are modern techniques that are good at finding latent information in images. Especially InfoGAN \cite{Chen2016} should be named, which can extract interpretable features such as pose, hairstyle, prensence of glasses and emotions from images unsupervisedly. 

\paragraph{LDA} 
\label{sec:lda}

In the realm of \gls{nlp}, this also relates to \textbf{Topic Modeling}, which aims to extract multiple hidden themes from a given text corpus by discovering groups of co-occuring words unsupervisedly. A well-known algorithm for this is Latent Dirichlet Allocation (LDA) \cite{Blei2003}, which represents documents by its salient \textit{topics}, each of which being a cluster of natural language terms. This technique bases on the assumption that each text consists of various topics, which are in turn made up by various keywords, making it possible to represent texts as multinomial distribution over latent topics which are aggregations of these keywords. Assuming a hierachical bayesian distribution where each text of a corpus is represented as mixture of topics it contains, their unsupervised algorithm extracts these by approximating the underlying infinite mixture of topic with an expectation-maximization (EM) algorithm. This yields a representation where each text is explicitly represented by the most propable words according to this distribution for a finite number of most probable topics. The algorithm finds use in text classification and collaborative filtering, but relies on unflexible \gls{bow} representations, making it hard to incorporate additional information such as correlations between topics \cite{Ager2018}.

\todoparagraph{LDA is still sometimes better than this algo} - see \autoref{tab:f1_mainalgos_me_short}


\paragraph{Academic Interests Recommender}
\label{sec:sidbert}
Regarding the used domain, there is already a system incorporated into the Siddata-\gls{dsa} that aids students by finding and recommending educational resources. SidBERT \cite{Schrumpf2021DELPHI} extracts implicit information from courses and other learning material by their title by categorizing them into one of 905 classes derived from the third or fourth level of the \gls{ddc} \cite{Dewey1876}, a hierachical tree stucture system commonly used to categorize library books. SidBERT uses the same dataset as this work and classifies with a custom classification head ontop of a \gls{bert}-encoder \gls{ann} which is trained on 1.3 million book titles collected from three universities as well as the German National Library, currently achieving 45.2\% test accuracy (62.2\% recall) among 905 classes.

\paragraph{Variations of this Algorithm}
\label{sec:algo_variants}

There are also techniques that extend the algorithm of \textcite{Derrac2015}: \textit{Alshaikh et al.} \cite{Alshaikh2019, Alshaikh2021} use this algorithm as one of their steps and create a similar algorithm to find disentangled features, to be more in line with the original definition of Conceptual Spaces,which requires low-dimensional domain-specific subspaces. Regarding other unsupervised ways to create Conceptual Spaces, Peter Gärdenfors himself suggested in his book \cite{Gardenfors2000a} to use self-organizing maps (Kohonen-Networks \cite{Kohonen1997}) instead of classical \gls{nlp} algorithms and \gls{mds} to unsupervisedly create concpetual spaces. And finally, the whole concepts of vector-space models for words \cite{Mikolov2013} and texts \cite{Le2014,Devlin2019} is related in that represents the meaing of terms, phrases or documents by embedding them in a vector space. However these have arbitrary non-interpretable dimensions and are no metric spaces, thus having no relation of geometry and meaning for \eg betweeness or analogical reasoning, which will be eloborated in the next section.

\todoparagraph{Keep Kohonen nets, Im mentioning that in discussion}
\todoparagraph{BERT as best language model}



\todoparagraph{For more related work read the intros and corresponding sections of Ager and Alshaikh!}











\section{Relevant Algorithms and Techniques}
\label{sec:required_algos}

\todoparagraph{dont forget the links for lsa und lda - are in lsa-long-md }

Thus far, we have described the base algorithm which this thesis replicates. Before describing each of its step in detail, it is useful to get a grasp of some general concepts relevant for it and its components. Furthermore, we will to put it into the context of the field of computational \gls{nlp} and elucidate some of the required theoretical foundation, and finally quickly look at what other techniques can be used for some of its components. \todoparagraph{Also, we need some general linear algebra bc after all these are vector spaces, so let's look quickly at projecting and playing around with coordinates}

\subsection{Classical Vector Space Construction}
\label{sec:vsm_construction}

\todoparagraph{Important that it becomes clear that the distribution of unimportant (stop-) words is arbitrary whereas important words have an influence on the position in the space}


\textcite{Lowe} conceived a general framework to construct vector spaces from texts, splitting the process into the steps of first counting the token frequencies, then transforming the raw counts into more useful \gls{quant} measures (see \ref{sec:word_count_techniques}) %adjust the weights of the elements in the matrix, because common words will have high frequencies yet are less informative than rare words.
 %TODO: talk about a fucking DOC-TERM-MATRIX, A MATRIX OF FREQUENCIES!!!
and smoothing the space using dimensionality reduction, before calculating the similarities on the resulting embedding. While the considered algorithm requires more steps before calculating the similarities to allow for domain-specific and more complex forms of reasoning, it also adheres to this structure, as will be shown in section~\nameparanref{sec:generate_vectorspaces}.

\paragraph{Bag-of-Words Hypothesis and Distributional Hypothesis}
\label{sec:bow_hypothesis}

\begin{quote}
    \q{you shall know a word by the company it keeps} \hfill \textcite{firth57synopsis}
\end{quote}

\todoparagraph{Grundlage fur vector-space models} - similar words occur in similar contexts.  % \url{https://en.wikipedia.org/wiki/Distributional_semantics#Distributional_hypothesis}
% Eg. LSA assumes that words that are close in meaning will occur in similar pieces of text

Idea: Words that occur in similar surroundings (pieces of text) have similar meanings (distributional hypothesis), and texts that contain similar distributions of words are similear (bag-of-words hypothesis).

% "Hunde sind süß" und "katzen sind süß" -> in some context, the  "hunde" and "katzen" are similar.

According to \cite{Turney2010}, vector-space models fall into different categories depending on if the similarity of documents (Term-Document-Model) or of words (Word-Context-Model) is in question.  The former relies on the \textbf{bag-of-word hypothesis}, stating that documents with similar words have similar meaning. The Word-Context-Model relies on the \textbf{distributional hypothesis}, stating words that occur in similar context have similar meanings.



\todoparagraph{zu 90 prozent raus hiermit, und falls doch drin dann kleiner}
\begin{quote}
	\q{After the text has been tokenized and (optionally) normalized and annotated, the first step is to generate a matrix of frequencies. Second, we may want to adjust the weights of the elements in the matrix, because common words will have high frequencies, yet they are less informative than rare words. Third, we may want to smooth the matrix, to reduce the amount of random noise and to fill in some of the zero elements in a sparse matrix. Fourth, there are many different ways to measure the similarity of two vectors.} \hfill \textcite{Turney2010}
\end{quote}

\todoparagraph{Mention Distributional Hypothesis}

The algorithm consists of several steps/components, each of it of course is an algorithm in itself. The first steps are basically classical linguistic tools: Preprocessing \textrightarrow ~Bag-of-NGrams \textrightarrow ~Quantification \textrightarrow ~Frequency Matrix \textrightarrow ~Dimensionality Reduction (MDS) - followed by SVM-Classification\&Scoring \textrightarrow ~Clustering \textrightarrow ~Re-Embedding.

Many \gls{nlp} tasks rely on documents being represented as vectors, such as such as Information Retrieval, Recommendation, Text Classification, Translation, Sentiment Analysis among others \cite{Smith2017,bird2009natural,Devlin2019,Le2014,Mikolov2013a,Turney2010,Guo,Chen2018,Maas2011}. The process of turning a collection of texts document into numerical feature vectors is referred to as \textit{vectorization}. Let us explore some theoretical basis for that.

According to \textcite{Turney2010} (who in turn base their work on \textcite{Lowe}), the construction of a \gls{vsm} from texts can be decomposed into a four steps:\footnote{When considering neural embeddings such as \gls{word2vec}, the separation of these steps is hidden in the algorithm and not as distinct, but the principles hold also in these techniques.}

\begin{description}
    \item[1) Building the Frequency Matrix] which starts with preprocessing such as tokenization followed by normalizing and possibly \gls{lemma}tizing the tokens amongst many other possible techniques, before counting frequencies of either words or \glspl{ngram}, yielding a matrix of \glspl{bow}.
    \item[2) Transforming Raw Frequency Counts] \q{[B]ecause common words will have high frequencies, yet they are less informative than rare words} \cite{Turney2010}, it may make sense to adjust the weights of the elements of the frequency matrix.
    \item[3) Smoothing the Frequency Matrix] So far, the matrix is noisy, extremely sparse and extremely high-dimensional. Dimensionality Reduction helps to counter all three issues.
    \item[4) Calculating Similarities] of individual vectors as final step and aim of most algorithms. This can be done in various ways, a classical technique is to use the \gls{cos}.
\end{description}

Importantly, our algorithm differs from this four-step-process by injecting several additional steps before calculating similarities. This is because we hold the notion that similarity is necessarily context-dependent and there is no overall similarity (see \autoref{sec:reasoning}), which requires additional dissection of the final step.

Lets look at what all this means.



\subsubsection{Bag-of-ngrams representation}
\label{sec:techniques:bow}

\includeMD{pandoc_generated_latex/2_bow}


\subsubsection{Word-weighting techniques}
\label{sec:word_count_techniques}

\todoparagraph{TODO: term? phrase? n-gram?}

When comparing the \gls{bow}-representations of texts, it is reasonable to give more weight to \emph{surprising} words than to expected ones. \q{The hypothesis is that surprising events, if shared by two vectors, are more discriminative of the similarity between the vectors than less surprising events.} \cite[156]{Turney2010} Another crucial reason is, that individual texts in the corpus are of drastically varying length, so longer entities would naturally dominate shorter ones when only comparing the raw counts - considering relative frequencies instead of absolute ones alleviates such problems. Because of these reasons, in the algorithm it will often be talked about \glspl{quant}. The algorithms explained below transforms the raw frequency-counts of a document and an \gls{ngram} into some \emph{score}, dependent on the number of occurences of this term in this document as well as the counts of other \glspl{ngram} and other documents. This score is henceforth called a \gls{quant}.

Let us consider term $t$, corpus $C$, document $d \in C$ (represented as \gls{bow}). Then:\\
term-frequency $\text{tf}_{t,d}$: How often $t$ occurs in $d$\\
document-frequency $\text{df}_t$: How many documents $\in C$ contain $t$\\
summed term-frequency $\text{df}_{t,*} = \sum_{d' \in C} \text{tf}_{t,d'}$: How often $t$ occurs in any document $\in C$

\paragraph{Tf-Idf} The most well-known technique formalizing this concept is the \gls{tf-idf}, which gives a term-document pair a higher weight if the term is generally rare in the corpus (low \textit{df}) and frequent in the respective document (high \textit{tf}):
\vspace{-2.5ex}
$$ w_{t,d} = \text{\textit{tf}}_{t,d} * log(\frac{|C|}{df_t}) $$

%see also: https://towardsdatascience.com/3-basic-approaches-in-bag-of-words-which-are-better-than-word-embeddings-c2cbc7398016
%\cite{Turney2010} (sec 4.2): Salton and Buckley (1988) defined a large family of tf-idf weighting functions and evaluated them on information re- trieval tasks, demonstrating that tf-idf weighting can yield significant improvements over raw frequency

\paragraph{PPMI}
% See also: https://stackoverflow.com/a/58725695/5122790

\textcite{Turney2010} suggested to use the \gls{ppmi} measure instead of \gls{tf-idf} to weight the counts in \glspl{doctermmat}, relying on \cite{Bullinaria2007}'s work taking into account psychological models to extract information about lexical semantics from co-occurence statistics. According to \cite{Turney2010,Bullinaria2007}, \gls{ppmi} performs most plausible to measure semantic similarity in word-context matrices compared to human evaluation. For that reason, \textcite{Derrac2015} and its follow-up works \cite{Ager2018,Alshaikh2020} rely solely on this technique. %(sec 4.2) of \cite{Turney2010}: Bullinaria and Levy (2007) demonstrated that PPMI performs better than a wide variety of other weighting approaches when measuring semantic similarity with word-context matrices.
Like tf-idf, it weights terms that are strongly associated with a document highly by favoring terms frequently associated with document $d$ while infrequent in the corpus overall. For that, it uses the \textit{log} of the probability of the term-document-combination, normalized by the probability of this $d$ with any $t$ and this $t$ with any $d$:
\vspace{-2.5ex}
\begin{align*}
   w_{t,d} &= max\left(0, log\left( \frac{p_{d,t}}{\sum_{t'}p_{d,t'}*\sum_{d'}p_{d',t}} \right) \right) \\
   p_{d,t} &= \frac{\text{tf}_{t,d}}{\sum_{d'}\sum_{t'} \text{tf}_{t',d'}}
\end{align*}

\todoparagraph{In practical terms}, efficient calculation of the PPMI-score for high-dimensional \glspl{doctermmat} requires excessives amounts of RAM, as it does not appear to be implemented in any major Python-library and its calculating requires multiplication of huge matrices.


\subsubsection{Dimensionality Reduction and Latent Space Embedding}
\label{sec:dim_red}

Next step is to smooth the frequency matrix. Goal is to keep comparison performance high, but make the  process faster and also ignore irrelevant noise \cite{Turney2010}.

We start with sparse vectors, and then \q{some form of dimensionality reduction is typically used to obtain vectors whose components correspond to concepts.} \cite{Derrac2015}


\paragraph*{LSI/LSA}

\textcite{Deerwester}

In fact, when using the right algorithms this has the additional effect to also find latent topics from tehse texts, which IMPOROVES similarity measurements bc now it's as if the latent stuff is in there

LSI relies on the \gls{distribhyp} in that it decomposes the matrix into the product of three matrices and then compresses them. Similar to AutoEncoders, if there is hidden information in the text than that will be used for this compression! 

\paragraph*{MDS}
\label{sec:mds}


\subsection{Other important Techniques}


\includeMD{pandoc_generated_latex/2_othertechniques}



%METHODS
	%(Algorithmus & Datensatz)
	\chapter{Methods}

Direkt am Anfang schreiben dass ich halt auf den main algorithmus eingehe und das laut meiner research diese 3 paper am besten den main algo beschreiben (bzw sinnvoll erweitern) - was nicht heißt dass das die einzigen in dem kontext sind, Alshaikh2019 bspw nutzen ja den main algorithmus, aber ja nur als komponente, und haben andere Ziele was sie dann damit machen

Im folgenden gibt es neben Datasets 2 main sections: algoritm and architecture. Dass Algorithm und Architecture 2 subsection von methods sind ist halt "Der allgemeine Algorithmus und die spezifische Anwendung" Warum Architecture section? es kostet extrem viel zeit die schwammigen formulierungen in den papern genau zu verstehen, man probiert super oft falsche parameter-kombinationen aus etc etc, es ist halt ein riesiger ewig langer lernprozess den man von vorne machen müsste wenn man es nachimplementieren möchte, ich hätte mir gewünscht die authors hätten darüber mehr worte verloren, and also the scalable reproducible open-science part. And also - it took me a shitton of time, way more than working on the algorithm (but NOW it can run so easily on the grid and all param-combis simultaneously, ..), so this is what you'll get.


\section{Datasets}

All considered algorithms \mainalgos use a dataset of 15.000 movies and their reviews on IMDB, as well as a placetypes-dataset to evaluate their methods. The former consists of the concatenation of all available reviews for movies from IMDB\footnote{\url{https://www.imdb.com/}}, whereas the latter a collection of tags from photos uploaded to Flickr\footnote{\url{https://www.flickr.com}} that co-occur with a certain placetype. Other considered datasets are wine reviews, posts to certain newsgroups and another IMDB-review-dataset (see \tref{tab:all_datasets}). 

All of these datasets have in common that they are made up from a collection of independent texts or tags, created by different people. This means, that the more obvious or distinct a property of the respective entity is, the more often words describing that property will be used as tag or as part of the review. For example, a movie that is \emph{scary} to a lot of people will lead to many reviews mentioning that, which means that the word scary (or other words commonly co-occuring with it) will have a high count in the concatenation of that review. The algorithm from \cite{Derrac2015} heavily leans on this property by using the (relative) frequency of certain words as signal for the importance of the concept they may refer to. 

The main considered dataset of this thesis unfortunately does not share this property, as the texts that belong to an entity are not collected from different independent texts, but solely from the description of that entity - while it may be the case that the more \emph{mathematical} a course is, the more often the word \emph{math} occurs in it description, but the correlation is likely not as prominent as in the aforementioned datasets. Interestingly, \cite{Alshaikh2020} also used three datasets that only use a sort of description for an entity: its Wikipedia\footnote{\url{https://en.wikipedia.org/}}-article.


\subsection{SIDDATA-courses}

% TODO: bei dataset section darauf verweisen dass große teile des siddata-datasets mit gtranslate übersetzt wurden und auf den entsprechenden anhang verweiseen

% * Steht ja schon woanders dass mein Datensatz anders ist als concatenated-movie-reviews und ich deswegen nicht einfach "je öfter 'scary' desco scarier" machen kann. Da gibt's several ways mit umzugehen
	% * Das sich-die-richtigen-wörter-per-candidate-svm-bootstrappen
	% * Mit LSI rausfinden welche Terme genausogut in dem Text hätten vorkommen können (hab ich auch irgendwo schon)
	% * Explizit einfach zu gucken "Welche Terme kommen oft in den gleichen dokumenten vor" (und das inverse (steht iwo im code)), und dann ne candidate SVM für grouped terms anstelle von einzelterms machen (auch schon iwo als code)
	% * Mit Wordnet hypernyms/hyponyns und synonyms zu finden damit ebenfalls zu arbeiten (kann man wit wordnet angeben welches abstraktionsniveau ich haben will?)
	%     * Abstraktionsniveau gibt's nicht in wordnet, das heißt das richtige layer zu finden ist schwer. Was man auf jeden Fall machen kann ist die Terme zu den bases ihrer synsets umzuwandeln (dadurch wird aus "math" und "mathematics" das gleiche), aber in anderen Fällen ist es halt so dass ich die Candidate-Terms schon vorher brauche und nur sagen kann "diese entity enhält X wörter die halt hyponyms von dem Term sind"

their algorithm is tailored to concatenated-reviews or concatenated-bags-of-tags. Take their success-metric for the SVMs splitting the embedding. The more often the word "scary" comes in the concatenated reviews, the more scary the movie is. Sounds legit. The more often the people that took pictures at a particular place mentioned the "nature" of that, the more relevant "nature" is to that place. Also legit. But in the descriptions for courses that involve a lot of mathematics, it is not necessarily the case that the term "mathematics" occurs often. So due to the different nature of my dataset I have to go beyond their algorithm at some points - in this case it is probably the case that different kinds of mathematical terms actually do occur more often, so I'd need calculate these kinds of kappas not based oon a single term but ALREADY on a cluster of terms (... and I can bootstrap my way there, because after I do this I get more words to add to my cluster, rinse and repeat!)


% * Dass man theoretisch sich den task einfacher machen kann indem man nur die correctly-classified Kurse des fb-classifiers verwendet
% * MEINEN DATENSATZ mal mit den anderen vergleichen!! 
% 	* die Plots die schon drin sind beschreiben und warum der Datensatz whack ist.
% * Den ganzen "wo ist mein dataset anders als deren" Kram (teilweise schon im text, teilweise very old)
% * Woher der Datensatz kommt, dass es ja version 2 des Kurs-Datensatzes von Johannes ist (kommt von: /home/chris/Documents/UNI_neu/Masterarbeit/OTHER/study_behavior_analysis/src/data/course_data/db_dump_new/course_dump_new.csv)
% * Meine Pre-Preprocessing Schritte die da ja auch noch gut rumfiltern und rummergen beschreiben
% * Candidate-Word-Threshold: movies has samples-to-threshold value of 100, placetypes has 35, 20newsgrups has 614, so for 8000 courses any threshold from 2 to 25 seems reasonable => \cite{Derrac2015} say they intentionally kept the number of candidateterms approximate equal (at around 22.000), so to do the same I'd need a threshold of [TODO: optimal value]
% * [AGKR18] use a dataset that has fucking scipy preprocessing
% * ausrechnen "um so gut zu sein wie die, müsste der datensatz größe xyz haben"
% * Die standard-whackities des datensatzes, dass halt viele nur sind "Tutoren sind: Susi Sorglos Willi Wacker", oder "Findet statt in Raum XYZ", oder dass alle Sprachkurse die gleichen Beschreibung haben (beispiel. `....len([i for i in descriptions._descriptions if "kompetenzen entwickelt befahigen akademischen berufstypischen" in i.processed_as_string()]) == 25  ... weil es genau 25 exakt gleiche Beschreibungen gibt, für die Fremdsprachkurse. Deswegen ist up to jede 5-wort-kombination davon ein extracted keyword`)
% * Der Kappa-Score der rankigns vergleicht ist für mich ne kack metric weil ich ebennicht reviews nehme und more-occurences better-candidate heißen -> gucken wie ich stattdessen gute dimensionen und cluster finde (klingt doch so als sei accuracy/f1/... doch wichtig)

%TODO https://tex.stackexchange.com/questions/526198/table-resize-table-and-automatic-line-breaks


% \afterpage{%

\newgeometry{
	top=21mm,
	bottom=16mm,
	inner=16mm,
	outer=16mm,
} 


\begin{landscape}
	\begin{table}[]
		\resizebox{.98\textwidth}{!}{%
        \begin{tabular}{@{}llllll@{}} 
        	\toprule
        		\textbf{dataset} &
        		\textbf{contents} &
        		\textbf{preprocessing} &
        		\textbf{size} &
        		\textbf{classification classes} &
        		\textbf{candidate word threshold}
        		% & \textbf{key feature sizes} 
        		 \\ \midrule
        	\textbf{movies\tablefootnote{\label{origdsets}\url{https://www.cs.cf.ac.uk/semanticspaces/}} \cite{Derrac2015,Ager2018,Alshaikh2020} } &
				\specialcell[l]{grouped-by-movie-concatenated\\reviews for movies} & 
        		\specialcell[l]{\tabitem removed stop-words\tablefootnote{\label{fnote:stopwordlist}\url{http://snowball.tartarus.org/algorithms/english/stop.txt}} \\ \tabitem lower-cased text \\ \tabitem removed diacritics  \\ \tabitem removed punctuation} &
        		\specialcell[l]{\cite{Derrac2015}: 15000 movies \\ \cite{Ager2018,Alshaikh2020}: 13978 movies } & %Ager2018 says 15.000 - 1022 duplicates, that's the number of Alshaikh2020
        		\specialcell[l]{ \tabitem genre (23 classes)\\ \tabitem plot keywords (eg. \textit{suicide, beach}) (100 classes) \\ \tabitem age-rating certificates (6 classes)} & \specialcell[l]{\acrshort{df} $\geq 100$ \\ \textrightarrow 22 903 candidates \\ variable-length \textbf{n-grams} considered}
        		
        		\\ \midrule
        	\textbf{place types\footref{origdsets} \cite{Derrac2015,Ager2018,Alshaikh2020} } &
				\specialcell[l]{Tags of Flickr-photos that are also\\tagged with a place-type}
        		% bag-of-tags from Flickr used to describe places of a certain place-type
        		& 
        		None &
        		1383 place-types & %both in DESC15 and the follow-up paper
        		\specialcell[l]{ \tabitem category from Geonames (7 classes)\\ \tabitem category from Foursquare (9 classes)\\ \tabitem category from OpenCYC (93\cite{Derrac2015}/20\cite{Ager2018,Alshaikh2020} classes) } &
        		\specialcell[l]{\acrshort{df} $\geq 50$ \\ \textrightarrow 21\,833 candidates \\ (all words from the BoW) \\ \textbf{n-grams}: squashed all words of a tag} 
        		% & candidate-terms: 6385
        		\\ \midrule
        	\textbf{wines\footref{origdsets}\tablefootnote{\url{https://snap.stanford.edu/data/web-CellarTracker.html}} \cite{Derrac2015}} &
				\specialcell[l]{grouped-by-wine-variant-concatenated\\reviews for wines} & \specialcell[l]{\tabitem removed stop-words\footnoteref{fnote:stopwordlist} \\ \tabitem lower-cased text \\ \tabitem removed diacritics  \\ \tabitem removed punctuation} & 330 wine-varieties &
        		\textit{not performed} &
        		\specialcell[l] {\acrshort{df} $\geq 50$ \\  \textrightarrow around 6k candidates \\ variable-length \textbf{n-grams} considered }
        		\\ \midrule
        	\textbf{20 newsgroups\tablefootnote{\url{http://qwone.com/~jason/20Newsgroups}} \cite{Ager2018}} &
				\specialcell[l]{posts partitioned roughly even\\across 20 different newsgroups} &
        		\specialcell[l]{ \tabitem Headers, footers and quote metadata removed\tablefootnote{Using the scikit-learn python package, see \url{https://scikit-learn.org/0.19/datasets/twenty_newsgroups.html}} \\ \tabitem removed stopwords (using NLTK's corpus \cite{loper-bird-2002-nltk})\\ \tabitem lowercased text\\ \tabitem candidate terms: all textual and numerical tokens} &
        		18446 posts &
        		\tabitem newgroup post was submitted to (20 classes) &
        		$\geq$ 30 occurences 
        		\\ \midrule
        	\textbf{imdb sentiment\tablefootnote{\url{http://ai.stanford.edu/~amaas/data/sentiment/} \cite{maas-EtAl:2011:ACL-HLT2011}} \cite{Ager2018}} &
				\specialcell[l]{highly polar movie reviews\\for binary sentiment classification}  &
        		\specialcell[l]{ \tabitem removed stopwords (using NLTK's corpus \cite{loper-bird-2002-nltk})\\ \tabitem lowercased text\\ \tabitem candidate terms: all textual and numerical tokens} &
        		50000 reviews &
        		\tabitem sentiment of the review (2 classes) &
        		$\geq$ 50 occurences
        		\\ \midrule
        	\textbf{Bands \cite{Alshaikh2020}} &
        		\specialcell[l]{All Wikipedia pages ($\geq 200$ words) whose \\ WikiData semantic type is "Band"} &
        		\specialcell[l]{ \tabitem removed HTML-tags and references \\ \tabitem \textit{"standard preprocessing strategy"} \cite[137]{Alshaikh2019} \\ \tabitem removed stopwords (using NLTK's corpus \cite{loper-bird-2002-nltk})\\ \tabitem POS-tagging and keeping only nouns and adjectives \\ \tabitem remove words with a rel. \acrshort{df}  $>$ 60\% or abs. \acrshort{df} $<$ 10 } &
        		11448 bands & \specialcell[l]{ \tabitem Genres (22 classes) \\ \tabitem Country of origin (6 classes) \\ \tabitem Loc. of formation (4 classes) }  & 
        		\specialcell[l]{ 10 $<$ \acrshort{df} $<$ 6869 \\ (all words from the BoW)}\\ \midrule
        	\textbf{Organisations\tablefootnote{\label{fnote:for_alshaikh2019}Originally created in and for \cite{Alshaikh2019}} \cite{Alshaikh2020}} &
        		\specialcell[l]{All Wikipedia pages ($\geq 200$ words) whose \\ WikiData semantic type is "Organisation"} &
        		\specialcell[l]{ \tabitem removed HTML-tags and references \\ \tabitem \textit{"standard preprocessing strategy"} \cite[137]{Alshaikh2019} \\ \tabitem removed stopwords (using NLTK's corpus \cite{loper-bird-2002-nltk})\\ \tabitem POS-tagging and keeping only nouns and adjectives \\ \tabitem remove words with a rel. \acrshort{df}  $>$ 60\% or abs. \acrshort{df} $<$ 10 } &
        		11800 organisations &
        		\specialcell[l]{ \tabitem Country (4 classes)\\ \tabitem Headquarter Loc. (2 classes)} &
        		\specialcell[l]{ 10 $<$ \acrshort{df} $<$ 7080 \\ (all words from the BoW)} \\ \midrule
        	\textbf{Buildings\footnoteref{fnote:for_alshaikh2019} \cite{Alshaikh2020}} &
        		\specialcell[l]{All Wikipedia pages ($\geq 200$ words) whose \\ WikiData semantic type is "Building"} &
        		\specialcell[l]{ \tabitem removed HTML-tags and references \\ \tabitem \textit{"standard preprocessing strategy"} \cite[137]{Alshaikh2019} \\ \tabitem removed stopwords (using NLTK's corpus \cite{loper-bird-2002-nltk})\\ \tabitem POS-tagging and keeping only nouns and adjectives \\ \tabitem remove words with a rel. \acrshort{df}  $>$ 60\% or abs. \acrshort{df} $<$ 10 } &
        		3721 buildings &
        		\specialcell[l]{ \tabitem Country (2 classes)\\ \tabitem Administrative loc. (2 classes)} &
        		\specialcell[l]{10 $<$ \acrshort{df} $<$ 2233 \\ (all words from the BoW) }\\ \Xhline{4\arrayrulewidth}
        	% \textbf{SIDDATA-Courses} &
        	% 	TODO &
        	% 	&
        	% 	&
        	% 	\tabitem Faculty (10 classes) 
        	% 	\\ \midrule 
        	% \textbf{100K Coursera reviews}\tablefootnote{\url{https://www.kaggle.com/septa97/100k-courseras-course-reviews-dataset}} &
        	% 	TODO &
        	% 	&
        	% 	&
        	% 	\specialcell[l]{ \tabitem Rating (5 classes) \\ \textit{\tabitem Major, Category, Offered-By,... (tbd)} }
        		\\ 
		\end{tabular}
		\caption[All datasets used by any of \mainalgos]{All datasets used by any of \mainalgos. Citations behind the dataset name denote which author used it. Other listed properties include dataset sources (where available), contents, sizes, the respectively used preprocessing-methods and candidate-word-thresholds, as well as the classes considered in the evaluation of the derived explainable classifiers.}
		\label{tab:all_datasets}
	}
	\end{table}
\end{landscape}


\restoregeometry % !!! when trying to add afterpage again, remove this!!


% \restoregeometry\clearpage % !!! Jörg's comment on https://tex.stackexchange.com/a/78285/108199 !!!!
% \aftergroup\restoregeometry  % see THE QUESTION of https://tex.stackexchange.com/q/139834/108199
% } %afterpage



% Empirie, auch specifics über den Datensatz

%To write:
% * where does the data come from
% * what size is the data, what is the distribution, ...
% * Preliminary analysis (if I delete all that are shorter than X, it are |Y|..)
% * Does it cluster and look nice?
% * Verteilung der Sprachen
% * Preprocessing in kurzem Fließtext beschreiben - "After throwing out all descriptions shorter than xyz chars, 2323 courses where left. 223 of these were ..."
% * That the type of dataset differs from DESC15 and followups - mainly used movie-dataset consists of concatenated reviews (which means relevant words occur more often!) 
%     (TODO: look/think was die anderen auszeichnet - bei dem placetypedataset ists ja gar kein fließtext sondern direkt ein bag-of-tags)
% Dass mein Datensatz kleiin ist! Bei keinem sonderlichen min-word-per-desc threshold hab ich halt 7588 samples, bei 50 schon nur noch 4123, das ist wirklich little
% Dass auch die Descriptions echt kurz sind! Ich hab rund 8k samples, um das selbe samples-to-threshold verhältnis zu haben wie DESC15 wäre rechnerisch ein wert von 2 bis 25 sinnvoll (wobei man beachten muss das 2 schon richtig kacke ist weil dann die SVM 2 vs 8000 klassifizieren muss and that will never work -> 25 ist minimum), ABER wenn ich dann 25 nehme hab ich nur 2.4k candidates statt the 22k DESC15 aimed at, which also sucks!! --> CONCLUSION: Datensatz scheint zu klein.

The main goal of this thesis was to create a conceptual space of courses, automatically generated by course descriptions.


For that, a dataset of courses and their descriptions was obtained as export from the Stud.IP system as used at the universities of Osnabrück, Hannover and Bremen.
%TODO wait, woher kam der datensatz überhaupt? Tobias hat mir den geschickt, aber kam er zustande im Rahmen von Siddata?

The dataset comes from Johannes' Repo at \url{https://git.siddata.de/jschrumpf/study_behavior_analysis} (requires authentification over UOS!)

\begin{figure}[H]
	\centering
	\includegraphics[width=\figwidth]{graphics/figures/courses_language_distribution.png}
	\slcaption{
		\label{fig:courses_language_distribution}
		Distribution of languages of course descriptions.
		%TODO figure if this is the correct amount of preprocessing/throwout to have done
		Of the 21337 courses left after preprocessing, 18,679 were in german language according to the \textit{langdetect} python-package (for details, see \aref{ap:translating}).
		}
\end{figure}


The faculty is easily obtainable from the dataset, as the first one or two digits of the course ID correspond to it. The distribution of the faculties is depicted in figure \ref{fig:faculty_plot}.

\begin{figure}[H]
	\centering
	\includegraphics[width=\figwidth]{graphics/figures/faculty_plot.png}
	\slcaption{
		\label{fig:faculty_plot}
		Distribution of faculties in the courses
		}
\end{figure}

The purpose of the Neural Network classifier is to check if it is anyhow possible to extract meaningful information from the descriptions: If it is possible to train a classifier on the data that can reasonably predict a qualitative feature, there is enough structure in the data such that the algorithm I'm about to produce can work.
Also, we have a lower bound for useful data: we can just throw away data that cannot be classified!
%TODO: train a second classifier on something else and throw away data that gets classified by neither and inspect it

(-> 91\% test accuracy)

% =============== Besonderheiten vom Siddata-datensatz

....len([i for i in descriptions._descriptions if "kompetenzen entwickelt befahigen akademischen berufstypischen" in i.processed_as_string()]) == 25  ... weil es genau 25 exakt gleiche Beschreibungen gibt, für die Fremdsprachkurse. Deswegen ist up to jede 5-wort-kombination davon ein extracted keyword
(und das obwohl sie verschiedene Namen haben! merging them doesn't make sense but they are almost equal)

% =============== Schreiben zum Thema Datensatz-Vergleich:

...ist es richtig dass nur 6000 verschiedene Terms >= 25 mal vorkommen?! 6000?!
=> auch in groß ist mein datensatz ja noch deutlich kleiner als placetypes, die haben immerhin 22k candidates
--> n-docs: 7596
--> 1-grams >= 25 times: 5054, 1-5-grams >= 25 times: 6717
--> unique 1-grams: 106235

bei placetypes sind es 
* unique 1-grams: 746180, davon 41320 >= 25 mal und 21833 >= 50 mal (their threshold)

--> das verhältnis Anzahl Texte zu Länge Texte ist bei mir halt komplett off 

% =============== 


\subsection{Place-Types}

\begin{itemize}
	\item Took it to be able to compare my results to the ones of \mainalgos
	\item Did NOT do the movies-dataset (also used by all \mainalgos, see \tref{tab:all_datasets}), because version available online does not contain n-grams so it will not be comparable
	\item Also took it to be able to sanity-check if my implementation was correct, which was extremely helpful
	\item Didn't do the openCYC taxonomy bc they say that they don't use one level of the taxonomy consistently but also never explain where they go to which level
\end{itemize}

%TODO: write IN THE ALGORITHM & ARCHITECTURE SECTIONS that I of course tried the placetypes-dataset as sanity-check to find errors - for that dataset, stuff like the good-candidates is known so as long as I don't reach their performances for that dataset I know my code is the problem, but as soon as I reach their performance I can savely say that the actual algorithm is correct and if it's still bad on the siddata dataset it's just not applicable to this kind of data

So, infos from \cite{Derrac2015}:
\begin{itemize}
	\item GeoNames has 667 place-types in 9 categories (403 used)
	\item Foursquare has 435 place-types in 9 top-level categories (391 used))
	\item content: tags of Flickr photos. Photos assumed to be of a type if one of the tags is the name of that type (so they queried for photos with that tag), and then all other tags of that picture make up the BoW.
	\item 22816139 photos, types with less than 1000 photos removed.
\end{itemize}


Also tried the Plactypes-Dataset used by all main-paper-authors. When doing so I noticed that there are definitely duplicates (which are consistently recognized as closest-terms in embedding):
  abandoned rail road and abandoned railroad
  boat yard and boatyard
  coral reef and reef
  court house and courthouse
  grass land and grassland
  sheep fold and sheepfold
  skate park and skatepark
  steak house and steakhouse
  water fall and waterfall
  wind mill and windmill

Next to that, the embedding however also sees very similar ones as very similar, which is a nice sanity-check, eg.

  abandoned farm and abandoned home
  airfield and airport
  airport and airport terminal
  ancient site and archaeological site
  arch and arch bridge
  art gallery and art museum
  coffee house and coffee shop
  aircraft cabin and airplane cabin
  apartment and apartment building
  bank and bank building
  field hockey field and hockey field

\subsection{Other Datasets}

Also tried a dataset of 100.000 coursera course reviews from \url{https://www.kaggle.com/septa97/100k-courseras-course-reviews-dataset}. Why? Because it's also eduactional resources, but as it's reviews it seems closer to the movies dataset
See \url{https://www.kaggle.com/roshansharma/coursera-course-reviews} for exploratory analysis of the dataset (there he also has another dataset he writes about, but you cannot merge them unfortunately, so besides course name the only possible task is the rating)
%TODO: I could try to merge it with this one https://www.kaggle.com/siddharthm1698/coursera-course-dataset or another one (see https://www.kaggle.com/mihirs16/coursera-course-data which links names to links, https://www.kaggle.com/search?q=coursera+in%3Adatasets for other places)

Also, there's the Large Movie Review Dataset\footnote{\url{http://ai.stanford.edu/~amaas/data/sentiment/}, \url{https://scikit-learn.org/0.19/datasets/twenty_newsgroups.html}}, also used by \cite{Ager2018}.




\section{Algorithm}

Let us finally go into detail about the main algorithm. The implementation of this thesis replicates and extends the algorithm proposed by \textcite{Derrac2015} with some novel contributions to deal with the given dataset. Further, some improvements from the works of \textcite{Ager2018} and \textcite{Alshaikh2020} are incorporated, who also replicated and improved the original algorithm and share Prof. Steven Schokaert as last author. According to our evaluation of the field, these two papers provide some useful improvements in several aspects, as they apply the algorithm to different datasets, suggest more straight-forward ways of evaluating their performance, and help in understanding important concepts. That we are focusing on only those three papers should by no means imply that they are the only ones that were considered and influcenced this implementation\footnote{See \autoref{sec:otherwork}}, however in contrast to the other pertinent literature these two works do not substantially divert from the algorithms core principles.
% see \autoref{sec:otherwork} - \cite{VISR12} and their tag genome, the fact that the algorithm detailed here is basically only one step in \cite{Alshaikh2019}, Gärdenfors himself suggested that one may use self-organizing maps instead of classical AI/NLP algorithms.

It is important to keep in mind that the algorithm is no rigid monolith but modularly consists of several components, such as \textit{dimensionality reduction}. Many of these components do not require specific algorithms, and \mainalgos also experiment with different components. The exact system for these components may be exchanged, and in the following these exchangeable algorithms are also referred to as hyperparameters. Note further that while this thesis mostly replicates the work of \textcite{Derrac2015}, the following will describe the algorithm as implemented here, which differs in some details from the original work. For the sake of overview, very specific implementation details will be left out in the following description, as however reproducibility is an important aim for us, implementation details are available in Appendix~\ref{AppendixB} and linked where relevant. \todoparagraph{Further, there is a table that compares the implementations here and in mainalgos, as well as another table what-configs-are-available where, the config-yamls and a section on what-other-things-could-one-have-done-hereandthere.}

\subsection*{Core Algorithm}

\todoparagraph{The following explanation assumes that we accept some things as given. For now we'll do that, but we will later revisit and critically question many of these assumptions!}

% The core idea of the algorithm is to unsupervisedly find a a set of features which can be modelled as directions for a vector-space representation of the respetive entities.
The main goal of the algorithm is to unsupervisedly use text-corpora associated with the considered from a certain domain\glspl{entity}\footnote{From now on, the term \textit{\glspl{entity}} refers to the sample described by one text from the corpus (description, concatenated reviews, ...). The corpus accordingly defines the domain: educational resources, movies, ...} to embed these into a vector-space where the axes correspond to human concepts/Properties.\footnote{\textit{Concepts} and \textit{Properties} explicitly refer to what is defined in Criterions C and P, see \ref{sec:csdefinition}} This is referred to as \textit{feature-based} representation: A high-dimensional vector that numerically encodes the degree (\textit{protoypicality}) to which the entity corresponds to a number of appropriate dimensions. This is generally referred to as Conceptual Space and can be used as basis for explainable reasoning.

The general idea to achieve that is as follows: First, the entities are embedded as fixed-dimensional vectors. To allow for the types of reasoning mentioned in Section \ref{sec:cs_reasoning}, it is embedded into metric spaces where the concepts of direction and distance are well-defined \gencite{Derrac2015} original algorithm uses MDS (see \ref{sec:mds}) for this matter, which enforces metric distances. \cite{Ager2018,Alshaikh2020} both soften this requirement and also use document embedding techniques such as doc2vec and averaged GloVe \todoparagraph{REFERENCE!} embeddings.

Additionally, words or phrases from the text are extracted as candidates for the names of the semantic dimensions. The underlying assumption is that \q{words describing semantically meaningful features can be identified by learning for each candidate word $w$ a linear classifier which separates the embeddings of entities that have $w$ in their description from the others} \cite[3574]{Alshaikh2020}. The better the performance of that classifier according to a chosen metric, the more evidence there is that $w$ describes a semantically meaningful feature. 
% * from Alshaikh2020: "Their core assumption is that words describing semantically meaningful features can be identified by learning for each candi- date word w a linear classifier which separates the embeddings of entities that have w in their description from the others. The performance of the classifier for w then tells us to what extent w describes a semantically meaningful feature"
In a final step, the candidate-words are clustered according to their similarity to find a fixed set of \emph{semantic directions}. A representative term for the directions is selected as dimension name, and the entities are re-embedded into a new space comprised of these dimensions, where the individual vector-components correspond to the ranking of an entity with respect to these dimensions.

The rest of this section goes into further detail for each of the individual components of the algorithm. \removeMe{An overview of which of the considered literature supports each components is given in \autoref{tab:compared_algos}.} Further, configuration files to enable exactly the respective components of the papers \mainalgos for the codebase of this thesis are listed in \aref{ap:yamls_for_origalgos}.

\todoparagraph{but before that, ager and alshaikh}

\removeMe{
\subsection{Regarding ager and alshaikh}

\todoparagraph{describe shortly what the improvements from [2,3] were}

\todoparagraph{Dass die den Zwischenschritt mit dem ganzen geeometric reasoning auf dem interim space nicht machen und DESWEGEN die requirement mit MDS soften konnen}

In principle Derrac2015, but with some components from Ager2018 and Alshaikh2020 as well as some own stuff. I'll be testing some claims or nonclaims of \mainalgos, bspw nutzen sie immer PPMI ohne je tf-idf zu testen. Also of course different nature of the dataset - their "how does this dimension correspond to the count in the reviews" doesn't make sense (their success-metric for the SVM is tailored to the one property, so I expect that one to be worse). I'll elaborate on different ways to deal with that later.
}

\subsection{Algorithm Steps}
\label{sec:algorithm_steps}

% The core idea of the algorithm is to (unsupervised, data-driven) find a a set of features which can be modelled as directions for a vector-space representation of the respective entities. For that, the steps are:

Let us finally describe the steps how to create an interpretable vector-space from the text corpus in detail. For that, we will explicitly elaborate on the parameter choices that branch up at every one. Note that that absolutely is a combinatorical explosion it is impossible to try out all. Further, this is about how this specific implementation does it, which may differ in some details from \mainalgos.

\label{sec:algorithmsteps}
\begin{enumerate}
	\item[\saveref{sec:algo_preproc}{1.}] \textbf{Preprocess} the corpus with default techniques and create a \textit{Bag-of-ngrams representation} (\ref{sec:techniques:bow}) of the texts.
	\item[\saveref{sec:extract_cands}{2.}] \textbf{Extract Candidate Feature} names from words/\glspl{ngram} of the corpus.
	\item[\saveref{sec:generate_vectorspaces}{3.}] \textbf{Embed all Entities} into a fixed-dimensional vector space with demanded properties that captures the respective semantics.
	\item[\saveref{sec:svm_filter_cands}{4.}] \textbf{Filter Candiate Features} by training a linear classifier for each candidate that seperates the vector representations of the entities that contain the term from those that do not. If a specified metric for this classifier is sufficiently high, assume that the candidate term captures a \textit{salient} feature - its direction is then characterized by the orthogonal of the classifier's separatrix.
	\item[\saveref{sec:algo:cluster}{5.}] \textbf{Cluster/Merge the candidates} and calculate the feature direction for each cluster from its components, and (optionally) find a representative cluster-name.
	\item[\saveref{sec:algo:postprocess}{6.}] (optionally) \textbf{Post-process} the candidate-clusters.
	\item[\saveref{sec:algo:reembed}{7.}] \textbf{Re-embed the entities} into a space of semantic directions by calculating their distance to each of the feature direction separatrices.	
\end{enumerate}

This techniques first embeds the collection of texts into a  vector space, to afterwards extract important features from this space using linear classifiers. The second step is an original idea of \cite{Derrac2015}, however creating vector space embeddings from texts is a very popular technique, used for many tasks in \gls{nlp} \cite{Mikolov:Regularities,Mikolov2013a,Guo,Lowe,Turney2010}. This implementation relies on classical creation of the \gls{vsm}, for which the general creation process was explained in \autoref{sec:vsm_construction}. The steps \textit{Build the Frequency Matrix}, \textit{Transform Raw Frequency Counts} and \textit{Smooth the Frequency Matrix} are squashed into the preprocessing and embedding of entities (steps 1 and 3).

 An explicit and simple implementation compliant with each step could be a simple word tokenization and count to generate a bag-of-words (step 1) where each sufficiently frequent word is used as candidate (step 2). A \gls{dissimmat} of the individual \gls{bow}-vectors is compressed using MDS (step 3). A \gls{svm} calculates the accuracy for each candidate (step 4), and k-means-clustering on the 500 top-scoring terms subsequently creates 100 clusters and averages their directions (step 5). The distance to each of the hyperplanes is calculate (step 6), yielding new space for the entities. The sequence of steps is also given as pseudocode in \autoref{ap:algorithm_pseudo}. 
 
 Again it should be stressed that many different components can be considered for each step and the distinction of steps is not rigid: Instead of creating a dissimilarity-matrix followed by dimensionality reduction, \cite{Ager2018,Alshaikh2020} use neural word or document embeddings.\footnote{see \autoref{sec:embeddings}} Instead of extracing candidates from corpus tokens and training a linear classifier for each of them and use their orthogonal as direction, techniques such as LSA or LDA can be employed to find topic vectors directly. We will come back to these ideas when discussing future research opportunities (\autoref{sec:futurework}) by listing what other ways of fulfilling each respective step could have been considered.

\autoref{fig:dependency_graph} shows an automatically exported dependency-graph, displaying the individual steps of the algorithm as done in the accompaning code, also showing where selected important parameters are first used. As explained in \autoref{sec:architecture}, the modularity of the provided architecture allows individual components to be exchanged as needed and run in parallel.


\begin{figure}[H]
	\begin{center}
	  \includegraphics[width=0.9\textwidth]{dependency_graph.pdf}
	  \caption[Dependency-Graph of the Algorithm]{Dependency-Graph of the Algorithm, displaying the individual steps of the algorithm as well as their dependencies and where selected important parameters are first used. \todoparagraph{Generated using command XYZ}}
	  \label{fig:dependency_graph}
	\end{center}
\end{figure}


Before looking at the steps in turn, it should be noted that even the preprocessing does not work on completely raw data, but on curated and processed corpora. This processing is however not considered part of the algorithm, as it is very specific to the respective datasets and manual dataset exploration, tweaking settings such that they are best for each corpus separately. The preprocessing for the Siddata-dataset is described in \autoref{sec:dataset_siddata} and its implementation is done in separate Jupyter Notebooks.\footnote{Such as \url{https://github.com/cstenkamp/derive_conceptualspaces/blob/main/notebooks/create_datasets/Preprocess_Siddata2022.ipynb}}. In the considered literature, the preprocessing is not considered part of the algorithm at all. Their implementations start from already fully processed datasets available as bag-of-words, each separately processed. Details of their individual processing per dataset is listed in \autoref{tab:all_datasets}. By incorporating the preprocessing into the pipeline, this work aims to increase adaptability and reproducibility, and also allows to experiment with different techniques such as translation or lemmatization or how duplicate entities with different associated texts are merged.
% In course-descriptions, I want some parts of the pre-preprocessing be part of the pipeline, like how we merge descriptions of different iterations of the same course that overlap to a high degree (sentwise-merge vs relative-term-frequencies)

\subsubsection{Preprocessing\arrowref{sec:algorithmsteps}}

\label{sec:algo_preproc}

A common prerequisit for NLP algorithms is to pre-process the text corpus. The preprocessing itself consists of multiple independent components chained after each other. Which components are necessary also depends on the processed dataset - as for example the \emph{placetypes}-dataset consists of a collection of \textit{tags} instead of full sentences, tokenizing sentences or removing \glspl{stopword} becomes irrelevant. Other datasets may require additional cleaning or are already available in preprocessed form.

\paragraph{Translation} As the main considered dataset of university-courses is highly multilingual (see \autoref{fig:sid_statistics}), one of the first questions that needs to be addressed is how entities of different langauges are handled. The algorithm consists of classical language processing algorithms such as comparing \gls{bow} representation of the entities, which means that the same text in two different languages may result in maximally different representations (see \autoref{sec:techniques:bow}). Because of this, before any other processing, the languages of each entity is checked, such that those of languages other than the demanded may be either translated, left out or used anyway. For details about the translation, it is referred to Appendix \ref{ap:translating}.\footnote{It should be noted that professional automatic translation is costly and thus not all texts are available in all languages.}

\paragraph{Components} The following components are developed for the preprocessing, every one of which can be individually enabled or disabled:

\begin{itemize}
	\item Prepend title and/or subtitle to the entities' associated text \itemtext{useful for the Siddata-Dataset, as the titles are often quite long and more descriptive than their descriptions}
	\item Remove HTML-Tags from texts 
	\itemtext{useful for the Siddata-dataset, as it includes descriptions for \glspl{mooc} which are crawled from websites and often contain such}
	\item Tokenize sentences 
	\itemtext{such that \glspl{ngram} across sentences are not considered}
	\item Lower-case all words
	\itemtext{reduces the amount of individual words and ensures that words at the beginning of sentences are mapped correctly}
	\item Remove stop-words / frequent phrases
	\item Tokenize words
	\itemtext{means splitting at the word-boundary, resulting in a list of words. Order must be kept in case n-grams are to be extracted.}
	\item \Gls{lemma}tize words
	\item Remove diacritics
	\itemtext{\emph{Diacritics} are glyphs added to basic letters, such as accents or German \emph{Umlaute}. Removing them converts for example the letter \emph{ä} to an \emph{a}}
	\item Remove punctuation 
\end{itemize}

The above can be done either be done with proprietary code for all of these steps,\footnote{Mostly relying on the python package \emph{nltk} \cite{bird2009natural}} or using \codeother{sklearn}\footnote{\url{https://scikit-learn.org/stable/}}s \codeother{CountVectorizer} (which is faster, but less configurable), as \cite{Ager2018} claim to have done.

\paragraph{On Stop-Words}
Removing stop-words from the texts is useful because it makes the resulting frequency more compact and thus less computationally intensive, and stop-words generally have very discriminative power, meaning their occurence among the entities is arbitrary, just making hte emeddings more noisy (cf. \autoref{sec:word_count_techniques}). There are however reasons to not remove them: Two words that are considered stop-words may posess relevant semantic content (such as a \textsc{Fällt aus} in a course title), and also stopwordslists are often incomplete and of low quality \cite{nothman-etal-2018-stop}. For these reasons it is also possible to instead remove \glspl{ngram} that exceeded a certain frequency (\gls{df}).

\paragraph{On Lemmatization}
The languages most prevalent in the considered datasets are considered \textit{agglomerative}, which means word stems are changed by the addition of affixes and suffixes. Consequently, the same word may be present in multiple different forms, which modelled as completely dissimilar vectors in the present \glspl{bow}-approach. Lemmatization is the process of mapping different forms of these words onto the same stem. Considering that the Siddata-dataset consists of far fewer words than the others, this has important implications. For the german descriptions, this implementation relies on the \textit{HanTa} lemmatizer. \todoparagraph{Correct citation for hanta!!} %https://textmining.wp.hs-hannover.de/Preprocessing.html#Lemmatisierung

The result of this step is a bag-of-ngrams representation for each entity (see \autoref{sec:techniques:bow})


\subsubsection{Extract Candidates\arrowref{sec:algorithmsteps}}
\label{sec:extract_cands}
% Section 4.2.1 of Derrac2015

The final result of the algorithm is a metric space in which the individual dimensions (\emph{\glspl{feature}}/\emph{Interpretable direcitons}) correspond to natural-language concepts and attributes. The candidates for these features are verbatim phrases extracted from the text-corpus of the \glspl{entity}, which are subsequently filtered and merged as necessary.

In \gencite{Derrac2015} work, the selection of phrases to be extracted depends on the dataset: For placetypes-dataset, all sufficiently frequent\footnote{\label{fnote:cand_thresholds}The respective thresholds are listed in \autoref{tab:all_datasets} as ``candidate word threshold''.} 1-grams\footnote{Note that in the case of the place-types dataset, a 1-gram corresponds to all merged words of a tag.} were considered. For the other two datasets, they applied a \gls{pos}-tagger that extracted all sufficiently frequent\footnoteref{fnote:cand_thresholds} \textbf{adjectives, nouns, adjective phrases} and \textbf{noun phrases}, assuming that adjectives would correspond to gradual properties (\eg \textit{violent, funny}) and nouns to topics (\eg the \textit{genre}) \cite[Sec. 4.2.1]{Derrac2015}. Also, the authors ensured that the number of extracted candidates for both datasets is roughly equal, getting around 20\,000 candidates for movies and placetypes.

% Their method depended on the dataset - as their placetypes-dataset was just a collection of tags and the number of tags with term-freq >= XYZ (docfreq>2?! hä?) corresponded to their desired number of candidates anyway (around 22k), they just took all of these as candidates. For their movie-reviews-dataset, they considered all nouns, adjectives, nounphrases 	and adjective-phrases as detected by a POS-tagger. Doing something similar in the scope of this thesis led to suboptimal results, which is why alternative methods were developed
For this step, the implementation of this thesis differs from the original algorithm, as both taking all words as candidate and running a \gls{pos}-tagger led to suboptimal results in previous experiments, which indicated that the robustness of the algorithm is increased if less candidates are considered in earlier steps. This will be further argued and elaborated in the discussion. To ensure comparability to these works however, in the case of the placetypes-dataset the original method of taking all words with a term-frequency of at least 50 was used. Similar techniques for the Siddata-dataset were also considered, but in constrast to the placetypes-dataset it is also important to consider various-length n-grams. While \textcite{Derrac2015} claim to have considered \glspl{ngram} for the movies-dataset, the published version of this dataset contains a \glspl{bow}-representation for each entity where the original word-order is lost, making it impossible to recover \glspl{ngram}, making comparisons with their results impossible for that dataset.\footnote{\url{https://www.cs.cf.ac.uk/semanticspaces/}}

% \todoparagraph{thing is I have less words but the algorithm seems to profit from less words as that makes it more robust}
% I would however argue that the difference here doesn't make a relevant difference 

In our implementation the candidate-extraction is split into four subsequently excecuted substeps, because depending on the algorithm used to extract the candidates the runtime of the individual components is comparably long and some settings are only relevant in later substeps. The steps are:
\begin{itemize}
	\item Extracting Candidate Terms
	\item Postprocessing the Candidates
	\item Creating the \gls{doctermmat} for the candidates and applying a \gls{quant}
\end{itemize}

As visualized in \autoref{fig:dependency_graph}, these substeps only depend on the preprocessed descriptions, which means they can be run in parallel to the creation of the embedding.\footnote{\todoparagraph{Another good reason for cluster exceution!}}

% This can be done either based on the frequency (meaning all terms with a minimal term-frequency), based on some notion of *importance* (based on scores like tf-idf or ppmi), or by more complex means of figuring out *important* keywords and keyphrases. An example of the latter would be KeyBERT
Three main techniques are implemented to extract candidates from the text-corpus. Irrespective of the algorithm, only words with a sufficiently high \gls{df} are extracted, which is important to ensure that the classifier that determines its meaningfulness has enough samples in both clases. This means that the minimal freqeuncy can be calculated from the dataset size: In \cite{Derrac2015}, the minimal frequency for the movies-dataset with 15\,000 entities was only 100, meaning that the algorithm even works if only 0.6\% of samples are in the positive class. 

\todoparagraph{We will come back to this later}
\todoparagraph{HAB ICH DF UND TF RICHTIG??}

\begin{description}
	\item[By frequency:] consider all phrases that exceed a specified document-frequency (like \cite{Derrac2015}).
	\item[By a \gls{quant}:] consider all phrases that are prominent by some notion of \textit{importance} , such as the \gls{ppmi} or \gls{tf-idf}-score. Note that the respective scores depend on the combination of document and term, such that candidates may be extracted for some documents. Of course, all their occurences are considered in the creation of the frequncy matrix.
	\item[Using \emph{KeyBERT}\cite{grootendorst2020keybert}:] consider phrases whose BERT-embedding \cite{Devlin2019} is most similar to the text they are in. 
\end{description}

Using KeyBERT results in candidate terms that are most appealing in qualitative inspection, however it is also most computational demanding, techniques and requires substantial amounts of post-processing for the resulting phrases. More details on KeyBERT and how it is incorporated into the algorithm are given in the implementation are given in Appendix~\ref{ap:details_keybert}

Finally, a \gls{doctermmat} is created from the postprocessed candidates, containing the frequency for each candidate-phrase in each entity. The creation of this frequency matrix mirrors the process described in \autoref{sec:vsm_construction}, however only for the extracted words. After filtering this matrix to ensure that only candidates with a minimal \gls{df} or \textit{stf} are considered, a quantification is applied as described in \autoref{sec:word_count_techniques}. Available Quantifications include raw count, binarization\footnote{meaning all counts are either one or zero. According to \cite{Alshaikh2020} this improves performance \todoparagraph{Aber ich hab logische Probleme damit}}, tf-idf or PPMI.

\cite{Derrac2015} \todoparagraph{always only use PPMI without ever testing tf-idf or giving a reason, I'll try both}

\todoparagraph{so the relation of term to document may be expressed by something else than count - so if we later compare the ranking induced by the svm to this maybe something else thatn the count stands there - I'm expecting that for my dataset tf-idf is much more valuable than the count bc no concatenated reviews or tags}


\subsubsection{Generating Vector Space Embeddings\arrowref{sec:algorithmsteps}}
\label{sec:generate_vectorspaces}

\todoparagraph{This is Turney2020s "Building the frequency matrix", BUT SO IS THE STEP ABOVE}

In this step, the individual \glspl{entity} are embedded into a fixed-dimensional vector space, making up a \emph{frequency matrix}. Importantly, while this matrix is a \gls{doctermmat}, it is only an interim result in the algorithm and the calculation of distances and directions will be done on another matrix from a later step - This is where our pipeline starts to diverge from what the pipeline specified in \nameparanref{sec:vsm_construction}. So we created a frequency matrix that encodes the relevance of a candidate-phrase for each entity in the previous step, and in this step we create another one that encodes each document as a vector. Neither of these matrices will be used to finally calculate similarities on, but both are important to get the dimensions necessary for for these similarities.

Embedding words, \glspl{ngram}/phrases or other tokens, as depicted by \cite{Turney2010,Lowe}, generally involves counting the token frequencies, transforming them to get relative frequencies, and performing dimensionality reduction on the resulting matrix.
So far (step 1), we have counted the token frequencies. 
\todoparagraph{yes we are talking about all tokens}
\textcite{Derrac2015} argued that this space must be a Euclidean \todoparagraph{which is invariant to affine transformations}, such that geometric/algebraic solutions correspond to commonsense commonsense reasoning tasks (see \autoref{sec:reasoning}). \todoparagraph{We will later look into this in more detail}. 
Another requirement is that the number of dimensions is can be chosen as hyperparameter to the algorithm to be able to find a compromise between \todoparagraph{powerfulness and compression... nee argh was fur worter suche ich hier... drauf zuruckkommen wenn ich den teil uber vsms fertig hab. ausserdem, reicht das schon als beschreibung dann?}. Because of these two requirements, \todoparagraph{and also because gardenfors said so} they selected \gls{mds} for dimensionality reduction.

As stated in \autoref{sec:mds}, \gls{mds} calculcates a Euclidean \gls{vsm} from a set of pairwise distances. This means that the algorithm first creates a \textit{Dissimilarity Matrix} that encodes the distance between all pairs of entities (represented as Bag-of-ngrams representation), from which subsequently the final embedding is generated.  
\todoparagraph{This technique of bag-of-ngrams-representation then dissimmat and quantication is not the only way to do it, ager and alshaikh both did it differently}

\todoparagraph{Note that we can use another quantification than in the step above! In my algo this sometimes performed best.}
\todoparagraph{do all this again argh}

In their algorithm, the dissimilarity-matrix is created using distance metrics for the bags-of-words of the respective entities. 

\paragraph{Document embeddings}
If the strict requirement for a metric space is dropped however, many different algorithms may instead be used at this point - not only different dimensionality reduction methods for the embedding, but also ones that do not rely on the distance matrix or even the \gls{bow} at all, like document-embedding-techniques such as \gls{doc2vec} \cite{Le2014} (as \eg used by \cite{Alshaikh2020}). This would change only these steps and the rest of the algo not too much.
However when \cite{Alshaikh2020} used doc2vec instead of dissimmat-mds, it performed worse (see tables in results), which is why it is not in this thesis 


\paragraph{Create Dissimilarity Matrix and Quantify}

The default way of doing it is to create a \gls{doctermmat} that counts the occurences for all words\footnote{Not just the candidates in step 2, but words that occur in any description} for all entities.

In their algorithm, the \gls{dissimmat} is created using the \emph{normalized angular distances} of the \glspl{bow} of the respective entities.  

From this quantified Doc-Term-Matrix, a dissimilarity-matrix is generated. This requires a measure for the dissimlarity - in the original paper, this is what they call "normalized angular difference" - according to \cite{Derrac2015}:

\begin{align}
	ang(e_i, e_j) &= \frac{2}{\pi} * \arccos \left( \frac{\vec[m]{v_{e_i}} * \vec[m]{v_{e_j}}} { \lVert \vec[m]{v_{e_i}} \rVert * \lVert \vec[m]{v_{e_j}} \rVert }  \right)  \label{eq:norm_ang_dist} \\
	&= \frac{2}{\pi} * \arccos(1-\cos(\vec[m]{v_{e_i}},\vec[m]{v_{e_j}})) \text{, where $\cos$ is the default cosine-distance} \nonumber
\end{align}

in \cite{Schockaert2011}, they define similarity through a variation of the Jaccard-distance (IoU, Overlap-Area divided by Union-Area)

\paragraph{Embed}

Because this dissimilarity-Matrix is far too high-dimensional and sparse, a dimensionality-reduction is applied - we discussed other raeasons why that is smart before.

Multidimensional scaling but also isomap yadda yadda


\subsubsection{Filter Candidates by Classifier Performance\arrowref{sec:algorithmsteps}}
\label{sec:svm_filter_cands}

\todoparagraph{Also known as: Creating Candidate SVMs and Filter Candidate Feature Direcitons}
\autoref{ap:algo_filter}

This step brings together the entity embeddings and the extracted keyphrases. To quantify how well each semantic directions captures semantic content of the entities, a linear classifier splits those entities where the keyphrase occurs from those where it does not. The best example for such a classifier is a \gls{svm} which does not rely on the kernel trick.\footnote{kerneltrick ist ja "Projecten in nem anderen space, damit das was da linear ist bei uns nonlinear ist" und ich will linear sein)} As visually exemplified in \autoref{fig:3d_hyperplane_ortho}, the result is a hyperplane that divides the positive and negative samples (the plot a toy-example - in practice it is highly unlikely that they are clearly distinct, but the properties hold in other cases as well). Regardless of the dimensionality of the original space, this hyperplane has a one-dimensional orthogonal vector. Each of the entity-embeddings is subsequently orthogonally projected onto this orthogonal. Now the distance of this projection to the plane offset (the coordinate where it crosses the decision surface) is a scalar that encodes the distance to this decision hyperplane. \footnote{Again, if you don't understand this, look at the plot}

The ranking of the entities in terms of this distance is now what is used as their values for the feature directions, in the sense that the further away an entity is from the decision surface on the positive side, the more it has the corresponding feature. The same holds for the negative direction. This may sound initially surprising, but the point is that the original space is created based on similarity measures of the entities. Given that the \textit{Bag-Words-Hypothesis} (\autoref{sec:bow_hypothesis}) holds, they should have similar words. And those that are maximally dissimilar are as far apart from these as the space allows. To stick with the example of movies, the assumption is that movies that are maximally unscary are maximally far from the away from scary ones, in the sense that you can assume that a maximally dissimilar distribution of words from the positive class means a maximally unscary movie. So the more dissimilar to that, the less scary, so the relationship holds in both directions, even if the word scary itself occurs zero times in the descriptions of any movie on the negative class it still holds that the further away the less the concept applies. Distributional Hypothesis and what we wrote for the logic of LSA. The classifier takes the other words into account for the classification as well. The logic of this is especially clear in the case of SVMs: This classifier works by creating the hyplane such that hte margin between the positive and negative class is maximized. 

Ok so the orthogonal to the resulting decision-hyperplane is then used as axis, onto which the entities are mapped - the further away from the plane the mapping of a point onto the orthogonal, the more the entity is said to have the attribute encoded by the phrase responsible for the hyperplane. A score function compares the ranking induced by this to the ranking induced by number of occurences (or quantification-value) of the respective keyphrase of all documents, such that only those terms where the correspondance of these rankings exceeds a certain threshold are considered as candidate directions henceforth.

\q{The higher the Kappa score of a term \textit{t}, the more we consider \vec{v_t} to be a faithful representation of the term \textit{t}} \q[20]{Derrac2015}. Subsequently only those directions are considered where this classifier exceeds a certain threshold. So what's the logic behind that? As we stated before \todoparagraph{When discussing bow-representations and the reasons for quantifications and also LSA and also stop-words}, unimportant words are more or less uniformly, in any case arbitrarily, distributed throughout the corpus. The vector space that we are doing this SVM on is created on the basis of distributional semantics. The entire basis of this is that there are latent obfuscated topics, and correlations of words for these. So if a topic is very prominent in some texts but not in others, that will influence the position in the vector space. In the case of unimportant words, they are arbitrarily distributed and are not signifying a latent topic, no correlation of other words, not important for similarity. Like no reason for dissimilarity. Not indicating a cluster, because all these stopwords occur random and are thus noise. These randomness does not go along with a cluster of positions in any of the dimensions, nose gets removed by dimensionality reduction. So yes, it does make sense that the better a classifier can split between does-the-word-occur and does-it-not, the more the word is an \textit{important topic} in the sense that it explains the dissimilarity in the entities.\footnote{For a better intuition why this makes sense it is referred to \cite{Lowe}}
\todoparagraph{Do I need a plot that shows a non-faithful direction? } % grafische Darstellung von "if the ranking induced by the SVM corresponds to the count/PPMI, we see it as faithful measure", also ein beispiel wo's passt und ein Beispiel wo's nicht passt

\cite{Ager2018}: \q{if this classifier is sufficiently accurate, it must mean that whether word w relates to object o (i.e. whether it is used in the description of o) is important enough to affect the semantic space representation of o. In such a case, it seems rea- sonable to assume that w describes an important feature for the given domain.}



Okay so lets continue.

Concretely, the score used by \cite{Derrac2015} to assess the performance is not the accuracy or some other measure of the bare performance of the classifer, but rather if the ranking by distance to decision hyperplane corresponds to ranking of number of occurences (or the PPMI-score, the authers are imprecise in their wording) of that word.
The reasoning behind that becomes especially clear when considering the root of their datasets - in the case of reviews or tags it is the case that the often a word is mentioned, the more relevant the word is for that entity. And because we are using the PPMI-score, it is even more: The more salient relevant for this entity but not for the others the word is, the higher the score. That is what how we created the semantic space in the first place, by saying important ones are weighted more, those are very prominent for some but not all were important for the dissimilarity that is the basis for our embedding. So this entire thing basically looks back at our embedding and tries ot figure out which words it were that were relevant for the dissimilarity. It dissects the overall dissimilarity we had before into its components.

Okay, enough for the theory, lets talk about the implementation. \cite{Derrac2015} say that they use the Kappa-score, which is a metric that compares rankings. With that, they compare the rankings of the svm with the ranking how-important that word is. They took kappa because that is good at dealing with high imbalances in class sizes, which are definitely given.

Yet another point where \cite{Derrac2015} are really low on information what parameters they used. Sklearn allows different weighting types\footnote{\url{https://scikit-learn.org/stable/modules/generated/sklearn.metrics.cohen_kappa_score.html\#sklearn.metrics.cohen_kappa_score}} 
\todoparagraph{explain what that changes respectively}

Unfortunately, they give hardly any details, and there are many different ways how to implement that. While \cite{Ager2018,Alshaikh2020} explicitly say that they are interested in the PPMI-scores\footnote{Though the uploaded code of \cite{Alshaikh2020} does not compare rankings but raw values}, from \cite{Derrac2015} it is not even clear if they take the count or the PPMI-score. As that is highly relevant, we try many different ways of this scoring and report them in the results. We also compare the overlaps of different kappa-scores to check if the choice is as imporant as we think it is. Which scores we used and how they are written here is listed in the implementation details: \autoref{tab:kappa_measures}.

\textcite{Ager2018} compare the kappa-score to accuracy and NDCG and say accuracy works better than kappa.

So, alogorithm: For every candidate-term, take the quantifications from the doc-term-matrix and binarize it, such that we have two classes. On that we then train a linear classifier such as an SVM. On that we calculate binary classificaiton-quality-metrics, and from the ranking the kappas. resulting SVM has a hyperplane as decision surface. The distance of a point to it's orthogonal projection onto that hyperplane can be seen as proportional to how much this point is considered to be in the respective class of the SVM. One can use these distances to enduce a ranking how prototypicality. compared to other heuristics encoding it, such as the ranking induced by the per-term-frequencies of the terms for all documents, or it's PPMI or tf-idf representations.
\cite{Derrac2015} call this "measure the faithfulness of representation"

\cite{Ager2018}: \q{We say Feature *Directions* and not feature *vectors* because they are supposed to rank, not measure degrees of similarity! it only tells us "this one has the feature to a higher degree"}


In the end we have a shitton of scores, and two threshold, yielding great ones and okay ones.
Ehm, why didn't I just take the ndims*2 best ones instead of hard-thresholding??
Well, I can say it was because:

At this point we already have an estimate of how good the parameter-combination so far was: if not enough great-ones were extracted, we don't need to bother continuing.



\includeMD{pandoc_generated_latex/3_1_algorithm}




\subsubsection{Merging the extracted candidate-directions\arrowref{sec:algorithmsteps}}
\label{sec:algo:cluster}

The previous step yielded many \textit{basic feature directions} that are defined as direction of the orthogonal vector for the hyperplanes splitting each individual candidate \gls{ngram}. The performance-thresholds are set such that many more directions are generated than the demanded dimansionality of the final embedding, such that they must be clustered and merged.

This is done via the following substeps, each of whch will be closer eloaborated:

\begin{itemize}
	\item Cluster good-performing candidates by their similarity
	\item (optional) Remove uninformative features
	\item Recalulate the direction of the cluster
	\item (optional) Find a representative name for the cluster
\end{itemize}

\paragraph{Clustering the candidates}

Clustering refers to an unsupervised algorithm that groups items based on some notion of similarity. In our case the assumption is that semantically similar concepts have \textit{close} vectors, which is given due to the \gls{bow}-hypothesis that states that the underlying structure of our dataset is expressed by the usage of related words ((\autoref{sec:bow_hypothesis}), \ref{sec:lsi}).\footnote{In case of the Siddata-dataset, it may mean that in courses that contain the word \textit{computer} have a high chance of also containing \textit{program}.} As these vectors in principle only encode a direction, their similarity can be calculated by their \gls{cos}.

The clustering should reduce the number of features and also ensure that the resulting directions are different enough. Note that unlike \eg in Principal Component Analysis (PCA), the suggested here techniques do not enforce orthogonality, such that the resulting directions may remain linearly dependent to a certain degree. As in the final embedding only the projection onto those directions is relevant, it must be ensured that enough of the data's original variation is covered by these directions. To ensure that, we follow \gencite{Derrac2015} suggestion to allow for redundancy by extracting twice as many directions than the original \gls{vsm} dimensionality. 

This implementation implements the original clustering-method of \cite{Derrac2015}:

First, we consider the best \textit{basic features} as \textit{main directions}. For that, we select one of the scores calculated in the previous step and select all candidates that exceed a threshold (\cite{Derrac2015} suggest $\kappa \geq 0.5$).
To get the directions, we follow the following algoritm:

\vspace{-1ex}
\begingroup
\verbatimfont{\footnotesize}%
\begin{verbatim}
	greats = filter(candidates, 0.5)
	directions = greats.argmax()
	for nterm in range(ndims*2):
	  greats = set(greats)-set(directions)
	  distances = {cand: min(comparer(cand, compareto) 
	                for compareto in directions) 
	                  for cand in greats}
		directions.append(compares.argmax)
\end{verbatim}
\endgroup
\vspace{-1ex}

This starts with the best candidate and then iteratively adds the one from the set of top-scoring candidates that most dissimilar to the set of final directions. The result is a set of $ndims*2$ main directions, which are henceforth considered the Cluster centers. Subsequently, all leftover terms from $T^{0.5}$ as well as all terms from $T^{0.5}$ are added to the respective cluster whose direction they are most similar to. 

\textcite{Derrac2015} used the \gls{cos} distance to measure the respective similarities. This may lead to unexpected situations (discussed in \autoref{sec:discuss_points}). As alternative similarity metric that does not rely on the angle between their vectors, \cite{Alshaikh2019} suggest to use the overlap of the positive-samples of two features as similarity. This was however not yet implemented in this thesis.

Alternatively to the described algorithm, is is also possible to use the popular \textit{k-means} algorithm for clustering, as done by \cite{Ager2018}. We do not present results for this approach here however, as it lead to a substantial increase in runtime, without affecting performance much. In the development we also noticed that many clusters contain a lot of irrelevant terms. To alleviate this, we experimented with different techniques, for example setting minimal similarity thershold that must be given for a term to be added to a cluster, however so far no formal evaluation to test how this affects performance was performed.

\removeMe{
\todoparagraph{Doesn't one bad word in the cluster destroy it?} No. It *IS* okay if common words (like "course") are in clusters, it is NOT the case that as soon as the word occurs once it is said to have a certain property. ("Wenn cluster-threshold zu groß, kommt “A1” in ein cluster mit “Course” and everything is over" is FALSE). However it IS not too good -> A cluster with many words like "course" in it has a high degree of randomness (there is no information gain by such words, it occurs random across courses, a cluster of courses that mention that they are courses is useless) The word occurs randomly, if a course is assumed to have a certain property because of that it's certainly wrong
}

\paragraph{Find Cluster-Direction}

So far, we have a set of clustered canidates terms, each of which has an individual direction. The final \textit{feature-direction} must subsequently be found from the elements of the cluster. For that, \cite{Derrac2015} and \cite{Ager2018} define the cluster centroids as the average of all (normalized) vectors per cluster. In our experiments, however, we noticed that the final direction tends to be too much affected by irrelevant cluster-elements. Because of this, we experimented with other techniques to determine the cluster direction. Two other considered methods include to just consider the direction of the cluster-center, or to weight the influcence of each cluster-element by their kappa-score.

The best performaning method however was the \textit{reclassify}-algorithm, which (similar to \cite{Alshaikh2020}) finds the cluster-direction by training a new classifier that splits those \glspl{entity} that contain \textit{any of the elements} from the cluster from those that do not, analogous to the previous step (except that it requires to generate and quantify a new frequency matrix from the sums of the individual counts). Doing this however often leads to the opposite problem than the previous step, namely that for many clusters there are almost no entities that do not contain at least one of the cluster elements. To counter this, we instead trained a classifier to split the 30\% of entities with the highest \glspl{quant} from the 30\% of entities with the lowest \glspl{quant}. A comparison of this algorithm with the method of \cite{Derrac2015} is given in the Appendix as \autoref{tab:text_per_dim}. As \cite{Alshaikh2020} already performed formal experiments with this that have shown its superior performance, all generated results of this work rely on this algorithm. 

\paragraph{Bad Clusters}

After these steps, we finally have the vectors that correspond to semantic directions. As there however still be clusters of uniformative terms, \textcite{Alshaikh2020} have an additional step to remove uninformative cluster. As this however bases on another clustering algorithm used by the author (namely \textit{affinity propagation}) which does not allow to specify the number of clusters, it was not implemented in the scope of this thesis.

\paragraph{Find a representative Cluster Name}

An important advantage of the clustering process is that it makes the extracted directions more \textit{descriptive} due to the fact that they are described by several phrases instead of only one. However, it may be helpful for an attractive user interface to find the single \textit{best} description of the cluster direction by its element.

An analysis of \cite{Carmel2009} showed that a statistical method to extract features from clustered text corpora identified the labels of human annotators as one of the top five most important terms in only 15\% of cases, implying \q{that human labels are not necessarily significant from a statistical perspective} \cite[139]{Carmel2009}. In their paper, they suggest several methods to find one representative name for the cluster. 

\cite{Derrac2015} and its follow-ups \cite{Ager2018,Alshaikh2020} did care about such methods and instead use either the name of the cluster center as its description or the cluster center plus two other sample elements. This work experimented with several techniques to get a more representative direction name. One of these techniques used the KeyBERT-algorithm (see \autoref{ap:details_keybert}) to find the term that is most similar to the set of terms making up the cluster. We also experimented with a method that embeds the cluster terms using \gls{word2vec} and returns  the word behind the vector that is closest to their average (which is not neccessarily part of the original set of words). Similarly to \gls{lsa} (\autoref{sec:lsi}), it is also possible to consider the entity whose \textit{pseudo-document} embeddings is closest in direction to the cluster direction.

S so far the best technique to find a cluster-name was not evaluated yet. All considered methods (of \mainalgos and here) that formally evaluate the corresponding feature-directions work independently of the actual cluster name. This is unfortunate, because \textit{subjectively}, the name of the respective directions is very important for the usability of any recommendation engine based on this work. Especially this subjectivity however indicates that the only way to evaluate the cluster-names is with study of human subjects.

\todoparagraph{Kappa in den Glossary!!}

\subsubsection{Postprocessing the Feature-Directions\arrowref{sec:algorithmsteps}}
\label{sec:algo:postprocess}

This step was the main contribution of the work of \textcite{Ager2018}. As has been shown that it increases the algorithm performance only slightly while adding a substantial amount of work, it was not implemented in the scope of this thesis. The Modification is described in \autoref{sec:ager}, its gist is to use the rank for quantified summed count of any cluster-word as weak supervision signal to distort the embeddings such that their feature directions better correspond to this value.

\subsubsection{Re-Embedding the entities into the new space\arrowref{sec:algorithmsteps}}
\label{sec:algo:reembed}

In the end we re-embed the entities into a space where each of the vector components is a semantic directions and the value are the respective \gls{rank}ings. That's what we then finally call its \textbf{feature-based representation} 

NOT a change of basis, only ordinal scale level bc rank, no linear independence.

from Alshaikh2020: "The learned vectors will be referred to as feature directions to emphasize the fact that only the ordering induced by the dot product d_i · e matters"

Maybe better idea is maths?






\subsection{Modifications from \textcite{Ager2018,Alshaikh2020}}

\subsubsection{\textcite{Ager2018}}
\label{sec:ager}


The main contribution of \textcite{Ager2018} is a postprocessing step that changes the final space such the ranking of entities \wrt each feature direction more closely mimics the ranking of frequencies of that direction's cluster words. The reasoning is that the original embeddings from which the feature directions are created are based on global similarity. This makes it very vulnerable to outliers which often take up extreme positions. If one now creates the feature directions from the space, these outliers are assumed to have certain properties. So the space is optimized for that, which limits the quality of feature directions in the space. Problem again is global similarity: If one entity ranks high for a feature, it is very likely that another entity that is close to that will also rank high for this feature, even though it may be something completely different. So to get better feature directions one has to distort the space. 

They do this fully unsupervisedly as an extra step after the full pipeline of \cite{Derrac2015}, by again using the BoW representation of the entites. After the clusters are collected and the entities re-embedded, for each feature a new ranking is computed by the summed frequency of any of a cluster's words per feature and entity. Each entity is thus represented as Bag-of-Clusters and again scored with PPMI to generate a ranking for each cluser/direction. This ranking is then used as a target for a simple \gls{ann} that distorts the space representation.

Generally this is a great idea. Among others the explicit usage of all cluster-words should help, as it is a lot less sparse than a single word that only occurs in 0.6\% of entites. However the results of their approach are mixed: For some of their considered datasets the fine-tuning even decreases performance - according to the authors especially \q{when the considered categorizes are too specialized} \cite{Ager2018}, because the resulting space is too much distorted towards the selected features.\footnote{See also \autoref{tab:f1_placetypes_long}} Considering its bad performance, this contribution was not considered in this work.

\textcite{Alshaikh2020}

\todo

\removeMe{
\subsection{Concluding stuff for algo}

\subsection{Features and differences to original algorithm}

\includeMD{pandoc_generated_latex/3_features_differences}

\subsection{Reasonable params}

\includeMD{pandoc_generated_latex/3_reasonableparams}

\subsection{Algorithm Complexity}
}


\section{Architecture}
\label{sec:architecture}

As elaborated in \autoref{sec:reproducibility}, one of the main motivations for this thesis was to create a publicly available \textit{open-source} version of the algorithm that is easily \textit{understood} and \textit{reproduced}, \textit{adaptable} for other datasets and methods, as well as fast and \textit{scalable}, meaning it can be run maximally efficient on single machines but also on compute clusters, such as the \acrshort{ikw} Grid.
%TODO: Hier schon eindeutig sagen dass es auf ner single machine infeasibly lange läuft und deswegen der ganze Bums fürs Grid nötig war!!

% Main goal: BETTER ARCHITECTURE. Most important things for that: scalability, modularity, transparency, reproducibility, understandability, objectiveness, systematicacy, sustainability, adaptability
% describing this because I want to encourage extending the code etc and for that not only the algorithm but also the architecture should be described 
% and I think that was successful: This codebase contains everything and finally fulfills code-standards! 

This section will outline the architecture that was developed in order to achieve the aforementioned results. The resulting pipeline is the result of a lot of trial-end-error, but fulfills all of the aformentioned criteria, dealing with vastly differing sizes and kinds of datasets, minimizing runtime wherever feasible and allowing for a multitude of parameters at every step of the process. %TODO: don't like this paragraph, lieber später nohcmal auf die design principles eingehen und sagen dass sie alle fulfilled sind.

The rest of this section will go into further detail regarding the architecture of the resulting code-base. \todoparagraph{it will start with xyz and then asdf and then yaddayadda}

\subsection{Implementation}

The associated program is written by the author of this work and licensed under the \emph{GNU General Public License} (GNU GPLv3). The source code is written in the Python Programming Language and available digitally on GitHub\footnote{Source code: \url{https://github.com/cstenkamp/derive_conceptualspaces/}\\Source of this Document: \url{https://github.com/cstenkamp/MastersThesisText/}\\Compiled Document: \url{https://nightly.link/cstenkamp/MastersThesisText/workflows/create_pdf_artifact/master/Thesis.zip}}. In order to ensure that no work after the deadline is considered, it is referred to the signed commits \todoparagraph{COMMIT} and \todoparagraph{COMMIT}. 

The code is a proper python-package that can be installed into any Python 3.10 environment using for example python's default package manager pip:\\ \mytokens{pip install git+https://github.com/cstenkamp/derive_conceptualspaces.git@main}~ .\\ It can then be run using \mytokens{python -m derive_conceptualspace <COMMAND>} \footnote{The command \mytokensfnote{python -m derive_conceptualspace --help} gives a peak into what sub-commands can be used}. For more information on how to invoke the code base with these commands it is referred to \autoref{ap:usecase_click}

To guarantee reusability of this code-base, there is also a \emph{Dockerfile}\footnote{{\url{https://github.com/cstenkamp/derive_conceptualspaces/blob/main/Dockerfile}}} that allows to easily create a \emph{Docker-Container\footnote{\url{https://www.docker.com/resources/what}}} from it\footnote{A Container can be thought of as a lightweight virtual operating system, in which the codebase is bundled together with all required dependencies, libraries and configurations, enabling users install this software on any system without having to download or install anything besides this container, irrespective of operating system or software versions on the host \acrshort{os}. For more info about the container, it is referred to \url{https://github.com/cstenkamp/derive_conceptualspaces/blob/main/doc/docker_intro.md}.}.

\subsection{Modularity}

The developed algorithm consists of clearly divisible components (as demonstrated in \autoref{fig:dependency_graph}), where the runtime for each of the steps is roughly in the same order of magnitude. All of the aforementioned (\autoref{sec:algorithm_steps}) steps are itself algorithms with many \gls{param} each. Furthermore, the framework described here does not even require particular algorithms for the individual components, but rather a classes of algorithms like \emph{dimensionality reduction techniques}. This means that in practice, there is a combinatorical explosion of settings and \glspl{param} that must be experimented with in order to find the best-performing one. Because of the clear modularity of the algorithm however, many of these become only relevant in a later step of the pipeline. Due to this, it is reasonable to make the architecture as modular as possible, storing interim results before every step, such that two parameter-combinations that differ only in \eg the fourth step of the pipeline can share the intermediate results up to that point, keeping the required computation to a minimum. 

The design principle of maximal modularity is the cornerstone of the developed pipeline. All of the interim results store the configurations that were required for the respective algorithm (and forward the ones of the input-files they transformed), as well as the created output and plots. When there are different possible algorithms for a step, it is ensured that its result are of the same format, as required by the next step. Many of the individual steps generate additional plots that can be used as sanity-checks to quickly inspect if the results so far are reasonable.

\subsubsection{Workflow Management}

A pipeline where multiple intermediate files for different parameter-combinations are created introduces the problem of \emph{dependency resolution}: Ultimately, there is supposed to be one final file for every combination. This file however relies on intermediate files, which in turn rely on intermediate files. To resolve these dependencies, there are many existing \textbf{Workflow Management Systems}. For this thesis, \textbf{Snakemake}\footnote{\url{https://snakemake.readthedocs.io/en/stable/}} \cite{Molder2021a} seemed the right choice.

Snakemake defines a small comprehensible domain-specific language ontop of python. With this, a workflow is described in terms of individual \textbf{rules}, each of which defining how an \textbf{output} is generated from several \textbf{inputs} using code or shell-commands. Through \textbf{wildcards}, these rules can be generalized and hyperparameters introduced \cite{Molder2021a}. The job of Snakemake is to infer a \gls{dag} from these, finding for every rule in the dependency tree for the demanded file an output that generates the required inputs, and to create jobs for all required instanciations of the wildcards if the required files are not already present. Importantly, Snakemake then also handles the inevitable scheduling problem: Due to (explicitly specified) restrictions of \acrshort{cpu} and \acrshort{ram} and the nature of the unresolved dependencies, not all jobs of the workflow can be executed simultaneously. Its scheduler favors maximal utilization of \acrshort{cpu} and parallelisation for minimal execution time \cite{Molder2021a}. Especially relevant was also that it allows to schedule these jobs on high performance clusters and computation grids, and supports among others the scheduling system \gls{sge} which is used to orchestrate jobs at the \gls{ikw} grid. Configurations for the grid, like the maximal runtime or the amount of \gls{ram} and \glspl{cpu} to request, can be specified per-rule as well as in special configuration files.
%TODO: gibt noch 1-2 buzzwords from paper, ich kann schonmal aufs Grid hinaus und dann halt der wann-ist-snakemake-sinnvoll-und-wann-nicht.

Snakemake was chosen because it is a lightweight system ontop of python, adding only a few lines of code to specify what inputs and outputs are created ontop of the \gls{cli} that is necessary to run and debug individual steps anyway. It is a useful tool if the workflow can be divided into rougly equally long steps which can run independently and heavily parallelized (possibly on multiple machines) with an optimal usage of resources. Its file-centric dependency resolution system allows to fill in missing steps seemlessly when working on specific configurations for later step, but on the other hand requires unintuitive customization if instead configuration-files with explicit parameter-choices declare the demanded output for dynamically generated filenames. Also it unfortunately doesn't allow debugging and has a comparably small community\footnote{As of \DTMdisplaydate{2022}{03}{16}{-1}, there are only 1256 question tagged ``snakemake'' on StackOverflow (\url{https://stackoverflow.com/questions/tagged/snakemake})}. \autoref{ap:usecase_snakemake} shows the different ways the full pipeline can be invoked using Snakemake.

%wenn viele parameter die an gwissen punkten relevant werden und später nicht mehr, wenn viele param-kombis, it's main thing is the automatic dependency resolvement (which means I can just tell it "hey I need this file" (automatically creating missing stuff), but with config-files you're abusing it. good for optimal CPU/RAM usage. Good if independent parallelzed steps, not if one main step. Have to abuse it for configs, no good way to debug, small comunities, nondynamic I need nondynamic filenames that are set from the start of the execution 


\subsection{Modes of Execution / Use-Cases}

It is possible to run the full pipeline for individual files as well as for a set of \gls{param}-configurations specified via configuration files, but also possible to run individual steps to inspect or debug the respective steps. To inspect and compare results it is possible to load all available parameter-configurations, as well as the complete history for a certain combination, listing the generated outputs and metrics. Further, individual configurations can be loaded in \emph{Jupyter Notebooks} to generate and export plots and tables from them (like the ones used in this text). The three main ways of exectution are:
%TODO: deutlicher drauf eingehen dass man wegen dem ganzen bums mit intermediate files undso speziell drauf achten muss dass 
% * keine plots/prints verloren gehen
% * man mitschreibt wann welche configs genutzt werden
% * immer eindeutig drauf geachtet wird dass dependencies für genau die konfigurationen as demanded verwendet werden!! 

\begin{description}[style=unboxed]
	\item[Running individual Steps per \gls{cli}] is the mode of choice when working on custom steps, as it allows to attach debuggers and executes in the main thread. If a later step is executed, it is also possible to automatically generated its required dependencies using the workflow-definition. Passing configurations is possible using configuration-files, command-line-arguments or enviroment-files/-variables. For usage-examples, see \autoref{ap:usecase_click}.
	\item[Loading existing Configurations for inspection] especially in Notebooks, allowing to easily load a complete configuration including all its dependencies to inspect and plot (intermediate) previously created results and outputs, also allowing to iterate over several configurations to compare their results\footnote{The tables used in thesis are also automatically exported as \LaTeX- code from the functions available there, as specified in their respetive references.}. For usage-examples, see \autoref{ap:usecase_notebook}.
	\item[Running/Scheduling a Workflow] This mode is used to execute several \gls{param}-combinations at once, specified via configuration-files. Thanks to heavy integration for cluster scheduling systems, this allows for heavily parallelisation of jobs. Executing such a workflow on computation clusters is special case of this and elaborated further in the following section. For usage-examples, see \autoref{ap:usecase_snakemake}.
\end{description}
% 3 ways: Snakemake for shitton of param-combinations, invididual steps via the CLI for looking, debugging, creating, and 
% context-loading for jupyter to inspect and plot results - allowing load-context, where you can call eg. `print(ctx.display_output("embedding"))` of every component, read in several configs, iterate over them, re-create plots, allow for show-data-info showing where plots are first used, ...


\subsubsection*{Running on the \gls{sge}}

Due to a combinatorical explosion in the \gls{param}-space as well as the computational complexity of the algorithm, running the pipeline a sufficient amount of parameter-combinations would take several weeks on a single machine\footnote{\todoparagraph{Give a few examples!}}. As the \gls{ikw} at the \gls{uos} owns a dedicated computation grid\footnote{\url{https://doc.ikw.uni-osnabrueck.de/content/grid-computing}} with considerable modern hardware\footnote{Currently comprising, among many others, of 26 machines with an i7-11700 \gls{cpu} and 64 GB \gls{ram}} which uses the \gls{sge} as workload manager, which is supported by snakemake, it was the obvious candidate. Snakemake encodes special configurations for clusters using \emph{profiles}\footnote{\url{https://snakemake.readthedocs.io/en/stable/executing/cluster.html}}, and there exists a profile for the Sun Grid engine\footnote{\url{https://github.com/Snakemake-Profiles/sge}}. Unfortunately, this default configuration does not take into account many of the pecularities of the \gls{ikw} grid and it needed to be heavily customized in order to work. Foremost, all available machines to \me have a runtime-limit of 90 minutes, which means all of the algorithm-steps that take longer than that must be able to be interrupted and gracefully shut down before getting killed and pick up the work on a new machine afterwards (including the job responsible for the workflow scheduling itself). Additionally, the arguments to request resources (such as \emph{memory} or \emph{parallel environments}) often differ from the documentation, and the \emph{accounting file} which keeps track if jobs succeeded is not available to users, so a custom one must be written. Resolving these and other issues required changing the available profile heavily, so the result was open-sourced\footnote{The resuling Snakemake-Profile is available and documented at \url{https://github.com/cstenkamp/Snakemake-IKW-SGE-Profile}. Note that it is heavily customized to the specific engine and thus includes explicit machine names or runtimes. This repository also contains convenience-terminal-commands to inspect failed pipeline-steps or to show the current progress of the current run. A sample output of the latter is presented in \autoref{lst:joblog}. Furthermore it contains \mytokensfnote{.sge}-files and shell-scripts to schedule or run a requested workflow (see \autoref{ap:usecase_snakemake})}. 


Scheduling on such engines interestingly unveils a whole new set of ``hyperparameters'' that have to be optimized to use the available hardware as efficiently as possible: there are limits of how many slots are available per user, there is a fixed walltime (and interrupting and restarting leads to overhead), and the effiency of multiprocessing is not linear in the number of threads per process. Thus, depending on the size of the dataset, resources must be divided among the steps with care. The required resources of the rules are accordingly dynamically allocated in the rule-descriptions of the workflow manager.

While the code required to scalably run on the \gls{ikw}-grid required much more work than expected, the result fulfills all demands perfectly, %TODO: WHAT demands
and the 64 allocated \emph{parallel environments} (slots) are maximally utilized, while most of the complexity of the scheduling system is abstracted away\footnote{To the best of \my knowledge, no attempts going beyond simple \mytokensfnote{.sge}-files as job-descriptions were attempted on the IKW-grid before, and much of the available documentation turned out to be false information (as consultations with the grid's administrator have shown).}. The workflow is installed and run with a single (well documented) command and can be customized using explicit configuration-files. A sample output of the custom-made watcher is listed in \autoref{lst:joblog}.  


\begin{widepage}
	\lstconsolestyle
	\lstinputlisting[
		caption={[Sample terminal output of the custom watcher, when running a full configuration on the grid.]Sample terminal output of the custom watcher, when running a full configuration on the \gls{ikw}-grid. The script lists the currently running jobs continously, including their progress and runtime and informs of finished jobs and failed jobs. There is another script that summarizes the progress as per snakemake's dependency-graph.}, 
		label={lst:joblog},
		float,
		floatplacement=h!,
		xleftmargin=-0.5cm, 
		xrightmargin=-0.5cm,
		]{listings/joblog\_grid.txt}
	\lstdefaultstyle
\end{widepage}

% \includeMD{pandoc_generated_latex/chapter_methods_section_architecture}

\subsection{Conclusion}

It was originally unexpected, but implementing an appropriate architecture for the present codebase has been a major focus of work for this thesis, and the result fulfills all of the desired design criteria: 

% reproducibility alone is not enough to sustain the hours of work that scientists invest in crafting data analyses. Here, we outlined how the interplay of automation, scalabil- ity, portability, readability, traceability, and documentation can help to reach beyond reproducibility, making data analyses adaptable and transparent.

\begin{description}[style=unboxed]
	\item[Modularity] has been the main focus in the design, so exchanging components or running individual steps is easy and intuitive.
	\item[Scalability] is reached thanks to massive parallelisation wherever possible as well as a professional workflow management system that is perfectly adjusted to the available cluster engine but also highly customizable for other engines.
	\item[Reproducibility and Adaptability] are guaranteed by rigorous encapsulation of components, completely automating the full data-analysis-pipeline, open-sourcing the code as proper package and containerization of the entire codebase for guaranteed and worry-free setup on any machine or compute cluster. The exact \gls{param}-combinations of \mainalgos are included (see \autoref{ap:yamls_for_origalgos}), allowing to re-create even the original papers using this code-base. Running the code on new datasets is extensively documented\footnote{\todoparagraph{TODO: link to that}} and a matter of minutes. Extending or exchanging steps of the pipeline is seamless due to a consistent and understandable data schema, and pre-existing analysis-notebooks can easily create informative plots and figures.
	\item[Transparency and Understandability] are ensured due to rigorous documentation\footnote{\todoparagraph{Link github-documentation!}} (among others in this thesis) at any level of detail, from rough descriptions to concrete code-examples. Code, documentation and used data are publicly and easily available. Many analyses are inlcuded with the source-codes, for example allowing to visualize all steps of the process that can work with arbitrary numbers of dimension interactively in 3D. Code, data and configurations are clearly divided. All steps of the pipeline are very explicit about the used configurations and dependencies (making them traceable) and generate output at configurable levels of verbosity. All intermediate output can be re-accessed using helper commands (see \footnote{\todoparagraph{ref the appendix with the show-info-command and a notebook with} \mytokensfnote{create_svm("mathematik", embedding, dcm, pp_descriptions, highlight=["Informatik A: Algorithmen", "Informatik B: Grundlagen der Software-Entwicklung"])}}), including clear traces of the first usage of parameters (as \eg in a plot as depicted in \autoref{fig:dependency_graph})
	% it is crucial that the analysis code is as readable as possible such that it can be easily modified (looking at you, 40 unnamed cmd-args!)
	% code is readable and well-documented 
	% mit 2 Zeilen code kannst du dir in nem Jupyternotebook nen 3D-Plot anzeigen mit ner SVM die "Mathematik" von nicht-mathe trennt, mit gehighlighted ob "Informatik A" und "Informatik B" beeinander sind
\end{description}

%RESULTS
	\chapter{Results}


\begin{table}[h]
	\resizebox{\textwidth}{!}{%
	\begin{tabular}{llllrrrrrrrrr}
	\toprule
% 	 &  &  &  & \rotatebox{70}{\textbf{k_r2r_d}} & \rotatebox{70}{\textbf{k_r2r_min}} & \rotatebox{70}{\textbf{k_dig}} & \rotatebox{70}{\textbf{k_r2r+_d}} & \rotatebox{70}{\textbf{k_r2r+_min}} & \rotatebox{70}{\textbf{k_r2r+_max}} & \rotatebox{70}{\textbf{k_dig+_2}} & \rotatebox{70}{\textbf{k_c2r+}} & \rotatebox{70}{\textbf{mean}} \\
	\textbf{Preprocessing} & \specialcell[b]{\textbf{Quanti-}\\ \textbf{fication}} & \textbf{\#Dims} & \specialcell[b]{\textbf{Doc-Term-}\\ \textbf{Matrix} \\ \textbf{Quanti-}\\ \textbf{fication}} & \rotatebox{70}{\textbf{r2r-d}} & \rotatebox{70}{\textbf{r2r-min}} & \rotatebox{70}{\textbf{dig}} & \rotatebox{70}{\textbf{r2r+d}} & \rotatebox{70}{\textbf{r2r+min}} & \rotatebox{70}{\textbf{r2r+max}} & \rotatebox{70}{\textbf{dig+2}} & \rotatebox{70}{\textbf{c2r+}} & \rotatebox{70}{\textbf{mean}} \\
	\midrule
	\multirow[t]{24}{*}{\mfauhcsdT} & \multirow[t]{8}{*}{\textbf{count}} & \multirow[t]{2}{*}{\textbf{3}} & \textbf{ppmi} & 0 & 1 & 0 & 145 & 370 & 510 & 191 & - & 174 \\
	 &  &  & \textbf{tfidf} & 0 & 1 & 0 & 110 & 237 & 278 & 83 & - & 101 \\
	\cline{3-4}
	 &  & \multirow[t]{3}{*}{\textbf{100}} & \textbf{count} & 0 & 5 & 0 & 0 & 114 & 52 & 290 & 0 & 58 \\
	 &  &  & \textbf{ppmi} & 0 & 6 & 27 & 139 & 224 & 247 & 120 & - & 109 \\
	 &  &  & \textbf{tfidf} & 0 & 6 & 5 & 246 & 270 & 281 & 201 & - & 144 \\
	\cline{3-4}
	 &  & \multirow[t]{3}{*}{\textbf{200}} & \textbf{count} & 0 & 5 & 1 & 0 & 133 & 52 & 509 & 0 & 88 \\
	 &  &  & \textbf{ppmi} & 0 & 6 & 57 & 196 & 315 & 344 & 90 & - & 144 \\
	 &  &  & \textbf{tfidf} & 0 & 6 & 17 & 357 & 370 & 372 & 433 & - & 222 \\
	\cline{2-4} \cline{3-4}
	 & \multirow[t]{8}{*}{\textbf{ppmi}} & \multirow[t]{2}{*}{\textbf{3}} & \textbf{ppmi} & 0 & 0 & 0 & 192 & 247 & 363 & 136 & - & 134 \\
	 &  &  & \textbf{tfidf} & 0 & 0 & 0 & 169 & 206 & 217 & 59 & - & 93 \\
	\cline{3-4}
	 &  & \multirow[t]{3}{*}{\textbf{100}} & \textbf{count} & 0 & 0 & 0 & 0 & 38 & 25 & 242 & 0 & 38 \\
	 &  &  & \textbf{ppmi} & 0 & 0 & 0 & 80 & 112 & 101 & 22 & - & 45 \\
	 &  &  & \textbf{tfidf} & 0 & 0 & 0 & 89 & 90 & 96 & 85 & - & 51 \\
	\cline{3-4}
	 &  & \multirow[t]{3}{*}{\textbf{200}} & \textbf{count} & 0 & 0 & 0 & 0 & 34 & 21 & 293 & 0 & 44 \\
	 &  &  & \textbf{ppmi} & 0 & 1 & 112 & 100 & 163 & 163 & 37 & - & 82 \\
	 &  &  & \textbf{tfidf} & 0 & 1 & {\cellcolor{lightgreen}} 127 & 99 & 107 & 106 & 131 & - & 82 \\
	\cline{2-4} \cline{3-4}
	 & \multirow[t]{8}{*}{\textbf{tfidf}} & \multirow[t]{2}{*}{\textbf{3}} & \textbf{ppmi} & 0 & 0 & 0 & 229 & 357 & 423 & 84 & - & 156 \\
	 &  &  & \textbf{tfidf} & 0 & 0 & 0 & 169 & 255 & 258 & 24 & - & 101 \\
	\cline{3-4}
	 &  & \multirow[t]{3}{*}{\textbf{100}} & \textbf{count} & 0 & 1 & 0 & 0 & 162 & 64 & 450 & 0 & 85 \\
	 &  &  & \textbf{ppmi} & 0 & 1 & 3 & 324 & 404 & 423 & 151 & - & 187 \\
	 &  &  & \textbf{tfidf} & 0 & 1 & 0 & 390 & 422 & 437 & 425 & - & 239 \\
	\cline{3-4}
	 &  & \multirow[t]{3}{*}{\textbf{200}} & \textbf{count} & 0 & 2 & 0 & 0 & 211 & 83 & {\cellcolor{lightgreen}} 869 & {\cellcolor{lightgreen}} 1 & 146 \\
	 &  &  & \textbf{ppmi} & 0 & 2 & 13 & 395 & {\cellcolor{lightgreen}} 559 & {\cellcolor{lightgreen}} 577 & 153 & - & 243 \\
	 &  &  & \textbf{tfidf} & 0 & 2 & 0 & {\cellcolor{lightgreen}} 531 & 554 & 572 & 794 & - & {\cellcolor{lightgreen}} 350 \\
	\cline{1-4} \cline{2-4} \cline{3-4}
	\multirow[t]{24}{*}{\mfauhtcsldp} & \multirow[t]{8}{*}{\textbf{count}} & \multirow[t]{2}{*}{\textbf{3}} & \textbf{ppmi} & 0 & 1 & 0 & 226 & 319 & 317 & 208 & - & 153 \\
	 &  &  & \textbf{tfidf} & 0 & 1 & 0 & 210 & 214 & 215 & 82 & - & 103 \\
	\cline{3-4}
	 &  & \multirow[t]{3}{*}{\textbf{100}} & \textbf{count} & 0 & 7 & 0 & 0 & 118 & 61 & 230 & 0 & 52 \\
	 &  &  & \textbf{ppmi} & 0 & 8 & 27 & 184 & 256 & 262 & 125 & - & 123 \\
	 &  &  & \textbf{tfidf} & 0 & 8 & 5 & 253 & 255 & 255 & 168 & - & 135 \\
	\cline{3-4}
	 &  & \multirow[t]{3}{*}{\textbf{200}} & \textbf{count} & 0 & 8 & 0 & 0 & 117 & 64 & 290 & 0 & 60 \\
	 &  &  & \textbf{ppmi} & 0 & {\cellcolor{lightgreen}} 11 & 41 & 200 & 319 & 325 & 88 & - & 141 \\
	 &  &  & \textbf{tfidf} & 0 & {\cellcolor{lightgreen}} 11 & 8 & 331 & 333 & 333 & 302 & - & 188 \\
	\cline{2-4} \cline{3-4}
	 & \multirow[t]{8}{*}{\textbf{ppmi}} & \multirow[t]{2}{*}{\textbf{3}} & \textbf{ppmi} & 0 & 0 & 0 & 138 & 310 & 321 & 254 & - & 146 \\
	 &  &  & \textbf{tfidf} & 0 & 0 & 0 & 143 & 148 & 150 & 187 & - & 90 \\
	\cline{3-4}
	 &  & \multirow[t]{3}{*}{\textbf{100}} & \textbf{count} & 0 & 0 & 0 & 0 & 29 & 11 & 186 & 0 & 28 \\
	 &  &  & \textbf{ppmi} & 0 & 1 & 0 & 117 & 142 & 142 & 20 & - & 60 \\
	 &  &  & \textbf{tfidf} & 0 & 1 & 0 & 122 & 124 & 124 & 103 & - & 68 \\
	\cline{3-4}
	 &  & \multirow[t]{3}{*}{\textbf{200}} & \textbf{count} & 0 & 1 & 0 & 0 & 25 & 10 & 272 & 0 & 38 \\
	 &  &  & \textbf{ppmi} & 0 & 1 & 48 & 126 & 161 & 165 & 28 & - & 76 \\
	 &  &  & \textbf{tfidf} & 0 & 1 & 17 & 143 & 144 & 148 & 133 & - & 84 \\
	\cline{2-4} \cline{3-4}
	 & \multirow[t]{8}{*}{\textbf{tfidf}} & \multirow[t]{2}{*}{\textbf{3}} & \textbf{ppmi} & 0 & 0 & 0 & 146 & 219 & 223 & 133 & - & 103 \\
	 &  &  & \textbf{tfidf} & 0 & 0 & 0 & 108 & 111 & 109 & 38 & - & 52 \\
	\cline{3-4}
	 &  & \multirow[t]{3}{*}{\textbf{100}} & \textbf{count} & 0 & 1 & 0 & 0 & 160 & 54 & 389 & 0 & 76 \\
	 &  &  & \textbf{ppmi} & 0 & 2 & 9 & 281 & 375 & 380 & 205 & - & 179 \\
	 &  &  & \textbf{tfidf} & 0 & 2 & 0 & 373 & 377 & 392 & 339 & - & 212 \\
	\cline{3-4}
	 &  & \multirow[t]{3}{*}{\textbf{200}} & \textbf{count} & 0 & 3 & 0 & 0 & 199 & 64 & 661 & 0 & 116 \\
	 &  &  & \textbf{ppmi} & 0 & 3 & 21 & 362 & 456 & 472 & 164 & - & 211 \\
	 &  &  & \textbf{tfidf} & 0 & 3 & 1 & 499 & 498 & 501 & 645 & - & 307 \\
	\bottomrule
	\end{tabular}
	}
	\caption{Number of Candidate-Phrases for different parameter-combinations and kappa-values \label{tab:kappa_table}}
	\label{tab:cands_per_config}
\end{table}

	
		
\section{Evaluation} %TODO what is this for a title? it sucks

% * viel showcasen. Also mir 10 Kurse raussuchen die laut algorithmus nahe beieinander sind und sagen "hier, wir sehen die sehen tatsächlich related aus".
% * Drauf eingehen dass ich nicht wirklich metriken habe. This is not classification, I don't get a "87% accuracy" and can compare that, ich krieg cluster und muss mir angucken ob die "ähnlich sind wie ein mensch das macht" -> qualitative und quantitative analyse, beides mit grains of salt (quantiativ ertrickst metriken halt nen bissle -> Clustert das so dass ich den fachbereich dadrin wiedererkenne? )
% * Brauch ich mehr/bessere Daten? Wenn ich nur die 1000 mit den längsten Beschreibungen behalten würde und dann 10 solcher subsets hätte wären halt die Fälle wie "Tutoren sind: Susi Sorglos Willi Wacker" etc raus
% * ...ich kann auch mit Johannes' Datensatz mit Mapping Kurstitel -> DDCs vergleichen und gucken ob die shallow decision trees was ähnliches wie die DDCs extracten können als weiteren Klassifizerungs-Task nehmen! (see Masterarbeit/OTHER/study_behavior_analysis/EducationalResource-2022-01-20.csv), dann kann ich auch das Siddata/SidBert-Paper von Johannnes Referenzieren!
% * Was man als testing halt machen kann ist nen decision tree based on their features zu machen und zu gucken ob der einen held out test dataset klassifizieren kann.
% * Ein anderer Weg zum testen wäre auch ein classifier der nur anhand der most salient generated features versucht den kurs wiederherzustellen (das zeigt natürlich nicht ob es similar to how humans do it but part of it)
% * Metrik: Gucken ob es ähnlich wie FB Clustert => Da kann ich dann die Parameterkombi die die im paper gemacht haben plus nen paar andere in ne tabelle packen und fertig
% * Check my claim in the results for place-types (chapter 6.1), that the classification based on word embeddings may even be better than their SVM_MDS!!!
% * Ich hab ja den Fachbereichs-Classifier gemacht, wenn ich jetzt noch die shallow decision trees mache kann ich ja legit accuracies vergleichen

\begin{itemize}
	\item \cite{Derrac2015} evaluated using a bunch of commonsense reasoning based classifiers (want to show that at least as performant than standard approaches, but can give intuitive explanations) (these reasoning-classifiers can be linked to intuitive explanations: 1-NN is "Y is of the same class as X because X closest to Y", but also more complex ones.) 
	\item 
\end{itemize}

* This is clustering and looking if it corresponds to human judgement, which unfortunately doesn't allow for a simple accuracy and be done with it.
* So, the papers that did this come up with a few things
* [TODO: the shallow decisiontrees of one of the followups]
* DESC15 "evaluate the practical usefulness of the considered semantic relations" by checking "their use in commonsense reasoning based classifiers", like interpolation and a fortiori inference (chap 5)


* DESC15 tests like this: Section 6.1: Evaluate whether the derived relations are sufficiently accurate for classification, and 6.2 is then comparison with crowdsourcing experiments (more subjective aspects, the question “are the relations useful explanations?”)



\section{Qualitative Analysis}

Qualitative Analysis in this case means "looking at stuff". Such a qualitative analysis is always to be taken with a grain of salt, because it is very prone to cherry-picking (both on purpose and not on purpose, the stuff you're looking at just doesn't need to be representative!). However it does help a lot and provides a lot of insights (and often helped me in the debugging process).
What can you look at for such a qualitative analysis?
\begin{itemize}
	\item The clusters, checking if things you know to be similar are actually in the same clusters
	\item If descriptions you know to be semantically similar are actually close in the embedding
	\item You can do the whole thing for only three dimensions instead of the 50/100/200 because there you can plot the stuff and interpret it
\end{itemize}

\begin{itemize}
	\item  Man kann ja schon nach dem Embedding anhand der nächsten Entities sehen ob das was werden kann - bei 100D sind dann halt "airplane cabin" und "aircraft cabin" die nächsten entities, bei 3D dann halt eher kram wie "area" and "moor", was schon eindeutig zeigt dass 3D offensichtlich nicht so der Hit ist
	\item Die vielen Sanity Checks die man machen kann, bspw dass ich ja in 3D gucken kann (und auch in höher-D ausrechnen) ob eben diese dinge (von item 1) im Embedding nah sind, und ob die SVM Dinge schön trennt ("howto_embed.ipynb")
	\item "placetypes_origconf.ipynb", was einfach von vorne bis hinten die original-config (ist ja auch im yaml) von DESC15 ausführt und interpretiert	
\end{itemize}

\begin{itemize}
	\item Ist "Mathe" ein Keyword, clustern "a1" und "a2", ...
	\item Ist "Codierungstheorie und Kryptographie" - mathe = "Kryptographische Methoden in der Informatik"?
	\item Question: Does the continuation thingy which they have (backtothefuture:backtothefutureII::terminator:terminator2) hold for courses as well - Verhält sich Informatik A zu Informatik B wie Mathe für Anwender 1 zu Mathe für Anwender 2?  Info B zu Info A genau wie Statistik 2 zu Statisik 1? 
	\item Paperlesen und den cluster von "pub" für placetypes angucken
\end{itemize}


\begin{figure}[H]
	\centering
	\includegraphics[width=\figwidth]{svm_mathematik_highlight_infoAB.png}
	\caption[3D-Plot with an SVM for the term "Mathematik"]{
		\label{fig:3dplot_mathe_infoab}
		3D-Plot with an SVM for the term "Mathematik", also highlighting the courses "Informatik A" and "Informatik B"
	}
\end{figure}

In figure \ref{fig:3dplot_mathe_infoab} we see a 3D-Embedding for courses, splitting courses which contain the term "mathematics" from those that don't, also hightlighting the terms "Informatik A" and "Informatik B". We see they are close we see the SVM is not to bad, and even though neiher Info A nor Info B contains the word "mathematik", thy are both on the "mathematical side" of courses. Negative samples are hidden for better visibility, and entities that contain the word more-often-than-the 75th (???) percentile have bigger markers.


\begin{itemize}
	\item In 3D ists immer ne Kugel, und ich würde behaupten in höheren Dimensionen ist es nicht extrem viel besser. dadrin ne SVM zu machen bringt echt wenig bis gar nix (Ich hab ja sogar Plots die zeigen dass die Movies viel besser clustern - TODO: die einbringen)
\end{itemize}


\section{Quantiative Results}

% Schreiben was die paper denen ich mostly folge zur evaluation gemacht haben! ("To evaluate whether the discovered features are semantically meaningful, we test how similar they are to natural categories, by training depth-1 decision trees")
% Ein anderer Weg zum testen wäre auch ein classifier der nur anhand der most salient generated features versucht den kurs wiederherzustellen (das zeigt natürlich nicht ob es similar to how humans do it but part of it)

Here I'll add the results of the low-depth-decision-trees for Fachbereich, and also compare the results of throwing my code onto their placetypes-dataset and how my results compare to theirs 
(set overlap of candidate terms!)

To see if it is possible to extract any kind of structured data from the unstructured course descriptions, a Neural Network classifier was trained on the dataset, classifying courses to the faculty they run under. 
$\rightarrow$ Der FB-Classifier kommt auf $95.33\%$ train, $90.96\%$ Test accuracy nach 10 epochs, that's a lot!!


Both \cite{Ager2018} and \cite{Alshaikh2020} train shallow decision-trees (depth 1 and depth 3 each), on their feature-based representations (such that the 1 or 3 most distinct interpretable dimensions are used) on a known property of the data (genres for movies, category in some taxonomy for placetypes, fachbereich for mine) - in the assumption that these eg in the movie domain the genre (or rather *terms accurately predicting it*) is among the features.


TODO die Plots mit den Boxen von display_desc15_top3.ipynb !!!

\begin{itemize}
	\item result: set overlap of my extracted candidates for placetypes and theirs (und auch die big_21222.yml ergebnisse danach auswerten) (nicht nur overlap, ich kann auch verhältnis set intersect zu set union machen, und die als true/false positive/negative deklarieren und dann accuracy, f1 etc analysieren und halt anhand dessen "die hyperparam kombi die am closesten zu deren ergebnissen ist" rausbekommen)
	\item result: kommt accuracy etc von den shallow decision trees für fachbereich close an die vom fb-classifier?
\end{itemize}

%DISCUSSION
	% (was sind die broaden takeaways von meinem Kram)
	% * Nochmal nen theoretisches Embedding, Kontextualisieren für Bildungsressourcen
	% * ...and conclusion
	
\chapter{Discussion and Conclusion}
% (was sind die broaden takeaways von meinem Kram)
% * Nochmal nen theoretisches Embedding, Kontextualisieren für Bildungsressourcen
% * ...and conclusion

\section{Future Work}

\includeMD{pandoc_generated_latex/5_0_futurework}



\section{Discussion}

\includeMD{pandoc_generated_latex/5_1_discussion}


\section{Conclusion}



% \chapter*{Acknowledgements}
%TODO A place to say thank you to everybody who helped you.


% START Acronym definitions
\newacronym{utc}{UTC}{Universal Time Coordinated}
\newacronym{ml}{ML}{Machine Learning}
\newacronym{svm}{SVM}{Support Vector Machine}
\newacronym{mds}{MDS}{Multi Dimensional Scaling}
\newacronym{ppmi}{PPMI}{Positive Pointwise Mutual Information}
\newacronym{bow}{BoW}{Bag Of Words}
\newacronym{imdb}{IMDB}{Internet Movie Database}
% END Acronym definitions

\glsaddall
\printglossaries %TODO: let glossary appear in TOC

%----------------------------------------------------------------------------------------
%	THESIS CONTENT - APPENDICES
%----------------------------------------------------------------------------------------
	
	\appendix % Cue to tell LaTeX that the following "chapters" are Appendices

	\cleardoubleoddpage
	\phantomsection
	\addcontentsline{toc}{part}{\appendixname}
	\setchapterpreamble[o]{% siehe KOMA-Script-Anleitung
  		\usekomafont{disposition}\usekomafont{part}\appendixname\bigskip} 
	% the three above are to show the appendix as section, see https://golatex.de/viewtopic.php?p=22119&sid=ebe25f27fce1765ab7d8b1d2e91ee979#p22119

	% Include the appendices of the thesis as separate files from the Appendices folder
	% Uncomment the lines as you write the Appendices
	
	% Appendix A

% \newgeometry{
% 	a4paper,
% 	top=21mm,
% 	bottom=11mm,
% 	inner=24mm,
% 	outer=9mm,
% } %bindingoffset=.5cm

%\newgeometry{
%	a4paper, inner=1.9cm, outer=1.9cm, bindingoffset=1.3cm, top=1.5cm, bottom=1.5cm, 
%} %bindingoffset=.5cm


% \lstset{
% 	numberblanklines=false
% 	,basicstyle=\ttfamily%
% 	,breaklines=true%
% 	,tabsize=1%
% 	,showstringspaces=false%
% 	,numbers=left%
% 	,numbersep=\lstnumbersep%
% 	,numberstyle=\lstnumberstyle%
% 	,framesep=0pt%
% 	,xleftmargin=\lstnumberwidth%
% 	,framexleftmargin=\lsthorizontalpadding%
% 	,xrightmargin=\lsthorizontalpadding%
% 	,framexrightmargin=\lsthorizontalpadding%
% 	,backgroundcolor=\color{verylightgray}%
% 	,postbreak=\ding{229}\space%
% 	,escapeinside={*(}{*)}
% 	\linespread{1.0}
% }


\chapter{Code Use-Cases (and how to call it)} % Main appendix title

\label{AppendixA} 

\vspace{-0.8cm}

\@input{pandoc_generated_latex/appendix_a}

%\lstinputlisting[language=Python, firstline=29]{codes/dqn.txt}


	% Appendix B
% https://tex.stackexchange.com/questions/152829/how-can-i-highlight-yaml-code-in-a-pretty-way-with-listings

% \newgeometry{
% 	a4paper,
% 	top=21mm,
% 	bottom=11mm,
% 	inner=24mm,
% 	outer=9mm,
% } %bindingoffset=.5cm

%\newgeometry{
%	a4paper, inner=1.9cm, outer=1.9cm, bindingoffset=1.3cm, top=1.5cm, bottom=1.5cm, 
%} %bindingoffset=.5cm



\chapter{Implementation Details} % Main appendix title

A main goal of this thesis is to provide a code base that makes it as simple as possible to get started with \gencite{Derrac2015} algorithm to derive rudimentary conceptual spaces for any kind of dataset. In order to achieve this, documenting some implementation details and design decisions is crucial.
% TODO: something a la "es ist aber zu detailliert für den hauptteil und zerstört den lesefluss, deswegen ist der aufbau halt so dass der Hauptteil/der methods-section sich möglichst kurz fasst, wie halt die methods-section von nem Paper, und ebendieser appendix für diejenigen gedacht ist die den spezifischen Algorithmus genauer wissen wollen ODER den code nutzen wollen ODER sich einfach fragen warum dinge so sind wie sie sind. Also I HAVE to cite some of the used techniques as per their licences.
This appendix goes into more detail for selected components of the algorithm.

\label{AppendixB} 

\section{Algorithm Implemetatation Details}

\subsection*{Preprocessing}

see \autoref{sec:algo_preproc}

\subsubsection*{Language-Detection and Translation}
\label{ap:translating}

To check the languages of the entities, the \codeother{langdetect}\footnote{\url{https://pypi.org/project/langdetect/}, \textcite{nakatani2010langdetect}} library is used, which is a direct port of a java library that claims to have 99.8\% accuracy on longer texts \cite{nakatani2010langdetect}. 
\newline

Depending on the translation-policy, it is possible to either take only those entities of the demanded language, ignore it and consider all entities in their original language, or enforce the demanded language by translating all entities from their original language to the demanded one. The accompaning code for this thesis contains extensive code to do that using the \emph{Google Cloud Translation API}\footnote{\url{https://cloud.google.com/translate}}. Many descriptions of the SIDDATA-dataset were translated using this technique\footnote{As, however, only 500.000 characters per google-account and month can be translated \href{https://cloud.google.com/translate/pricing}{free of charge}, the translation-process for the descriptions is still in progress.}. As of now, Google's Cloud Translation Service uses an embedding-based neural model of a hybrid architecture that has a transformer encoder, followed by an RNN decoder \cite{Chen2018}. All of the languages detected in the SIDDATA-dataset are supported by the system - translating between the languages German, English and Spanish, which make up \todoparagraph{HOWMANY} percent of the SIDDATA-descriptions, is what the system is particularly optimized for. 
\todoparagraph{write short about their percentage, bleu score etc}

\includeMD{pandoc_generated_latex/6_1_implementationdetails}

\subsection*{Candidate Extraction}

see \autoref{sec:extract_cands}

\subsubsection*{KeyBERT}
\label{ap:details_keybert}

The \emph{KeyBERT}-algorithm\footnote{\label{fnote:keybertgibhut}\fullcite{MaartenGr2021}} \cite{grootendorst2020keybert} is one of the techniques used to select phrases of the text-corpus as candidates for \gls{feature}-directions. 

KeyBERT is a keyword-extraction technique \q{that leverages BERT embeddings to create keywords and keyphrases that are most similar to a document}\footnoteref{fnote:keybertgibhut}. \Gls{bert} is a neural language representation model that is able to embed both words and documents. Its embeddings are obtained by training a multi-layer bidirectional transformer encoder \gls{ann} architecture on a task in which a masked word must be predicted from the its bidirectional context as well subsequent fine-tuning tasks \cite{Devlin2019}. To extract keywords, the KeyBERT algorithm embeds both the document as well as its containing \glspl{ngram} of a configurable length using BERT and returns those phrases whose embedding ist most similar to the document-embedding according to the cosine-similiarity\footnoteref{fnote:keybertgibhut}.

The KeyBERT-model was incorporated to extract key-phrases for this codebase in two ways: 

\paragraph{KeyBERT-original} runs the algorithm on the unprocessed original texts. This is reasonable, as this is what BERT-embeddings are trained on, however it has the disadvantage that it requires a lot of post-processing to match the extracted phrases to the processed descriptions (which \eg may contain only lemmas or have their \glspl{stopword} removed)
\paragraph{KeyBERT-preprocessed} alleviates this problems by running the algorithm on already preprocessed texts. This may however lead to worse results, as the algorithm was trained on unprocessed natural sentences.

In practice, though both variants extracted different phrases, the results for either of the technqiues did not differ significantly.


\subsection*{Candidate Filtering}
\label{ap:algo_filter}



\begin{figure}[H]
	\begin{center}
	  \includegraphics[width=\textwidth]{3dplot_hyperplane_and_orthogonal}
	  \caption[Visual representation of the Hyperplane of an SVM splitting a dataset]{ \label{fig:3d_hyperplane_ortho} Visual representation of the Hyperplane of a Support-Vector-Machine splitting a dataset, as well as it's orthogonal and the orthogonal projection of a set of samples onto the plane. For an interactive version of this plot, visit  {\small \url{https://nbviewer.org/github/cstenkamp/derive_conceptualspaces/blob/main/notebooks/text_referenced_plots/hyperplane_orthogonal_3d.ipynb?flush_cache}}}
	\end{center}
\end{figure}


\begin{table}[H]
    \centering
    \resizebox{\textwidth}{!}{%
    \begin{tabular}{llllll}
    \textbf{Long}                    & \textbf{Short} & \textbf{Data} & \textbf{Quantifications} & \textbf{Distances} & \textbf{Comments}                \\ \midrule
    rank2rank\_dense          & r2r-d          & all           & Dense-Ranked           & Dense-Ranked     &                                  \\
    rank2rank\_min            & r2r-min        & all           & Min-Ranked               & Dense-Ranked     &                                  \\
    bin2bin                   & b2b            & all           & Binary                   & Binary             & Disregards rankings              \\
    digitized &
      dig &
      all &
      Digitized &
      Digitized &
      \begin{tabular}[c]{@{}l@{}}Bins decided by np.histogram\_bin\_edges \\ from min and max of all data\end{tabular} \\
    count2rank\_onlypos       & c2r+           & positive      & Unchanged                & Dense-Ranked     & Only for Count as Quantification \\
    rank2rank\_onlypos\_dense & r2r+d          & positive      & Dense-Ranked           & Dense-Ranked     &                                  \\
    rank2rank\_onlypos\_min   & r2r+min        & positive      & Min-Ranked               & Min-Ranked         &                                  \\
    rank2rank\_onlypos\_max   & r2r+max        & positive      & Max-Ranked               & Max-Ranked         &                                  \\
    digitized\_onlypos\_1 &
      dig+1 &
      positive &
      Digitized &
      Digitized &
      \begin{tabular}[c]{@{}l@{}}Bins decided by np.histogram\_bin\_edges \\ from min and max of all data\end{tabular} \\
    digitized\_onlypos\_2 &
      dig+2 &
      positive &
      Digitized &
      Digitized &
      \begin{tabular}[c]{@{}l@{}}Bins decided by np.histogram\_bin\_edges \\ from min and max of all positive data\end{tabular}
    \end{tabular}%
    }
    \caption{Dense means: if there are 14.900 zeros, the next is a 1
    Min means: if there are 14.900 zeros, the next one is a 14.901
    Max means: if there are 14.900 zeros, they all get the label 14.900
    These scores are weighted}
    \label{tab:kappa_measures}
\end{table}





\subsection*{Faculty-Classifier}
\label{sec:faculty_classifier}

As one of the evaluations is to compare the results of classifiers based on the algorithm here with a powerful classification algorithm, a neural network that classifies the Faculty of a course in the Siddata-Dataset was also implemented. The implementation for that will not be elaborated upon except that it is available at \todoparagraph{Link to the repo}, it relies on sacred,\footnote{\todoparagraph{link to sacred, note that to get the results like I did you'll need a MongoDB in a docker container, see this link}} and that it uses Google's `universal-sentence-encoder-multilingual` in Version 3 (linear in textlength, thus managable time and space requirements) plus a few classification layers ontop. The encoder is trained \q{on a number of natural language prediction tasks that require modeling the meaning of word sequences rather than just individual words},\footnote{Quote from their description at \url{https://tfhub.dev/google/collections/universal-sentence-encoder/1} (accessed \date{2022}{03}{25})} aimed being the base for architectures for many NLP tasks through the usage of sentence embeddings \cite{Guo}. It was trained on with a train-test-split of 90/10 (the results being consitent through sampled cross-validation)


\todoparagraph{Another purpose of the classifier} is to check if it is anyhow possible to extract meaningful information from the descriptions: If it is possible to train a classifier on the data that can reasonably predict a qualitative feature, there is enough structure in the data such that the algorithm I'm about to produce can work. 

Also, we have a lower bound for useful data: we can just throw away data that cannot be classified!!

% #########################################################################################################################################################################################################################################################################################################################################################################################################################################################################################################################################################################################################################################################################################################################################################################################################################################################

\section{Used Software}

\includeMD{pandoc_generated_latex/6_2_usedsoftware}

% #########################################################################################################################################################################################################################################################################################################################################################################################################################################################################################################################################################################################################################################################################################################################################################################################################################################################

\section{Configurations to run \mainalgos}
\label{ap:yamls_for_origalgos}

% \vspace{-0.8cm}

% \lstinputlisting[language=, firstline=29]{codes/dqn.txt}

\subsection{\textcite{Derrac2015}}

\begin{lstlisting}[language=yaml, caption={YAML for \textcite{Derrac2015}}]
    pp_components:          mfautcsdp
    translate_policy:       translate
    quantification_measure: ppmi
    dissim_measure:         norm_ang_dist
    embed_algo:             mds
    embed_dimensions:       [20, 50, 100, 200]
    extraction_method:      pp_keybert
    max_ngram:              5                   
    dcm_quant_measure:      count
    classifier:             SVM
    kappa_weights:          quadratic
    classifier_succmetric:  [kappa_count2rank_onlypos, kappa_rank2rank_onlypos_min] 
    prim_lambda:            0.5
    sec_lambda:             0.1
    __perdataset__:
      placetypes:
        extraction_method:  all 
        pp_components:      none
\end{lstlisting}

\subsection{\textcite{Ager2018}}

\input{pandoc_generated_latex/ZZ_listingreplacement_ager}

\removeMe{
\begin{lstlisting}[language=yaml, caption={YAML for \textcite{Ager2018}}]
    max_ngram:              1
    classifier_succmetric:  [cohen_kappa, accuracy, ndcg]
    dcm_quant_measure:      ppmi    
\end{lstlisting}
}


\todo 

% \removeMe{

% \subsection{\textcite{Alshaikh2020}}

% \begin{lstlisting}[language=yaml, caption={YAML for \textcite{Alshaikh2020}}]
%     TODO: do
% \end{lstlisting}

% }


	\cleardoublepage
\pagestyle{plain}
\bookmarksetup{startatroot} 

\printbibliography[heading=bibintoc]

\end{document}