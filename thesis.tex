%TODO once the text is "finished": 
% * Figure out first appearance of all words in the glossary and make sure that the first mention of it is written out with the abbreviation in parantheses
% * Ensure all links to the repo of the code refer to a signed commit and not just to the main-branch 

\documentclass[11pt,
  paper=a4, 
  hidelinks,
  bibliography=totocnumbered,
	captions=tableheading,
	BCOR=10mm
]{scrreprt}

\usepackage[utf8]{inputenc}
 

\usepackage{makecell} % linebreaks in tables, see https://tex.stackexchange.com/a/176780/108199
\renewcommand{\cellalign}{vh}
\usepackage{lscape} % landscape tables
\usepackage{footnote}
\makesavenoteenv{tabular} % this line and line above see https://tex.stackexchange.com/a/109471/108199
\newcommand{\specialcell}[2][l]{%
  \begin{tabular}[#1]{@{}l@{}}#2\end{tabular}} %https://tex.stackexchange.com/a/19678/108199
\newcommand{\tabitem}{\textbullet~~}


%\usepackage[text={7in,10in},centering]{geometry}  %such that appendices etc can define new margins etc
\usepackage{caption}  % https://tex.stackexchange.com/a/176175/108199
\captionsetup[table]{position=below} 

\usepackage[anythingbreaks]{breakurl}
\usepackage[onehalfspacing]{setspace}
\usepackage{amsmath} % Standard math.
\usepackage{amsthm} % Math theorems.
\usepackage{amssymb} % More math symbols.
\usepackage{dsfont} % Render |R and the like
\usepackage[british]{babel} %was [english] (see note of underscore), is now [british] 
\usepackage{underscore} % I need underscore to not have to write "\_" for underscores, but that would break labels with unterscores in it unless I also include babel, see https://tex.stackexchange.com/a/121438/108199
\theoremstyle{definition}
\newtheorem{definition}{Definition}[chapter]
 
% for https://pandas.pydata.org/docs/reference/api/pandas.DataFrame.to_latex.html, https://pandas.pydata.org/docs/reference/api/pandas.io.formats.style.Styler.to_latex.html:
\usepackage{booktabs} 
\usepackage{multirow} 
\usepackage[table,dvipsnames]{xcolor} %dvipsnames for yaml-code-listings
\usepackage{siunitx}
\colorlet{lightgreen}{green!40!white}
\usepackage{etoolbox}
\robustify\bfseries
\robustify\itshape
% end 

\usepackage{pdflscape} % if this is used, those pages within {landscape} are turned when viewed digitally (https://tex.stackexchange.com/a/141444/108199)


\usepackage{url}
\usepackage[section]{placeins} % Keep floats in the section they were defined in.
\usepackage{tabularx}
\usepackage{booktabs} % Scientific table styling.
\usepackage{floatrow} % Option for keeping floats in the place they were defined in the code.
\floatsetup[table]{style=plaintop}
\usepackage[breaklinks=true]{hyperref} % Hyperlinks.
\usepackage[all]{nowidow} % Prevent widows and orphans.
\usepackage{xstring} % logic string operations
\usepackage{bbm} % \mathbb on numerals.
\usepackage{mathtools}
\usepackage[ruled,vlined]{algorithm2e} % Pseudocode
\usepackage{scrhack} % Make warning go away.
\usepackage{graphicx}
\usepackage{subcaption} % Subfigures with subcaptions.
\usepackage{authoraftertitle} % Make author, etc., available after \maketitle
\usepackage{listofitems}
\usepackage{blindtext} % Placeholder text.
\usepackage[automake, nopostdot, nonumberlist]{glossaries} % glossary for definitions and acronyms, without dot after entry and page reference 
\makeglossaries % Generate the glossary

% ================== biblatex stuff ==================
\newcommand\posscite[1]{\citeauthor{#1}'s \cite{#1}}
\newcommand\gencite[1]{\citeauthor{#1}'s \cite{#1}} %https://tex.stackexchange.com/a/22279/108199

% \PassOptionsToPackage{obeyspaces}{url}%
\usepackage[
	backend=bibtex,% 
	style=nature,% 
	doi=true,
	isbn=false,
	url=false, 
	eprint=false
	]{biblatex}
% \renewbibmacro*{url}{\printfield{urlraw}}

\addbibresource{mendeley_bibs/Masterarbeit.bib}

\DeclareStyleSourcemap{
  \maps[datatype=bibtex, overwrite=true]{
    \map{
      \step[fieldsource=url, final]
      \step[typesource=misc, typetarget=online]
    }
    \map{
      \step[typesource=misc, typetarget=patent, final]
      \step[fieldsource=institution, final]
      \step[fieldset=holder, origfieldval]
    }
  }
}

% ================
% https://tex.stackexchange.com/a/468286/108199 
\DeclareCiteCommand{\fancyquotecite}
  {\usebibmacro{prenote}}
  {\usebibmacro{citeindex}%
   \usebibmacro{fancyquotecite}}
  {\multicitedelim}
  {\usebibmacro{postnote}}

\newbibmacro{fancyquotecite}{%
  \printnames[given-family]{labelname}%
  \setunit{\addcomma\space}%
  \printfield{maintitle}%
  \setunit{\addcomma\space}%
  \printfield{booktitle}%
  \setunit{\addcomma\space}%
  \printfield{title}%
}
% ================

\usepackage{dirtytalk} %https://de.overleaf.com/learn/latex/Typesetting_quotations
% \usepackage{csquotes} % Context sensitive quotation.
\usepackage[autostyle=false, style=english]{csquotes}
\newcommand{\q}[1]{\enquote{#1}}
% \MakeOuterQuote{"} %https://tex.stackexchange.com/a/216166/108199 to auto-replace " with `` '' (parity must be given)
% See also regarding quotation:
% IEEE Standard: https://libraryguides.vu.edu.au/ieeereferencing/gettingstarted
% https://de.overleaf.com/learn/latex/Typesetting_quotations, https://www.andy-roberts.net/writing/latex/formatting, https://wiki.carleton.edu/download/attachments/20155418/textguide.pdf?version=1&modificationDate=1387231254000&api=v2
% ================== END biblatex stuff ==================

%\linespread{1.5} % set line spacing

\DeclareFontFamily{U}{mathx}{\hyphenchar\font45}
\DeclareFontShape{U}{mathx}{m}{n}{
      <5> <6> <7> <8> <9> <10>
      <10.95> <12> <14.4> <17.28> <20.74> <24.88>
      mathx10
      }{}
\DeclareSymbolFont{mathx}{U}{mathx}{m}{n}
\DeclareFontSubstitution{U}{mathx}{m}{n}
\DeclareMathSymbol{\bigtimes}{1}{mathx}{"91}


 

%%% Custom definitions %%%
% Shorthands
\newcommand{\ie}{i.\,e.~}
\newcommand{\eg}{e.\,g.~}
\newcommand{\wrt}{w.\,r.\,t.~}
\newcommand{\ind}{\mathbbm{1}}
\DeclarePairedDelimiter{\norm}{\lVert}{\rVert} 
% Functions
\newcommand{\tpow}[1]{\cdot 10^{#1}}
\newcommand{\fref}[1]{(Figure~\ref{#1})}
\newcommand{\figref}[1]{(Figure~\ref{#1})}
\newcommand{\figureref}[1]{Figure~\ref{#1}}
\newcommand{\tref}[1]{Table~\ref{#1}}
\newcommand{\tabref}[1]{(Table~\ref{#1})}
\newcommand{\tableref}[1]{Table~\ref{#1}}
\newcommand{\secref}[1]{%
	\IfBeginWith{#1}{chap:}{%
		(cf. Chapter \ref{#1})}%
		{(cf. Section \ref{#1})}%
		}
\newcommand{\sectionref}[1]{%
	\IfBeginWith{#1}{chap:}{%
		Chapter \ref{#1}}%
		{\IfBeginWith{#1}{s}{%
			Section \ref{#1}}%
			{[\PackageError{sectionref}{Undefined option to sectionref: #1}{}]}}}
\newcommand{\chapref}[1]{(see chapter \ref{#1})}
% \newcommand{\unit}[1]{\,\mathrm{#1}}
\newcommand{\unitfrac}[2]{\,\mathrm{\frac{#1}{#2}}}
\newcommand{\codeil}[1]{\lstinline{#1}}{} % wrapper for preventing syntax highlight error
\newcommand{\techil}[1]{\texttt{#1}}
\newcommand{\Set}[2]{%
  \{\, #1 \mid #2 \, \}%
}
% Line for signature.
\newcommand{\namesigdate}[1][5cm]{%
	\vspace{5cm}
	{\setlength{\parindent}{0cm}
	\begin{minipage}{0.3\textwidth}
		\hrule 
		\vspace{0.5cm}
		{\small city, date}
	\end{minipage}
	 \hfill
	\begin{minipage}{0.3\textwidth}
		\hrule
		\vspace{0.5cm}
	    {\small signature}
	\end{minipage}
	}
}
% Automatically use the first sentence in a caption as the short caption.
\newcommand\slcaption[1]{\setsepchar{.}\readlist*\pdots{#1}\caption[{\pdots[1].}]{#1}}

% Variables. 
% Adapt if necessary, use to refer to figures and graphics.
\def \figwidth {0.9\linewidth}
\graphicspath{ {./graphics/figures/}{./graphics/figures/} } % Path to figures and images.

% Pandoc creates tightlists (https://tex.stackexchange.com/a/258486/108199)
\providecommand{\tightlist}{%
  \setlength{\itemsep}{0pt}\setlength{\parskip}{0pt}}

% Customizations of existing commands.

% vec-command to be used in text and mathmode. If called with \vec[m]{a} it's math-mode, default text.
\renewcommand{\vec}[2][t]{%
	\IfEqCase{#1}{%
		{m}{\mathbf{#2}}%
		{t}{\textbf{#2}}%
	}[\PackageError{tree}{Undefined option to vec: #1}{}]%
}%



% Capitalized \autoref names.
\renewcommand*{\chapterautorefname}{Chapter}
\renewcommand*{\sectionautorefname}{Section}

%have multiple references to the same footnote, see https://tex.stackexchange.com/a/35044/108199
\usepackage{cleveref}
\crefformat{footnote}{#2\footnotemark[#1]#3} %https://tex.stackexchange.com/a/10116/108199
\makeatletter 
\newcommand\footnoteref[1]{\protected@xdef\@thefnmark{\ref{#1}}\@footnotemark}
\makeatother
%....but the above doesn't work for tables, so we need something else as well, see https://tex.stackexchange.com/a/95905/108199
\usepackage{scrextend}


\title{Data-Driven Embedding of Educational Resources in a Vector Space with Interpretable Dimensions for Explainable Recommendation}
\author{Christoph Stenkamp}


% Scale images if necessary, so that they will not overflow the page
% margins by default, and it is still possible to overwrite the defaults
% using explicit options in \includegraphics[width, height, ...]{}
\makeatletter
\def\maxwidth{\ifdim\Gin@nat@width>\linewidth\linewidth\else\Gin@nat@width\fi}
\def\maxheight{\ifdim\Gin@nat@height>\textheight\textheight\else\Gin@nat@height\fi}
\makeatother
% Scale images if necessary, so that they will not overflow the page
% margins by default, and it is still possible to overwrite the defaults
% using explicit options in \includegraphics[width, height, ...]{}
\setkeys{Gin}{width=\maxwidth,height=\maxheight,keepaspectratio}
% see https://github.com/jgm/pandoc/issues/4941#issuecomment-425975499, https://github.com/jgm/pandoc/issues/4384#issuecomment-367585913


% ##################################################################################
% GENERAL code listing

 
\usepackage{listings} % rendering program code
\lstset{% general command to set parameter(s)
	basicstyle=\ttfamily\color{grey},          % print whole listing small
	keywordstyle=\color{black}\bfseries\underbar,
	% underlined bold black keywords
	identifierstyle=,           % nothing happens
	commentstyle=\color{white}, % white comments
	stringstyle=\ttfamily,      % typewriter type for strings
	showstringspaces=false}     % no special string spaces



% YAML code listing (see https://tex.stackexchange.com/a/152856/108199)


\newcommand\YAMLcolonstyle{\color{red}\mdseries}
\newcommand\YAMLkeystyle{\color{black}\bfseries}
\newcommand\YAMLvaluestyle{\color{blue}\mdseries}

\makeatletter

% here is a macro expanding to the name of the language
% (handy if you decide to change it further down the road)
\newcommand\language@yaml{yaml}

\expandafter\expandafter\expandafter\lstdefinelanguage
\expandafter{\language@yaml}
{
  keywords={true,false,null,y,n},
  keywordstyle=\color{darkgray}\bfseries,
  basicstyle=\YAMLkeystyle,                                 % assuming a key comes first
  sensitive=false,
  comment=[l]{\#},
  morecomment=[s]{/*}{*/},
  commentstyle=\color{purple}\ttfamily,
  stringstyle=\YAMLvaluestyle\ttfamily,
  moredelim=[l][\color{orange}]{\&},
  moredelim=[l][\color{magenta}]{*},
  moredelim=**[il][\YAMLcolonstyle{:}\YAMLvaluestyle]{:},   % switch to value style at :
  morestring=[b]',
  morestring=[b]",
  literate =    {---}{{\ProcessThreeDashes}}3
                {>}{{\textcolor{red}\textgreater}}1     
                {|}{{\textcolor{red}\textbar}}1 
                {\ -\ }{{\mdseries\ -\ }}3,
}

% switch to key style at EOL
\lst@AddToHook{EveryLine}{\ifx\lst@language\language@yaml\YAMLkeystyle\fi}
\makeatother

\newcommand\ProcessThreeDashes{\llap{\color{cyan}\mdseries-{-}-}}




% ##################################################################################
% specialstuff mentioned later
\newcommand{\mainalgos}{\cite{Derrac2015,Ager2018,Alshaikh2020} }


\input{special_highlight.tex}
\newcommand{\todoparagraph}[1]{\highlight[red]{#1}}

% ################################################################################## 
% ################################################################################## 
% ################################################################################## 

\makeatletter

\begin{document}


\begin{titlepage}
	\begin{flushleft}
		Universität Osnabrück\\
		Fachbereich Humanwissenschaften\\
		Institute of Cognitive Science
	\end{flushleft}

	\vspace{2cm}
	\centering{
		Master's thesis\vspace{1cm}\\
		\textbf{\Large{\MyTitle}}
		\vspace{1cm}\\
		\begin{tabular}{c}
			\MyAuthor                          \\
			955004                             \\
			Master's Program Cognitive Science \\
			April 2017 - April 2022
		\end{tabular}}
	\vspace{1cm}

	\begin{tabular}{ll}
		First supervisor:  & Dr. Tobias Thelen          \\
		                   & Institute of Cognitive Science \\
		                   & University of Osnabrück  \\\\
		Second supervisor: & Johannes Schrumpf, M.Sc.         \\
		                   & Institute of Cognitive Science \\
		                   & Osnabrück
	\end{tabular}

\end{titlepage}


\chapter*{Declaration of Authorship}
I hereby certify that the work presented here is, to the best of my knowledge and belief, original and the result of my own investigations, except as acknowledged, and has not been submitted, either in part or whole, for a degree at this or any other university.

\namesigdate
\pagenumbering{gobble}
\pagebreak

\begin{abstract}
	\textbf{\LARGE{Abstract}}\\\\
	%TODO summarize the main objectives and outcomes of your work. The abstract should fit on one page.
	In this thesis, I want to generate a conceptual space for the domain of educational reasources such as university courses, automatically created in data-driven way from their descriptions.

	Conceptual Spaces are seen as something that may be able to link sub-symbolic and symbolic approaches by standing in between them: In Conceptual Spaces, Concepts are represented as convex regions in high-dimensional spaces. Optimally, these spaces are cartesian, and the axes correspond to human-interpretable dimensions. If that is the case, you could for example classify the concept of "Apple" as a region that is in the color-dimension somwhere between green and red, and in the form-dimension roughly at "round".
	Creating these concpetual spaces is a very cumbersome task, which is why an automated method may lead to reasonable results. Unfortunately, this is still computationally very complex.
	The method of [DESC15] uses MDS, blablabla, then a Support-Vector-Machine separating concepts, and the orthogonal of the separating hyperplane is then an axis
\end{abstract}




\tableofcontents
\listoffigures
\listoftables
\listofalgorithms
\lstlistoflistings %TODO: figure out which of these two is correct

\chapter{Introduction}
\pagenumbering{arabic}

% Broad - "Initialkontextualisierung" - warum mach ich das, aus was für ner domain kommen die daten, was will ich damit machen (use rrecommendation, ich bau AI part, ..)

In this thesis, I want to generate a conceptual space for the domain of university courses, automatically created in data-driven way from their descriptions.

\section{Motivation}

Dass die paar leute die in dem Bereich veröffentlichen echt aktiv sind and all und coole Ideen haben, dass die aber immer nur sich selbst zitieren (und alle auf DESC15 basieren und einen der autoren als co-author haben), und das es sinnvoll ist da mal nen sanity-check reinzubringen und als externe person die validität vom DESC15 algorithmus zu prüfen (...und dass sie ja auch alle die selben 2 datensätze nutzen und dass man eben da auch mal prüfen sollte ob deren kram so sinnvoll ist BEYOND this one dataset) - also in kurz "If this algorithm is as good as they claim, that would be great, but we have reasons to not trust their claim so we're checking them." Im Prozess dafür soll halt auch eine Pipeline rauskommen die es future research leichter macht ebengenau das zu tun was ich hier tu und die validiät der claims zu prüfen etc.

Es gibt auch arbeiten bei denen die formulierte These ein Beispiel-Anwendungsfall für die Software-Grundlage ist

in meinem Fall "Meine Ausgangsfrage war ob man die Methoden von diesem einem Paper auf educational resources anwenden kann um regelmäßigkeiten zu finden für recommendations für educational resources. Um das rauszufinden war es wichtig den Algorithmus zu entwickeln, und im verlauf der thesis ist rausgekommen das ein system dafür zu entwickeln sehr komplex ist (...dass dafür halt eine solide Software-Grundlage gegeben sein muss), also ist diese thesis primär dafür da um das system zu beshcreiben um dann mit diesem system prototypisch die fragestellung anzugehen, und die results für die originalfrage sind dann eher als priliminary results zu betrachten - fokus-shift von results zu methodik".

Das ist eine Motivation für die Thesis ganz klar ist "der Algorithmus den ich da gesehen habe ist ganz nice, das wäre doch cool wenn der so modular und reproduzierbar undundund wäre dass jemand wie ich ankommen kann und den auf andere Datensätze schmeißen kann, aber leider sind die bei open source/open data/details nennen leider nicht so super, so I'm making that instead - the delivarable is a scalable, modular, ... system that makes it easy to exchange components of it, has many analysis-scripts, etc etc etc" 

Das was ich mache ist ja eine Replikationsstudie -> Dann darf ich auch gerne diese Dinge über die Architektur undso schreiben. "Ist ja schon irgendwie ingeneurwissenschaft", dazu gehört also auch mal mehr detail wie man das gemacht hat - hängt natürlich von der gewünschten seitenzahl und dem raum den ich hab ab. In der Arbeit sollte alles drinstehen was man für die Beurteilung braucht, also quellcode oder so darf auch gerne mal im Hauptteil stehen

feature directions allow us to rank objects according to how much they have the corresponding feature, and can thus play an important role in interpretable classifiers, recommendation systems, or entity-oriented search engines, among others  [AGKS18] has many sources for these!!
	* Recommender systems (gerade critique-based ones thanks to the keyword-extraction etc)
		-> see example of [VIGSR12]
	* Semantic Search Engines (can use directions in case of gradual and possibly ill-defined features, like "popular holiday destinations")
	* Represent examples in classification tasks
	* Rule-Based Classifiers from the rankings

TODO: direkt repeatability problems ansprechen, see \url{https://cs.carleton.edu/cs_comps/1920/replication/index.php} (the paper states there is problem X, makes a claim that algorithm Y may be good at problem X, create datasets Z for X, and then test the code on these datasets. " In that test of performance, the goal is typically to identify how well the proposed algorithm works versus alternative approaches and additionally to explore what kinds of examples one's algorithm can successfully classify versus what examples it makes errors on  Future research and applications often build on these experiments, relying on their results when deciding what algorithm is most appropriate for a new task or determining whether a new algorithm is better than existing work. For instance, based on the paper above, one might conclude that to test if one has a better sarcasm detector, one need only compare against the new algorithm, since the older approach performed less well in their experiments. Yet, it's rare that people directly try to replicate other's work to confirm that the results are valid and evaluate whether the trends in the results hold in other datasets. In psychology, there has been concern in recent years that many purported psychological phenomena may be overblown, as some attempts to replicate them have been unsuccessful. While computer science experiments are not the same as psychology experiments, there is still reason to be concerned about the lack of work focused on replicating computer science experiments. Often, the details of experiments in published work are opaque, and sometimes important information for reproducing the work in not included. Replicating previous work offers the opportunity to better understand that work, and to investigate the robustness of the algorithm to changes in parameters or dataset. If the exact parameters used have major impacts on the results or the same approach on a different dataset produces very different results, it suggests that caution should be used in generalizing the results and adding nuance to the original conclusions." [quote from webpage])


\section{What are conceptual spaces? }

Conceptual spaces (Gärdenfors, blabla) want to stand in between subsymbolic processing and symbolic processing: Like in subsymbolism, concepts are represented in high-dimensional spaces, but because the dimensions of these spaces are not arbitrary but human-interpretable, it allows for symbolistic high-level reasoning.

So, in conceptual spaces, concepts are represented as convex regions in high-dimensional, human interpretable spaces. For example, the concept of "apple" is a region that in the dimension "color" is somewhere between red and green, in the dimension "form" at roughly round, in the dimension "taste" somwhere between sweet and sour, etc. 
Every instance of an apple is thus a vector that lies inside the high-dimensional region of the concept. This allows for high-level reasoning, such as the question "does any Instance of concept X fit into my bag?" -> If the "size" dimension of the whole region is smaller than the size of my bag, it will.

Conceptual spaces sounds similar to word2vec or other word embedding approaches, however there are a few important distinctions - first, the domain of a conceptual space does not include all kinds of words or concepts, but only concepts of a certain domain (like movies or university courses). 
Second, conceptual spaces are convex regions, not mere vectors (which allows for easy extraction of is-a and part-of relations or prototypical examples vs edge examples, but makes the generation computationally vastly more expensive). And, most importantly, while the geometry of word2vec is roughly euclidian (otherwise the famous vec(king)-vec(man)+vec(woman)==vec(queen) wouldn't work), the dimensions are not interpretable but arbitrarily depend on the random initial setup, so the concepts king and queen differ not only in a single "gender" dimension [..and also its not really euclidian, is it?! sonst wäre die betweeness doch nicht so special, oder?].

Now the standard problem with conceptual spaces is that they would have to be manually generated, which of courses is a lot of work, which is where the work of [Schokeart et al] comes in - to generate them in a data-driven fashion.
For that, the authors look at three different domains: movies, wines and places. For each of these domains, they collected many samples (like movies) together with descriptions from places where people can leave them (like reviews from IMDB). A representation of a movie is then generated from the bag-of-words of the descriptions of the individual movies, leading to a very high-dimensional, very sparse representation for all movies. 
To make the representations less sparse and more meaningful, the words in the BOW are subsequently PPMI-weighted, which weights words that appear often in the description of a particular movie while being infrequent in the corpus overall higher while setting the representation of stopwords to almost zero. 
This PPMI-weighted BOW is however not yet a euclidian space yet, which is why the authors subsequently use multidimensional scaling (MDS). MDS is a diminsionality reduction technique that attempts to create a euclidian space of lower dimensionality than the original one in which the individual distances of the items are preserved as well as possible. 

With such a space, the concepts of betweeness already makes sense, but so far, the dimensions are not interpretable. So how does one automatically find such directions? In the case of movies, good dimensions may be "scariness", "funniness", "epicness", "family-friendlyness" etc. 
To find these dimensions, the authors look for these words (as well as similar words thanks to clustering) in the reviews. Then the movies are grouped into those that contain the words from the cluster often enough vs those that don't. A support-vector-machine subsquently finds a hyperplane that best divides the two groups (eg. scary and non-scary), and the orthogonal of that hyperplane is used as one axis of the new coordinate basis. 

% * Dass der tatsächliche Anwendungsbereich von CS noch sehr begrenzt ist - RaZb20 mention "they are commonly used in perceptual domains, e.g. for music cognition [Forth et al., 2010; Chella, 2015], where quality dimensions are carefully chosen to maximize how well the resulting conceptual spaces can predict human similarity judgements"
% * Dass Word Embeddings ja relativ nah an CS sind - For ex- ample, a well-known property of word embeddings is that many syntactic and semantic relationships can be captured in terms of word vector differences [Mikolov et al., 2013].
% TODO: Ist word2vec schon nen euclidian space? Why/Why not?



% \begin{figure}[H]
% 	\centering
% 	\includegraphics[width=\figwidth]{scientific_paper_graph_quality}
% 	\slcaption{
% 		Developmemt of scientific paper graph quality. A dip in the
% 		quality of scientific graphs is observed from the early 1990s to the early 2010s.
% 		During this time Microsoft Paint and PowerPoint were often used to create graphs in scientific papers.\label{fig:scientific_graph_quality}}
% \end{figure}

% \begin{table}[H]
% 	\begin{tabular}{@{}ll@{}}
% 		\toprule
% 		year & quality \\ \midrule
% 		1985 & good    \\
% 		2000 & bad     \\ \midrule
% 		2015 & better  \\ \bottomrule
% 	\end{tabular}
% 	\caption{
% 		Empirical measurements of scientific graph quality. Data points were collected using
% 		a systematic literature review.\label{tab:scientific_graph_quality}}
% \end{table}
% This references a \figref{fig:scientific_graph_quality} while this references a table \tabref{tab:scientific_graph_quality}.

% A citation looks like this \cite{hadash2018estimate}. To embed a citation in the text flow use textcite,
% \eg \textcite{hadash2018estimate} said you should use a lot of citations.

\section{What do I want to do in this thesis?}


\begin{figure}[H]
	\centering
	\includegraphics[width=0.7\textwidth]{graphics/stolenfigures/movietuner.png}
	\slcaption{
		The Movie-Tuner Interface from \cite{VISR12} %TODO: cite exact figure (" … as seen in [8, Fig. 33]")
		\label{fig:movetuner}}
\end{figure}

\subsection{Open Science and Reproducibility}

I think it is absolutely crucial for all branches of science to adhere to the principles of open science and to ensure that all claims that are made in publications are reproducible and testable. This thesis will mostly copy the work of somebody else, but doing so was incredibly tedious, much more so than it would have to be.

Dabei ist mir aufgefallen dass die schon einige DInge machen die ich aus wissenschaftlicher Sicht für ziemlich kritikwürdig halte, zum Beispiel sind die so schwammig in den Formulierungen dass man beim Versuch den Code zu reproduzieren echt viel raten muss, haben geschrieben dass der Code open ist verweisen aber auf ein leeres Repo, haben ihre Daten veröffentlicht aber wenn man damit arbeitet merkt man dass das die selbst definitiv nicht mit dem Datensatz den sie veröffentlicht haben gearbeitet haben könne, sind sehr hart am cherry-picken in ihrer qualitativ  analysis etc etc et

So one main motivation is to reproduce the code for the paper I liked in a way that adheres to the principles of open science, such that others that find it interesting don't have to go through the shit I had to go through.

Principles of open science (TODO: which are: [see thisandthis paper]) are very important to me, so I want to ensure that the claims I am making in this thesis are backed by code that is scalable, reproducible, modular, easily-understood, easily set up and run, well documented, ... . To support this, I will as often as necessary refer to the actual code in this thesis, to allow to understand and reproduce the claims and results, and also highly encourage to critically read everything here and check the respective code (...and let me know if you spot any errors! Just open a Github Issue!)

% TODO: also make the data available somewhere open!


\chapter{Background}
% (zuspitzung von generell auf spezifisch, sowohl technisch (conceputalspaces -> was macht das paper konkret), (und bei dem anderen teil was sind educational resources, was sind die schwierigkeiten dabei, warum möchte man überhaupt empfehlen))

\section{Use Case: Educational Resources}
% Anwendungsfall (->e-learning, recommenden von bildungsressourcen, ...) -> nicht-technisch, aber nötig zum verstehen wo passiert das 

\section{Conceptual Spaces}

% \cite{Alshaikh2019} (verbatim!):
% * vector space models that are aimed at representing the entities of a given kind (e.g. movies), to- gether with their associated properties (e.g. scary) and concepts (e.g. thrillers).
% * As such, they are similar in spirit to the vector space models that have been proposed in information retrieval (Deer- wester et al., 1990) and natural language pro- cessing (Turney and Pantel, 2010; Mikolov et al., 2013), but there are also notable differences.
% * First, in the context of conceptual spaces, an explicit dis- tinction is made between the entities from the do- main of discourse, which are represented as vec- tors, and the corresponding properties and con- cepts, which are represented as regions (e.g. poly- topes) or soft regions (e.g. characterized by a Gaussian). 
% * Second, conceptual spaces are organ- ised into a set of facets [domains], each of which captures a different aspect of meaning. For instance, in a conceptual space of movies, we may have facets such as genre, language, geographic location, etc. Each facet is associated with its own vector space, which intuitively captures similarity w.r.t. the corresponding facet. Most of these facet spaces tend to be low-dimensional (e.g. modelling budget only needs a single dimension). This clearly dif- ferentiates them from traditional semantic spaces, which often have hundreds of dimensions






\@input{pandoc_generated_latex/chapter_theobg_section_cs}


\section{Automatic Data-Driven Generation of Conceptual Spaces}

% \cite{Alshaikh2019} geht drauf ein warum man infoGAN und VAEs für bilder als pretty much sowas betrachten kann

%Wie funktioniert die Idee des data-driven generieren 

% Base idea: [Derrac and Schockaert, 2015] proposed an unsupervised method which uses text descriptions of the considered entities to identify se- mantic features that can be characterized as directions. Their core assumption is that words describing semantically mean- ingful features can be identified by learning for each candi- date word w a linear classifier which separates the embed- dings of entities that have w in their description from the oth- ers. The performance of the classifier for w then tells us to what extent w describes a semantically meaningful feature. 
% This method trains for each word w in the vocab- ulary a linear classifier which predicts from the embedding of an entity whether w occurs in its description. The words w1, ..., wn for which this classifier performs sufficiently well are then used as basic features. To assess classifier perfor- mance, Cohen’s Kappa score, which can be seen as a correc- tion of classification accuracy to deal with class imbalance, is used. Each of the basic features w is associated with a cor- responding vector dw (i.e. the normal vector of the separat- ing hyperplane learned by the classifier). These directions are subsequently clustered, which serves to reduce the total num- ber of features.

% TODO: Have to write here:
% * that in a CS the axes correspond to human concepts, "concepts" meaning attributes and what-was-the-other-again, according to CS lingo corresponding to nouns and adjectives yadda yadda, darauf referenzier ich mich im Text


\section{Types of Reasoning}

\@input{pandoc_generated_latex/chapter_theobg_section_reasoning}

\section{Related Work}

% dass das alles vergleichbar mit InfoGAN etc für Bilder ist.. (siehe auch text von \cite{Alshaikh2019})

\@input{pandoc_generated_latex/chapter_related_work}



\section{Required Algorithms and Techniques}

\subsubsection*{Germanet}

\cite{hamp-feldweg-1997-germanet}
\cite{Henrich}

The Paper uses the following PPMI definition:\\ 

\noindent $e \in E$ is an entity, $D_e$ a document (bag of words) where that entity occurs.\\
We want to quantify for each term occuring in the corpus $\{D_e | e \in E\}$ how strongly it is associated with $e$.\\
$c(e,t)$ is the number of times term $t$ occurs in document $D_e$. \\
The weight $ppmi(e,t)$ for term $t$ in the vector representing $e$ is then:
\begin{align*}
ppmi(e,t) &= max\left(0, log\left( \frac{p_{et}}{p_{e*}*p_{*t}} \right) \right) \\
          &= max\left(0, pmi(e,t) \right) \\
 pmi(e,t) &= log\left( \frac{p_{et}}{p_{e*}*p_{*t}} \right) \\          
   p_{et} &= \frac{c(e,t)}{\sum_{e'}\sum_{t'} c(e',t')} \\
   p_{e*} &= \sum_{t'}p_{et'} \\
   p_{*t} &= \sum_{e'}p_{e't} \\
\end{align*}

\noindent log of the probability of the $e$-$t$-combination (count of this vs count of all), normalized by the probability of this $e$ with any $t$ times this $t$ with any $e$.\\
To quote the paper: "PPMI will favor terms which are frequently associated with the entity $e$ while being relatively infrequent in the corpus overall"

\vspace{30px}

I found this definition:
\begin{align*}
ppmi(X,i,j) &= max(0, pmi(X,i,j)) \\
pmi(X,i,j)  &= log\left( \frac{X_{ij}}{expected(X,i,j)} \right) \\
            &= log\left( \frac{P(X_{ij})}{P(X_{i*}) * P(X_{*j})} \right)
\end{align*}


\@input{pandoc_generated_latex/chapter_methods_section_required_algorithms}


\chapter{Methods}
%(Algorithmus & Datensatz)

Direkt am Anfang schreiben dass ich halt auf den main algorithmus eingehe und das laut meiner research diese 3 paper am besten den main algo beschreiben (bzw sinnvoll erweitern) - was nicht heißt dass das die einzigen in dem kontext sind, Alshaikh2019 bspw nutzen ja den main algorithmus, aber ja nur als komponente, und haben andere Ziele was sie dann damit machen

Im folgenden gibt es neben Datasets 2 main sections: algoritm and architecture. Dass Algorithm und Architecture 2 subsection von methods sind ist halt "Der allgemeine Algorithmus und die spezifische Anwendung" Warum Architecture session? es kostet extrem viel zeit die schwammigen formulierungen in den papern genau zu verstehen, man probiert super oft falsche parameter-kombinationen aus etc etc, es ist halt ein riesiger ewig langer lernprozess den man von vorne machen müsste wenn man es nachimplementieren möchte, ich hätte mir gewünscht die authors hätten darüber mehr worte verloren, and also the scalable reproducible open-science part. And also - it took me a shitton of time, way more than working on the algorithm (but NOW it can run so easily on the grid and all param-combis simultaneously, ..), so this is what you'll get.


\section{Datasets}

% * Steht ja schon woanders dass mein Datensatz anders ist als concatenated-movie-reviews und ich deswegen nicht einfach "je öfter 'scary' desco scarier" machen kann. Da gibt's several ways mit umzugehen
	% * Das sich-die-richtigen-wörter-per-candidate-svm-bootstrappen
	% * Mit LSI rausfinden welche Terme genausogut in dem Text hätten vorkommen können (hab ich auch irgendwo schon)
	% * Explizit einfach zu gucken "Welche Terme kommen oft in den gleichen dokumenten vor" (und das inverse (steht iwo im code)), und dann ne candidate SVM für grouped terms anstelle von einzelterms machen (auch schon iwo als code)
	% * Mit Wordnet hypernyms/hyponyns und synonyms zu finden damit ebenfalls zu arbeiten (kann man wit wordnet angeben welches abstraktionsniveau ich haben will?)
	%     * Abstraktionsniveau gibt's nicht in wordnet, das heißt das richtige layer zu finden ist schwer. Was man auf jeden Fall machen kann ist die Terme zu den bases ihrer synsets umzuwandeln (dadurch wird aus "math" und "mathematics" das gleiche), aber in anderen Fällen ist es halt so dass ich die Candidate-Terms schon vorher brauche und nur sagen kann "diese entity enhält X wörter die halt hyponyms von dem Term sind"

their algorithm is tailored to concatenated-reviews or concatenated-bags-of-tags. Take their success-metric for the SVMs splitting the embedding. The more often the word "scary" comes in the concatenated reviews, the more scary the movie is. Sounds legit. The more often the people that took pictures at a particular place mentioned the "nature" of that, the more relevant "nature" is to that place. Also legit. But in the descriptions for courses that involve a lot of mathematics, it is not necessarily the case that the term "mathematics" occurs often. So due to the different nature of my dataset I have to go beyond their algorithm at some points - in this case it is probably the case that different kinds of mathematical terms actually do occur more often, so I'd need calculate these kinds of kappas not based oon a single term but ALREADY on a cluster of terms (... and I can bootstrap my way there, because after I do this I get more words to add to my cluster, rinse and repeat!)

%TODO https://tex.stackexchange.com/questions/526198/table-resize-table-and-automatic-line-breaks


% \afterpage{%

\newgeometry{
	top=21mm,
	bottom=16mm,
	inner=16mm,
	outer=16mm,
} 


\begin{landscape}
	\begin{table}[]
		\resizebox{.98\textwidth}{!}{%
        \begin{tabular}{@{}llllll@{}} 
        	\toprule
        		\textbf{dataset} &
        		\textbf{contents} &
        		\textbf{preprocessing} &
        		\textbf{size} &
        		\textbf{classification classes} &
        		\textbf{candidate word threshold}
        		% & \textbf{key feature sizes} 
        		 \\ \midrule
        	\textbf{movies\tablefootnote{\label{origdsets}\url{https://www.cs.cf.ac.uk/semanticspaces/}} \cite{Derrac2015,Ager2018,Alshaikh2020} } &
				\specialcell[l]{grouped-by-movie-concatenated\\reviews for movies} & 
        		\specialcell[l]{\tabitem removed stop-words\tablefootnote{\label{fnote:stopwordlist}\url{http://snowball.tartarus.org/algorithms/english/stop.txt}} \\ \tabitem lower-cased text \\ \tabitem removed diacritics  \\ \tabitem removed punctuation} &
        		\specialcell[l]{\cite{Derrac2015}: 15000 movies \\ \cite{Ager2018,Alshaikh2020}: 13978 movies } & %Ager2018 says 15.000 - 1022 duplicates, that's the number of Alshaikh2020
        		\specialcell[l]{ \tabitem genre (23 classes)\\ \tabitem plot keywords (eg. \textit{suicide, beach}) (100 classes) \\ \tabitem age-rating certificates (6 classes)} & \specialcell[l]{\acrshort{df} $\geq 100$ \\ \textrightarrow 22 903 candidates \\ variable-length \textbf{n-grams} considered}
        		
        		\\ \midrule
        	\textbf{place types\footref{origdsets} \cite{Derrac2015,Ager2018,Alshaikh2020} } &
				\specialcell[l]{Tags of Flickr-photos that are also\\tagged with a place-type}
        		% bag-of-tags from Flickr used to describe places of a certain place-type
        		& 
        		None &
        		1383 place-types & %both in DESC15 and the follow-up paper
        		\specialcell[l]{ \tabitem category from Geonames (7 classes)\\ \tabitem category from Foursquare (9 classes)\\ \tabitem category from OpenCYC (93\cite{Derrac2015}/20\cite{Ager2018,Alshaikh2020} classes) } &
        		\specialcell[l]{\acrshort{df} $\geq 50$ \\ \textrightarrow 21\,833 candidates \\ (all words from the BoW) \\ \textbf{n-grams}: squashed all words of a tag} 
        		% & candidate-terms: 6385
        		\\ \midrule
        	\textbf{wines\footref{origdsets}\tablefootnote{\url{https://snap.stanford.edu/data/web-CellarTracker.html}} \cite{Derrac2015}} &
				\specialcell[l]{grouped-by-wine-variant-concatenated\\reviews for wines} & \specialcell[l]{\tabitem removed stop-words\footnoteref{fnote:stopwordlist} \\ \tabitem lower-cased text \\ \tabitem removed diacritics  \\ \tabitem removed punctuation} & 330 wine-varieties &
        		\textit{not performed} &
        		\specialcell[l] {\acrshort{df} $\geq 50$ \\  \textrightarrow around 6k candidates \\ variable-length \textbf{n-grams} considered }
        		\\ \midrule
        	\textbf{20 newsgroups\tablefootnote{\url{http://qwone.com/~jason/20Newsgroups}} \cite{Ager2018}} &
				\specialcell[l]{posts partitioned roughly even\\across 20 different newsgroups} &
        		\specialcell[l]{ \tabitem Headers, footers and quote metadata removed\tablefootnote{Using the scikit-learn python package, see \url{https://scikit-learn.org/0.19/datasets/twenty_newsgroups.html}} \\ \tabitem removed stopwords (using NLTK's corpus \cite{loper-bird-2002-nltk})\\ \tabitem lowercased text\\ \tabitem candidate terms: all textual and numerical tokens} &
        		18446 posts &
        		\tabitem newgroup post was submitted to (20 classes) &
        		$\geq$ 30 occurences 
        		\\ \midrule
        	\textbf{imdb sentiment\tablefootnote{\url{http://ai.stanford.edu/~amaas/data/sentiment/} \cite{maas-EtAl:2011:ACL-HLT2011}} \cite{Ager2018}} &
				\specialcell[l]{highly polar movie reviews\\for binary sentiment classification}  &
        		\specialcell[l]{ \tabitem removed stopwords (using NLTK's corpus \cite{loper-bird-2002-nltk})\\ \tabitem lowercased text\\ \tabitem candidate terms: all textual and numerical tokens} &
        		50000 reviews &
        		\tabitem sentiment of the review (2 classes) &
        		$\geq$ 50 occurences
        		\\ \midrule
        	\textbf{Bands \cite{Alshaikh2020}} &
        		\specialcell[l]{All Wikipedia pages ($\geq 200$ words) whose \\ WikiData semantic type is "Band"} &
        		\specialcell[l]{ \tabitem removed HTML-tags and references \\ \tabitem \textit{"standard preprocessing strategy"} \cite[137]{Alshaikh2019} \\ \tabitem removed stopwords (using NLTK's corpus \cite{loper-bird-2002-nltk})\\ \tabitem POS-tagging and keeping only nouns and adjectives \\ \tabitem remove words with a rel. \acrshort{df}  $>$ 60\% or abs. \acrshort{df} $<$ 10 } &
        		11448 bands & \specialcell[l]{ \tabitem Genres (22 classes) \\ \tabitem Country of origin (6 classes) \\ \tabitem Loc. of formation (4 classes) }  & 
        		\specialcell[l]{ 10 $<$ \acrshort{df} $<$ 6869 \\ (all words from the BoW)}\\ \midrule
        	\textbf{Organisations\tablefootnote{\label{fnote:for_alshaikh2019}Originally created in and for \cite{Alshaikh2019}} \cite{Alshaikh2020}} &
        		\specialcell[l]{All Wikipedia pages ($\geq 200$ words) whose \\ WikiData semantic type is "Organisation"} &
        		\specialcell[l]{ \tabitem removed HTML-tags and references \\ \tabitem \textit{"standard preprocessing strategy"} \cite[137]{Alshaikh2019} \\ \tabitem removed stopwords (using NLTK's corpus \cite{loper-bird-2002-nltk})\\ \tabitem POS-tagging and keeping only nouns and adjectives \\ \tabitem remove words with a rel. \acrshort{df}  $>$ 60\% or abs. \acrshort{df} $<$ 10 } &
        		11800 organisations &
        		\specialcell[l]{ \tabitem Country (4 classes)\\ \tabitem Headquarter Loc. (2 classes)} &
        		\specialcell[l]{ 10 $<$ \acrshort{df} $<$ 7080 \\ (all words from the BoW)} \\ \midrule
        	\textbf{Buildings\footnoteref{fnote:for_alshaikh2019} \cite{Alshaikh2020}} &
        		\specialcell[l]{All Wikipedia pages ($\geq 200$ words) whose \\ WikiData semantic type is "Building"} &
        		\specialcell[l]{ \tabitem removed HTML-tags and references \\ \tabitem \textit{"standard preprocessing strategy"} \cite[137]{Alshaikh2019} \\ \tabitem removed stopwords (using NLTK's corpus \cite{loper-bird-2002-nltk})\\ \tabitem POS-tagging and keeping only nouns and adjectives \\ \tabitem remove words with a rel. \acrshort{df}  $>$ 60\% or abs. \acrshort{df} $<$ 10 } &
        		3721 buildings &
        		\specialcell[l]{ \tabitem Country (2 classes)\\ \tabitem Administrative loc. (2 classes)} &
        		\specialcell[l]{10 $<$ \acrshort{df} $<$ 2233 \\ (all words from the BoW) }\\ \Xhline{4\arrayrulewidth}
        	% \textbf{SIDDATA-Courses} &
        	% 	TODO &
        	% 	&
        	% 	&
        	% 	\tabitem Faculty (10 classes) 
        	% 	\\ \midrule 
        	% \textbf{100K Coursera reviews}\tablefootnote{\url{https://www.kaggle.com/septa97/100k-courseras-course-reviews-dataset}} &
        	% 	TODO &
        	% 	&
        	% 	&
        	% 	\specialcell[l]{ \tabitem Rating (5 classes) \\ \textit{\tabitem Major, Category, Offered-By,... (tbd)} }
        		\\ 
		\end{tabular}
		\caption[All datasets used by any of \mainalgos]{All datasets used by any of \mainalgos. Citations behind the dataset name denote which author used it. Other listed properties include dataset sources (where available), contents, sizes, the respectively used preprocessing-methods and candidate-word-thresholds, as well as the classes considered in the evaluation of the derived explainable classifiers.}
		\label{tab:all_datasets}
	}
	\end{table}
\end{landscape}


\restoregeometry % !!! when trying to add afterpage again, remove this!!


% \restoregeometry\clearpage % !!! Jörg's comment on https://tex.stackexchange.com/a/78285/108199 !!!!
% \aftergroup\restoregeometry  % see THE QUESTION of https://tex.stackexchange.com/q/139834/108199
% } %afterpage



% Empirie, auch specifics über den Datensatz

%To write:
% * where does the data come from
% * what size is the data, what is the distribution, ...
% * Preliminary analysis (if I delete all that are shorter than X, it are |Y|..)
% * Does it cluster and look nice?
% * Verteilung der Sprachen
% * Preprocessing in kurzem Fließtext beschreiben - "After throwing out all descriptions shorter than xyz chars, 2323 courses where left. 223 of these were ..."
% * That the type of dataset differs from DESC15 and followups - mainly used movie-dataset consists of concatenated reviews (which means relevant words occur more often!) 
%     (TODO: look/think was die anderen auszeichnet - bei dem placetypedataset ists ja gar kein fließtext sondern direkt ein bag-of-tags)
% Dass mein Datensatz kleiin ist! Bei keinem sonderlichen min-word-per-desc threshold hab ich halt 7588 samples, bei 50 schon nur noch 4123, das ist wirklich little
% Dass auch die Descriptions echt kurz sind! Ich hab rund 8k samples, um das selbe samples-to-threshold verhältnis zu haben wie DESC15 wäre rechnerisch ein wert von 2 bis 25 sinnvoll (wobei man beachten muss das 2 schon richtig kacke ist weil dann die SVM 2 vs 8000 klassifizieren muss and that will never work -> 25 ist minimum), ABER wenn ich dann 25 nehme hab ich nur 2.4k candidates statt the 22k DESC15 aimed at, which also sucks!! --> CONCLUSION: Datensatz scheint zu klein.

The main goal of this thesis was to create a conceptual space of courses, automatically generated by course descriptions.


For that, a dataset of courses and their descriptions was obtained as export from the Stud.IP system as used at the universities of Osnabrück, Hannover and Bremen.
%TODO wait, woher kam der datensatz überhaupt? Tobias hat mir den geschickt, aber kam er zustande im Rahmen von Siddata?

The dataset comes from Johannes' Repo at \url{https://git.siddata.de/jschrumpf/study_behavior_analysis} (requires authentification over UOS!)

\begin{figure}[H]
	\centering
	\includegraphics[width=\figwidth]{graphics/figures/courses_language_distribution.png}
	\slcaption{
		\label{fig:courses_language_distribution}
		Distribution of languages of course descriptions.
		%TODO figure if this is the correct amount of preprocessing/throwout to have done
		Of the 21337 courses left after preprocessing, 18,679 were in german language according to the \textit{langdetect} python-package\footnote{\url{https://pypi.org/project/langdetect/}, which is a direct port of a java library\ which claims to have 99.8\% accuracy on longer texts\cite{nakatani2010langdetect}}.
		}
\end{figure}


The faculty is easily obtainable from the dataset, as the first one or two digits of the course ID correspond to it. The distribution of the faculties is depicted in figure \ref{fig:faculty_plot}.

\begin{figure}[H]
	\centering
	\includegraphics[width=\figwidth]{graphics/figures/faculty_plot.png}
	\slcaption{
		\label{fig:faculty_plot}
		Distribution of faculties in the courses
		}
\end{figure}

The purpose of the Neural Network classifier is to check if it is anyhow possible to extract meaningful information from the descriptions: If it is possible to train a classifier on the data that can reasonably predict a qualitative feature, there is enough structure in the data such that the algorithm I'm about to produce can work.
Also, we have a lower bound for useful data: we can just throw away data that cannot be classified!
%TODO: train a second classifier on something else and throw away data that gets classified by neither and inspect it

(-> 91\% test accuracy)

% =============== Besonderheiten vom Siddata-datensatz

....len([i for i in descriptions._descriptions if "kompetenzen entwickelt befahigen akademischen berufstypischen" in i.processed_as_string()]) == 25  ... weil es genau 25 exakt gleiche Beschreibungen gibt, für die Fremdsprachkurse. Deswegen ist up to jede 5-wort-kombination davon ein extracted keyword
(und das obwohl sie verschiedene Namen haben! merging them doesn't make sense but they are almost equal)

% =============== Schreiben zum Thema Datensatz-Vergleich:

...ist es richtig dass nur 6000 verschiedene Terms >= 25 mal vorkommen?! 6000?!
=> auch in groß ist mein datensatz ja noch deutlich kleiner als placetypes, die haben immerhin 22k candidates
--> n-docs: 7596
--> 1-grams >= 25 times: 5054, 1-5-grams >= 25 times: 6717
--> unique 1-grams: 106235

bei placetypes sind es 
* unique 1-grams: 746180, davon 41320 >= 25 mal und 21833 >= 50 mal (their threshold)

--> das verhältnis Anzahl Texte zu Länge Texte ist bei mir halt komplett off 

% =============== 


\subsection{Other Datasets}

%TODO: write IN THE ALGORITHM & ARCHITECTURE SECTIONS that I of course tried the placetypes-dataset as sanity-check to find errors - for that dataset, stuff like the good-candidates is known so as long as I don't reach their performances for that dataset I know my code is the problem, but as soon as I reach their performance I can savely say that the actual algorithm is correct and if it's still bad on the siddata dataset it's just not applicable to this kind of data

Also tried the Plactypes-Dataset used by all main-paper-authors. When doing so I noticed that there are definitely duplicates (which are consistently recognized as closest-terms in embedding):
  abandoned rail road and abandoned railroad
  boat yard and boatyard
  coral reef and reef
  court house and courthouse
  grass land and grassland
  sheep fold and sheepfold
  skate park and skatepark
  steak house and steakhouse
  water fall and waterfall
  wind mill and windmill

Next to that, the embedding however also sees very similar ones as very similar, which is a nice sanity-check, eg.

  abandoned farm and abandoned home
  airfield and airport
  airport and airport terminal
  ancient site and archaeological site
  arch and arch bridge
  art gallery and art museum
  coffee house and coffee shop
  aircraft cabin and airplane cabin
  apartment and apartment building
  bank and bank building
  field hockey field and hockey field


Also tried a dataset of 100.000 coursera course reviews from \url{https://www.kaggle.com/septa97/100k-courseras-course-reviews-dataset}. Why? Because it's also eduactional resources, but as it's reviews it seems closer to the movies dataset
See \url{https://www.kaggle.com/roshansharma/coursera-course-reviews} for exploratory analysis of the dataset (there he also has another dataset he writes about, but you cannot merge them unfortunately, so besides course name the only possible task is the rating)
%TODO: I could try to merge it with this one https://www.kaggle.com/siddharthm1698/coursera-course-dataset or another one (see https://www.kaggle.com/mihirs16/coursera-course-data which links names to links, https://www.kaggle.com/search?q=coursera+in%3Adatasets for other places)

Also, there's the Large Movie Review Dataset\footnote{\url{http://ai.stanford.edu/~amaas/data/sentiment/}, \url{https://scikit-learn.org/0.19/datasets/twenty_newsgroups.html}}, also used by \cite{Ager2018}.


\section{Architecture}
%Doing this section before Algorithm-section such that I can reference how plots are created or general code examples with the real thing, including how what you're seeing was generated and can be reproduced

%TODO: Also in this chapter:
% * Source-code is ofc open, available under github under this link, it is referred to the signed commit xyz
% * Reference Snakemake-Paper (and at least look a the abstract of that, they also talk about that in science you need reproducible, adaptable and transparency including definitions of what that means!)
\cite{Molder2021a}
% 		* good way to bash the original paper who either didn't publish their sourcecode or link a github-repo in their paper that is fucking empty, or did at least opensource their code but have just one fucking file in there that expects >40 unnamed command-line-args


\@input{pandoc_generated_latex/chapter_methods_section_architecture}



\section{Algorithm}

The algorithm that is implemented in the scope of the thesis is in principle the one from \textcite{Derrac2015}, but includes some improvements from \textcite{Ager2018} and \textcite{Alshaikh2020} - both of which directly builing on the work of the former. As all of these publications share Prof. Steven Schockaert as last author, it seems plausible that a) the latter ones are legit improvements upon the first, b) at least to a certain degree they can share code and data, c) this field of work is constrained to a small community, without any alternative implementations or substantial improvements from outside of it.
% sollte ja schon vorher erwähnt haben dass das thema leider nicht über diese small community hinaus geht
% and includes my own improvements und anpassungen für den speziellen datensatz
% und ist so modular wie möglich
\newline

The main goal of the algorithm is to unsupervisedly use the considered text-corpora associated with the respective entities from a certain domain % descriptions, reviews, ...
to embed the entities in to a vector-space where the axes correspond to human concepts, % "concepts" meaning attributes and what-was-the-other-again, according to CS lingo corresponding to nouns and adjectives, TODO: see above where I described that already.
allowing a \textit{feature-based} representation of them - a high-dimensional vector that numerically encodes the degree % protypicality
to which the entity corresponds to a number of appropriate dimensions. % Das haupt-ziel der algorithmen ist es, am ende die entities "feature-based" darstellen zu können, also als high-dim-vector mit floats per human-interpretable dimension. 
% Alshaikh2020 hat dafür wegen den subfeatures noch kleine specials
\newline

The general idea to achieve that is as follows: First, the entities %TODO: at this point already defined that entities = texts = descriptions/concatenatedreviews/... 
are embedded as fixed-dimensional vectors. This already softens the definition of a conceptual space, as the considered entities are modelled as vectors instead of regions. % entities were supposed to be regions, however here we assume vectors because it's computationally a lot easier (TODO: are the reasons mentioned elsewhere? also the distiction between types and tokens?)
To allow for the types of reasoning mentioned in Section \ref{sec:cs_reasoning}, %TODO: section where the entire mapping from logic-reasoning to CS-reasoning is explained, mostly from DESC15 and Gärdenfors himself
it is embedded into metric spaces where the concepts of direction and distance are well-defined. \gencite{Derrac2015} original algorithm uses MDS (see \ref{MDS}) for this matter, which enforces equal distances  %TODO: link the distance-matrix-notepad FROM MY REPO in nbviewer+binder
(however as shown by \textcite{Mikolov:Regularities}, such things are also possible with neural word-embedding techniques such as word2vec \cite{Mikolov:Word2Vec}) %TODO: word2vec cite
. Additionally, words from the text are extracted as candidates for the names of the semantic dimensions. The core assumption is then, that \q{words describing semantically meaningful features can be identified by learning for each candidate word $w$ a linear classifier which separates the embeddings of entities that have $w$ in their description from the others} \cite[3574]{Alshaikh2020}. The better the performance of that classifier according to a chosen metric, the more evidence there is that $w$ describes a semnatically meaningful feature. 
% * from Alshaikh2020: "Their core assumption is that words describing semantically mean- ingful features can be identified by learning for each candi- date word w a linear classifier which separates the embed- dings of entities that have w in their description from the oth- ers. The performance of the classifier for w then tells us to what extent w describes a semantically meaningful feature"
% TODO: make the direct quote shorter, such things should only be done "to support an argument", which I don't here.
In a final step, the candidate-words are clustered according to their similarity to find a fixed set of dimensions. A representative term for the directions is selected, and the entities are re-embedded into a new space comprised of these dimensions, where the individual vector-components correspond to the ranking of an entity with respect to these dimensions.
% * from Alshaikh2020: "The learned vectors will be referred to as feature directions to emphasize the fact that only the ordering induced by the dot product d_i · e matters"
\newline

\todoparagraph{Important features:}
\begin{itemize}
	\item Unsupervised (in contrast to \textcite{VISR12})
	\item Modular (subcomponents may be exchanged)
\end{itemize}


Many of the just mentioned components don't refer to a specific algorithms, and the existing implementations \mainalgos differ in many of these implementations. The rest of this sections goes into further detail for each of these components, and an overview of the components supported in the respective implementations is given in \tref{tab:compared_algos}

% * Yamls für die Configs von DESC15 und den anderen beiden in den Text packen
% * Im Text link zu binder bei results section der auf die notebooks/analyze_results/analyze_pipeline_results.ipynb referenziert, und für die tables auch!
% * Schreiben dass ich einige Claims oder nonclaims von denen prüfe, bspw nutzen sie immer PPMI ohne je tf-idf zu testen
% * RaZb20 tried Doc2Vec instead of MDS and it performed worse!
% * Wie lange der ganze Kram dauert - MDS hat quadratic complexity etc
% * [AGKS18] haben den candidate-filter-teil konfigurierbar (und sagen bei denen ist accuracy even better than kappa)
% * Das mit dem Koordinatensystem drehen passiert gar nicht so wie ich dachte dass es passiert...?!
% * Tabelle
% 	* Die Parameter von [AGKS18] und [RaZb20] sowohl in die Tabelle als auch ins yaml packen
% 	* Einduetig rausschreiben welche der 3 paper [DESC15] [AGKS18] [RaZb20] welche parameter-werte verlangen und !!welche optimal waren!! angucken welche Kombi die Beste Performance hatte und die entsprechend markieren (und im yaml haben!)!
% * Regarding DESC15 vs AGKS18 vs RaZb20:
% 	* didn't somebody say that cohen's kappa sucks!?!
% 	* RaZb20: 
% 		* use affinity propagation "for getting rid of the clusters of informative words", similar to how they did it in their 2019 paper
% 			-> affinity propagation has a so-called preference parameter, den als config lassen - usual, this parameter is chosen relative to the median µ of the affinity scores. For the methods Sub and Or- tho, we considered values from {0.7µ, 0.9µ, µ, 1.1µ, 1.3µ}
% 		* do kappa ON BINARY!!!
% 		* say that for them, the binary "does the word occur in the description" is the only sensible signal, no ppmi or anything! (page 2, footnote 1 of RaZb20)
% 	* DESC15: 
% 		* "Here we use the assumption that the better Ht separates entities to which t applies from the others in S,the better \vec{v_t} models the term t." --> allein von der aussage muss man das mit den induzierten rankings echt nicht machen, sondern halt nur auf classification quality (-> metrics like accuracy) gucken, bzw kappa anhand der binären klasse berechnen --> the ranking induced by count, or the baremetal count?

% 	* POSSIBLE EXTRA-STEPS FOR ALGORITHM
% 		* BOOTSTRAP MORE CANDIDATES (AFTER EXTRACT CANDIDATES)
% 			* [VISR12]: LSI
% 				* Options:
% 					* What to take for the term-document-matrix
% 						* [VISR12]: 
% 							* tag-applied
% 							* tag-count
% 							* tag-share (the number of times tag t has been applied to item i, divided by the number of times any tag has been applied to item i)
% 						* relative-tag-count (tag-count / text-len) or tag-count / distinct-words-in-text
% 						* See also: https://en.wikipedia.org/wiki/Latent_semantic_analysis#Term-document_matrix
% 				* Parameters:
% 					#dims for the rank reduction (see https://en.wikipedia.org/wiki/Latent_semantic_analysis#Rank-reduced_singular_value_decomposition)

% * Schritte des Algorithmus bewerten:
% 	* "that the better Ht separates entities to which t applies from the others in S,the better \vec{v_t} models the term t." --> wie sinnvoll ist diese measure wenn das verhältnis literally 14.900 zu 100 ist, dann haben halt 99.33% der Daten einen rank von 0 ?!
% 	* DESC15 write they select Kappa "due to its tolerance to class imbalance." -> Does that mean I have to set the weight?! Außerdem weiß ich ja superviel ebennicht, like which weighting they use! I don't like
% 	* Der letzte Schritt mit dem Clustern der good-kappa-ones ist wirklich very basic und hat very much room for improvement


% General idea of the algorithm: Alshaikh2020: "Their core assumption is that words describing semantically meaningful features can be identified by learning for each candidate word w a linear classifier which separates the embeddings of entities that have w in their description from the others. The performance of the classifier for w then tells us to what extent w describes a semantically meaningful feature" (TODO: my concise formulation)
% TODO: despite my citation style, write the names of the three main papers at least once




%TODO: maybe describe shortly what the improvements from  \cite{Ager2018} and \cite{Alshaikh2020} were? 
% Alshaikh2020: 
% * "When representing a particular entity in a conceptual space, we need to specify which domains it belongs to, and for each of these domains we need to provide a corresponding vector." 
% * then they show their example of something that is not seperable with a hyperplane unless we specify subdomains, maybe just steal their plot that explains their one contribution to 99%
% * The "Disentangled" from their title means "feature-based"

The core idea of the algorithm is to (unsupervised, data-driven) find a a set of features which can be modelled as directions for a vector-space representation of the respetive entities. For that, the steps are:
\begin{enumerate}
	\item preprocess the descriptions using default techniques and create a bag-of-words representation for the texts
	\item extract candidate features from these texts (easist variant is to just consider each sufficiently frequent word as candidate)
	\item create an fixed-dimensional embedding for the texts (optimally a metric space, optimally based on their dissimilarity)
	\item for each candidate term, train a linear classifier to seperate the vector representationgs of the entities that contain the term vs those that don't
	\item if some metric for this classifier is sufficiently high, assume that the candidate term captures a salient feature - it's direction is then characterized by the normal vector for that hyperplane (for an intuition see \ref{fig:3d_hyperplane_ortho})
	\item Cluster the candidates and calculate the direction of the cluster from the directions of it's contents
\end{enumerate}

\begin{figure}
	\begin{center}
	  \includegraphics[width=0.9\textwidth]{3dplot_hyperplane_and_orthogonal}
	  \caption[Visual representation of the Hyperplane of an SVM splitting a dataset]{ \label{fig:3d_hyperplane_ortho} Visual representation of the Hyperplane of a Support-Vector-Machine splitting a dataset, as well as it's orthogonal and the orthogonal projection of a set of samples onto the plane. For an interactive version of this plot, visit  {\small \url{https://nbviewer.org/github/cstenkamp/derive_conceptualspaces/blob/main/notebooks/text_referenced_plots/hyperplane_orthogonal_3d.ipynb?flush_cache}}}
	\end{center}
\end{figure}

\begin{figure}[htp]
	\begin{center}
	  \includegraphics[width=0.9\textwidth]{dependency_graph_2022-02-14_11-36-45}
	  \caption[Dependency-Graph of the Algorithm]{(automatically generated) dependency-graph, displaying the individual steps of the algorithm as well as their dependencies and where selected important parameters are first used.}
	  \label{fig:depdendency_graph}
	\end{center}
\end{figure}

\@input{pandoc_generated_latex/section_algorithm}


\afterpage{%

\newgeometry{
	a4paper,
	top=18mm,
	bottom=8mm,
} 


\begin{landscape}
	\begin{table}[]
		%Preprocessing siehe other table
		\resizebox{\textwidth}{!}{%
			\begin{tabular}{lllll} % statt textcite mit \fancyquotecite https://tex.stackexchange.com/a/468286/108199
			& \textbf{\textcite{Derrac2015}} & \textbf{\textcite{Ager2018}} & \textbf{\textcite{Alshaikh2020}} & \textbf{This codebase} \\
			\textbf{Step 1: Pre-process text corpus} & \multicolumn{3}{c}{\textit{see \tref{tab:all_datasets}}} & \specialcell[l]{placetypes: \textit{see \tref{tab:all_datasets}} \\ SIDDATA: \textit{see section \ref{sec:algo_preproc}} }
			\\ \midrule
			\textbf{Step 2: Generate Vector Spaces (Embeddings) from text corpus} 
			& 
			\specialcell[l]{MDS trained with the angular differences $\frac{2}{\pi}* arccos\left(\frac{\vec{v}_{e_i}*\vec{v}_{e_j}}{\norm{\vec{v}_{e_i}}*\norm{\vec{v}_{e_j}}}\right)$ \\
				between the PPMI weighted BoW Vectors (all terms) \\
				Spaces of dim 20, 50, 100 and 200 \\
				Previous Experiments also considered SVD and Isomap}
			&
			\specialcell[l]{ \tabitem like \cite{Derrac2015} 
				\\ \tabitem PCA from PPMI weighted BoW vectors (no quadratic complexity) \\ \tabitem Doc2Vec\footnote{\label{foot:doc2vec} TODO: Explain and cite!} Document Embeddings \\ \tabitem thresholded\footnote{\label{foot:threshold}Only words that occur more than 2 times (15 for movies-dataset)}, averaged pre-trained GloVe word embeddings \\ \tabitem thresholded\footref{foot:threshold}, PPMI-weighted-averaged pre-trained GloVe word embeddings \\
				Number of dimensions one of (50, 100, 200)}
			&
			\specialcell[l]{ \textbf{movies and placetypes:} Re-used the 100D-embeddings of \textcite{Derrac2015} \\ \textbf{other datasets}: \\ \tabitem 100d Document Embeddings from angular differences and MDS\\  \tabitem 100d-Doc2Vec\footref{foot:doc2vec} }
			& 
			\specialcell[l]{ \tabitem MDS, \tabitem t-SNE or \tabitem Isomap with arbitrary number of dimensions \\ on a dissimilarity-matrix based on \tabitem BoW (raw counts), \tabitem BoW (binarized), \tabitem tf-idf-weighted BoW, \tabitem ppmi-weighted BoW, \tabitem tf-weighted BoW \\ distance measure \tabitem normalized angular distance \tabitem cosine distance \tabitem TODO: the other distance measures}
			\\ \midrule

			\specialcell[l]{ \textbf{Step 3: Generate Candidate Words} \\ \textbf{for Feature Directions} } 
			& 
			\specialcell[l]{ All sufficiently frequent\footnote{For the placetypes-dataset: all tags that co-occur with at least 50 place types} adjectives, nouns, adjective phrases and noun phrases \\
				(using POS-Tagger and Chunker from openNLP) } 
			&  
			\specialcell[l]{All sufficiently frequent words\footnote{For the thresholds, see table \ref{tab:all_datasets}} (use PPMI in a later step so possibly PPMI?!) \\ weighted Logitistic Regression Classifier for Vector Direction} 
			&
			\specialcell[l]{ 
				\textbf{Candidates:} movies and placetypes: see \textcite{Derrac2015}, other datasets: all occuring\footnote{see Datasets-Table} 1-grams \\
				\textbf{Classifier:} logistic regression classifier (similar performance to SVM but faster training)
			}
			&
			\specialcell[l]{ Keywords extracted using \\ KeyBERT (on \tabitem raw or on \tabitem preprocessed texts) \\ \tabitem all sufficiently frequent phrases \\ Those with a minimal score for \tabitem PPMI \tabitem tf-idf \tabitem tf}
			\\ \midrule


			\textbf{Step 4: Filter Candidate Feature Directions} 
			&                   
			\specialcell[l]{ linear SVM for all candidates (pos samples: $\forall e: c \in e$) with pos/neg-count-ratio as instance cost \\
				Only take candidates where the correlation according to Cohen's Kappa between the ranking induced by the SVM's hyperplane and count(t, e) is $\geq \lambda$ (0.5/0.1)\\
				also tried Spearman's and Kendall's correlation coeffcients }
			&
			\specialcell[l]{ Classifier-Performance as measured by \\ 
				\tabitem Cohen's Kappa (compared to the ranking induced by the PPMI) \\ \tabitem Accuracy (binary)\\ \tabitem Normalized Discounted Cumulative Gain (NDCG) (TODO: binary? ranking?) } 
			& 
			\specialcell[l]{
				Cohen's Kappa (threshold=0.3 in iteration 1 and 0.1 in iteration 2), only the top 5000 scoring features
			}
			& 
			\specialcell[l]{ 
				Various classifiers such as \tabitem linear SVM \tabitem squared-hinge-loss-SVC \\ %TODO: add SVC to acronyms. TODO: see Complexities of SVM implementations in python: https://stackoverflow.com/a/64274403/5122790 TODO: it's really easy to add the classifiers of \cite{Ager2018} & \cite{Alshaikh2020}, low-hanging fruit! 
				Compare classifier performance with \tabitem count \tabitem tf-idf \tabitem PPMI \tabitem \dots \\
				Comparison Functions: \tabitem Accuracy, Precision, Recall, F1 \tabitem Cohen's Kappa (rank2rank) \tabitem Cohen's Kappa (various other ways)
			}
			\\ \midrule


			\textbf{Step 5: Merge Feature Directions} 
			&   
			\specialcell[l]{ According to \cite{Alshaikh2020} "a variant of \textit{k}-means" \\ Cluster centers: Select Term with highest $\kappa$, then i=2*ndims times select the term from $T^{0.5}$ minimizing $max_{j<i}cos(\vec[m]{v_{t_j}},\vec[m]{v_t})$ \\ 
				Others: assign all terms from $T^{0.1}$ to the closest cluster and define $\vec[m]{v^*_i} = \frac{1}{|C_i|} \sum_{t\in C_i} \vec[m]{v_t}$ as cluster direction (average direction of cluster's elements)} 
			&
			\specialcell[l]{Input-ndims for clustering algorithm one of (500, 1000, 2000) \\
			Number of clusters one of (ndims, 2*ndims) \\
				Centroid of the cluster computed as $v_{C_j} = \frac{1}{|C_j|}\sum_{w_l \in C_j} v_l$, provided $\vec[m]{v_w}$ all normalized} 
			& 
			\specialcell[l]{  \textbullet\, \textbf{sub, ortho, primary}:   
				Affinity propagation instead of \textit{k}-means (no need to specify the ndims, helps with the issue that there are some non-informative clusters in \cite{Derrac2015}'s algorithm) \\ ~~ n-dims for this not directly configurable, only over preference parameter relative to median $\mu$, tried for (0.7$\mu$, 0.9$\mu$, $\mu$, 1.1$\mu$, 1.3$\mu$) \\  \textbullet\, \textbf{AHC}: Agglomerative Hierachical Clustering to cluster word directions with distance cut-offs  \\ \textbullet\, also tried Hierachical LDA \\ Cluster direction (AHC and Affinity Propagation): normal vector of the hyperplane of a linear classifier separating entities whose description contains at least one of the words from the cluster from the others  } %TODO find a short-term notation for "entities whose description contains at least one of the words from XYZ" ("for a Cluster C, we write pos_C and neg_C for the set of positively and negatively classified entities")
			&
			TODO: my stuff
			\\  \midrule

			\textbf{Step 6: Post-Processing} 
			&                   
			None 
			&                 
			TODO describe fine-tuning! 
			& 
			\specialcell[l]{
				Perform steps 1-4 a second time (only for positively classified entities), such that there are primary features (domains) and sub-features \\ representation kept flat (values for the sub-features is same dot-product as for domains) \\ \textbullet\, \textbf{sub}: sub-features extracted equal to first-order-features \\ \textbullet\, \textbf{ortho}: sub-feature directions orthogonal to corresponding primary feature direction (enforce complementary information) \\ ~~ by computing orthogonal decomposition of feature w.r.t. its domain (pg. 4, equation 1) \\ \textbullet\, also tried to combine mother-domain with sub-feature, but that performed poorly \\ \textbullet\, \textbf{primary}: model with only primary features}
			&
			TODO: my stuff
			\end{tabular}
		}
		\caption{Compared algorithms from \mainalgos}
		\label{tab:compared_algos}
	\end{table}
\end{landscape}

% Alshaikh2020 say: "It may seem counter-intuitive to use binary classifiers to learn representations of ordinal features. However, the occurrence or non- occurrence of a word in the description is binary, and this is the most important available signal. We experimented with statistics such as pointwise mutual information, which did not lead to better results." -> does that mean DCM_QUANT_MEASURE or QUANTIFICATION_MEASURE or both? and also does that mean they use binary or count as this measure?

% Alshaikh2020 use: sub, ortho, primary, random (coordinates uniformly random), 

} %afterpage
\restoregeometry

\subsection{Preprocessing}

There are issues with using stop-word lists, see \cite{nothman-etal-2018-stop} ( SkLearn references this paper why their own/stopwordslists in general suck)


%Was für Schritte hat der Algo 

%TODO something along the lines of "Da, based on [source die die accuracy von dem gtranslate algorithm mit denen von menschen vergleicht], a gtranslate translation is as good as the average lecturer, it is assumed that translating the texts to english before using an english model can lead to better results


\subsection{Extracting Candidate Terms}
\label{sec:extract_candidates}

* KeyBERT

%TODO theoretisch ist es auch möglich bspw nen network mit attention auf gewisse dinge wie fachbereich und anderes zu trainieren und dann rauszusuchen was die wichtigen ausschlaggebenden dinge für das Netzwerk waren

%TODO extract using TF/IDF as well

%TODO a source: https://github.com/MaartenGr/KeyBERT#citation

* After you figure out which candidate term appears in which texts, figure out which other terms are frequent in these texts while infrequent in texts of the other class and then add these to the candidate-term-set (other way may even be to classify the texts according to if the candidateterm appears in them, and then take the misclassified one also as positive samples)

\begin{figure}[H]
	\centering
	\includegraphics[width=\figwidth]{graphics/figures/keyphrases_histogram.png}
	%plot created with scripts/create_siddata_dataset.py filter-keyphrases /home/chris/Documents/UNI_neu/Masterarbeit/DATA_CLONE --verbose
	\caption[Occurences in all Documents per Keyphrase]{
		\label{fig:keyphrases_histogram}
		Occurences in all Documents per Keyphrase (for all keyphrases that occur $\geq$ 5 times, cut off at the 93th percentile).
		7007 of 45295 terms occur at least 5 times.
		Most frequent phrases: seminar (4173), course (3722), students (2923), it (2671), language (2071), work (1980), event (1842), research (1731), lecture (1723), law (1719).
		}
\end{figure}


\subsection{Calculating the distance to the SVMs Separatrix}
\label{sec:calculate_distance}

%TODO: before this, explain 
% * extraction of the candidate term set
% * how vectors are made from texts
In order to tell how much a text is prototypical of a category, all texts are split depending on whether they contain words of a set as described in \ref{sec:extract_candidates}, before a linear Support Vector Machine Classifier is trained on the vector-representation of all of the texts, splitting them into two classes: those that contain one of the candidate terms and those that do not. Due to the linear kernel, the SVM finds a hyperplane (\textit{separatrix}) that splits the positive from the negative samples in a way that maximizes the distance between the respective classmembers and the separatrix, using samples close to the margin as \textit{support vectors}. %TODO: what if the data is not linearly seperable?

Following the SVMs logic, one could argue that the further away a sample is from this separatrix, the more prototypical it is of its class. Thus, the distance of a sample to its \textit{orthogonal projection onto the hyperplane} %TODO explain orthogonal projection
may serve as metric for how prototypical a sample is for the respective category. 

Translated into terms relevant to the aim of this thesis, the classes may be those educational resources whose description contains the word "\textit{hard}" as one class vs. those that do not as the other class. Now according to [TODO], one may use the distance of the sample towards the separatrix as a measure of how hard a class is: For all positive samples, a longer distance means a harder class, for all negative samples a longer distance means an easier one, whereas those samples close to the separatrix can be considered average.

\subsubsection*{How to calculate the distance}

%TODO mention that I'm in a cartesian coordinate system
%TODO mention that I'm thinking in euclidian coordinates, see https://en.wikipedia.org/wiki/Plane_(geometry)
\noindent In the following paragraphs, I will visualize how to calculate the orthogonal distance from a sample to the hyperplane exemplary for the case of three-dimensional text-vectors.

Generally, the separatrix splitting positive from negative samples for an $n$-dimensional space $\mathds{R}^n$ is an $n-1$-dimensional subspace (called \textit{hyperplane}), which in the case of $\mathds{R}^3$ corresponds to a plane. 
%https://en.wikipedia.org/wiki/Plane_(geometry)#Representation

The general form of the equation of a plane is given as the following linear equation, where parameters $a, b, c$ and $d$ encode the position of the plane:

\begin{equation}
	\label{eq:general_plane}
	ax + by + cz = d
	%TODO source? mein Tafelwerk? :D
\end{equation}

This reads as "All points $(x,y,z)$ for which \ref{eq:general_plane} holds are on the plane". 

In this representation of the plane, the vector $(a,b,c)$ encodes a normal vector orthogonal to this plane, whereas $d$ serves as intercept, encoding the intersection of plane and normal. (specifically: the perpendicular distance you would need to translate the plane along its normal to have the plane pass through the origin) In higher dimensions, the formula for the hyperplane would become $a_1x_1+a_2x_2+a_3x_3+...+a_nx_n = b$, which means that encoding the hyperplane for a space $\mathds{R}^n$ requires $n+1$ parameters.
% one rough quote in this paragraph from https://stackoverflow.com/a/17661431

%TODO explain that it's not even harder in higher-dimensional spaces
%TODO explain that nicely, in python the separatrix is perfectly specified using the normal and the intercept, so we have everything we need 
%TODO die handschriftlichen notizen aus den beiden notebooks einbauen
%TODO die plots aus dem notebook einbauen

% The distance from any point of this $\mathds{R}^n$ to the hyperplane is then calculated as the length of the vector that is the \textit{orthogonal projection} from that point onto the hyperplane. The orthogonal projection from one vector onto another can be calculated as 

% \begin{equation}
% 	\label{eq:orthogonal_projection}
% 	\hat{\vec{a}} = \frac{\vec{a}\cdot\vec{b}}{\vec{b}\cdot\vec{b}}\cdot\vec{b}
% 	%TODO source https://en.wikipedia.org/wiki/Vector_projection
% \end{equation}

% ...as we however have a plane we want to project to, not a vector, what I wrote here is rather useless, isn't it?


The distance from any point of this $\mathds{R}^n$ to the hyperplane can then be calculated as 
% As... * dist(point, point_projected_onplane)     						 (`project[1,3]_pre`)
%       * abs(trafo(point)[0])     										 (`protoypicality_pre`)
%       * np.dot(plane.normal, point) + plane.d							 (`project2_pre')
% ...normiert sind die alle gleich, aber for some reason differn die um nen multiplicator..?!
% And the projections...:
%       * back_trafo([0, trafo(point)[1], trafo(point)[2]]
% 		* plane.project(point): k = (ax+by+cz+d)/(a²+b²+c²); result = [x-ka, y-kb, z-kc]
%       * point - distance * plane.normal  (...aber nur mit protoypicality_pre als distance! )
% 		...note that second and third are basically equal - both calculate "how much do I need to go into the direction of the orthogonal" and then do so  (point - distance * normal). The difference is that in plane.project the distance is divided by (a²+b²+c²). Originally sagt der typ von SO (https://stackoverflow.com/a/17661431) die distance ist einfach n*p-d, dann fehlt nur der normierungsterm. 
% TODO: figure out the explanation of the difference from this to the result of using forward and backward??
% See get_svm_decisionboundary.ipynb, den kram zwischen `#deleteme from here', commit d46a8300dae81adee


\subsection{Clustering the extracted candidates}

An analysis of \cite{Carmel2009} showed that a statistical method to extract features from clustered text corpora identified the labels of human annotators as one of the top five most important terms in only 15\% of cases, implying ``that human labels are not necessarily significant from a statistical perspective" \cite[139]{Carmel2009}
%TODO: die eigentliche Methode (JSD) mehr erklären!!
%(the JSD method for feature selection identifies human labels as “significant” (appearing in the top five most important terms) for only 15% of the categories. This result implies that human labels are not necessarily significant from a statistical perspective.z)


\chapter{Results}



\begin{table}[h]
	\resizebox{\textwidth}{!}{%
	\begin{tabular}{llllrrrrrrrrr}
	\toprule
% 	 &  &  &  & \rotatebox{70}{\textbf{k_r2r_d}} & \rotatebox{70}{\textbf{k_r2r_min}} & \rotatebox{70}{\textbf{k_dig}} & \rotatebox{70}{\textbf{k_r2r+_d}} & \rotatebox{70}{\textbf{k_r2r+_min}} & \rotatebox{70}{\textbf{k_r2r+_max}} & \rotatebox{70}{\textbf{k_dig+_2}} & \rotatebox{70}{\textbf{k_c2r+}} & \rotatebox{70}{\textbf{mean}} \\
	\textbf{Preprocessing} & \specialcell[b]{\textbf{Quanti-}\\ \textbf{fication}} & \textbf{\#Dims} & \specialcell[b]{\textbf{Doc-Term-}\\ \textbf{Matrix} \\ \textbf{Quanti-}\\ \textbf{fication}} & \rotatebox{70}{\textbf{r2r-d}} & \rotatebox{70}{\textbf{r2r-min}} & \rotatebox{70}{\textbf{dig}} & \rotatebox{70}{\textbf{r2r+d}} & \rotatebox{70}{\textbf{r2r+min}} & \rotatebox{70}{\textbf{r2r+max}} & \rotatebox{70}{\textbf{dig+2}} & \rotatebox{70}{\textbf{c2r+}} & \rotatebox{70}{\textbf{mean}} \\
	\midrule
	\multirow[t]{24}{*}{\mfauhcsdT} & \multirow[t]{8}{*}{\textbf{count}} & \multirow[t]{2}{*}{\textbf{3}} & \textbf{ppmi} & 0 & 1 & 0 & 145 & 370 & 510 & 191 & - & 174 \\
	 &  &  & \textbf{tfidf} & 0 & 1 & 0 & 110 & 237 & 278 & 83 & - & 101 \\
	\cline{3-4}
	 &  & \multirow[t]{3}{*}{\textbf{100}} & \textbf{count} & 0 & 5 & 0 & 0 & 114 & 52 & 290 & 0 & 58 \\
	 &  &  & \textbf{ppmi} & 0 & 6 & 27 & 139 & 224 & 247 & 120 & - & 109 \\
	 &  &  & \textbf{tfidf} & 0 & 6 & 5 & 246 & 270 & 281 & 201 & - & 144 \\
	\cline{3-4}
	 &  & \multirow[t]{3}{*}{\textbf{200}} & \textbf{count} & 0 & 5 & 1 & 0 & 133 & 52 & 509 & 0 & 88 \\
	 &  &  & \textbf{ppmi} & 0 & 6 & 57 & 196 & 315 & 344 & 90 & - & 144 \\
	 &  &  & \textbf{tfidf} & 0 & 6 & 17 & 357 & 370 & 372 & 433 & - & 222 \\
	\cline{2-4} \cline{3-4}
	 & \multirow[t]{8}{*}{\textbf{ppmi}} & \multirow[t]{2}{*}{\textbf{3}} & \textbf{ppmi} & 0 & 0 & 0 & 192 & 247 & 363 & 136 & - & 134 \\
	 &  &  & \textbf{tfidf} & 0 & 0 & 0 & 169 & 206 & 217 & 59 & - & 93 \\
	\cline{3-4}
	 &  & \multirow[t]{3}{*}{\textbf{100}} & \textbf{count} & 0 & 0 & 0 & 0 & 38 & 25 & 242 & 0 & 38 \\
	 &  &  & \textbf{ppmi} & 0 & 0 & 0 & 80 & 112 & 101 & 22 & - & 45 \\
	 &  &  & \textbf{tfidf} & 0 & 0 & 0 & 89 & 90 & 96 & 85 & - & 51 \\
	\cline{3-4}
	 &  & \multirow[t]{3}{*}{\textbf{200}} & \textbf{count} & 0 & 0 & 0 & 0 & 34 & 21 & 293 & 0 & 44 \\
	 &  &  & \textbf{ppmi} & 0 & 1 & 112 & 100 & 163 & 163 & 37 & - & 82 \\
	 &  &  & \textbf{tfidf} & 0 & 1 & {\cellcolor{lightgreen}} 127 & 99 & 107 & 106 & 131 & - & 82 \\
	\cline{2-4} \cline{3-4}
	 & \multirow[t]{8}{*}{\textbf{tfidf}} & \multirow[t]{2}{*}{\textbf{3}} & \textbf{ppmi} & 0 & 0 & 0 & 229 & 357 & 423 & 84 & - & 156 \\
	 &  &  & \textbf{tfidf} & 0 & 0 & 0 & 169 & 255 & 258 & 24 & - & 101 \\
	\cline{3-4}
	 &  & \multirow[t]{3}{*}{\textbf{100}} & \textbf{count} & 0 & 1 & 0 & 0 & 162 & 64 & 450 & 0 & 85 \\
	 &  &  & \textbf{ppmi} & 0 & 1 & 3 & 324 & 404 & 423 & 151 & - & 187 \\
	 &  &  & \textbf{tfidf} & 0 & 1 & 0 & 390 & 422 & 437 & 425 & - & 239 \\
	\cline{3-4}
	 &  & \multirow[t]{3}{*}{\textbf{200}} & \textbf{count} & 0 & 2 & 0 & 0 & 211 & 83 & {\cellcolor{lightgreen}} 869 & {\cellcolor{lightgreen}} 1 & 146 \\
	 &  &  & \textbf{ppmi} & 0 & 2 & 13 & 395 & {\cellcolor{lightgreen}} 559 & {\cellcolor{lightgreen}} 577 & 153 & - & 243 \\
	 &  &  & \textbf{tfidf} & 0 & 2 & 0 & {\cellcolor{lightgreen}} 531 & 554 & 572 & 794 & - & {\cellcolor{lightgreen}} 350 \\
	\cline{1-4} \cline{2-4} \cline{3-4}
	\multirow[t]{24}{*}{\mfauhtcsldp} & \multirow[t]{8}{*}{\textbf{count}} & \multirow[t]{2}{*}{\textbf{3}} & \textbf{ppmi} & 0 & 1 & 0 & 226 & 319 & 317 & 208 & - & 153 \\
	 &  &  & \textbf{tfidf} & 0 & 1 & 0 & 210 & 214 & 215 & 82 & - & 103 \\
	\cline{3-4}
	 &  & \multirow[t]{3}{*}{\textbf{100}} & \textbf{count} & 0 & 7 & 0 & 0 & 118 & 61 & 230 & 0 & 52 \\
	 &  &  & \textbf{ppmi} & 0 & 8 & 27 & 184 & 256 & 262 & 125 & - & 123 \\
	 &  &  & \textbf{tfidf} & 0 & 8 & 5 & 253 & 255 & 255 & 168 & - & 135 \\
	\cline{3-4}
	 &  & \multirow[t]{3}{*}{\textbf{200}} & \textbf{count} & 0 & 8 & 0 & 0 & 117 & 64 & 290 & 0 & 60 \\
	 &  &  & \textbf{ppmi} & 0 & {\cellcolor{lightgreen}} 11 & 41 & 200 & 319 & 325 & 88 & - & 141 \\
	 &  &  & \textbf{tfidf} & 0 & {\cellcolor{lightgreen}} 11 & 8 & 331 & 333 & 333 & 302 & - & 188 \\
	\cline{2-4} \cline{3-4}
	 & \multirow[t]{8}{*}{\textbf{ppmi}} & \multirow[t]{2}{*}{\textbf{3}} & \textbf{ppmi} & 0 & 0 & 0 & 138 & 310 & 321 & 254 & - & 146 \\
	 &  &  & \textbf{tfidf} & 0 & 0 & 0 & 143 & 148 & 150 & 187 & - & 90 \\
	\cline{3-4}
	 &  & \multirow[t]{3}{*}{\textbf{100}} & \textbf{count} & 0 & 0 & 0 & 0 & 29 & 11 & 186 & 0 & 28 \\
	 &  &  & \textbf{ppmi} & 0 & 1 & 0 & 117 & 142 & 142 & 20 & - & 60 \\
	 &  &  & \textbf{tfidf} & 0 & 1 & 0 & 122 & 124 & 124 & 103 & - & 68 \\
	\cline{3-4}
	 &  & \multirow[t]{3}{*}{\textbf{200}} & \textbf{count} & 0 & 1 & 0 & 0 & 25 & 10 & 272 & 0 & 38 \\
	 &  &  & \textbf{ppmi} & 0 & 1 & 48 & 126 & 161 & 165 & 28 & - & 76 \\
	 &  &  & \textbf{tfidf} & 0 & 1 & 17 & 143 & 144 & 148 & 133 & - & 84 \\
	\cline{2-4} \cline{3-4}
	 & \multirow[t]{8}{*}{\textbf{tfidf}} & \multirow[t]{2}{*}{\textbf{3}} & \textbf{ppmi} & 0 & 0 & 0 & 146 & 219 & 223 & 133 & - & 103 \\
	 &  &  & \textbf{tfidf} & 0 & 0 & 0 & 108 & 111 & 109 & 38 & - & 52 \\
	\cline{3-4}
	 &  & \multirow[t]{3}{*}{\textbf{100}} & \textbf{count} & 0 & 1 & 0 & 0 & 160 & 54 & 389 & 0 & 76 \\
	 &  &  & \textbf{ppmi} & 0 & 2 & 9 & 281 & 375 & 380 & 205 & - & 179 \\
	 &  &  & \textbf{tfidf} & 0 & 2 & 0 & 373 & 377 & 392 & 339 & - & 212 \\
	\cline{3-4}
	 &  & \multirow[t]{3}{*}{\textbf{200}} & \textbf{count} & 0 & 3 & 0 & 0 & 199 & 64 & 661 & 0 & 116 \\
	 &  &  & \textbf{ppmi} & 0 & 3 & 21 & 362 & 456 & 472 & 164 & - & 211 \\
	 &  &  & \textbf{tfidf} & 0 & 3 & 1 & 499 & 498 & 501 & 645 & - & 307 \\
	\bottomrule
	\end{tabular}
	}
	\caption{Number of Candidate-Phrases for different parameter-combinations and kappa-values \label{tab:kappa_table}}
	\label{tab:cands_per_config}
\end{table}

	
		
	
	
	


\section{Evaluation} %TODO what is this for a title? it sucks


* This is clustering and looking if it corresponds to human judgement, which unfortunately doesn't allow for a simple accuracy and be done with it.
* So, the papers that did this come up with a few things
* [TODO: the shallow decisiontrees of one of the followups]
* DESC15 "evaluate the practical usefulness of the considered semantic relations" by checking "their use in commonsense reasoning based classifiers", like interpolation and a fortiori inference (chap 5)


* DESC15 tests like this: Section 6.1: Evaluate whether the derived relations are sufficiently accurate for classification, and 6.2 is then comparison with crowdsourcing experiments (more subjective aspects, the question “are the relations useful explanations?”)



\section{Qualitative Analysis}

Qualitative Analysis in this case means "looking at stuff". Such a qualitative analysis is always to be taken with a grain of salt, because it is very prone to cherry-picking (both on purpose and not on purpose, the stuff you're looking at just doesn't need to be representative!). However it does help a lot and provides a lot of insights (and often helped me in the debugging process).
What can you look at for such a qualitative analysis?
\begin{itemize}
	\item The clusters, checking if things you know to be similar are actually in the same clusters
	\item If descriptions you know to be semantically similar are actually close in the embedding
	\item You can do the whole thing for only three dimensions instead of the 50/100/200 because there you can plot the stuff and interpret it
\end{itemize}

\begin{itemize}
	\item  Man kann ja schon nach dem Embedding anhand der nächsten Entities sehen ob das was werden kann - bei 100D sind dann halt "airplane cabin" und "aircraft cabin" die nächsten entities, bei 3D dann halt eher kram wie "area" and "moor", was schon eindeutig zeigt dass 3D offensichtlich nicht so der Hit ist
	\item Die vielen Sanity Checks die man machen kann, bspw dass ich ja in 3D gucken kann (und auch in höher-D ausrechnen) ob eben diese dinge (von item 1) im Embedding nah sind, und ob die SVM Dinge schön trennt ("howto_embed.ipynb")
	\item "placetypes_origconf.ipynb", was einfach von vorne bis hinten die original-config (ist ja auch im yaml) von DESC15 ausführt und interpretiert	
\end{itemize}

\begin{figure}[H]
	\centering
	\includegraphics[width=\figwidth]{svm_mathematik_highlight_infoAB.png}
	\caption[3D-Plot with an SVM for the term "Mathematik"]{
		\label{fig:3dplot_mathe_infoab}
		3D-Plot with an SVM for the term "Mathematik", also highlighting the courses "Informatik A" and "Informatik B"
	}
\end{figure}

In figure \ref{fig:3dplot_mathe_infoab} we see a 3D-Embedding for courses, splitting courses which contain the term "mathematics" from those that don't, also hightlighting the terms "Informatik A" and "Informatik B". We see they are close we see the SVM is not to bad, and even though neiher Info A nor Info B contains the word "mathematik", thy are both on the "mathematical side" of courses. Negative samples are hidden for better visibility, and entities that contain the word more-often-than-the 75th (???) percentile have bigger markers.


\begin{itemize}
	\item In 3D ists immer ne Kugel, und ich würde behaupten in höheren Dimensionen ist es nicht extrem viel besser. dadrin ne SVM zu machen bringt echt wenig bis gar nix (Ich hab ja sogar Plots die zeigen dass die Movies viel besser clustern - TODO: die einbringen)
\end{itemize}


\section{Quanitative Results}

% Schreiben was die paper denen ich mostly folge zur evaluation gemacht haben! ("To evaluate whether the discovered features are semantically meaningful, we test how similar they are to natural categories, by training depth-1 decision trees")
% Ein anderer Weg zum testen wäre auch ein classifier der nur anhand der most salient generated features versucht den kurs wiederherzustellen (das zeigt natürlich nicht ob es similar to how humans do it but part of it)

Here I'll add the results of the low-depth-decision-trees for Fachbereich, and also compare the results of throwing my code onto their placetypes-dataset and how my results compare to theirs 
(set overlap of candidate terms!)

To see if it is possible to extract any kind of structured data from the unstructured course descriptions, a Neural Network classifier was trained on the dataset, classifying courses to the faculty they run under. 
$\rightarrow$ Der FB-Classifier kommt auf $95.33\%$ train, $90.96\%$ Test accuracy nach 10 epochs, that's a lot!!


Both \cite{Ager2018} and \cite{Alshaikh2020} train shallow decision-trees (depth 1 and depth 3 each), on their feature-based representations (such that the 1 or 3 most distinct interpretable dimensions are used) on a known property of the data (genres for movies, category in some taxonomy for placetypes, fachbereich for mine) - in the assumption that these eg in the movie domain the genre (or rather *terms accurately predicting it*) is among the features.



\chapter{Discussion and Conclusion}
% (was sind die broaden takeaways von meinem Kram)
% * Nochmal nen theoretisches Embedding, Kontextualisieren für Bildungsressourcen
% * ...and conclusion

\section{Future Work}

\@input{pandoc_generated_latex/section_future_work}



\section{Discussion}


In die Conclusion auch die Frage inwieweit das jetzt conceptual spaces sind (sehr viele vereinfachende sachen, like no convex regions but simply dots)
[AGKS18] ist da auch mehr humble als [DESC15] und sagt "The idea of learning semantic spaces with accurate fea- ture directions can be seen as a first step towards methods for learning conceptual space representa- tions from data"


\chapter*{Acknowledgements}
%TODO A place to say thank you to everybody who helped you.


% START Acronym definitions
\newacronym{utc}{UTC}{Universal Time Coordinated}
\newacronym{ml}{ML}{Machine Learning}
\newacronym{svm}{SVM}{Support Vector Machine}
\newacronym{mds}{MDS}{Multi Dimensional Scaling}
\newacronym{ppmi}{PPMI}{Positive Pointwise Mutual Information}
\newacronym{bow}{BoW}{Bag Of Words}
\newacronym{imdb}{IMDB}{Internet Movie Database}
% END Acronym definitions

\glsaddall
\printglossaries %TODO let glossary appear in TOC

%----------------------------------------------------------------------------------------
%	THESIS CONTENT - APPENDICES
%----------------------------------------------------------------------------------------
	
	\appendix % Cue to tell LaTeX that the following "chapters" are Appendices
	
	% Include the appendices of the thesis as separate files from the Appendices folder
	% Uncomment the lines as you write the Appendices
	
	% Appendix A

% \newgeometry{
% 	a4paper,
% 	top=21mm,
% 	bottom=11mm,
% 	inner=24mm,
% 	outer=9mm,
% } %bindingoffset=.5cm

%\newgeometry{
%	a4paper, inner=1.9cm, outer=1.9cm, bindingoffset=1.3cm, top=1.5cm, bottom=1.5cm, 
%} %bindingoffset=.5cm


% \lstset{
% 	numberblanklines=false
% 	,basicstyle=\ttfamily%
% 	,breaklines=true%
% 	,tabsize=1%
% 	,showstringspaces=false%
% 	,numbers=left%
% 	,numbersep=\lstnumbersep%
% 	,numberstyle=\lstnumberstyle%
% 	,framesep=0pt%
% 	,xleftmargin=\lstnumberwidth%
% 	,framexleftmargin=\lsthorizontalpadding%
% 	,xrightmargin=\lsthorizontalpadding%
% 	,framexrightmargin=\lsthorizontalpadding%
% 	,backgroundcolor=\color{verylightgray}%
% 	,postbreak=\ding{229}\space%
% 	,escapeinside={*(}{*)}
% 	\linespread{1.0}
% }


\chapter{Code Use-Cases (and how to call it)} % Main appendix title

\label{AppendixA} 

\vspace{-0.8cm}

\@input{pandoc_generated_latex/appendix_a}

%\lstinputlisting[language=Python, firstline=29]{codes/dqn.txt}


	% Appendix B
% https://tex.stackexchange.com/questions/152829/how-can-i-highlight-yaml-code-in-a-pretty-way-with-listings

% \newgeometry{
% 	a4paper,
% 	top=21mm,
% 	bottom=11mm,
% 	inner=24mm,
% 	outer=9mm,
% } %bindingoffset=.5cm

%\newgeometry{
%	a4paper, inner=1.9cm, outer=1.9cm, bindingoffset=1.3cm, top=1.5cm, bottom=1.5cm, 
%} %bindingoffset=.5cm



\chapter{Implementation Details} % Main appendix title

A main goal of this thesis is to provide a code base that makes it as simple as possible to get started with \gencite{Derrac2015} algorithm to derive rudimentary conceptual spaces for any kind of dataset. In order to achieve this, documenting some implementation details and design decisions is crucial.
% TODO: something a la "es ist aber zu detailliert für den hauptteil und zerstört den lesefluss, deswegen ist der aufbau halt so dass der Hauptteil/der methods-section sich möglichst kurz fasst, wie halt die methods-section von nem Paper, und ebendieser appendix für diejenigen gedacht ist die den spezifischen Algorithmus genauer wissen wollen ODER den code nutzen wollen ODER sich einfach fragen warum dinge so sind wie sie sind. Also I HAVE to cite some of the used techniques as per their licences.
This appendix goes into more detail for selected components of the algorithm.

\label{AppendixB} 

\section{Algorithm Implemetatation Details}

\subsection*{Preprocessing}

see \autoref{sec:algo_preproc}

\subsubsection*{Language-Detection and Translation}
\label{ap:translating}

To check the languages of the entities, the \codeother{langdetect}\footnote{\url{https://pypi.org/project/langdetect/}, \textcite{nakatani2010langdetect}} library is used, which is a direct port of a java library that claims to have 99.8\% accuracy on longer texts \cite{nakatani2010langdetect}. 
\newline

Depending on the translation-policy, it is possible to either take only those entities of the demanded language, ignore it and consider all entities in their original language, or enforce the demanded language by translating all entities from their original language to the demanded one. The accompaning code for this thesis contains extensive code to do that using the \emph{Google Cloud Translation API}\footnote{\url{https://cloud.google.com/translate}}. Many descriptions of the SIDDATA-dataset were translated using this technique\footnote{As, however, only 500.000 characters per google-account and month can be translated \href{https://cloud.google.com/translate/pricing}{free of charge}, the translation-process for the descriptions is still in progress.}. As of now, Google's Cloud Translation Service uses an embedding-based neural model of a hybrid architecture that has a transformer encoder, followed by an RNN decoder \cite{Chen2018}. All of the languages detected in the SIDDATA-dataset are supported by the system - translating between the languages German, English and Spanish, which make up \todoparagraph{HOWMANY} percent of the SIDDATA-descriptions, is what the system is particularly optimized for. 
\todoparagraph{write short about their percentage, bleu score etc}

\includeMD{pandoc_generated_latex/6_1_implementationdetails}

\subsection*{Candidate Extraction}

see \autoref{sec:extract_cands}

\subsubsection*{KeyBERT}
\label{ap:details_keybert}

The \emph{KeyBERT}-algorithm\footnote{\label{fnote:keybertgibhut}\fullcite{MaartenGr2021}} \cite{grootendorst2020keybert} is one of the techniques used to select phrases of the text-corpus as candidates for \gls{feature}-directions. 

KeyBERT is a keyword-extraction technique \q{that leverages BERT embeddings to create keywords and keyphrases that are most similar to a document}\footnoteref{fnote:keybertgibhut}. \Gls{bert} is a neural language representation model that is able to embed both words and documents. Its embeddings are obtained by training a multi-layer bidirectional transformer encoder \gls{ann} architecture on a task in which a masked word must be predicted from the its bidirectional context as well subsequent fine-tuning tasks \cite{Devlin2019}. To extract keywords, the KeyBERT algorithm embeds both the document as well as its containing \glspl{ngram} of a configurable length using BERT and returns those phrases whose embedding ist most similar to the document-embedding according to the cosine-similiarity\footnoteref{fnote:keybertgibhut}.

The KeyBERT-model was incorporated to extract key-phrases for this codebase in two ways: 

\paragraph{KeyBERT-original} runs the algorithm on the unprocessed original texts. This is reasonable, as this is what BERT-embeddings are trained on, however it has the disadvantage that it requires a lot of post-processing to match the extracted phrases to the processed descriptions (which \eg may contain only lemmas or have their \glspl{stopword} removed)
\paragraph{KeyBERT-preprocessed} alleviates this problems by running the algorithm on already preprocessed texts. This may however lead to worse results, as the algorithm was trained on unprocessed natural sentences.

In practice, though both variants extracted different phrases, the results for either of the technqiues did not differ significantly.


\subsection*{Candidate Filtering}
\label{ap:algo_filter}



\begin{figure}[H]
	\begin{center}
	  \includegraphics[width=\textwidth]{3dplot_hyperplane_and_orthogonal}
	  \caption[Visual representation of the Hyperplane of an SVM splitting a dataset]{ \label{fig:3d_hyperplane_ortho} Visual representation of the Hyperplane of a Support-Vector-Machine splitting a dataset, as well as it's orthogonal and the orthogonal projection of a set of samples onto the plane. For an interactive version of this plot, visit  {\small \url{https://nbviewer.org/github/cstenkamp/derive_conceptualspaces/blob/main/notebooks/text_referenced_plots/hyperplane_orthogonal_3d.ipynb?flush_cache}}}
	\end{center}
\end{figure}


\begin{table}[H]
    \centering
    \resizebox{\textwidth}{!}{%
    \begin{tabular}{llllll}
    \textbf{Long}                    & \textbf{Short} & \textbf{Data} & \textbf{Quantifications} & \textbf{Distances} & \textbf{Comments}                \\ \midrule
    rank2rank\_dense          & r2r-d          & all           & Dense-Ranked           & Dense-Ranked     &                                  \\
    rank2rank\_min            & r2r-min        & all           & Min-Ranked               & Dense-Ranked     &                                  \\
    bin2bin                   & b2b            & all           & Binary                   & Binary             & Disregards rankings              \\
    digitized &
      dig &
      all &
      Digitized &
      Digitized &
      \begin{tabular}[c]{@{}l@{}}Bins decided by np.histogram\_bin\_edges \\ from min and max of all data\end{tabular} \\
    count2rank\_onlypos       & c2r+           & positive      & Unchanged                & Dense-Ranked     & Only for Count as Quantification \\
    rank2rank\_onlypos\_dense & r2r+d          & positive      & Dense-Ranked           & Dense-Ranked     &                                  \\
    rank2rank\_onlypos\_min   & r2r+min        & positive      & Min-Ranked               & Min-Ranked         &                                  \\
    rank2rank\_onlypos\_max   & r2r+max        & positive      & Max-Ranked               & Max-Ranked         &                                  \\
    digitized\_onlypos\_1 &
      dig+1 &
      positive &
      Digitized &
      Digitized &
      \begin{tabular}[c]{@{}l@{}}Bins decided by np.histogram\_bin\_edges \\ from min and max of all data\end{tabular} \\
    digitized\_onlypos\_2 &
      dig+2 &
      positive &
      Digitized &
      Digitized &
      \begin{tabular}[c]{@{}l@{}}Bins decided by np.histogram\_bin\_edges \\ from min and max of all positive data\end{tabular}
    \end{tabular}%
    }
    \caption{Dense means: if there are 14.900 zeros, the next is a 1
    Min means: if there are 14.900 zeros, the next one is a 14.901
    Max means: if there are 14.900 zeros, they all get the label 14.900
    These scores are weighted}
    \label{tab:kappa_measures}
\end{table}





\subsection*{Faculty-Classifier}
\label{sec:faculty_classifier}

As one of the evaluations is to compare the results of classifiers based on the algorithm here with a powerful classification algorithm, a neural network that classifies the Faculty of a course in the Siddata-Dataset was also implemented. The implementation for that will not be elaborated upon except that it is available at \todoparagraph{Link to the repo}, it relies on sacred,\footnote{\todoparagraph{link to sacred, note that to get the results like I did you'll need a MongoDB in a docker container, see this link}} and that it uses Google's `universal-sentence-encoder-multilingual` in Version 3 (linear in textlength, thus managable time and space requirements) plus a few classification layers ontop. The encoder is trained \q{on a number of natural language prediction tasks that require modeling the meaning of word sequences rather than just individual words},\footnote{Quote from their description at \url{https://tfhub.dev/google/collections/universal-sentence-encoder/1} (accessed \date{2022}{03}{25})} aimed being the base for architectures for many NLP tasks through the usage of sentence embeddings \cite{Guo}. It was trained on with a train-test-split of 90/10 (the results being consitent through sampled cross-validation)


\todoparagraph{Another purpose of the classifier} is to check if it is anyhow possible to extract meaningful information from the descriptions: If it is possible to train a classifier on the data that can reasonably predict a qualitative feature, there is enough structure in the data such that the algorithm I'm about to produce can work. 

Also, we have a lower bound for useful data: we can just throw away data that cannot be classified!!

% #########################################################################################################################################################################################################################################################################################################################################################################################################################################################################################################################################################################################################################################################################################################################################################################################################################################################

\section{Used Software}

\includeMD{pandoc_generated_latex/6_2_usedsoftware}

% #########################################################################################################################################################################################################################################################################################################################################################################################################################################################################################################################################################################################################################################################################################################################################################################################################################################################

\section{Configurations to run \mainalgos}
\label{ap:yamls_for_origalgos}

% \vspace{-0.8cm}

% \lstinputlisting[language=, firstline=29]{codes/dqn.txt}

\subsection{\textcite{Derrac2015}}

\begin{lstlisting}[language=yaml, caption={YAML for \textcite{Derrac2015}}]
    pp_components:          mfautcsdp
    translate_policy:       translate
    quantification_measure: ppmi
    dissim_measure:         norm_ang_dist
    embed_algo:             mds
    embed_dimensions:       [20, 50, 100, 200]
    extraction_method:      pp_keybert
    max_ngram:              5                   
    dcm_quant_measure:      count
    classifier:             SVM
    kappa_weights:          quadratic
    classifier_succmetric:  [kappa_count2rank_onlypos, kappa_rank2rank_onlypos_min] 
    prim_lambda:            0.5
    sec_lambda:             0.1
    __perdataset__:
      placetypes:
        extraction_method:  all 
        pp_components:      none
\end{lstlisting}

\subsection{\textcite{Ager2018}}

\input{pandoc_generated_latex/ZZ_listingreplacement_ager}

\removeMe{
\begin{lstlisting}[language=yaml, caption={YAML for \textcite{Ager2018}}]
    max_ngram:              1
    classifier_succmetric:  [cohen_kappa, accuracy, ndcg]
    dcm_quant_measure:      ppmi    
\end{lstlisting}
}


\todo 

% \removeMe{

% \subsection{\textcite{Alshaikh2020}}

% \begin{lstlisting}[language=yaml, caption={YAML for \textcite{Alshaikh2020}}]
%     TODO: do
% \end{lstlisting}

% }
	% \restoregeometry %TODO do I need separate geometries for the appendices? Bc if I use the package I have to change how I set margins etc!


\printbibliography[heading=bibintoc]

\end{document}
