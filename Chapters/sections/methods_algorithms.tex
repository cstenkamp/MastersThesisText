Let us finally go into detail about the main algorithm. The implementation of this thesis replicates and extends the algorithm proposed by \textcite{Derrac2015} with some novel contributions to deal with the given dataset. Further, some improvements from the works of \textcite{Ager2018} and \textcite{Alshaikh2020} are incorporated, who also replicated and improved the original algorithm and published together with Prof. Steven Schokaert. According to our evaluation of the field, these two papers provide some useful improvements in several aspects, as they apply the algorithm to different datasets, suggest more straight-forward ways of evaluating their performance, and help in understanding important concepts. That we are focusing on only those three papers should by no means imply that they are the only ones that were considered and influcenced this implementation,\footnote{See \autoref{sec:otherwork}} however in contrast to the other pertinent literature these two works do not substantially divert from the algorithm's core principles.
% see \autoref{sec:otherwork} - \cite{VISR12} and their tag genome, the fact that the algorithm detailed here is basically only one step in \cite{Alshaikh2019}, Gärdenfors himself suggested that one may use self-organizing maps instead of classical AI/NLP algorithms.

It is important to keep in mind that the algorithm is no rigid monolith but modularly consists of several components, such as \textit{dimensionality reduction}. Many of these components do not require specific algorithms, and \mainalgos also experiment with different components. The exact system for these components may be exchanged, and in the following these exchangeable algorithms are also referred to as hyperparameters. Note further that while this thesis mostly replicates the work of \textcite{Derrac2015}, the following will describe the algorithm as implemented here, which differs in some details from the original work. For the sake of clarity, very specific implementation details will be left out in the following description, as however reproducibility is an important aim for us, implementation details are available in Appendix~\ref{AppendixB} and linked where relevant. %\todoparagraph{Further, there is a table that compares the implementations here and in mainalgos, as well as another table what-configs-are-available where, the config-yamls and a section on what-other-things-could-one-have-done-hereandthere.}

\subsection*{Core Algorithm}

%\todoparagraph{The following explanation assumes that we accept some things as given. For now we'll do that, but we will later revisit and critically question many of these assumptions!}

% The core idea of the algorithm is to unsupervisedly find a a set of features which can be modelled as directions for a vector-space representation of the respetive entities.
The main goal of the algorithm is to unsupervisedly use text-corpora associated with the considered \glspl{entity}\footnote{From now on, the term \textit{\glspl{entity}} refers to the sample described by one text from the corpus (description, concatenated reviews, ...) from a certain domain. The corpus accordingly defines the domain: educational resources, movies, ...} to embed these into a vector-space where the axes correspond to human concepts/properties.\footnote{\textit{Concepts} and \textit{Properties} explicitly refer to what is defined in Criterions C and P, see \ref{sec:csdefinition}} This is referred to as \textit{feature-based} representation: A high-dimensional vector that numerically encodes the degree (\textit{protoypicality}) to which the entity corresponds to a number of appropriate dimensions. This is generally referred to as Conceptual Space and can be used as basis for explainable reasoning.

The general idea to achieve that is as follows: First, the entities are embedded as fixed-dimensional vectors. To allow for the types of reasoning from \autoref{sec:cs_reasoning}, it is embedded into metric spaces where the concepts of direction and distance are well-defined. \gencite{Derrac2015} original algorithm uses MDS (see \ref{sec:mds}) for this matter, which enforces metric distances. \cite{Ager2018,Alshaikh2020} both soften this requirement and also use document embedding techniques such as \gls{doc2vec} and averaged GloVe \cite{pennington2014glove} embeddings.

Additionally, words or phrases from the text are extracted as candidates for the names of the semantic dimensions. The underlying assumption is that \q{words describing semantically meaningful features can be identified by learning for each candidate word $w$ a linear classifier which separates the embeddings of entities that have $w$ in their description from the others} \cite[3574]{Alshaikh2020}: The better the performance of that classifier according to a chosen metric, the more evidence there is that $w$ describes a semantically meaningful feature. 
% * from Alshaikh2020: "Their core assumption is that words describing semantically meaningful features can be identified by learning for each candi- date word w a linear classifier which separates the embeddings of entities that have w in their description from the others. The performance of the classifier for w then tells us to what extent w describes a semantically meaningful feature"
In a final step, the candidate-words are clustered according to their similarity to find a fixed set of \emph{semantic directions}. A representative term for the directions is selected as dimension name, and the entities are re-embedded into a new space comprised of these dimensions, where the individual vector-components correspond to the ranking of an entity with respect to these dimensions.

The rest of this section goes into further detail for each of the individual algorithm components. \removeMe{An overview of which of the considered literature supports each components is given in \autoref{tab:compared_algos}.} Further, configuration files to enable exactly the respective components of the papers \mainalgos for the codebase of this thesis are listed in \aref{ap:yamls_for_origalgos}.

% \todoparagraph{but before that, ager and alshaikh}

% \removeMe{
% \subsection{Regarding ager and alshaikh}

% \todoparagraph{describe shortly what the improvements from [2,3] were}

% \todoparagraph{Dass die den Zwischenschritt mit dem ganzen geeometric reasoning auf dem interim space nicht machen und DESWEGEN die requirement mit MDS soften konnen}

% In principle Derrac2015, but with some components from Ager2018 and Alshaikh2020 as well as some own stuff. We will testi some claims or nonclaims of \mainalgos, bspw nutzen sie immer PPMI ohne je tf-idf zu testen. Also of course different nature of the dataset - their "how does this dimension correspond to the count in the reviews" doesn't make sense (their success-metric for the SVM is tailored to the one property, so we expect that one to be worse). We will elaborate on different ways to deal with that later.
% }

\subsection{Algorithm Steps}
\label{sec:algorithm_steps}

% The core idea of the algorithm is to (unsupervised, data-driven) find a a set of features which can be modelled as directions for a vector-space representation of the respective entities. For that, the steps are:

This section describes the steps how to create an interpretable vector-space from the text corpus in detail. For that, we will explicitly elaborate on the parameter choices that branch up at every step for this specific implementation. %Note that absolutely is a combinatorical explosion it is impossible to try out all. %Further, this is about how this specific implementation does it, which may differ in some details from \mainalgos.

\label{sec:algorithmsteps}
\begin{enumerate}
	\item[\saveref{sec:algo_preproc}{1.}] \textbf{Preprocess} the corpus with default techniques and create a \textit{Bag-of-ngrams representation} (\ref{sec:techniques:bow}) of the texts.
	\item[\saveref{sec:extract_cands}{2.}] \textbf{Extract Candidate Feature Names} from words/\glspl{ngram} of the corpus.
	\item[\saveref{sec:generate_vectorspaces}{3.}] \textbf{Embed all Entities} into a fixed-dimensional vector space with demanded properties that captures the respective semantics.
	\item[\saveref{sec:svm_filter_cands}{4.}] \textbf{Filter Candiate Features} by training a linear classifier for each candidate that seperates the vector representations of the entities that contain the term from those that do not. If a specified metric for this classifier is sufficiently high, assume that the candidate term captures a \textit{salient} feature - its direction is then characterized by the orthogonal of the classifier's separatrix.
	\item[\saveref{sec:algo:cluster}{5.}] \textbf{Cluster/Merge the candidates} and calculate the feature direction for each cluster from its components, and (optionally) find a representative cluster-name.
	\item[\saveref{sec:algo:postprocess}{6.}] (optionally) \textbf{Post-process} the candidate-clusters.
	\item[\saveref{sec:algo:reembed}{7.}] \textbf{Re-embed the entities} into a space of semantic directions by calculating their distance to each of the feature direction separatrices.	
\end{enumerate}

This techniques first embeds the collection of texts into a  vector space, to afterwards extract important features from this space using linear classifiers. The second step is an original idea of \cite{Derrac2015}, however creating vector space embeddings from texts is a very popular technique, used for many tasks in \gls{nlp} \cite{Mikolov:Regularities,Mikolov2013a,Guo,Lowe,Turney2010}. This implementation relies on classical creation of the \gls{vsm}, for which the general creation process was explained in \autoref{sec:vsm_construction}. The steps \textit{Build the Frequency Matrix}, \textit{Transform Raw Frequency Counts} and \textit{Smooth the Frequency Matrix} are squashed into the preprocessing and embedding of entities. When \textit{extracing candidate features}, their frequencies must additionally be quantified - which may differ from the quantification when \textit{embedding all entities}.

An explicit and simple implementation compliant with each step could be a simple word tokenization and count to generate a bag-of-words (step 1) where each sufficiently frequent word is used as candidate (step 2). A \gls{dissimmat} of the individual \gls{bow}-vectors is compressed using MDS (step 3). A \gls{svm} calculates the accuracy for each candidate (step 4), and k-means-clustering on the 500 top-scoring terms subsequently creates 100 clusters and averages their directions (step 5). The distance to each of the hyperplanes is calculated (step 6), yielding new space for the entities. The sequence of steps is also given as pseudocode in \autoref{ap:algorithm_pseudo}. 
 
 The distinction of steps is not always this rigid: Instead of creating a dissimilarity matrix followed by dimensionality reduction, \cite{Ager2018,Alshaikh2020} use neural word or document embeddings.\footnote{see \autoref{sec:embeddings}} %Instead of extracing candidates from corpus tokens and training a linear classifier for each of them and use their orthogonal as direction, techniques such as LSA or LDA can be employed to find topic vectors directly. 
 We will come back to other ideas ideas when discussing future research opportunities (\autoref{sec:futurework}) by listing what other ways of fulfilling each respective step could have been considered.

\autoref{fig:dependency_graph} shows an automatically exported dependency-graph, displaying the individual steps of the algorithm as done in the accompaning code, also showing where selected important parameters are first used. As explained in \autoref{sec:architecture}, the modularity of the provided architecture allows individual components to be exchanged as needed and run in parallel.


\begin{figure}[h]
	\begin{center}
	  \includegraphics[width=0.9\textwidth]{dependency_graph.pdf}
	  \caption[Dependency-graph of the implementation]{Dependency-graph of the implementation, displaying the individual steps of the algorithm as well as their dependencies and where selected important parameters are first used.}
	  \label{fig:dependency_graph} % \todoparagraph{Generated using command XYZ}
	\end{center}
\end{figure}


\subsubsection{Preprocessing\arrowref{sec:algorithmsteps}}

\label{sec:algo_preproc}

Before looking at the steps in turn, it should be noted that even the preprocessing does not work on completely raw data, but on curated and processed corpora. This processing is however not considered part of the algorithm, as it is very specific to the respective datasets and manual dataset exploration, tweaking settings such that they are best for each corpus separately. The preprocessing for the Siddata-dataset is described in \autoref{sec:dataset_siddata} and its implementation is done in separate Jupyter Notebooks.\footnote{Such as \url{https://github.com/cstenkamp/derive_conceptualspaces/blob/main/notebooks/create_datasets/Preprocess_Siddata2022.ipynb}} In the considered literature, the preprocessing is not considered part of the algorithm at all. Their implementations start from already fully processed datasets available as bag-of-words, each separately processed. Details of their individual processing per dataset is listed in \autoref{tab:all_datasets}. By incorporating the preprocessing into the pipeline, this work aims to increase adaptability and reproducibility, and also allows to experiment with different techniques such as translation or lemmatization or how duplicate entities with different associated texts are merged.
% In course-descriptions, I want some parts of the pre-preprocessing be part of the pipeline, like how we merge descriptions of different iterations of the same course that overlap to a high degree (sentwise-merge vs relative-term-frequencies)


A common prerequisit for NLP algorithms is to pre-process the text corpus. The preprocessing itself consists of multiple independent components chained after each other. Which components are necessary also depends on the processed dataset - as for example the \emph{placetypes}-dataset consists of a collection of \textit{tags} instead of full sentences, tokenising sentences or removing \glspl{stopword} becomes irrelevant. Other datasets may require additional cleaning or are already available in preprocessed form.

\paragraph{Translation} As the main considered dataset of university-courses is highly multilingual (see \autoref{fig:sid_statistics}), one of the first questions that needs to be addressed is how entities of different languages are handled. The algorithm consists of classical language processing algorithms such as comparing \gls{bow} representation of the entities, which means that the same text in two different languages may result in maximally different representations (see \autoref{sec:techniques:bow}). Because of this, before any other processing, the languages of each entity is checked, such that those of languages other than the demanded may be either translated, left out or used anyway. For details about the translation, it is referred to Appendix \ref{ap:translating}.\footnote{It should be noted that professional automatic translation is costly and thus not all texts are available in all languages.}

\paragraph{Components} The following components are developed for the preprocessing, every one of which can be individually enabled or disabled:

\begin{itemize}
	\item Prepend title and/or subtitle to the entities' associated text \itemtext{useful for the Siddata-Dataset, as the titles are often quite long and more descriptive than their descriptions}
	\item Remove HTML-Tags from texts 
	\itemtext{useful for the Siddata-dataset, as it includes descriptions for \glspl{mooc} which are crawled from websites and often contain such}
	\item Tokenise sentences 
	\itemtext{such that \glspl{ngram} across sentences are not considered}
	\item Lower-case all words
	\itemtext{reduces the amount of individual words and ensures that words at the beginning of sentences are mapped correctly}
	\item Remove stop-words / frequent phrases
	\item Tokenise words
	\itemtext{means splitting at the word-boundary, resulting in a list of words. Order must be kept in case n-grams are to be extracted.}
	\item \Gls{lemma}tize words
	\item Remove diacritics
	\itemtext{\emph{Diacritics} are glyphs added to basic letters, such as accents or German \emph{Umlaute}. Removing them converts for example the letter \emph{ä} to an \emph{a}}
	\item Remove punctuation 
\end{itemize}

The above can be done either be done with proprietary code for all of these steps,\footnote{Mostly relying on the python package \emph{nltk} \cite{bird2009natural}} or using \codeother{sklearn}\footnote{\url{https://scikit-learn.org/stable/}}s \codeother{CountVectorizer} (which is faster, but less configurable), as \cite{Ager2018} claim to have done.

\paragraph{On Stop-Words}
Removing stop-words from the texts is useful because it makes the resulting frequency more compact and thus less computationally intensive, and stop-words generally have very discriminative power, meaning their occurence among the entities is arbitrary, just making hte emeddings more noisy (cf. \autoref{sec:word_count_techniques}). There are however reasons to not remove them: Two words that are considered stop-words may posess relevant semantic content (such as a \textsc{Fällt aus} in a course title), and stopwordslists are often incomplete and of low quality \cite{nothman-etal-2018-stop}. For these reasons it is also possible to instead remove \glspl{ngram} that exceeded a certain frequency (\gls{df}).

\paragraph{On Lemmatization}
The languages most prevalent in the considered datasets are considered \textit{agglomerative}, which means word stems are changed by the addition of affixes and suffixes. Consequently, the same word may be present in multiple different forms, which modelled as completely dissimilar vectors in the present \glspl{bow}-approach. Lemmatization is the process of mapping different forms of these words onto the same stem. Considering that the Siddata-dataset consists of far fewer words than the others, this has important implications. For the german descriptions, this implementation relies on the \textit{HanTa} lemmatizer \cite{Wartena2020}.\footnote{\url{https://textmining.wp.hs-hannover.de/Preprocessing.html\#Lemmatisierung}}

The result of this step is a bag-of-ngrams representation for each entity (see \autoref{sec:techniques:bow}).


\subsubsection{Extract Candidates\arrowref{sec:algorithmsteps}}
\label{sec:extract_cands}
% Section 4.2.1 of Derrac2015

The final result of the algorithm is a metric space in which the individual dimensions (\emph{features}/\emph{interpretable directions}) correspond to natural-language concepts and attributes. The candidates for these features are verbatim phrases extracted from the text-corpus of the \glspl{entity}, which are subsequently filtered and merged as necessary.

In \gencite{Derrac2015} work, the selection of phrases to be extracted depends on the dataset: For placetypes-dataset, all sufficiently frequent\footnote{\label{fnote:cand_thresholds}The respective thresholds are listed in \autoref{tab:all_datasets} as ``candidate word threshold''.} 1-grams\footnote{Note that in the case of the placetypes dataset, a 1-gram corresponds to all merged words of a tag.} were considered. For the other two datasets, they applied a \gls{pos}-tagger that extracted all sufficiently frequent\footnoteref{fnote:cand_thresholds} \textbf{adjectives, nouns, adjective phrases} and \textbf{noun phrases}, assuming that adjectives would correspond to gradual properties (\eg \textit{violent, funny}) and nouns to topics (\eg the \textit{genre}) \cite[Sec. 4.2.1]{Derrac2015}. Also, the authors ensured that the number of extracted candidates for both datasets is roughly equal, getting around 20\,000 candidates for movies and placetypes.

% Their method depended on the dataset - as their placetypes-dataset was just a collection of tags and the number of tags with term-freq >= XYZ (docfreq>2?! hä?) corresponded to their desired number of candidates anyway (around 22k), they just took all of these as candidates. For their movie-reviews-dataset, they considered all nouns, adjectives, nounphrases 	and adjective-phrases as detected by a POS-tagger. Doing something similar in the scope of this thesis led to suboptimal results, which is why alternative methods were developed
For this step, the implementation of this thesis differs from the original algorithm, as both taking all words as candidate and running a \gls{pos}-tagger led to suboptimal results in previous experiments, which indicated that the robustness of the algorithm is increased if less candidates are considered in earlier steps. This will be further argued and elaborated in the discussion. To ensure comparability to these works however, in the case of the placetypes-dataset the original method of taking all words with a term-frequency of at least 50 was used. Similar techniques for the Siddata-dataset were also considered, but in constrast to the placetypes-dataset it is also important to consider various-length n-grams. 

% \todoparagraph{thing is I have less words but the algorithm seems to profit from less words as that makes it more robust}
% I would however argue that the difference here doesn't make a relevant difference 

In our implementation the candidate-extraction is split into four subsequently excecuted substeps, because depending on the algorithm used to extract the candidates the runtime of the individual components is comparably long and some settings are only relevant in later substeps. The steps are:
\begin{itemize}
	\item Extracting Candidate Terms
	\item Postprocessing the Candidates
	\item Creating the \gls{doctermmat} for the candidates and applying a \gls{quant}
\end{itemize}

As visualised in \autoref{fig:dependency_graph}, these substeps only depend on the preprocessed descriptions, which means they can be run in parallel to the creation of the embedding if \eg running on compute clusters.

% This can be done either based on the frequency (meaning all terms with a minimal term-frequency), based on some notion of *importance* (based on scores like tf-idf or ppmi), or by more complex means of figuring out *important* keywords and keyphrases. An example of the latter would be KeyBERT
Three main techniques are implemented to extract candidates from the text-corpus. Irrespective of the algorithm, only words with a sufficiently high \gls{df} are extracted, which is important to ensure that the classifier that determines its meaningfulness has enough samples in both clases. This means that the minimal freqeuncy can be calculated from the dataset size: In \cite{Derrac2015}, the minimal frequency for the movies-dataset with 15\,000 entities was only 100, meaning that the algorithm even works if only 0.6\% of samples are in the positive class. 


\begin{description}
	\item[By frequency:] consider all phrases that exceed a specified document-frequency (like \cite{Derrac2015}).
	\item[By a \gls{quant}:] consider all phrases that are prominent by some notion of \textit{importance} , such as the \gls{ppmi} or \gls{tf-idf}-score. Note that the respective scores depend on the combination of document and term, such that candidates may be extracted for some documents. Of course, all their occurences are considered in the creation of the frequncy matrix.
	\item[Using \emph{KeyBERT}\cite{grootendorst2020keybert}:] consider phrases whose BERT-embedding \cite{Devlin2019} is most similar to the text they are in. 
\end{description}

Using KeyBERT results in candidate terms that are most appealing in qualitative inspection, however it is also most computational demanding, techniques and requires substantial amounts of post-processing for the resulting phrases. More details on KeyBERT and how it is incorporated into the algorithm are given in the implementation are given in Appendix~\ref{ap:details_keybert}

Finally, a \gls{doctermmat} is created from the postprocessed candidates, containing the frequency for each candidate-phrase in each entity. The creation of this frequency matrix mirrors the process described in \autoref{sec:vsm_construction}, however only for the extracted words. After filtering this matrix to ensure that only candidates with a minimal \gls{df} or \textit{stf} are considered, a quantification is applied as described in \autoref{sec:word_count_techniques}. Available Quantifications include raw count, binarization\footnote{meaning all counts are either one or zero. According to \textcite{Alshaikh2020} this improves performance, however we could not confirm this.}, tf-idf or PPMI. We expect that expressing the relation of terms and documents by something else than the raw count will prove especailly useful for the Siddata-dataset: As it does not consist of concatenated reviews but short descriptions, each individual word will only occur a few times per description. Because of that, a \gls{rank} induced from the count will be not very meaningful when later comparing it to the ranking induced by the classifier.

\subsubsection{Generating Vector Space Embeddings\arrowref{sec:algorithmsteps}}
\label{sec:generate_vectorspaces}

In this step, the individual \glspl{entity} are embedded into a fixed-dimensional vector space, making up a \emph{frequency matrix}.\footnote{The previous step already created a frequency matrix that encodes the relevance of a candidate-phrase for each entity- this is a spearate one that later serves to encode each document as a vector} Neither this nor the frequency matrix from the previous step will used to finally calculate similarities on, but both are interim results to get the dimensions necessary for these similarities.

% Importantly, while this matrix is a \gls{doctermmat}, it is only an interim result in the algorithm and the calculation of distances and directions will be done on another matrix from a later step. %This is where our pipeline starts to diverge from what the pipeline specified in \nameparanref{sec:vsm_construction}.  in the previous step, and in this step we create another one that encodes each document as a vector. 

Embedding words, \glspl{ngram}/phrases or other tokens, as depicted by \cite{Turney2010,Lowe}, generally involves counting the token frequencies, transforming them to get relative frequencies, and performing dimensionality reduction on the resulting matrix.

So far (step 1), we have counted the token frequencies. \textcite{Derrac2015} argued that this space must be Euclidean, such that geometric/algebraic solutions correspond to commonsense reasoning tasks (see \autoref{sec:reasoning}). Another requirement is that the number of dimensions is be chosen as hyperparameter to the algorithm to be able to find a compromise between compression and expressive power. Because of these two requirements, they selected \gls{mds} for dimensionality reduction.

As stated in \autoref{sec:mds}, \gls{mds} calculcates a Euclidean \gls{vsm} from a set of pairwise distances. This means that the algorithm first creates a \textit{Dissimilarity Matrix} that encodes the distance between all pairs of entities (represented as Bag-of-ngrams representation), from which subsequently the final embedding is generated. 

In this implementation, the dissimilarity matrix is created using distance metrics for the quantified bags-of-ngrams of the respective entities. Note that this quantifiation does not need to be the same one as the one used for quantifying the relatedness of the extracted candidates and the documents.\footnote{As will be shown in the result, doing so sometimes even outperformed other configurations.}

\paragraph{Document embeddings}
If the strict requirement for a metric space is dropped however, many different algorithms may instead be used at this point - not only different dimensionality reduction methods for the embedding, but also ones that do not rely on the distance matrix or even the \gls{bow} at all, like document-embedding-techniques such as \gls{doc2vec} \cite{Le2014} (as \eg used by \cite{Alshaikh2020}). This would not require the count of token frequencies followed by a \gls{quant}, but not affect the rest of the algorithm. As, however, document embeddings tend to be created on the basis of \gls{cos}, this will not result in a Euclidean metric. \textcite{Alshaikh2020} experimented with this, however it performed worse than the original algorithm relying on MDS (see \autoref{tab:f1_placetypes_long}), which is why this technique was not futher pursued in this work.

\paragraph{Classical Processing}
The default way of doing it is, however, to create frequency matrix counting the token frequencies of each word of an entity and using a \gls{quant} to transform these raw counts. In this implementation, the \gls{dissimmat} that is required for the MDS-algorithm created on basis of the \emph{normalized angular distances} of the \glspl{bow} of the respective entities, which is relates to the \gls{cos} and is defined by \textcite{Derrac2015} as:

\vspace{-5ex}
\begin{align}
	ang(e_i, e_j) &= \frac{2}{\pi} * \arccos \left( \frac{\vec[m]{v_{e_i}} * \vec[m]{v_{e_j}}} { \lVert \vec[m]{v_{e_i}} \rVert * \lVert \vec[m]{v_{e_j}} \rVert }  \right)  \label{eq:norm_ang_dist} \\
	&= \frac{2}{\pi} * \arccos(1-\cos(\vec[m]{v_{e_i}},\vec[m]{v_{e_j}})) \nonumber
\end{align}

Finally, the dimensionality of the frequency matrix is reduced using the MDS-algorithm described in \autoref{sec:mds}. \cite{Derrac2015} also experiment with the \textit{Isomap} algorithm, which does not yield a Euclidean space but one of geodesic distances. As the results for their implementation are, however, of worse quality than those for MDS, it is not considered further in this work. 

In the publication of \textcite{Derrac2015}, the result of this embedding is used to experiment with the explainable classifiers (discussed in \autoref{sec:reasoning}). Due to the usage of MDS, this space has clearly defined concepts of \textit{betweeness} and \textit{parallesism} between the entities, however it is important to stress that it does not have interpretable dimensions and is, so far, a plain \gls{vsm} as described in \autoref{sec:vsm_construction}. 


\subsubsection{Filter Candidates by Classifier Performance\arrowref{sec:algorithmsteps}}
\label{sec:svm_filter_cands}

This step brings together the entity embeddings and the extracted keyphrases, to finally find semantically meaningful directions in the space the entities are embedded in. For that, classifiers are trained for all previously extracted candidates, each yielding a \textit{Candidate Feature Direction}, which are subsequently filtered to detect those corresponding to semantically meaningful features. The algorithm is executed separately for each candidate term and looks as follows:

All entities are split into two classes: Those that contain the respectively handled candidate term and those that do not. To quantify how well each candidate word captures semantic content of the entities, a linear classifier is trained to separate the embeddings of these two classes. The best example for such a classifier is a \gls{svm} which does not rely on the kernel trick. As visually exemplified in \autoref{fig:3d_hyperplane_ortho}, the result is a hyperplane that best divides the positive from the negative samples. Regardless of the dimensionality of the original space, this hyperplane has a one-dimensional orthogonal vector. Each of the entity-embeddings is subsequently orthogonally projected onto it. The distance of this projection to the plane offset\footnote{The coordinate where it crosses the decision surface} is a scalar that encodes the distance to this decision hyperplane. 

This distance encodes \textit{how much} this point is considered to be protoypical of the respective class by the classifier, for the positive as well as the negative class. Given that the original space is created based on similarity measures of the entities, the \textit{Bag-Words-Hypothesis} (\autoref{sec:bow_hypothesis}) states that similar words should have similar words - meaning maximally dissimilar entities are far apart from positive ones.\footnote{To stick with the example of movies, the assumption is that movies that are maximally unscary are maximally far from the away from scary ones: An entity that has a maximally dissimilar distribution of words than those that are \textit{scary} means a maximally maximally dissimilar movie, as little of its \textit{latent topics} (see \autoref{sec:lsi}) agrees with the scary ones: The more dissimilar, the less scary.} Because of this, the hyperplane's orthogonal can be considered an axis that encodes how much each entity agrees with the feature that is the basis of the classification. Accordingly, a ranking of the entities in terms of this distance should encode their degree of having the respective feature.

According to \textcite{Derrac2015}, the performance of this classification encodes how good a candidate serves as feature direction. This can be explained like this: As discussed earlier (\autoref{sec:word_count_techniques}), unimportant words are arbitrarily throughout the corpus. Due to distibutional semantics, topics that are very prominent in some texts but not in others will influence the position of a texts' embedding in the vector space more than \textit{unimportant} words, which do not signify a latent topic of the corpus. This means, they do not correlate with other words and do not explain much of the dissimilarity of the embeddings. Accordingly, unimportant words do not influence the positions of the entities' embedding other than being \textit{noise}. This randomness does not go along with a cluster of positions in any of the dimensions, as noise gets removed in process of dimensionality reduction. This makes it reasonable to assume that the a classifier can split entities that contain a word from those that do not, the more the word is an \textit{important topic} in the sense that it explains the dissimilarity in the entities.

Because of this, \cite{Derrac2015} henceforth only consider those terms as \textit{faithful directions} that explain a lot for variance in the data whose performance exceeds a certain threshold.\footnote{\q{if this classifier is sufficiently accurate, it must mean that whether word w relates to object o (i.e. whether it is used in the description of o) is important enough to affect the semantic space representation of o. In such a case, it seems reasonable to assume that w describes an important feature for the given domain.} \cite{Ager2018}}

Concretely, the score used by \cite{Derrac2015} to assess the performance is not the accuracy or some other measure of the binary performance of the classifer, but rather if the ranking induced by the classifier corresponds to ranking of number of occurences\footnote{or the PPMI-score, the authers are imprecise in their wording - we will elaborate more on that later} of that word: The more these agree, the more we consider this direction \q{to be a faithful representation of the term} \cite[20]{Derrac2015}. The reasoning behind that becomes especially clear when considering the root of their datasets - in the case of reviews or tags it is the case that the more often a word is mentioned, the more relevant the word is for that entity. In the case of using a \gls{quant} such as the \gls{ppmi}-score, this instead becomes: The more salient relevant for this entity but not for the others the word is, the higher the score.

A score function compares these rankings, such that only those terms where the correspondance of these rankings exceeds a certain threshold are considered as candidate directions henceforth. For that, \textcite{Derrac2015} say that they use the Cohen's kappa, which is a metric to compare rankings that can deal with highly imbalanced data. Considering that for most candidates, most of the entities will be in the negative class this makes sense.\footnote{\textcite{Ager2018} compare the kappa-score to accuracy and argue that accuracy works better than the kappa-scores.} Unfortunately, the authors do not give many details on how this scoring is implemented. While \cite{Ager2018,Alshaikh2020} explicitly say that they are interested in the PPMI-scores\footnote{Though the uploaded code of \cite{Alshaikh2020} does not compare rankings but raw values}, from \cite{Derrac2015} it is not even clear if they take the count or the PPMI-score. As that is highly relevant, we try many different ways of this scoring and report them in the results. We also compare the overlaps of different kappa-scores to check if the choice is as imporant as we think it is. Which scores we used and how they are written here is listed in the implementation details: \autoref{tab:kappa_measures}.

Subsequently, only those candidates with high enough scores are considered, and the othogonals of the respective classifiers considered their respective \textit{candidate feature direction}, encoding the degree how much an entity corresponds to this feature. The number of features with a sufficiently high score can also serve as estimate of how good the parameter-combination so far was: if not enough well-scoring features were extracted, the final embedding which is created through only a subset of these features does not explain much of the original variability in the data.

% \todoparagraph{Tpower0.5 etc in den Glossary!}

\subsubsection{Merging the extracted candidate-directions\arrowref{sec:algorithmsteps}}
\label{sec:algo:cluster}

The previous step yielded many \textit{basic feature directions} that are defined as direction of the orthogonal vector for the hyperplanes splitting each individual candidate \gls{ngram}. The performance-thresholds are set such that many more directions are generated than the demanded dimansionality of the final embedding, such that they must be clustered and merged.

This is done via the following substeps, each of whch will be closer eloaborated:

\begin{itemize}
	\item Cluster good-performing candidates by their similarity
	\item (optional) Remove uninformative features
	\item Recalulate the direction of the cluster
	\item (optional) Find a representative name for the cluster
\end{itemize}

\paragraph{Clustering the candidates}

Clustering refers to an unsupervised algorithm that groups items based on some notion of similarity. In our case the assumption is that semantically similar concepts have \textit{close} vectors, which is given due to the \gls{bow}-hypothesis that states that the underlying structure of our dataset is expressed by the usage of related words (\autoref{sec:bow_hypothesis}, \ref{sec:lsi}).\footnote{In case of the Siddata-dataset, it may mean that in courses that contain the word \textit{computer} have a high chance of also containing \textit{program}.} As these vectors in principle only encode a direction, their similarity can be calculated by their \gls{cos}.

The clustering should reduce the number of features and also ensure that the resulting directions are different enough. Note that unlike \eg in Principal Component Analysis (PCA), the suggested here techniques do not enforce orthogonality, such that the resulting directions may remain linearly dependent to a certain degree. As in the final embedding only the projection onto those directions is relevant, it must be ensured that enough of the data's original variation is covered by these directions. To ensure that, we follow \gencite{Derrac2015} suggestion to allow for redundancy by extracting twice as many directions than the original \gls{vsm} dimensionality. 

The following steps describe the implementation of the original clustering method of \textcite{Derrac2015}, also used in our work:

First, we consider the best \textit{basic features} as \textit{main directions}. For that, we select one of the scores calculated in the previous step and select all candidates that exceed a threshold (\cite{Derrac2015} suggest $\kappa \geq 0.5$).
To get the directions, we follow the following algoritm:

\vspace{-1ex}
\begingroup
\verbatimfont{\footnotesize}%
\begin{verbatim}
	greats = filter(candidates, 0.5)
	directions = greats.argmax()
	for nterm in range(ndims*2):
	  greats = set(greats)-set(directions)
	  distances = {cand: min(comparer(cand, compareto) 
	                for compareto in directions) 
	                  for cand in greats}
		directions.append(compares.argmax)
\end{verbatim}
\endgroup
\vspace{-1ex}

This starts with the best candidate and then iteratively adds the one from the set of top-scoring candidates that most dissimilar to the set of final directions. The result is a set of $ndims*2$ main directions, which are henceforth considered the Cluster centers. Subsequently, all leftover terms from $T^{0.5}$ as well as all terms from $T^{0.5}$ are added to the respective cluster whose direction they are most similar to. 

\textcite{Derrac2015} used the \gls{cos} to measure the respective similarities. This may lead to unexpected situations (discussed in \autoref{sec:discuss_points}). As alternative similarity metric that does not rely on the angle between their vectors, \cite{Alshaikh2019} suggest to use the overlap of the positive-samples of two features as similarity. This was however not yet implemented in this thesis.

Alternatively, to the described algorithm, is also possible to use the popular \textit{k-means} algorithm for clustering, as done by \cite{Ager2018}. We do not present results for this approach here however, as it lead to a substantial increase in runtime, without affecting performance much. In the development we also noticed that many clusters contain a lot of irrelevant terms. To alleviate this, we experimented with different techniques, for example setting minimal similarity thershold that must be given for a term to be added to a cluster, however so far no formal evaluation to test how this affects performance was performed.


\paragraph{Find Cluster-Direction}

So far, we have a set of clustered canidates terms, each of which has an individual direction. The final \textit{feature-direction} must subsequently be found from the elements of the cluster. For that, \cite{Derrac2015} and \cite{Ager2018} define the cluster centroids as the average of all (normalized) vectors per cluster. In our experiments, however, we noticed that the final direction tends to be too much affected by irrelevant cluster-elements. Because of this, we experimented with other techniques to determine the cluster direction. Other considered methods include \eg to 1. just consider the direction of the cluster-center, or 2. to weight the influcence of each cluster-element by their kappa-score.

The best performaning method however was the \textit{reclassify}-algorithm, which (similar to \cite{Alshaikh2020}) finds the cluster-direction by training a new classifier that splits those \glspl{entity} that contain \textit{any of the elements} from the cluster from those that do not, analogous to the previous step.\footnote{except that it requires to generate and quantify a new frequency matrix from the sums of the individual counts.}. Doing this however often leads to the opposite problem than the previous step, namely that for many clusters there are almost no entities that do not contain at least one of the cluster elements. To counter this, we instead trained a classifier to split the 30\% of entities with the highest \glspl{quant} from the 30\% of entities with the lowest \glspl{quant}. A comparison of this algorithm with the method of \cite{Derrac2015} is given in the Appendix as \autoref{tab:text_per_dim}. As \cite{Alshaikh2020} already performed formal experiments with this that have shown its superior performance, all generated results of this work rely on this algorithm. 

\paragraph{Bad Clusters}

After these steps, we finally have the vectors that correspond to semantic directions. However, as there were still clusters of uninformative terms, \textcite{Alshaikh2020} have an additional step to remove uninformative cluster. As this bases on another clustering algorithm used by the author (namely \textit{affinity propagation}) which does not allow to specify the number of clusters, it was not implemented in the scope of this thesis.

\paragraph{Find a representative Cluster Name}

An important advantage of the clustering process is that it makes the extracted directions more \textit{descriptive} due to the fact that they are described by several phrases instead of only one. However, it may be helpful for an attractive user interface to find the single \textit{best} description of the cluster direction by its element.

An analysis of \cite{Carmel2009} showed that a statistical method to extract features from clustered text corpora identified the labels of human annotators as one of the top five most important terms in only 15\% of cases, implying \q{that human labels are not necessarily significant from a statistical perspective} \cite[139]{Carmel2009}. In their paper, they suggest several methods to find one representative name for the cluster. 

\textcite{Derrac2015} and its follow-ups \cite{Ager2018,Alshaikh2020} did not care about such methods and instead use either the name of the cluster center as its description or the cluster center plus two other sample elements. This work experimented with several techniques to get a more representative direction name. One of these techniques used the KeyBERT-algorithm (see \autoref{ap:details_keybert}) to find the term that is most similar to the set of terms making up the cluster. We also experimented with a method that embeds the cluster terms using \gls{word2vec} and returns  the word behind the vector that is closest to their average, which is not neccessarily part of the original set of words. Similarly to \gls{lsa} (\autoref{sec:lsi}), it is also possible to consider the entity whose \textit{pseudo-document} embeddings is closest in direction to the cluster direction.

The best technique to find a cluster-name was not evaluated yet. All considered methods that formally evaluate the corresponding feature-directions work independently of the actual cluster name. This is unfortunate, because \textit{subjectively}, the name of the respective directions is very important for the usability of any recommendation engine based on this work. Especially this subjectivity indicates that the only way to evaluate the cluster-names is with a study of human subjects.

% \todoparagraph{Kappa in den Glossary!!}

\subsubsection{Postprocessing the Feature-Directions\arrowref{sec:algorithmsteps}}
\label{sec:algo:postprocess}


The main contribution of \textcite{Ager2018} is a postprocessing step that changes the final space such that the ranking of entities \wrt each feature direction more closely mimics the ranking of frequencies of that direction's cluster words. The reasoning behind this is that the original embeddings from which the feature directions are created are based on global similarity. This makes it very vulnerable to outliers which often take up extreme positions. If one now creates the feature directions from the space, these outliers are assumed to have certain properties. The space is optimized for that, which limits the quality of feature directions in the space. The problem here is again the global similarity: If one entity ranks high for a feature, it is very likely that another entity that is close to that will also rank high for this feature, even though it may be something completely different. So to get better feature directions one has to distort the space. The authors accordingly again use the BoW representation of the entites to fine-tune the positions of the embeddings in the final space: After the clusters are collected and the entities re-embedded, for each feature a new ranking is computed by the summed frequency of any of a cluster's words per feature and entity. Each entity is thus represented as Bag-of-Clusters and again scored with PPMI to generate a ranking for each cluser/direction. This ranking is then used as a target for a simple \gls{ann} that distorts the space representation. As has been shown that it increases the algorithm performance only slightly while adding a substantial amount of work, it was not implemented in the scope of this thesis. 

\subsubsection{Re-Embedding the entities into the new space\arrowref{sec:algorithmsteps}}
\label{sec:algo:reembed}

To finally get the \textbf{feature-based representation} of the entities, they are re-embedded into a space where each of the vector components is a semantic directions and the value are the respective \gls{rank}ings. According to the authors, the degrees of similarity are not supposed to be encoded, which is why only the respetive ranking is considered.



% \subsection{Modifications from \textcite{Ager2018,Alshaikh2020}}

% \subsubsection{\textcite{Ager2018}}
% \label{sec:ager}


% \textcite{Alshaikh2020}

% \todo

% \removeMe{
% \subsection{Concluding stuff for algo}

% \subsection{Features and differences to original algorithm}

% \includeMD{pandoc_generated_latex/3_features_differences}

% \subsection{Reasonable params}

% \includeMD{pandoc_generated_latex/3_reasonableparams}

% \subsection{Algorithm Complexity}
% }