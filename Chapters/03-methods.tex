\chapter{Methods}

Direkt am Anfang schreiben dass ich halt auf den main algorithmus eingehe und das laut meiner research diese 3 paper am besten den main algo beschreiben (bzw sinnvoll erweitern) - was nicht heißt dass das die einzigen in dem kontext sind, Alshaikh2019 bspw nutzen ja den main algorithmus, aber ja nur als komponente, und haben andere Ziele was sie dann damit machen

Im folgenden gibt es neben Datasets 2 main sections: algoritm and architecture. Dass Algorithm und Architecture 2 subsection von methods sind ist halt "Der allgemeine Algorithmus und die spezifische Anwendung" Warum Architecture section? es kostet extrem viel zeit die schwammigen formulierungen in den papern genau zu verstehen, man probiert super oft falsche parameter-kombinationen aus etc etc, es ist halt ein riesiger ewig langer lernprozess den man von vorne machen müsste wenn man es nachimplementieren möchte, ich hätte mir gewünscht die authors hätten darüber mehr worte verloren, and also the scalable reproducible open-science part. And also - it took me a shitton of time, way more than working on the algorithm (but NOW it can run so easily on the grid and all param-combis simultaneously, ..), so this is what you'll get.


\section{Datasets}

All considered algorithms \mainalgos use a dataset of 15.000 movies and their reviews on IMDB, as well as a placetypes-dataset to evaluate their methods. The former consists of the concatenation of all available reviews for movies from IMDB\footnote{\url{https://www.imdb.com/}}, whereas the latter a collection of tags from photos uploaded to Flickr\footnote{\url{https://www.flickr.com}} that co-occur with a certain placetype. Other considered datasets are wine reviews, posts to certain newsgroups and another IMDB-review-dataset (see \tref{tab:all_datasets}). 

All of these datasets have in common that they are made up from a collection of independent texts or tags, created by different people. This means, that the more obvious or distinct a property of the respective entity is, the more often words describing that property will be used as tag or as part of the review. For example, a movie that is \emph{scary} to a lot of people will lead to many reviews mentioning that, which means that the word scary (or other words commonly co-occuring with it) will have a high count in the concatenation of that review. The algorithm from \cite{Derrac2015} heavily leans on this property by using the (relative) frequency of certain words as signal for the importance of the concept they may refer to. 

The main considered dataset of this thesis unfortunately does not share this property, as the texts that belong to an entity are not collected from different independent texts, but solely from the description of that entity - while it may be the case that the more \emph{mathematical} a course is, the more often the word \emph{math} occurs in it description, but the correlation is likely not as prominent as in the aforementioned datasets. Interestingly, \cite{Alshaikh2020} also used three datasets that only use a sort of description for an entity: its Wikipedia\footnote{\url{https://en.wikipedia.org/}}-article.


\subsection{SIDDATA-courses}

% TODO: bei dataset section darauf verweisen dass große teile des siddata-datasets mit gtranslate übersetzt wurden und auf den entsprechenden anhang verweiseen

% * Steht ja schon woanders dass mein Datensatz anders ist als concatenated-movie-reviews und ich deswegen nicht einfach "je öfter 'scary' desco scarier" machen kann. Da gibt's several ways mit umzugehen
	% * Das sich-die-richtigen-wörter-per-candidate-svm-bootstrappen
	% * Mit LSI rausfinden welche Terme genausogut in dem Text hätten vorkommen können (hab ich auch irgendwo schon)
	% * Explizit einfach zu gucken "Welche Terme kommen oft in den gleichen dokumenten vor" (und das inverse (steht iwo im code)), und dann ne candidate SVM für grouped terms anstelle von einzelterms machen (auch schon iwo als code)
	% * Mit Wordnet hypernyms/hyponyns und synonyms zu finden damit ebenfalls zu arbeiten (kann man wit wordnet angeben welches abstraktionsniveau ich haben will?)
	%     * Abstraktionsniveau gibt's nicht in wordnet, das heißt das richtige layer zu finden ist schwer. Was man auf jeden Fall machen kann ist die Terme zu den bases ihrer synsets umzuwandeln (dadurch wird aus "math" und "mathematics" das gleiche), aber in anderen Fällen ist es halt so dass ich die Candidate-Terms schon vorher brauche und nur sagen kann "diese entity enhält X wörter die halt hyponyms von dem Term sind"

their algorithm is tailored to concatenated-reviews or concatenated-bags-of-tags. Take their success-metric for the SVMs splitting the embedding. The more often the word "scary" comes in the concatenated reviews, the more scary the movie is. Sounds legit. The more often the people that took pictures at a particular place mentioned the "nature" of that, the more relevant "nature" is to that place. Also legit. But in the descriptions for courses that involve a lot of mathematics, it is not necessarily the case that the term "mathematics" occurs often. So due to the different nature of my dataset I have to go beyond their algorithm at some points - in this case it is probably the case that different kinds of mathematical terms actually do occur more often, so I'd need calculate these kinds of kappas not based oon a single term but ALREADY on a cluster of terms (... and I can bootstrap my way there, because after I do this I get more words to add to my cluster, rinse and repeat!)


% * Dass man theoretisch sich den task einfacher machen kann indem man nur die correctly-classified Kurse des fb-classifiers verwendet
% * MEINEN DATENSATZ mal mit den anderen vergleichen!! 
% 	* die Plots die schon drin sind beschreiben und warum der Datensatz whack ist.
% * Den ganzen "wo ist mein dataset anders als deren" Kram (teilweise schon im text, teilweise very old)
% * Woher der Datensatz kommt, dass es ja version 2 des Kurs-Datensatzes von Johannes ist (kommt von: /home/chris/Documents/UNI_neu/Masterarbeit/OTHER/study_behavior_analysis/src/data/course_data/db_dump_new/course_dump_new.csv)
% * Meine Pre-Preprocessing Schritte die da ja auch noch gut rumfiltern und rummergen beschreiben
% * Candidate-Word-Threshold: movies has samples-to-threshold value of 100, placetypes has 35, 20newsgrups has 614, so for 8000 courses any threshold from 2 to 25 seems reasonable => \cite{Derrac2015} say they intentionally kept the number of candidateterms approximate equal (at around 22.000), so to do the same I'd need a threshold of [TODO: optimal value]
% * [AGKR18] use a dataset that has fucking scipy preprocessing
% * ausrechnen "um so gut zu sein wie die, müsste der datensatz größe xyz haben"
% * Die standard-whackities des datensatzes, dass halt viele nur sind "Tutoren sind: Susi Sorglos Willi Wacker", oder "Findet statt in Raum XYZ", oder dass alle Sprachkurse die gleichen Beschreibung haben (beispiel. `....len([i for i in descriptions._descriptions if "kompetenzen entwickelt befahigen akademischen berufstypischen" in i.processed_as_string()]) == 25  ... weil es genau 25 exakt gleiche Beschreibungen gibt, für die Fremdsprachkurse. Deswegen ist up to jede 5-wort-kombination davon ein extracted keyword`)
% * Der Kappa-Score der rankigns vergleicht ist für mich ne kack metric weil ich ebennicht reviews nehme und more-occurences better-candidate heißen -> gucken wie ich stattdessen gute dimensionen und cluster finde (klingt doch so als sei accuracy/f1/... doch wichtig)

%TODO https://tex.stackexchange.com/questions/526198/table-resize-table-and-automatic-line-breaks


% \afterpage{%

\newgeometry{
	top=21mm,
	bottom=16mm,
	inner=16mm,
	outer=16mm,
} 


\begin{landscape}
	\begin{table}[]
		\resizebox{.98\textwidth}{!}{%
        \begin{tabular}{@{}llllll@{}} 
        	\toprule
        		\textbf{dataset} &
        		\textbf{contents} &
        		\textbf{preprocessing} &
        		\textbf{size} &
        		\textbf{classification classes} &
        		\textbf{candidate word threshold}
        		% & \textbf{key feature sizes} 
        		 \\ \midrule
        	\textbf{movies\tablefootnote{\label{origdsets}\url{https://www.cs.cf.ac.uk/semanticspaces/}} \cite{Derrac2015,Ager2018,Alshaikh2020} } &
				\specialcell[l]{grouped-by-movie-concatenated\\reviews for movies} & 
        		\specialcell[l]{\tabitem removed stop-words\tablefootnote{\label{fnote:stopwordlist}\url{http://snowball.tartarus.org/algorithms/english/stop.txt}} \\ \tabitem lower-cased text \\ \tabitem removed diacritics  \\ \tabitem removed punctuation} &
        		\specialcell[l]{\cite{Derrac2015}: 15000 movies \\ \cite{Ager2018,Alshaikh2020}: 13978 movies } & %Ager2018 says 15.000 - 1022 duplicates, that's the number of Alshaikh2020
        		\specialcell[l]{ \tabitem genre (23 classes)\\ \tabitem plot keywords (eg. \textit{suicide, beach}) (100 classes) \\ \tabitem age-rating certificates (6 classes)} & \specialcell[l]{\acrshort{df} $\geq 100$ \\ \textrightarrow 22 903 candidates \\ variable-length \textbf{n-grams} considered}
        		
        		\\ \midrule
        	\textbf{place types\footref{origdsets} \cite{Derrac2015,Ager2018,Alshaikh2020} } &
				\specialcell[l]{Tags of Flickr-photos that are also\\tagged with a place-type}
        		% bag-of-tags from Flickr used to describe places of a certain place-type
        		& 
        		None &
        		1383 place-types & %both in DESC15 and the follow-up paper
        		\specialcell[l]{ \tabitem category from Geonames (7 classes)\\ \tabitem category from Foursquare (9 classes)\\ \tabitem category from OpenCYC (93\cite{Derrac2015}/20\cite{Ager2018,Alshaikh2020} classes) } &
        		\specialcell[l]{\acrshort{df} $\geq 50$ \\ \textrightarrow 21\,833 candidates \\ (all words from the BoW) \\ \textbf{n-grams}: squashed all words of a tag} 
        		% & candidate-terms: 6385
        		\\ \midrule
        	\textbf{wines\footref{origdsets}\tablefootnote{\url{https://snap.stanford.edu/data/web-CellarTracker.html}} \cite{Derrac2015}} &
				\specialcell[l]{grouped-by-wine-variant-concatenated\\reviews for wines} & \specialcell[l]{\tabitem removed stop-words\footnoteref{fnote:stopwordlist} \\ \tabitem lower-cased text \\ \tabitem removed diacritics  \\ \tabitem removed punctuation} & 330 wine-varieties &
        		\textit{not performed} &
        		\specialcell[l] {\acrshort{df} $\geq 50$ \\  \textrightarrow around 6k candidates \\ variable-length \textbf{n-grams} considered }
        		\\ \midrule
        	\textbf{20 newsgroups\tablefootnote{\url{http://qwone.com/~jason/20Newsgroups}} \cite{Ager2018}} &
				\specialcell[l]{posts partitioned roughly even\\across 20 different newsgroups} &
        		\specialcell[l]{ \tabitem Headers, footers and quote metadata removed\tablefootnote{Using the scikit-learn python package, see \url{https://scikit-learn.org/0.19/datasets/twenty_newsgroups.html}} \\ \tabitem removed stopwords (using NLTK's corpus \cite{loper-bird-2002-nltk})\\ \tabitem lowercased text\\ \tabitem candidate terms: all textual and numerical tokens} &
        		18446 posts &
        		\tabitem newgroup post was submitted to (20 classes) &
        		$\geq$ 30 occurences 
        		\\ \midrule
        	\textbf{imdb sentiment\tablefootnote{\url{http://ai.stanford.edu/~amaas/data/sentiment/} \cite{maas-EtAl:2011:ACL-HLT2011}} \cite{Ager2018}} &
				\specialcell[l]{highly polar movie reviews\\for binary sentiment classification}  &
        		\specialcell[l]{ \tabitem removed stopwords (using NLTK's corpus \cite{loper-bird-2002-nltk})\\ \tabitem lowercased text\\ \tabitem candidate terms: all textual and numerical tokens} &
        		50000 reviews &
        		\tabitem sentiment of the review (2 classes) &
        		$\geq$ 50 occurences
        		\\ \midrule
        	\textbf{Bands \cite{Alshaikh2020}} &
        		\specialcell[l]{All Wikipedia pages ($\geq 200$ words) whose \\ WikiData semantic type is "Band"} &
        		\specialcell[l]{ \tabitem removed HTML-tags and references \\ \tabitem \textit{"standard preprocessing strategy"} \cite[137]{Alshaikh2019} \\ \tabitem removed stopwords (using NLTK's corpus \cite{loper-bird-2002-nltk})\\ \tabitem POS-tagging and keeping only nouns and adjectives \\ \tabitem remove words with a rel. \acrshort{df}  $>$ 60\% or abs. \acrshort{df} $<$ 10 } &
        		11448 bands & \specialcell[l]{ \tabitem Genres (22 classes) \\ \tabitem Country of origin (6 classes) \\ \tabitem Loc. of formation (4 classes) }  & 
        		\specialcell[l]{ 10 $<$ \acrshort{df} $<$ 6869 \\ (all words from the BoW)}\\ \midrule
        	\textbf{Organisations\tablefootnote{\label{fnote:for_alshaikh2019}Originally created in and for \cite{Alshaikh2019}} \cite{Alshaikh2020}} &
        		\specialcell[l]{All Wikipedia pages ($\geq 200$ words) whose \\ WikiData semantic type is "Organisation"} &
        		\specialcell[l]{ \tabitem removed HTML-tags and references \\ \tabitem \textit{"standard preprocessing strategy"} \cite[137]{Alshaikh2019} \\ \tabitem removed stopwords (using NLTK's corpus \cite{loper-bird-2002-nltk})\\ \tabitem POS-tagging and keeping only nouns and adjectives \\ \tabitem remove words with a rel. \acrshort{df}  $>$ 60\% or abs. \acrshort{df} $<$ 10 } &
        		11800 organisations &
        		\specialcell[l]{ \tabitem Country (4 classes)\\ \tabitem Headquarter Loc. (2 classes)} &
        		\specialcell[l]{ 10 $<$ \acrshort{df} $<$ 7080 \\ (all words from the BoW)} \\ \midrule
        	\textbf{Buildings\footnoteref{fnote:for_alshaikh2019} \cite{Alshaikh2020}} &
        		\specialcell[l]{All Wikipedia pages ($\geq 200$ words) whose \\ WikiData semantic type is "Building"} &
        		\specialcell[l]{ \tabitem removed HTML-tags and references \\ \tabitem \textit{"standard preprocessing strategy"} \cite[137]{Alshaikh2019} \\ \tabitem removed stopwords (using NLTK's corpus \cite{loper-bird-2002-nltk})\\ \tabitem POS-tagging and keeping only nouns and adjectives \\ \tabitem remove words with a rel. \acrshort{df}  $>$ 60\% or abs. \acrshort{df} $<$ 10 } &
        		3721 buildings &
        		\specialcell[l]{ \tabitem Country (2 classes)\\ \tabitem Administrative loc. (2 classes)} &
        		\specialcell[l]{10 $<$ \acrshort{df} $<$ 2233 \\ (all words from the BoW) }\\ \Xhline{4\arrayrulewidth}
        	% \textbf{SIDDATA-Courses} &
        	% 	TODO &
        	% 	&
        	% 	&
        	% 	\tabitem Faculty (10 classes) 
        	% 	\\ \midrule 
        	% \textbf{100K Coursera reviews}\tablefootnote{\url{https://www.kaggle.com/septa97/100k-courseras-course-reviews-dataset}} &
        	% 	TODO &
        	% 	&
        	% 	&
        	% 	\specialcell[l]{ \tabitem Rating (5 classes) \\ \textit{\tabitem Major, Category, Offered-By,... (tbd)} }
        		\\ 
		\end{tabular}
		\caption[All datasets used by any of \mainalgos]{All datasets used by any of \mainalgos. Citations behind the dataset name denote which author used it. Other listed properties include dataset sources (where available), contents, sizes, the respectively used preprocessing-methods and candidate-word-thresholds, as well as the classes considered in the evaluation of the derived explainable classifiers.}
		\label{tab:all_datasets}
	}
	\end{table}
\end{landscape}


\restoregeometry % !!! when trying to add afterpage again, remove this!!


% \restoregeometry\clearpage % !!! Jörg's comment on https://tex.stackexchange.com/a/78285/108199 !!!!
% \aftergroup\restoregeometry  % see THE QUESTION of https://tex.stackexchange.com/q/139834/108199
% } %afterpage



% Empirie, auch specifics über den Datensatz

%To write:
% * where does the data come from
% * what size is the data, what is the distribution, ...
% * Preliminary analysis (if I delete all that are shorter than X, it are |Y|..)
% * Does it cluster and look nice?
% * Verteilung der Sprachen
% * Preprocessing in kurzem Fließtext beschreiben - "After throwing out all descriptions shorter than xyz chars, 2323 courses where left. 223 of these were ..."
% * That the type of dataset differs from DESC15 and followups - mainly used movie-dataset consists of concatenated reviews (which means relevant words occur more often!) 
%     (TODO: look/think was die anderen auszeichnet - bei dem placetypedataset ists ja gar kein fließtext sondern direkt ein bag-of-tags)
% Dass mein Datensatz kleiin ist! Bei keinem sonderlichen min-word-per-desc threshold hab ich halt 7588 samples, bei 50 schon nur noch 4123, das ist wirklich little
% Dass auch die Descriptions echt kurz sind! Ich hab rund 8k samples, um das selbe samples-to-threshold verhältnis zu haben wie DESC15 wäre rechnerisch ein wert von 2 bis 25 sinnvoll (wobei man beachten muss das 2 schon richtig kacke ist weil dann die SVM 2 vs 8000 klassifizieren muss and that will never work -> 25 ist minimum), ABER wenn ich dann 25 nehme hab ich nur 2.4k candidates statt the 22k DESC15 aimed at, which also sucks!! --> CONCLUSION: Datensatz scheint zu klein.

The main goal of this thesis was to create a conceptual space of courses, automatically generated by course descriptions.


For that, a dataset of courses and their descriptions was obtained as export from the Stud.IP system as used at the universities of Osnabrück, Hannover and Bremen.
%TODO wait, woher kam der datensatz überhaupt? Tobias hat mir den geschickt, aber kam er zustande im Rahmen von Siddata?

The dataset comes from Johannes' Repo at \url{https://git.siddata.de/jschrumpf/study_behavior_analysis} (requires authentification over UOS!)

\begin{figure}[H]
	\centering
	\includegraphics[width=\figwidth]{graphics/figures/courses_language_distribution.png}
	\slcaption{
		\label{fig:courses_language_distribution}
		Distribution of languages of course descriptions.
		%TODO figure if this is the correct amount of preprocessing/throwout to have done
		Of the 21337 courses left after preprocessing, 18,679 were in german language according to the \textit{langdetect} python-package (for details, see \aref{ap:translating}).
		}
\end{figure}


The faculty is easily obtainable from the dataset, as the first one or two digits of the course ID correspond to it. The distribution of the faculties is depicted in figure \ref{fig:faculty_plot}.

\begin{figure}[H]
	\centering
	\includegraphics[width=\figwidth]{graphics/figures/faculty_plot.png}
	\slcaption{
		\label{fig:faculty_plot}
		Distribution of faculties in the courses
		}
\end{figure}

The purpose of the Neural Network classifier is to check if it is anyhow possible to extract meaningful information from the descriptions: If it is possible to train a classifier on the data that can reasonably predict a qualitative feature, there is enough structure in the data such that the algorithm I'm about to produce can work.
Also, we have a lower bound for useful data: we can just throw away data that cannot be classified!
%TODO: train a second classifier on something else and throw away data that gets classified by neither and inspect it

(-> 91\% test accuracy)

% =============== Besonderheiten vom Siddata-datensatz

....len([i for i in descriptions._descriptions if "kompetenzen entwickelt befahigen akademischen berufstypischen" in i.processed_as_string()]) == 25  ... weil es genau 25 exakt gleiche Beschreibungen gibt, für die Fremdsprachkurse. Deswegen ist up to jede 5-wort-kombination davon ein extracted keyword
(und das obwohl sie verschiedene Namen haben! merging them doesn't make sense but they are almost equal)

% =============== Schreiben zum Thema Datensatz-Vergleich:

...ist es richtig dass nur 6000 verschiedene Terms >= 25 mal vorkommen?! 6000?!
=> auch in groß ist mein datensatz ja noch deutlich kleiner als placetypes, die haben immerhin 22k candidates
--> n-docs: 7596
--> 1-grams >= 25 times: 5054, 1-5-grams >= 25 times: 6717
--> unique 1-grams: 106235

bei placetypes sind es 
* unique 1-grams: 746180, davon 41320 >= 25 mal und 21833 >= 50 mal (their threshold)

--> das verhältnis Anzahl Texte zu Länge Texte ist bei mir halt komplett off 

% =============== 


\subsection{Place-Types}

\begin{itemize}
	\item Took it to be able to compare my results to the ones of \mainalgos
	\item Did NOT do the movies-dataset (also used by all \mainalgos, see \tref{tab:all_datasets}), because version available online does not contain n-grams so it will not be comparable
	\item Also took it to be able to sanity-check if my implementation was correct, which was extremely helpful
	\item Didn't do the openCYC taxonomy bc they say that they don't use one level of the taxonomy consistently but also never explain where they go to which level
\end{itemize}

%TODO: write IN THE ALGORITHM & ARCHITECTURE SECTIONS that I of course tried the placetypes-dataset as sanity-check to find errors - for that dataset, stuff like the good-candidates is known so as long as I don't reach their performances for that dataset I know my code is the problem, but as soon as I reach their performance I can savely say that the actual algorithm is correct and if it's still bad on the siddata dataset it's just not applicable to this kind of data

So, infos from \cite{Derrac2015}:
\begin{itemize}
	\item GeoNames has 667 place-types in 9 categories (403 used)
	\item Foursquare has 435 place-types in 9 top-level categories (391 used))
	\item content: tags of Flickr photos. Photos assumed to be of a type if one of the tags is the name of that type (so they queried for photos with that tag), and then all other tags of that picture make up the BoW.
	\item 22816139 photos, types with less than 1000 photos removed.
\end{itemize}


Also tried the Plactypes-Dataset used by all main-paper-authors. When doing so I noticed that there are definitely duplicates (which are consistently recognized as closest-terms in embedding):
  abandoned rail road and abandoned railroad
  boat yard and boatyard
  coral reef and reef
  court house and courthouse
  grass land and grassland
  sheep fold and sheepfold
  skate park and skatepark
  steak house and steakhouse
  water fall and waterfall
  wind mill and windmill

Next to that, the embedding however also sees very similar ones as very similar, which is a nice sanity-check, eg.

  abandoned farm and abandoned home
  airfield and airport
  airport and airport terminal
  ancient site and archaeological site
  arch and arch bridge
  art gallery and art museum
  coffee house and coffee shop
  aircraft cabin and airplane cabin
  apartment and apartment building
  bank and bank building
  field hockey field and hockey field

\subsection{Other Datasets}

Also tried a dataset of 100.000 coursera course reviews from \url{https://www.kaggle.com/septa97/100k-courseras-course-reviews-dataset}. Why? Because it's also eduactional resources, but as it's reviews it seems closer to the movies dataset
See \url{https://www.kaggle.com/roshansharma/coursera-course-reviews} for exploratory analysis of the dataset (there he also has another dataset he writes about, but you cannot merge them unfortunately, so besides course name the only possible task is the rating)
%TODO: I could try to merge it with this one https://www.kaggle.com/siddharthm1698/coursera-course-dataset or another one (see https://www.kaggle.com/mihirs16/coursera-course-data which links names to links, https://www.kaggle.com/search?q=coursera+in%3Adatasets for other places)

Also, there's the Large Movie Review Dataset\footnote{\url{http://ai.stanford.edu/~amaas/data/sentiment/}, \url{https://scikit-learn.org/0.19/datasets/twenty_newsgroups.html}}, also used by \cite{Ager2018}.




\section{Algorithm}

Let us finally go into detail about the main algorithm. The implementation of this thesis replicates and extends the algorithm proposed by \textcite{Derrac2015} with some novel contributions to deal with the given dataset. Further, some improvements from the works of \textcite{Ager2018} and \textcite{Alshaikh2020} are incorporated, who also replicated and improved the original algorithm and share Prof. Steven Schokaert as last author. According to our evaluation of the field, these two papers provide some useful improvements in several aspects, as they apply the algorithm to different datasets, suggest more straight-forward ways of evaluating their performance, and help in understanding important concepts. That we are focusing on only those three papers should by no means imply that they are the only ones that were considered and influcenced this implementation\footnote{See \autoref{sec:otherwork}}, however in contrast to the other pertinent literature these two works do not substantially divert from the algorithms core principles.
% see \autoref{sec:otherwork} - \cite{VISR12} and their tag genome, the fact that the algorithm detailed here is basically only one step in \cite{Alshaikh2019}, Gärdenfors himself suggested that one may use self-organizing maps instead of classical AI/NLP algorithms.

It is important to keep in mind that the algorithm is no rigid monolith but modularly consists of several components, such as \textit{dimensionality reduction}. Many of these components do not require specific algorithms, and \mainalgos also experiment with different components. The exact system for these components may be exchanged, and in the following these exchangeable algorithms are also referred to as hyperparameters. Note further that while this thesis mostly replicates the work of \textcite{Derrac2015}, the following will describe the algorithm as implemented here, which differs in some details from the original work. For the sake of overview, very specific implementation details will be left out in the following description, as however reproducibility is an important aim for us, implementation details are available in Appendix~\ref{AppendixB} and linked where relevant. \todoparagraph{Further, there is a table that compares the implementations here and in mainalgos, as well as another table what-configs-are-available where, the config-yamls and a section on what-other-things-could-one-have-done-hereandthere.}

\subsection*{Core Algorithm}

\todoparagraph{The following explanation assumes that we accept some things as given. For now we'll do that, but we will later revisit and critically question many of these assumptions!}

% The core idea of the algorithm is to unsupervisedly find a a set of features which can be modelled as directions for a vector-space representation of the respetive entities.
The main goal of the algorithm is to unsupervisedly use text-corpora associated with the considered from a certain domain\glspl{entity}\footnote{From now on, the term \textit{\glspl{entity}} refers to the sample described by one text from the corpus (description, concatenated reviews, ...). The corpus accordingly defines the domain: educational resources, movies, ...} to embed these into a vector-space where the axes correspond to human concepts/Properties.\footnote{\textit{Concepts} and \textit{Properties} explicitly refer to what is defined in Criterions C and P, see \ref{sec:csdefinition}} This is referred to as \textit{feature-based} representation: A high-dimensional vector that numerically encodes the degree (\textit{protoypicality}) to which the entity corresponds to a number of appropriate dimensions. This is generally referred to as Conceptual Space and can be used as basis for explainable reasoning.

The general idea to achieve that is as follows: First, the entities are embedded as fixed-dimensional vectors. To allow for the types of reasoning mentioned in Section \ref{sec:cs_reasoning}, it is embedded into metric spaces where the concepts of direction and distance are well-defined \gencite{Derrac2015} original algorithm uses MDS (see \ref{sec:mds}) for this matter, which enforces metric distances. \cite{Ager2018,Alshaikh2020} both soften this requirement and also use document embedding techniques such as doc2vec and averaged GloVe \todoparagraph{REFERENCE!} embeddings.

Additionally, words or phrases from the text are extracted as candidates for the names of the semantic dimensions. The underlying assumption is that \q{words describing semantically meaningful features can be identified by learning for each candidate word $w$ a linear classifier which separates the embeddings of entities that have $w$ in their description from the others} \cite[3574]{Alshaikh2020}. The better the performance of that classifier according to a chosen metric, the more evidence there is that $w$ describes a semantically meaningful feature. 
% * from Alshaikh2020: "Their core assumption is that words describing semantically meaningful features can be identified by learning for each candi- date word w a linear classifier which separates the embeddings of entities that have w in their description from the others. The performance of the classifier for w then tells us to what extent w describes a semantically meaningful feature"
In a final step, the candidate-words are clustered according to their similarity to find a fixed set of \emph{semantic directions}. A representative term for the directions is selected as dimension name, and the entities are re-embedded into a new space comprised of these dimensions, where the individual vector-components correspond to the ranking of an entity with respect to these dimensions.

The rest of this section goes into further detail for each of the individual components of the algorithm. \removeMe{An overview of which of the considered literature supports each components is given in \autoref{tab:compared_algos}.} Further, configuration files to enable exactly the respective components of the papers \mainalgos for the codebase of this thesis are listed in \aref{ap:yamls_for_origalgos}.

\todoparagraph{but before that, ager and alshaikh}

\removeMe{
\subsection{Regarding ager and alshaikh}

\todoparagraph{describe shortly what the improvements from [2,3] were}

\todoparagraph{Dass die den Zwischenschritt mit dem ganzen geeometric reasoning auf dem interim space nicht machen und DESWEGEN die requirement mit MDS soften konnen}

In principle Derrac2015, but with some components from Ager2018 and Alshaikh2020 as well as some own stuff. I'll be testing some claims or nonclaims of \mainalgos, bspw nutzen sie immer PPMI ohne je tf-idf zu testen. Also of course different nature of the dataset - their "how does this dimension correspond to the count in the reviews" doesn't make sense (their success-metric for the SVM is tailored to the one property, so I expect that one to be worse). I'll elaborate on different ways to deal with that later.
}

\subsection{Algorithm Steps}
\label{sec:algorithm_steps}

% The core idea of the algorithm is to (unsupervised, data-driven) find a a set of features which can be modelled as directions for a vector-space representation of the respective entities. For that, the steps are:

Let us finally describe the steps how to create an interpretable vector-space from the text corpus in detail. For that, we will explicitly elaborate on the parameter choices that branch up at every one. Note that that absolutely is a combinatorical explosion it is impossible to try out all. Further, this is about how this specific implementation does it, which may differ in some details from \mainalgos.

\label{sec:algorithmsteps}
\begin{enumerate}
	\item[\saveref{sec:algo_preproc}{1.}] \textbf{Preprocess} the corpus with default techniques and create a \textit{Bag-of-ngrams representation} (\ref{sec:techniques:bow}) of the texts.
	\item[\saveref{sec:extract_cands}{2.}] \textbf{Extract Candidate Feature} names from words/\glspl{ngram} of the corpus.
	\item[\saveref{sec:generate_vectorspaces}{3.}] \textbf{Embed all Entities} into a fixed-dimensional vector space with demanded properties that captures the respective semantics.
	\item[\saveref{sec:svm_filter_cands}{4.}] \textbf{Filter Candiate Features} by training a linear classifier for each candidate that seperates the vector representations of the entities that contain the term from those that do not. If a specified metric for this classifier is sufficiently high, assume that the candidate term captures a \textit{salient} feature - its direction is then characterized by the orthogonal of the classifier's separatrix.
	\item[\saveref{sec:algo:cluster}{5.}] \textbf{Cluster/Merge the candidates} and calculate the feature direction for each cluster from its components, and (optionally) find a representative cluster-name.
	\item[\saveref{sec:algo:postprocess}{6.}] (optionally) \textbf{Post-process} the candidate-clusters.
	\item[\saveref{sec:algo:reembed}{7.}] \textbf{Re-embed the entities} into a space of semantic directions by calculating their distance to each of the feature direction separatrices.	
\end{enumerate}

This techniques first embeds the collection of texts into a  vector space, to afterwards extract important features from this space using linear classifiers. The second step is an original idea of \cite{Derrac2015}, however creating vector space embeddings from texts is a very popular technique, used for many tasks in \gls{nlp} \cite{Mikolov:Regularities,Mikolov2013a,Guo,Lowe,Turney2010}. This implementation relies on classical creation of the \gls{vsm}, for which the general creation process was explained in \autoref{sec:vsm_construction}. The steps \textit{Build the Frequency Matrix}, \textit{Transform Raw Frequency Counts} and \textit{Smooth the Frequency Matrix} are squashed into the preprocessing and embedding of entities (steps 1 and 3).

 An explicit and simple implementation compliant with each step could be a simple word tokenization and count to generate a bag-of-words (step 1) where each sufficiently frequent word is used as candidate (step 2). A \gls{dissimmat} of the individual \gls{bow}-vectors is compressed using MDS (step 3). A \gls{svm} calculates the accuracy for each candidate (step 4), and k-means-clustering on the 500 top-scoring terms subsequently creates 100 clusters and averages their directions (step 5). The distance to each of the hyperplanes is calculate (step 6), yielding new space for the entities. The sequence of steps is also given as pseudocode in \autoref{ap:algorithm_pseudo}. 
 
 Again it should be stressed that many different components can be considered for each step and the distinction of steps is not rigid: Instead of creating a dissimilarity-matrix followed by dimensionality reduction, \cite{Ager2018,Alshaikh2020} use neural word or document embeddings.\footnote{see \autoref{sec:embeddings}} Instead of extracing candidates from corpus tokens and training a linear classifier for each of them and use their orthogonal as direction, techniques such as LSA or LDA can be employed to find topic vectors directly. We will come back to these ideas when discussing future research opportunities (\autoref{sec:futurework}) by listing what other ways of fulfilling each respective step could have been considered.

\autoref{fig:dependency_graph} shows an automatically exported dependency-graph, displaying the individual steps of the algorithm as done in the accompaning code, also showing where selected important parameters are first used. As explained in \autoref{sec:architecture}, the modularity of the provided architecture allows individual components to be exchanged as needed and run in parallel.


\begin{figure}[H]
	\begin{center}
	  \includegraphics[width=0.9\textwidth]{dependency_graph.pdf}
	  \caption[Dependency-Graph of the Algorithm]{Dependency-Graph of the Algorithm, displaying the individual steps of the algorithm as well as their dependencies and where selected important parameters are first used. \todoparagraph{Generated using command XYZ}}
	  \label{fig:dependency_graph}
	\end{center}
\end{figure}


Before looking at the steps in turn, it should be noted that even the preprocessing does not work on completely raw data, but on curated and processed corpora. This processing is however not considered part of the algorithm, as it is very specific to the respective datasets and manual dataset exploration, tweaking settings such that they are best for each corpus separately. The preprocessing for the Siddata-dataset is described in \autoref{sec:dataset_siddata} and its implementation is done in separate Jupyter Notebooks.\footnote{Such as \url{https://github.com/cstenkamp/derive_conceptualspaces/blob/main/notebooks/create_datasets/Preprocess_Siddata2022.ipynb}}. In the considered literature, the preprocessing is not considered part of the algorithm at all. Their implementations start from already fully processed datasets available as bag-of-words, each separately processed. Details of their individual processing per dataset is listed in \autoref{tab:all_datasets}. By incorporating the preprocessing into the pipeline, this work aims to increase adaptability and reproducibility, and also allows to experiment with different techniques such as translation or lemmatization or how duplicate entities with different associated texts are merged.
% In course-descriptions, I want some parts of the pre-preprocessing be part of the pipeline, like how we merge descriptions of different iterations of the same course that overlap to a high degree (sentwise-merge vs relative-term-frequencies)

\subsubsection{Preprocessing\arrowref{sec:algorithmsteps}}

\label{sec:algo_preproc}

A common prerequisit for NLP algorithms is to pre-process the text corpus. The preprocessing itself consists of multiple independent components chained after each other. Which components are necessary also depends on the processed dataset - as for example the \emph{placetypes}-dataset consists of a collection of \textit{tags} instead of full sentences, tokenizing sentences or removing \glspl{stopword} becomes irrelevant. Other datasets may require additional cleaning or are already available in preprocessed form.

\paragraph{Translation} As the main considered dataset of university-courses is highly multilingual (see \autoref{fig:sid_statistics}), one of the first questions that needs to be addressed is how entities of different langauges are handled. The algorithm consists of classical language processing algorithms such as comparing \gls{bow} representation of the entities, which means that the same text in two different languages may result in maximally different representations (see \autoref{sec:techniques:bow}). Because of this, before any other processing, the languages of each entity is checked, such that those of languages other than the demanded may be either translated, left out or used anyway. For details about the translation, it is referred to Appendix \ref{ap:translating}.\footnote{It should be noted that professional automatic translation is costly and thus not all texts are available in all languages.}

\paragraph{Components} The following components are developed for the preprocessing, every one of which can be individually enabled or disabled:

\begin{itemize}
	\item Prepend title and/or subtitle to the entities' associated text \itemtext{useful for the Siddata-Dataset, as the titles are often quite long and more descriptive than their descriptions}
	\item Remove HTML-Tags from texts 
	\itemtext{useful for the Siddata-dataset, as it includes descriptions for \glspl{mooc} which are crawled from websites and often contain such}
	\item Tokenize sentences 
	\itemtext{such that \glspl{ngram} across sentences are not considered}
	\item Lower-case all words
	\itemtext{reduces the amount of individual words and ensures that words at the beginning of sentences are mapped correctly}
	\item Remove stop-words / frequent phrases
	\item Tokenize words
	\itemtext{means splitting at the word-boundary, resulting in a list of words. Order must be kept in case n-grams are to be extracted.}
	\item \Gls{lemma}tize words
	\item Remove diacritics
	\itemtext{\emph{Diacritics} are glyphs added to basic letters, such as accents or German \emph{Umlaute}. Removing them converts for example the letter \emph{ä} to an \emph{a}}
	\item Remove punctuation 
\end{itemize}

The above can be done either be done with proprietary code for all of these steps,\footnote{Mostly relying on the python package \emph{nltk} \cite{bird2009natural}} or using \codeother{sklearn}\footnote{\url{https://scikit-learn.org/stable/}}s \codeother{CountVectorizer} (which is faster, but less configurable), as \cite{Ager2018} claim to have done.

\paragraph{On Stop-Words}
Removing stop-words from the texts is useful because it makes the resulting frequency more compact and thus less computationally intensive, and stop-words generally have very discriminative power, meaning their occurence among the entities is arbitrary, just making hte emeddings more noisy (cf. \autoref{sec:word_count_techniques}). There are however reasons to not remove them: Two words that are considered stop-words may posess relevant semantic content (such as a \textsc{Fällt aus} in a course title), and also stopwordslists are often incomplete and of low quality \cite{nothman-etal-2018-stop}. For these reasons it is also possible to instead remove \glspl{ngram} that exceeded a certain frequency (\gls{df}).

\paragraph{On Lemmatization}
The languages most prevalent in the considered datasets are considered \textit{agglomerative}, which means word stems are changed by the addition of affixes and suffixes. Consequently, the same word may be present in multiple different forms, which modelled as completely dissimilar vectors in the present \glspl{bow}-approach. Lemmatization is the process of mapping different forms of these words onto the same stem. Considering that the Siddata-dataset consists of far fewer words than the others, this has important implications. For the german descriptions, this implementation relies on the \textit{HanTa} lemmatizer. \todoparagraph{Correct citation for hanta!!} %https://textmining.wp.hs-hannover.de/Preprocessing.html#Lemmatisierung

The result of this step is a bag-of-ngrams representation for each entity (see \autoref{sec:techniques:bow})


\subsubsection{Extract Candidates\arrowref{sec:algorithmsteps}}
\label{sec:extract_cands}
% Section 4.2.1 of Derrac2015

The final result of the algorithm is a metric space in which the individual dimensions (\emph{\glspl{feature}}/\emph{Interpretable direcitons}) correspond to natural-language concepts and attributes. The candidates for these features are verbatim phrases extracted from the text-corpus of the \glspl{entity}, which are subsequently filtered and merged as necessary.

In \gencite{Derrac2015} work, the selection of phrases to be extracted depends on the dataset: For placetypes-dataset, all sufficiently frequent\footnote{\label{fnote:cand_thresholds}The respective thresholds are listed in \autoref{tab:all_datasets} as ``candidate word threshold''.} 1-grams\footnote{Note that in the case of the place-types dataset, a 1-gram corresponds to all merged words of a tag.} were considered. For the other two datasets, they applied a \gls{pos}-tagger that extracted all sufficiently frequent\footnoteref{fnote:cand_thresholds} \textbf{adjectives, nouns, adjective phrases} and \textbf{noun phrases}, assuming that adjectives would correspond to gradual properties (\eg \textit{violent, funny}) and nouns to topics (\eg the \textit{genre}) \cite[Sec. 4.2.1]{Derrac2015}. Also, the authors ensured that the number of extracted candidates for both datasets is roughly equal, getting around 20\,000 candidates for movies and placetypes.

% Their method depended on the dataset - as their placetypes-dataset was just a collection of tags and the number of tags with term-freq >= XYZ (docfreq>2?! hä?) corresponded to their desired number of candidates anyway (around 22k), they just took all of these as candidates. For their movie-reviews-dataset, they considered all nouns, adjectives, nounphrases 	and adjective-phrases as detected by a POS-tagger. Doing something similar in the scope of this thesis led to suboptimal results, which is why alternative methods were developed
For this step, the implementation of this thesis differs from the original algorithm, as both taking all words as candidate and running a \gls{pos}-tagger led to suboptimal results in previous experiments, which indicated that the robustness of the algorithm is increased if less candidates are considered in earlier steps. This will be further argued and elaborated in the discussion. To ensure comparability to these works however, in the case of the placetypes-dataset the original method of taking all words with a term-frequency of at least 50 was used. Similar techniques for the Siddata-dataset were also considered, but in constrast to the placetypes-dataset it is also important to consider various-length n-grams. While \textcite{Derrac2015} claim to have considered \glspl{ngram} for the movies-dataset, the published version of this dataset contains a \glspl{bow}-representation for each entity where the original word-order is lost, making it impossible to recover \glspl{ngram}, making comparisons with their results impossible for that dataset.\footnote{\url{https://www.cs.cf.ac.uk/semanticspaces/}}

% \todoparagraph{thing is I have less words but the algorithm seems to profit from less words as that makes it more robust}
% I would however argue that the difference here doesn't make a relevant difference 

In our implementation the candidate-extraction is split into four subsequently excecuted substeps, because depending on the algorithm used to extract the candidates the runtime of the individual components is comparably long and some settings are only relevant in later substeps. The steps are:
\begin{itemize}
	\item Extracting Candidate Terms
	\item Postprocessing the Candidates
	\item Creating the \gls{doctermmat} for the candidates and applying a \gls{quant}
\end{itemize}

As visualized in \autoref{fig:dependency_graph}, these substeps only depend on the preprocessed descriptions, which means they can be run in parallel to the creation of the embedding.\footnote{\todoparagraph{Another good reason for cluster exceution!}}

% This can be done either based on the frequency (meaning all terms with a minimal term-frequency), based on some notion of *importance* (based on scores like tf-idf or ppmi), or by more complex means of figuring out *important* keywords and keyphrases. An example of the latter would be KeyBERT
Three main techniques are implemented to extract candidates from the text-corpus. Irrespective of the algorithm, only words with a sufficiently high \gls{df} are extracted, which is important to ensure that the classifier that determines its meaningfulness has enough samples in both clases. This means that the minimal freqeuncy can be calculated from the dataset size: In \cite{Derrac2015}, the minimal frequency for the movies-dataset with 15\,000 entities was only 100, meaning that the algorithm even works if only 0.6\% of samples are in the positive class. 

\todoparagraph{We will come back to this later}
\todoparagraph{HAB ICH DF UND TF RICHTIG??}

\begin{description}
	\item[By frequency:] consider all phrases that exceed a specified document-frequency (like \cite{Derrac2015}).
	\item[By a \gls{quant}:] consider all phrases that are prominent by some notion of \textit{importance} , such as the \gls{ppmi} or \gls{tf-idf}-score. Note that the respective scores depend on the combination of document and term, such that candidates may be extracted for some documents. Of course, all their occurences are considered in the creation of the frequncy matrix.
	\item[Using \emph{KeyBERT}\cite{grootendorst2020keybert}:] consider phrases whose BERT-embedding \cite{Devlin2019} is most similar to the text they are in. 
\end{description}

Using KeyBERT results in candidate terms that are most appealing in qualitative inspection, however it is also most computational demanding, techniques and requires substantial amounts of post-processing for the resulting phrases. More details on KeyBERT and how it is incorporated into the algorithm are given in the implementation are given in Appendix~\ref{ap:details_keybert}

Finally, a \gls{doctermmat} is created from the postprocessed candidates, containing the frequency for each candidate-phrase in each entity. The creation of this frequency matrix mirrors the process described in \autoref{sec:vsm_construction}, however only for the extracted words. After filtering this matrix to ensure that only candidates with a minimal \gls{df} or \textit{stf} are considered, a quantification is applied as described in \autoref{sec:word_count_techniques}. Available Quantifications include raw count, binarization\footnote{meaning all counts are either one or zero. According to \cite{Alshaikh2020} this improves performance \todoparagraph{Aber ich hab logische Probleme damit}}, tf-idf or PPMI.

\cite{Derrac2015} \todoparagraph{always only use PPMI without ever testing tf-idf or giving a reason, I'll try both}

\todoparagraph{so the relation of term to document may be expressed by something else than count - so if we later compare the ranking induced by the svm to this maybe something else thatn the count stands there - I'm expecting that for my dataset tf-idf is much more valuable than the count bc no concatenated reviews or tags}


\subsubsection{Generating Vector Space Embeddings\arrowref{sec:algorithmsteps}}
\label{sec:generate_vectorspaces}

\todoparagraph{This is Turney2020s "Building the frequency matrix", BUT SO IS THE STEP ABOVE}

In this step, the individual \glspl{entity} are embedded into a fixed-dimensional vector space, making up a \emph{frequency matrix}. Importantly, while this matrix is a \gls{doctermmat}, it is only an interim result in the algorithm and the calculation of distances and directions will be done on another matrix from a later step - This is where our pipeline starts to diverge from what the pipeline specified in \nameparanref{sec:vsm_construction}. So we created a frequency matrix that encodes the relevance of a candidate-phrase for each entity in the previous step, and in this step we create another one that encodes each document as a vector. Neither of these matrices will be used to finally calculate similarities on, but both are important to get the dimensions necessary for for these similarities.

Embedding words, \glspl{ngram}/phrases or other tokens, as depicted by \cite{Turney2010,Lowe}, generally involves counting the token frequencies, transforming them to get relative frequencies, and performing dimensionality reduction on the resulting matrix.
So far (step 1), we have counted the token frequencies. 
\todoparagraph{yes we are talking about all tokens}
\textcite{Derrac2015} argued that this space must be a Euclidean \todoparagraph{which is invariant to affine transformations}, such that geometric/algebraic solutions correspond to commonsense commonsense reasoning tasks (see \autoref{sec:reasoning}). \todoparagraph{We will later look into this in more detail}. 
Another requirement is that the number of dimensions is can be chosen as hyperparameter to the algorithm to be able to find a compromise between \todoparagraph{powerfulness and compression... nee argh was fur worter suche ich hier... drauf zuruckkommen wenn ich den teil uber vsms fertig hab. ausserdem, reicht das schon als beschreibung dann?}. Because of these two requirements, \todoparagraph{and also because gardenfors said so} they selected \gls{mds} for dimensionality reduction.

As stated in \autoref{sec:mds}, \gls{mds} calculcates a Euclidean \gls{vsm} from a set of pairwise distances. This means that the algorithm first creates a \textit{Dissimilarity Matrix} that encodes the distance between all pairs of entities (represented as Bag-of-ngrams representation), from which subsequently the final embedding is generated.  
\todoparagraph{This technique of bag-of-ngrams-representation then dissimmat and quantication is not the only way to do it, ager and alshaikh both did it differently}

\todoparagraph{Note that we can use another quantification than in the step above! In my algo this sometimes performed best.}
\todoparagraph{do all this again argh}

In their algorithm, the dissimilarity-matrix is created using distance metrics for the bags-of-words of the respective entities. 

\paragraph{Document embeddings}
If the strict requirement for a metric space is dropped however, many different algorithms may instead be used at this point - not only different dimensionality reduction methods for the embedding, but also ones that do not rely on the distance matrix or even the \gls{bow} at all, like document-embedding-techniques such as \gls{doc2vec} \cite{Le2014} (as \eg used by \cite{Alshaikh2020}). This would change only these steps and the rest of the algo not too much.
However when \cite{Alshaikh2020} used doc2vec instead of dissimmat-mds, it performed worse (see tables in results), which is why it is not in this thesis 


\paragraph{Create Dissimilarity Matrix and Quantify}

The default way of doing it is to create a \gls{doctermmat} that counts the occurences for all words\footnote{Not just the candidates in step 2, but words that occur in any description} for all entities.

In their algorithm, the \gls{dissimmat} is created using the \emph{normalized angular distances} of the \glspl{bow} of the respective entities.  

From this quantified Doc-Term-Matrix, a dissimilarity-matrix is generated. This requires a measure for the dissimlarity - in the original paper, this is what they call "normalized angular difference" - according to \cite{Derrac2015}:

\begin{align}
	ang(e_i, e_j) &= \frac{2}{\pi} * \arccos \left( \frac{\vec[m]{v_{e_i}} * \vec[m]{v_{e_j}}} { \lVert \vec[m]{v_{e_i}} \rVert * \lVert \vec[m]{v_{e_j}} \rVert }  \right)  \label{eq:norm_ang_dist} \\
	&= \frac{2}{\pi} * \arccos(1-\cos(\vec[m]{v_{e_i}},\vec[m]{v_{e_j}})) \text{, where $\cos$ is the default cosine-distance} \nonumber
\end{align}

in \cite{Schockaert2011}, they define similarity through a variation of the Jaccard-distance (IoU, Overlap-Area divided by Union-Area)

\paragraph{Embed}

Because this dissimilarity-Matrix is far too high-dimensional and sparse, a dimensionality-reduction is applied - we discussed other raeasons why that is smart before.

Multidimensional scaling but also isomap yadda yadda


\subsubsection{Filter Candidates by Classifier Performance\arrowref{sec:algorithmsteps}}
\label{sec:svm_filter_cands}

\todoparagraph{Also known as: Creating Candidate SVMs and Filter Candidate Feature Direcitons}
\autoref{ap:algo_filter}

This step brings together the entity embeddings and the extracted keyphrases. To quantify how well each semantic directions captures semantic content of the entities, a linear classifier splits those entities where the keyphrase occurs from those where it does not. The best example for such a classifier is a \gls{svm} which does not rely on the kernel trick.\footnote{kerneltrick ist ja "Projecten in nem anderen space, damit das was da linear ist bei uns nonlinear ist" und ich will linear sein)} As visually exemplified in \autoref{fig:3d_hyperplane_ortho}, the result is a hyperplane that divides the positive and negative samples (the plot a toy-example - in practice it is highly unlikely that they are clearly distinct, but the properties hold in other cases as well). Regardless of the dimensionality of the original space, this hyperplane has a one-dimensional orthogonal vector. Each of the entity-embeddings is subsequently orthogonally projected onto this orthogonal. Now the distance of this projection to the plane offset (the coordinate where it crosses the decision surface) is a scalar that encodes the distance to this decision hyperplane. \footnote{Again, if you don't understand this, look at the plot}

The ranking of the entities in terms of this distance is now what is used as their values for the feature directions, in the sense that the further away an entity is from the decision surface on the positive side, the more it has the corresponding feature. The same holds for the negative direction. This may sound initially surprising, but the point is that the original space is created based on similarity measures of the entities. Given that the \textit{Bag-Words-Hypothesis} (\autoref{sec:bow_hypothesis}) holds, they should have similar words. And those that are maximally dissimilar are as far apart from these as the space allows. To stick with the example of movies, the assumption is that movies that are maximally unscary are maximally far from the away from scary ones, in the sense that you can assume that a maximally dissimilar distribution of words from the positive class means a maximally unscary movie. So the more dissimilar to that, the less scary, so the relationship holds in both directions, even if the word scary itself occurs zero times in the descriptions of any movie on the negative class it still holds that the further away the less the concept applies. Distributional Hypothesis and what we wrote for the logic of LSA. The classifier takes the other words into account for the classification as well. The logic of this is especially clear in the case of SVMs: This classifier works by creating the hyplane such that hte margin between the positive and negative class is maximized. 

Ok so the orthogonal to the resulting decision-hyperplane is then used as axis, onto which the entities are mapped - the further away from the plane the mapping of a point onto the orthogonal, the more the entity is said to have the attribute encoded by the phrase responsible for the hyperplane. A score function compares the ranking induced by this to the ranking induced by number of occurences (or quantification-value) of the respective keyphrase of all documents, such that only those terms where the correspondance of these rankings exceeds a certain threshold are considered as candidate directions henceforth.

\q{The higher the Kappa score of a term \textit{t}, the more we consider \vec{v_t} to be a faithful representation of the term \textit{t}} \q[20]{Derrac2015}. Subsequently only those directions are considered where this classifier exceeds a certain threshold. So what's the logic behind that? As we stated before \todoparagraph{When discussing bow-representations and the reasons for quantifications and also LSA and also stop-words}, unimportant words are more or less uniformly, in any case arbitrarily, distributed throughout the corpus. The vector space that we are doing this SVM on is created on the basis of distributional semantics. The entire basis of this is that there are latent obfuscated topics, and correlations of words for these. So if a topic is very prominent in some texts but not in others, that will influence the position in the vector space. In the case of unimportant words, they are arbitrarily distributed and are not signifying a latent topic, no correlation of other words, not important for similarity. Like no reason for dissimilarity. Not indicating a cluster, because all these stopwords occur random and are thus noise. These randomness does not go along with a cluster of positions in any of the dimensions, nose gets removed by dimensionality reduction. So yes, it does make sense that the better a classifier can split between does-the-word-occur and does-it-not, the more the word is an \textit{important topic} in the sense that it explains the dissimilarity in the entities.\footnote{For a better intuition why this makes sense it is referred to \cite{Lowe}}
\todoparagraph{Do I need a plot that shows a non-faithful direction? } % grafische Darstellung von "if the ranking induced by the SVM corresponds to the count/PPMI, we see it as faithful measure", also ein beispiel wo's passt und ein Beispiel wo's nicht passt

\cite{Ager2018}: \q{if this classifier is sufficiently accurate, it must mean that whether word w relates to object o (i.e. whether it is used in the description of o) is important enough to affect the semantic space representation of o. In such a case, it seems rea- sonable to assume that w describes an important feature for the given domain.}



Okay so lets continue.

Concretely, the score used by \cite{Derrac2015} to assess the performance is not the accuracy or some other measure of the bare performance of the classifer, but rather if the ranking by distance to decision hyperplane corresponds to ranking of number of occurences (or the PPMI-score, the authers are imprecise in their wording) of that word.
The reasoning behind that becomes especially clear when considering the root of their datasets - in the case of reviews or tags it is the case that the often a word is mentioned, the more relevant the word is for that entity. And because we are using the PPMI-score, it is even more: The more salient relevant for this entity but not for the others the word is, the higher the score. That is what how we created the semantic space in the first place, by saying important ones are weighted more, those are very prominent for some but not all were important for the dissimilarity that is the basis for our embedding. So this entire thing basically looks back at our embedding and tries ot figure out which words it were that were relevant for the dissimilarity. It dissects the overall dissimilarity we had before into its components.

Okay, enough for the theory, lets talk about the implementation. \cite{Derrac2015} say that they use the Kappa-score, which is a metric that compares rankings. With that, they compare the rankings of the svm with the ranking how-important that word is. They took kappa because that is good at dealing with high imbalances in class sizes, which are definitely given.

Yet another point where \cite{Derrac2015} are really low on information what parameters they used. Sklearn allows different weighting types\footnote{\url{https://scikit-learn.org/stable/modules/generated/sklearn.metrics.cohen_kappa_score.html\#sklearn.metrics.cohen_kappa_score}} 
\todoparagraph{explain what that changes respectively}

Unfortunately, they give hardly any details, and there are many different ways how to implement that. While \cite{Ager2018,Alshaikh2020} explicitly say that they are interested in the PPMI-scores\footnote{Though the uploaded code of \cite{Alshaikh2020} does not compare rankings but raw values}, from \cite{Derrac2015} it is not even clear if they take the count or the PPMI-score. As that is highly relevant, we try many different ways of this scoring and report them in the results. We also compare the overlaps of different kappa-scores to check if the choice is as imporant as we think it is. Which scores we used and how they are written here is listed in the implementation details: \autoref{tab:kappa_measures}.

\textcite{Ager2018} compare the kappa-score to accuracy and NDCG and say accuracy works better than kappa.

So, alogorithm: For every candidate-term, take the quantifications from the doc-term-matrix and binarize it, such that we have two classes. On that we then train a linear classifier such as an SVM. On that we calculate binary classificaiton-quality-metrics, and from the ranking the kappas. resulting SVM has a hyperplane as decision surface. The distance of a point to it's orthogonal projection onto that hyperplane can be seen as proportional to how much this point is considered to be in the respective class of the SVM. One can use these distances to enduce a ranking how prototypicality. compared to other heuristics encoding it, such as the ranking induced by the per-term-frequencies of the terms for all documents, or it's PPMI or tf-idf representations.
\cite{Derrac2015} call this "measure the faithfulness of representation"

\cite{Ager2018}: \q{We say Feature *Directions* and not feature *vectors* because they are supposed to rank, not measure degrees of similarity! it only tells us "this one has the feature to a higher degree"}


In the end we have a shitton of scores, and two threshold, yielding great ones and okay ones.
Ehm, why didn't I just take the ndims*2 best ones instead of hard-thresholding??
Well, I can say it was because:

At this point we already have an estimate of how good the parameter-combination so far was: if not enough great-ones were extracted, we don't need to bother continuing.



\includeMD{pandoc_generated_latex/3_1_algorithm}




\subsubsection{Merging the extracted candidate-directions\arrowref{sec:algorithmsteps}}
\label{sec:algo:cluster}

The previous step yielded many \textit{basic feature directions} that are defined as direction of the orthogonal vector for the hyperplanes splitting each individual candidate \gls{ngram}. The performance-thresholds are set such that many more directions are generated than the demanded dimansionality of the final embedding, such that they must be clustered and merged.

This is done via the following substeps, each of whch will be closer eloaborated:

\begin{itemize}
	\item Cluster good-performing candidates by their similarity
	\item (optional) Remove uninformative features
	\item Recalulate the direction of the cluster
	\item (optional) Find a representative name for the cluster
\end{itemize}

\paragraph{Clustering the candidates}

Clustering refers to an unsupervised algorithm that groups items based on some notion of similarity. In our case the assumption is that semantically similar concepts have \textit{close} vectors, which is given due to the \gls{bow}-hypothesis that states that the underlying structure of our dataset is expressed by the usage of related words ((\autoref{sec:bow_hypothesis}), \ref{sec:lsi}).\footnote{In case of the Siddata-dataset, it may mean that in courses that contain the word \textit{computer} have a high chance of also containing \textit{program}.} As these vectors in principle only encode a direction, their similarity can be calculated by their \gls{cos}.

The clustering should reduce the number of features and also ensure that the resulting directions are different enough. Note that unlike \eg in Principal Component Analysis (PCA), the suggested here techniques do not enforce orthogonality, such that the resulting directions may remain linearly dependent to a certain degree. As in the final embedding only the projection onto those directions is relevant, it must be ensured that enough of the data's original variation is covered by these directions. To ensure that, we follow \gencite{Derrac2015} suggestion to allow for redundancy by extracting twice as many directions than the original \gls{vsm} dimensionality. 

This implementation implements the original clustering-method of \cite{Derrac2015}:

First, we consider the best \textit{basic features} as \textit{main directions}. For that, we select one of the scores calculated in the previous step and select all candidates that exceed a threshold (\cite{Derrac2015} suggest $\kappa \geq 0.5$).
To get the directions, we follow the following algoritm:

\vspace{-1ex}
\begingroup
\verbatimfont{\footnotesize}%
\begin{verbatim}
	greats = filter(candidates, 0.5)
	directions = greats.argmax()
	for nterm in range(ndims*2):
	  greats = set(greats)-set(directions)
	  distances = {cand: min(comparer(cand, compareto) 
	                for compareto in directions) 
	                  for cand in greats}
		directions.append(compares.argmax)
\end{verbatim}
\endgroup
\vspace{-1ex}

This starts with the best candidate and then iteratively adds the one from the set of top-scoring candidates that most dissimilar to the set of final directions. The result is a set of $ndims*2$ main directions, which are henceforth considered the Cluster centers. Subsequently, all leftover terms from $T^{0.5}$ as well as all terms from $T^{0.5}$ are added to the respective cluster whose direction they are most similar to. 

\textcite{Derrac2015} used the \gls{cos} distance to measure the respective similarities. This may lead to unexpected situations (discussed in \autoref{sec:discuss_points}). As alternative similarity metric that does not rely on the angle between their vectors, \cite{Alshaikh2019} suggest to use the overlap of the positive-samples of two features as similarity. This was however not yet implemented in this thesis.

Alternatively to the described algorithm, is is also possible to use the popular \textit{k-means} algorithm for clustering, as done by \cite{Ager2018}. We do not present results for this approach here however, as it lead to a substantial increase in runtime, without affecting performance much. In the development we also noticed that many clusters contain a lot of irrelevant terms. To alleviate this, we experimented with different techniques, for example setting minimal similarity thershold that must be given for a term to be added to a cluster, however so far no formal evaluation to test how this affects performance was performed.

\removeMe{
\todoparagraph{Doesn't one bad word in the cluster destroy it?} No. It *IS* okay if common words (like "course") are in clusters, it is NOT the case that as soon as the word occurs once it is said to have a certain property. ("Wenn cluster-threshold zu groß, kommt “A1” in ein cluster mit “Course” and everything is over" is FALSE). However it IS not too good -> A cluster with many words like "course" in it has a high degree of randomness (there is no information gain by such words, it occurs random across courses, a cluster of courses that mention that they are courses is useless) The word occurs randomly, if a course is assumed to have a certain property because of that it's certainly wrong
}

\paragraph{Find Cluster-Direction}

So far, we have a set of clustered canidates terms, each of which has an individual direction. The final \textit{feature-direction} must subsequently be found from the elements of the cluster. For that, \cite{Derrac2015} and \cite{Ager2018} define the cluster centroids as the average of all (normalized) vectors per cluster. In our experiments, however, we noticed that the final direction tends to be too much affected by irrelevant cluster-elements. Because of this, we experimented with other techniques to determine the cluster direction. Two other considered methods include to just consider the direction of the cluster-center, or to weight the influcence of each cluster-element by their kappa-score.

The best performaning method however was the \textit{reclassify}-algorithm, which (similar to \cite{Alshaikh2020}) finds the cluster-direction by training a new classifier that splits those \glspl{entity} that contain \textit{any of the elements} from the cluster from those that do not, analogous to the previous step (except that it requires to generate and quantify a new frequency matrix from the sums of the individual counts). Doing this however often leads to the opposite problem than the previous step, namely that for many clusters there are almost no entities that do not contain at least one of the cluster elements. To counter this, we instead trained a classifier to split the 30\% of entities with the highest \glspl{quant} from the 30\% of entities with the lowest \glspl{quant}. A comparison of this algorithm with the method of \cite{Derrac2015} is given in the Appendix as \autoref{tab:text_per_dim}. As \cite{Alshaikh2020} already performed formal experiments with this that have shown its superior performance, all generated results of this work rely on this algorithm. 

\paragraph{Bad Clusters}

After these steps, we finally have the vectors that correspond to semantic directions. As there however still be clusters of uniformative terms, \textcite{Alshaikh2020} have an additional step to remove uninformative cluster. As this however bases on another clustering algorithm used by the author (namely \textit{affinity propagation}) which does not allow to specify the number of clusters, it was not implemented in the scope of this thesis.

\paragraph{Find a representative Cluster Name}

An important advantage of the clustering process is that it makes the extracted directions more \textit{descriptive} due to the fact that they are described by several phrases instead of only one. However, it may be helpful for an attractive user interface to find the single \textit{best} description of the cluster direction by its element.

An analysis of \cite{Carmel2009} showed that a statistical method to extract features from clustered text corpora identified the labels of human annotators as one of the top five most important terms in only 15\% of cases, implying \q{that human labels are not necessarily significant from a statistical perspective} \cite[139]{Carmel2009}. In their paper, they suggest several methods to find one representative name for the cluster. 

\cite{Derrac2015} and its follow-ups \cite{Ager2018,Alshaikh2020} did care about such methods and instead use either the name of the cluster center as its description or the cluster center plus two other sample elements. This work experimented with several techniques to get a more representative direction name. One of these techniques used the KeyBERT-algorithm (see \autoref{ap:details_keybert}) to find the term that is most similar to the set of terms making up the cluster. We also experimented with a method that embeds the cluster terms using \gls{word2vec} and returns  the word behind the vector that is closest to their average (which is not neccessarily part of the original set of words). Similarly to \gls{lsa} (\autoref{sec:lsi}), it is also possible to consider the entity whose \textit{pseudo-document} embeddings is closest in direction to the cluster direction.

S so far the best technique to find a cluster-name was not evaluated yet. All considered methods (of \mainalgos and here) that formally evaluate the corresponding feature-directions work independently of the actual cluster name. This is unfortunate, because \textit{subjectively}, the name of the respective directions is very important for the usability of any recommendation engine based on this work. Especially this subjectivity however indicates that the only way to evaluate the cluster-names is with study of human subjects.

\todoparagraph{Kappa in den Glossary!!}

\subsubsection{Postprocessing the Feature-Directions\arrowref{sec:algorithmsteps}}
\label{sec:algo:postprocess}

This step was the main contribution of the work of \textcite{Ager2018}. As has been shown that it increases the algorithm performance only slightly while adding a substantial amount of work, it was not implemented in the scope of this thesis. The Modification is described in \autoref{sec:ager}, its gist is to use the rank for quantified summed count of any cluster-word as weak supervision signal to distort the embeddings such that their feature directions better correspond to this value.

\subsubsection{Re-Embedding the entities into the new space\arrowref{sec:algorithmsteps}}
\label{sec:algo:reembed}

In the end we re-embed the entities into a space where each of the vector components is a semantic directions and the value are the respective \gls{rank}ings. That's what we then finally call its \textbf{feature-based representation} 

NOT a change of basis, only ordinal scale level bc rank, no linear independence.

from Alshaikh2020: "The learned vectors will be referred to as feature directions to emphasize the fact that only the ordering induced by the dot product d_i · e matters"

Maybe better idea is maths?






\subsection{Modifications from \textcite{Ager2018,Alshaikh2020}}

\subsubsection{\textcite{Ager2018}}
\label{sec:ager}


The main contribution of \textcite{Ager2018} is a postprocessing step that changes the final space such the ranking of entities \wrt each feature direction more closely mimics the ranking of frequencies of that direction's cluster words. The reasoning is that the original embeddings from which the feature directions are created are based on global similarity. This makes it very vulnerable to outliers which often take up extreme positions. If one now creates the feature directions from the space, these outliers are assumed to have certain properties. So the space is optimized for that, which limits the quality of feature directions in the space. Problem again is global similarity: If one entity ranks high for a feature, it is very likely that another entity that is close to that will also rank high for this feature, even though it may be something completely different. So to get better feature directions one has to distort the space. 

They do this fully unsupervisedly as an extra step after the full pipeline of \cite{Derrac2015}, by again using the BoW representation of the entites. After the clusters are collected and the entities re-embedded, for each feature a new ranking is computed by the summed frequency of any of a cluster's words per feature and entity. Each entity is thus represented as Bag-of-Clusters and again scored with PPMI to generate a ranking for each cluser/direction. This ranking is then used as a target for a simple \gls{ann} that distorts the space representation.

Generally this is a great idea. Among others the explicit usage of all cluster-words should help, as it is a lot less sparse than a single word that only occurs in 0.6\% of entites. However the results of their approach are mixed: For some of their considered datasets the fine-tuning even decreases performance - according to the authors especially \q{when the considered categorizes are too specialized} \cite{Ager2018}, because the resulting space is too much distorted towards the selected features.\footnote{See also \autoref{tab:f1_placetypes_long}} Considering its bad performance, this contribution was not considered in this work.

\textcite{Alshaikh2020}

\todo

\removeMe{
\subsection{Concluding stuff for algo}

\subsection{Features and differences to original algorithm}

\includeMD{pandoc_generated_latex/3_features_differences}

\subsection{Reasonable params}

\includeMD{pandoc_generated_latex/3_reasonableparams}

\subsection{Algorithm Complexity}
}


\section{Architecture}
\label{sec:architecture}

As elaborated in \autoref{sec:reproducibility}, one of the main motivations for this thesis was to create a publicly available \textit{open-source} version of the algorithm that is easily \textit{understood} and \textit{reproduced}, \textit{adaptable} for other datasets and methods, as well as fast and \textit{scalable}, meaning it can be run maximally efficient on single machines but also on compute clusters, such as the \acrshort{ikw} Grid.
%TODO: Hier schon eindeutig sagen dass es auf ner single machine infeasibly lange läuft und deswegen der ganze Bums fürs Grid nötig war!!

% Main goal: BETTER ARCHITECTURE. Most important things for that: scalability, modularity, transparency, reproducibility, understandability, objectiveness, systematicacy, sustainability, adaptability
% describing this because I want to encourage extending the code etc and for that not only the algorithm but also the architecture should be described 
% and I think that was successful: This codebase contains everything and finally fulfills code-standards! 

This section will outline the architecture that was developed in order to achieve the aforementioned results. The resulting pipeline is the result of a lot of trial-end-error, but fulfills all of the aformentioned criteria, dealing with vastly differing sizes and kinds of datasets, minimizing runtime wherever feasible and allowing for a multitude of parameters at every step of the process. %TODO: don't like this paragraph, lieber später nohcmal auf die design principles eingehen und sagen dass sie alle fulfilled sind.

The rest of this section will go into further detail regarding the architecture of the resulting code-base. \todoparagraph{it will start with xyz and then asdf and then yaddayadda}

\subsection{Implementation}

The associated program is written by the author of this work and licensed under the \emph{GNU General Public License} (GNU GPLv3). The source code is written in the Python Programming Language and available digitally on GitHub\footnote{Source code: \url{https://github.com/cstenkamp/derive_conceptualspaces/}\\Source of this Document: \url{https://github.com/cstenkamp/MastersThesisText/}\\Compiled Document: \url{https://nightly.link/cstenkamp/MastersThesisText/workflows/create_pdf_artifact/master/Thesis.zip}}. In order to ensure that no work after the deadline is considered, it is referred to the signed commits \todoparagraph{COMMIT} and \todoparagraph{COMMIT}. 

The code is a proper python-package that can be installed into any Python 3.10 environment using for example python's default package manager pip:\\ \mytokens{pip install git+https://github.com/cstenkamp/derive_conceptualspaces.git@main}~ .\\ It can then be run using \mytokens{python -m derive_conceptualspace <COMMAND>} \footnote{The command \mytokensfnote{python -m derive_conceptualspace --help} gives a peak into what sub-commands can be used}. For more information on how to invoke the code base with these commands it is referred to \autoref{ap:usecase_click}

To guarantee reusability of this code-base, there is also a \emph{Dockerfile}\footnote{{\url{https://github.com/cstenkamp/derive_conceptualspaces/blob/main/Dockerfile}}} that allows to easily create a \emph{Docker-Container\footnote{\url{https://www.docker.com/resources/what}}} from it\footnote{A Container can be thought of as a lightweight virtual operating system, in which the codebase is bundled together with all required dependencies, libraries and configurations, enabling users install this software on any system without having to download or install anything besides this container, irrespective of operating system or software versions on the host \acrshort{os}. For more info about the container, it is referred to \url{https://github.com/cstenkamp/derive_conceptualspaces/blob/main/doc/docker_intro.md}.}.

\subsection{Modularity}

The developed algorithm consists of clearly divisible components (as demonstrated in \autoref{fig:dependency_graph}), where the runtime for each of the steps is roughly in the same order of magnitude. All of the aforementioned (\autoref{sec:algorithm_steps}) steps are itself algorithms with many \gls{param} each. Furthermore, the framework described here does not even require particular algorithms for the individual components, but rather a classes of algorithms like \emph{dimensionality reduction techniques}. This means that in practice, there is a combinatorical explosion of settings and \glspl{param} that must be experimented with in order to find the best-performing one. Because of the clear modularity of the algorithm however, many of these become only relevant in a later step of the pipeline. Due to this, it is reasonable to make the architecture as modular as possible, storing interim results before every step, such that two parameter-combinations that differ only in \eg the fourth step of the pipeline can share the intermediate results up to that point, keeping the required computation to a minimum. 

The design principle of maximal modularity is the cornerstone of the developed pipeline. All of the interim results store the configurations that were required for the respective algorithm (and forward the ones of the input-files they transformed), as well as the created output and plots. When there are different possible algorithms for a step, it is ensured that its result are of the same format, as required by the next step. Many of the individual steps generate additional plots that can be used as sanity-checks to quickly inspect if the results so far are reasonable.

\subsubsection{Workflow Management}

A pipeline where multiple intermediate files for different parameter-combinations are created introduces the problem of \emph{dependency resolution}: Ultimately, there is supposed to be one final file for every combination. This file however relies on intermediate files, which in turn rely on intermediate files. To resolve these dependencies, there are many existing \textbf{Workflow Management Systems}. For this thesis, \textbf{Snakemake}\footnote{\url{https://snakemake.readthedocs.io/en/stable/}} \cite{Molder2021a} seemed the right choice.

Snakemake defines a small comprehensible domain-specific language ontop of python. With this, a workflow is described in terms of individual \textbf{rules}, each of which defining how an \textbf{output} is generated from several \textbf{inputs} using code or shell-commands. Through \textbf{wildcards}, these rules can be generalized and hyperparameters introduced \cite{Molder2021a}. The job of Snakemake is to infer a \gls{dag} from these, finding for every rule in the dependency tree for the demanded file an output that generates the required inputs, and to create jobs for all required instanciations of the wildcards if the required files are not already present. Importantly, Snakemake then also handles the inevitable scheduling problem: Due to (explicitly specified) restrictions of \acrshort{cpu} and \acrshort{ram} and the nature of the unresolved dependencies, not all jobs of the workflow can be executed simultaneously. Its scheduler favors maximal utilization of \acrshort{cpu} and parallelisation for minimal execution time \cite{Molder2021a}. Especially relevant was also that it allows to schedule these jobs on high performance clusters and computation grids, and supports among others the scheduling system \gls{sge} which is used to orchestrate jobs at the \gls{ikw} grid. Configurations for the grid, like the maximal runtime or the amount of \gls{ram} and \glspl{cpu} to request, can be specified per-rule as well as in special configuration files.
%TODO: gibt noch 1-2 buzzwords from paper, ich kann schonmal aufs Grid hinaus und dann halt der wann-ist-snakemake-sinnvoll-und-wann-nicht.

Snakemake was chosen because it is a lightweight system ontop of python, adding only a few lines of code to specify what inputs and outputs are created ontop of the \gls{cli} that is necessary to run and debug individual steps anyway. It is a useful tool if the workflow can be divided into rougly equally long steps which can run independently and heavily parallelized (possibly on multiple machines) with an optimal usage of resources. Its file-centric dependency resolution system allows to fill in missing steps seemlessly when working on specific configurations for later step, but on the other hand requires unintuitive customization if instead configuration-files with explicit parameter-choices declare the demanded output for dynamically generated filenames. Also it unfortunately doesn't allow debugging and has a comparably small community\footnote{As of \DTMdisplaydate{2022}{03}{16}{-1}, there are only 1256 question tagged ``snakemake'' on StackOverflow (\url{https://stackoverflow.com/questions/tagged/snakemake})}. \autoref{ap:usecase_snakemake} shows the different ways the full pipeline can be invoked using Snakemake.

%wenn viele parameter die an gwissen punkten relevant werden und später nicht mehr, wenn viele param-kombis, it's main thing is the automatic dependency resolvement (which means I can just tell it "hey I need this file" (automatically creating missing stuff), but with config-files you're abusing it. good for optimal CPU/RAM usage. Good if independent parallelzed steps, not if one main step. Have to abuse it for configs, no good way to debug, small comunities, nondynamic I need nondynamic filenames that are set from the start of the execution 


\subsection{Modes of Execution / Use-Cases}

It is possible to run the full pipeline for individual files as well as for a set of \gls{param}-configurations specified via configuration files, but also possible to run individual steps to inspect or debug the respective steps. To inspect and compare results it is possible to load all available parameter-configurations, as well as the complete history for a certain combination, listing the generated outputs and metrics. Further, individual configurations can be loaded in \emph{Jupyter Notebooks} to generate and export plots and tables from them (like the ones used in this text). The three main ways of exectution are:
%TODO: deutlicher drauf eingehen dass man wegen dem ganzen bums mit intermediate files undso speziell drauf achten muss dass 
% * keine plots/prints verloren gehen
% * man mitschreibt wann welche configs genutzt werden
% * immer eindeutig drauf geachtet wird dass dependencies für genau die konfigurationen as demanded verwendet werden!! 

\begin{description}[style=unboxed]
	\item[Running individual Steps per \gls{cli}] is the mode of choice when working on custom steps, as it allows to attach debuggers and executes in the main thread. If a later step is executed, it is also possible to automatically generated its required dependencies using the workflow-definition. Passing configurations is possible using configuration-files, command-line-arguments or enviroment-files/-variables. For usage-examples, see \autoref{ap:usecase_click}.
	\item[Loading existing Configurations for inspection] especially in Notebooks, allowing to easily load a complete configuration including all its dependencies to inspect and plot (intermediate) previously created results and outputs, also allowing to iterate over several configurations to compare their results\footnote{The tables used in thesis are also automatically exported as \LaTeX- code from the functions available there, as specified in their respetive references.}. For usage-examples, see \autoref{ap:usecase_notebook}.
	\item[Running/Scheduling a Workflow] This mode is used to execute several \gls{param}-combinations at once, specified via configuration-files. Thanks to heavy integration for cluster scheduling systems, this allows for heavily parallelisation of jobs. Executing such a workflow on computation clusters is special case of this and elaborated further in the following section. For usage-examples, see \autoref{ap:usecase_snakemake}.
\end{description}
% 3 ways: Snakemake for shitton of param-combinations, invididual steps via the CLI for looking, debugging, creating, and 
% context-loading for jupyter to inspect and plot results - allowing load-context, where you can call eg. `print(ctx.display_output("embedding"))` of every component, read in several configs, iterate over them, re-create plots, allow for show-data-info showing where plots are first used, ...


\subsubsection*{Running on the \gls{sge}}

Due to a combinatorical explosion in the \gls{param}-space as well as the computational complexity of the algorithm, running the pipeline a sufficient amount of parameter-combinations would take several weeks on a single machine\footnote{\todoparagraph{Give a few examples!}}. As the \gls{ikw} at the \gls{uos} owns a dedicated computation grid\footnote{\url{https://doc.ikw.uni-osnabrueck.de/content/grid-computing}} with considerable modern hardware\footnote{Currently comprising, among many others, of 26 machines with an i7-11700 \gls{cpu} and 64 GB \gls{ram}} which uses the \gls{sge} as workload manager, which is supported by snakemake, it was the obvious candidate. Snakemake encodes special configurations for clusters using \emph{profiles}\footnote{\url{https://snakemake.readthedocs.io/en/stable/executing/cluster.html}}, and there exists a profile for the Sun Grid engine\footnote{\url{https://github.com/Snakemake-Profiles/sge}}. Unfortunately, this default configuration does not take into account many of the pecularities of the \gls{ikw} grid and it needed to be heavily customized in order to work. Foremost, all available machines to \me have a runtime-limit of 90 minutes, which means all of the algorithm-steps that take longer than that must be able to be interrupted and gracefully shut down before getting killed and pick up the work on a new machine afterwards (including the job responsible for the workflow scheduling itself). Additionally, the arguments to request resources (such as \emph{memory} or \emph{parallel environments}) often differ from the documentation, and the \emph{accounting file} which keeps track if jobs succeeded is not available to users, so a custom one must be written. Resolving these and other issues required changing the available profile heavily, so the result was open-sourced\footnote{The resuling Snakemake-Profile is available and documented at \url{https://github.com/cstenkamp/Snakemake-IKW-SGE-Profile}. Note that it is heavily customized to the specific engine and thus includes explicit machine names or runtimes. This repository also contains convenience-terminal-commands to inspect failed pipeline-steps or to show the current progress of the current run. A sample output of the latter is presented in \autoref{lst:joblog}. Furthermore it contains \mytokensfnote{.sge}-files and shell-scripts to schedule or run a requested workflow (see \autoref{ap:usecase_snakemake})}. 


Scheduling on such engines interestingly unveils a whole new set of ``hyperparameters'' that have to be optimized to use the available hardware as efficiently as possible: there are limits of how many slots are available per user, there is a fixed walltime (and interrupting and restarting leads to overhead), and the effiency of multiprocessing is not linear in the number of threads per process. Thus, depending on the size of the dataset, resources must be divided among the steps with care. The required resources of the rules are accordingly dynamically allocated in the rule-descriptions of the workflow manager.

While the code required to scalably run on the \gls{ikw}-grid required much more work than expected, the result fulfills all demands perfectly, %TODO: WHAT demands
and the 64 allocated \emph{parallel environments} (slots) are maximally utilized, while most of the complexity of the scheduling system is abstracted away\footnote{To the best of \my knowledge, no attempts going beyond simple \mytokensfnote{.sge}-files as job-descriptions were attempted on the IKW-grid before, and much of the available documentation turned out to be false information (as consultations with the grid's administrator have shown).}. The workflow is installed and run with a single (well documented) command and can be customized using explicit configuration-files. A sample output of the custom-made watcher is listed in \autoref{lst:joblog}.  


\begin{widepage}
	\lstconsolestyle
	\lstinputlisting[
		caption={[Sample terminal output of the custom watcher, when running a full configuration on the grid.]Sample terminal output of the custom watcher, when running a full configuration on the \gls{ikw}-grid. The script lists the currently running jobs continously, including their progress and runtime and informs of finished jobs and failed jobs. There is another script that summarizes the progress as per snakemake's dependency-graph.}, 
		label={lst:joblog},
		float,
		floatplacement=h!,
		xleftmargin=-0.5cm, 
		xrightmargin=-0.5cm,
		]{listings/joblog\_grid.txt}
	\lstdefaultstyle
\end{widepage}

% \includeMD{pandoc_generated_latex/chapter_methods_section_architecture}

\subsection{Conclusion}

It was originally unexpected, but implementing an appropriate architecture for the present codebase has been a major focus of work for this thesis, and the result fulfills all of the desired design criteria: 

% reproducibility alone is not enough to sustain the hours of work that scientists invest in crafting data analyses. Here, we outlined how the interplay of automation, scalabil- ity, portability, readability, traceability, and documentation can help to reach beyond reproducibility, making data analyses adaptable and transparent.

\begin{description}[style=unboxed]
	\item[Modularity] has been the main focus in the design, so exchanging components or running individual steps is easy and intuitive.
	\item[Scalability] is reached thanks to massive parallelisation wherever possible as well as a professional workflow management system that is perfectly adjusted to the available cluster engine but also highly customizable for other engines.
	\item[Reproducibility and Adaptability] are guaranteed by rigorous encapsulation of components, completely automating the full data-analysis-pipeline, open-sourcing the code as proper package and containerization of the entire codebase for guaranteed and worry-free setup on any machine or compute cluster. The exact \gls{param}-combinations of \mainalgos are included (see \autoref{ap:yamls_for_origalgos}), allowing to re-create even the original papers using this code-base. Running the code on new datasets is extensively documented\footnote{\todoparagraph{TODO: link to that}} and a matter of minutes. Extending or exchanging steps of the pipeline is seamless due to a consistent and understandable data schema, and pre-existing analysis-notebooks can easily create informative plots and figures.
	\item[Transparency and Understandability] are ensured due to rigorous documentation\footnote{\todoparagraph{Link github-documentation!}} (among others in this thesis) at any level of detail, from rough descriptions to concrete code-examples. Code, documentation and used data are publicly and easily available. Many analyses are inlcuded with the source-codes, for example allowing to visualize all steps of the process that can work with arbitrary numbers of dimension interactively in 3D. Code, data and configurations are clearly divided. All steps of the pipeline are very explicit about the used configurations and dependencies (making them traceable) and generate output at configurable levels of verbosity. All intermediate output can be re-accessed using helper commands (see \footnote{\todoparagraph{ref the appendix with the show-info-command and a notebook with} \mytokensfnote{create_svm("mathematik", embedding, dcm, pp_descriptions, highlight=["Informatik A: Algorithmen", "Informatik B: Grundlagen der Software-Entwicklung"])}}), including clear traces of the first usage of parameters (as \eg in a plot as depicted in \autoref{fig:dependency_graph})
	% it is crucial that the analysis code is as readable as possible such that it can be easily modified (looking at you, 40 unnamed cmd-args!)
	% code is readable and well-documented 
	% mit 2 Zeilen code kannst du dir in nem Jupyternotebook nen 3D-Plot anzeigen mit ner SVM die "Mathematik" von nicht-mathe trennt, mit gehighlighted ob "Informatik A" und "Informatik B" beeinander sind
\end{description}