\chapter{Introduction}

In this thesis, I want to generate a conceptual space for the domain of university courses, automatically created in data-driven way from their descriptions.

\section{Reading Instructions}

Throughout this thesis we'll use the "we" als \emph{Pluralis Auctoris}, signifying objectivity in science. This doesn't mean that anybody but the author of this thesis did anything for it TODO: see how rüdiger said that in his acknowledgements \url{https://www.rki.de/EN/Content/infections/epidemiology/signals/projects/Optimization_Outbreak_Detection_MasterThesis_Busche_2019.pdf?__blob=publicationFile}

\paragraph*{Document Structure}
\todoparagraph{
	Chapter 1 is Intro, with motivations etc. 
	Chapter 2 is Background, explaining the the required concepts - what are conceptual spaces generally, how does the base algorithm work, what kinds of algorithms occur in it. I am explaining the rquired algorithm before the main algo such that I can rely on definitions there.
	Chapter 3 is then methods. Dataset, algorithm, architecture. 
	4 results, 5 conclusion, that's it.
}

\paragraph*{Regarding Terminology}

Throughout this thesis, many abbreviations, symbols and technical terms will be used. \todoparagraph{I hope that all of that cannot be exptected to be known by the reader are defined. At the end of this thesis there is a} \nameref{sec:glossary} \todoparagraph{with the subsections yadda yadda. If you are reading this document digitally, all occurrences of the terms described there should be a hyperlink (as are all section, table, figure, etc references). If you don't have the version with colored hyperlinks, you can download it here:} \url{https://nightly.link/cstenkamp/MastersThesisText/workflows/create_pdf_artifact/master/Thesis.zip}

\section{Motivation}

\includeMD{pandoc_generated_latex/1_1_motivation}

We see evidence that VSMs capture the meaning of words by how simple vector arithmetic corresponds to semantic: 
\begin{align}
	vec(king) - vec(man) + vec(woman) &\approx vec(queen) \nonumber \\ 
	vec(planet) + vec(water) &\approx vec(earth)  \label{eq:w2vregularity}\\
	vec(house) + vec(movie) &\approx vec(cinema) \nonumber
\end{align}
%TODO: the latter two https://devmount.github.io/GermanWordEmbeddings/


\section{Research Questions \& Thesis Goals}
% Goals of this work & Research-Questions
\label{sec:goals_research_questions}

\includeMD{pandoc_generated_latex/1_2_thesisgoals}
