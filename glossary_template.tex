% see https://tex.stackexchange.com/q/180660/108199, https://tex.stackexchange.com/questions/1226/how-to-make-latexmk-use-makeglossaries, https://de.overleaf.com/learn/latex/Glossaries, https://tex.stackexchange.com/a/182765/108199, https://tex.stackexchange.com/q/270098/108199, https://tex.stackexchange.com/q/316777/108199, https://tex.stackexchange.com/a/509325/108199, https://texblog.org/2014/04/01/multiple-glossaries-in-latex/, https://de.overleaf.com/learn/latex/Glossaries, https://texwelt.de/fragen/24504/glossaries-extra-liste-der-seiten-kleiner-setzen-und-vertikalen-leerraum-entfernen, rc-file: https://tex.stackexchange.com/a/39782/108199 & https://tex.stackexchange.com/a/541990/108199

% especially https://tex.stackexchange.com/a/364059/108199, https://tex.stackexchange.com/a/360368/108199, https://www.dickimaw-books.com/gallery/glossaries-styles/

%TODO: If I don't want to have a COMPLETE pagelist, see https://tex.stackexchange.com/q/246818/108199

\usepackage[
  acronym,
  style=long3colheader,
  hyperfirst=true,
  toc,
  section=section,
  acronym,
  numberedsection=nameref, %https://tex.stackexchange.com/a/253961/108199
%   nopostdot,
%   nonumberlist, 
  ]{glossaries}
\usepackage{glossary-longbooktabs}
\usepackage{tabu}

\setlength{\glsdescwidth}{11cm}
\newlength{\glsnamewidth}
\setlength{\glsnamewidth}{2cm}

\renewcommand*{\glsautoprefix}{glo:}
\renewcommand{\glsnamefont}[1]{\textbf{#1}}
\renewcommand{\glspagelistwidth}{1.8cm} %such that the "Page List" is one line
% \renewcommand{\glossarypreamble}{\vspace*{-0.2cm}} % space after caption
% \renewcommand{\glsgroupskip}{\vspace*{0.2cm}} 
\let\oldglossaryheader\glossaryheader
\renewcommand{\glossaryheader}{\vspace{0.25cm}\oldglossaryheader}

\setglossarypreamble[\acronymtype]{
    The abbreviations used throughout the work are compiled in the following list below. Note that the abbreviations denote the singular form of the abbreviated words. Whenever the plural forms is needed, an \textit{s} is added. Thus, for example, whereas ANN abbreviates \textit{artificial neural network}, the abbreviation of \textit{artificial neural networks} is written as ANNs.
}

% =========================================================

\newglossary[tlg]{customs}{syi}{syg}{Custom Terms}
\newglossary[dlg]{symbols}{dld}{ddn}{Symbols} 
\newglossary[tlg]{defs}{tld}{tdn}{Definitions} 
% \glsaddkey{unit}{\glsentrytext{\glslabel}}{\glsentryunit}{\GLsentryunit}{\glsunit}{\Glsunit}{\GLSunit}
% \newglossary[slg]{units}{syi}{syg}{Units} %TODO: I'd like to be able to use this one, but latex...

% =========================================================


\newglossarystyle{long3colheaderMINE}{%
    \setglossarystyle{long3colheader}% base this style on the list style
    \renewenvironment{theglossary}{
        \begin{longtable}{@{}p{\glsnamewidth} p{\glsdescwidth} p{\glspagelistwidth}@{}}}%
        {\end{longtable}}%
    \renewcommand*{\glossaryheader}{%  Change the table header
        \bfseries Notation & \bfseries Description & \bfseries Page List \\[0.5em]
        %       \hline
        \endhead}
}

\newglossarystyle{3colger}{%
    \setglossarystyle{longragged3col}% base this style on the list style
    \renewenvironment{theglossary}{% Change the table type --> 3 columns
        \settowidth{\dimen0}{\bfseries Notation}%
        \settowidth{\dimen1}{\bfseries Unit}%
        \glsdescwidth=\dimexpr\linewidth-\dimen0-\dimen1-4\tabcolsep\relax
        \begin{longtable}{@{}l l p{\glsdescwidth}@{}}}%
        {\end{longtable}}%
    %
    \renewcommand*{\glossaryheader}{%  Change the table header
        \hspace{0.08cm} \bfseries Notation & \bfseries Unit & \bfseries Description \\
        %       \hline
        \vspace{0.05cm}
        \endhead}
    \renewcommand*{\glossentry}[2]{%  Change the displayed items
        \hspace{0.08cm} \glstarget{##1}{\glossentryname{##1}} %
        & \glsunit{##1}
        &  \glossentrydesc{##1}  \tabularnewline
    }
}


\newglossarystyle{2colacro}{%
    \setglossarystyle{longragged}% base this style on the list style
    \renewenvironment{theglossary}{% Change the table type --> 3 columns
        \settowidth{\dimen0}{\bfseries Acronym}%
%       \settowidth{\dimen1}{\bfseries Description}%
        \glsdescwidth=\dimexpr\linewidth-\dimen0-\dimen1-4\tabcolsep\relax
        \begin{longtable}{@{}l p{\glsdescwidth}@{}}}%
        {\end{longtable}}%
    %
    \renewcommand*{\glossaryheader}{%  Change the table header
        \bfseries Acronym & \bfseries Description \\
        %       \hline
        \vspace{0.05cm}
        \endhead}
    \renewcommand*{\glossentry}[2]{%  Change the displayed items
        \glstarget{##1}{\glossentryname{##1}} %
        &  \glossentrydesc{##1}  \tabularnewline
    }
}


% -----------------
% Acronym-styles
% -----------------

\newglossarystyle{myacronymstyle}{%
  \renewenvironment{theglossary}%
    {\begin{longtabu} to \linewidth {lX}}%
    {\end{longtabu}}%
  % Header line
  \renewcommand*{\glossaryheader}{}%
  % indicate what to do at the start of each logical group
  \renewcommand*{\glsgroupskip}{\tabularnewline}% What to do between groups
  \renewcommand*{\glossaryentryfield}[5]{%
    \glsentryitem{##1}\glstarget{##1}{##2}
    & ##3\glspostdescription ##5% Description
    \\% end of row
  }
}

% -----------------
% Symbols-styles
% -----------------

\newglossarystyle{mysymbolstyle}{%
  %\renewcommand{\glossarysection}[2][]{}% no title
  \renewcommand*{\glsclearpage}{}% avoid page break before glossary
  \renewenvironment{theglossary}%
    {\begin{longtabu} to \linewidth {clXc}}%
    {\end{longtabu}}%
  % Header line
  \renewcommand*{\glossaryheader}{%
    % Requires booktabs
    %\toprule%
    \textbf{Symbol} & \textbf{Name} & \textbf{Description} & \textbf{Unit}%
    \tabularnewline%
    \tabularnewline%
    %\midrule%
    \endhead%
    %\bottomrule%
    \endfoot%
  }%
  % indicate what to do at the start of each logical group
  \renewcommand*{\glsgroupskip}{\tabularnewline}% What to do between groups
  \renewcommand*{\glossentry}[1]{%
    \glsentryitem{##1}% Entry number if required
    \glstarget{##1}{\glossentrysymbol{##1}} &
    %\glossentrysymbol{##1} & % Symbol
    \glossentryname{##1}    & % Name
    \glossentrydesc{##1}    & % Description
    \glsentryuseri{##1}%      % Unit in User1-Variable
    \tabularnewline%
  }%
}