\usepackage{amsfonts}
\usepackage{siunitx}
\newcommand*{\Z}{\mathbb{Z}}
\newcommand*{\Q}{\mathbb{Q}}

%TODO: If I don't want to have a COMPLETE pagelist, see https://tex.stackexchange.com/q/246818/108199

% =========================================================
% UNITS

% \newglossaryentry{m}{
%     name=\ensuremath{m},
%     description={Meter},
%     unit={\si{m}},
%     type=units
% }

% \newglossaryentry{energyconsump}{
%     name=\ensuremath{P},
%     description={Power},
%     unit={\si{kW}},
%     type=units
% }

% =========================================================
% SYMBOLS

% \newglossaryentry{symb:Pi}{
%     name=\ensuremath{\pi},
%     description={geometric value},
%     type=symbols,
%     sort=P
% }

% \newglossaryentry{integers}{
%     name=\ensuremath{\Z},
%     type=symbols,
%     description={the ring of integers}, 
%     sort=Z
% }

% \newglossaryentry{rationals}{
%     name=\ensuremath{\Q},
%     type=symbols,
%     description={the field of rational numbers}, 
%     sort=Q,
%     % nonumberlist,
% } 

\newglossaryentry{vector-space}{
    name=\ensuremath{V},
    type=symbols,
    description={a vector space}, 
    sort=V
}

% =========================================================
% DEFINITIONS

\newglossaryentry{ngram}{
    type=defs,
    name=n-gram,
    description={
        n-grams are sequences of consecutive words of length $n$. For example, the text "I eat lunch" contains the 1-grams ["I", "eat", "lunch"], the 2-grams ["I eat", "eat lunch"] and the 3-gram ["I eat lunch"]
        }
}


\newglossaryentry{stopword}{
    type=defs,
    name=stopword,
    description={
        TODO: do
        }
}

\newglossaryentry{lemma}{
    type=defs,
    name=lemma,
    description={
        The \textbf{lemma} of a word is the canonical, base form of a set of words belonging to the same lexeme. \textbf{Lemmatizing} a word refers to the process of finding this base form for (possibly inflected) words. For example, the lemma of the words \textit{going, went, gone} is \textit{go}.
        }
}

\newglossaryentry{doctermmat}{
    type=defs,
    name=document-term \mbox{matrix},
    plural=document-term matrices,
    description={
        A document-term matrix encodes the frequency of terms (words, n-grams or other) for a collection of texts in a matrix. The (often very sparse) matrix has a rows represending the documents and columns corresponding to terms, the individual values encoding the pure counts, frequencies or quantifications of all combinations of document and term.
        }
}


\newglossaryentry{dissimmat}{
    type=defs,
    name=dissimilarity matrix,
    plural=dissimilarity matrices,
    description={
        \hspace{0.2em} A square matrix where both rows and columns represent entities, the cells being to their pairwise dissimilarities as calculated by an arbitrary distance function. For metric distances, distance matrices are mirrored along their main diagonal, which is made up solely from zeros. Also called \textbf{distance matrix}.
        }
}


\newglossaryentry{word2vec}{
    type=defs,
    name=Word2Vec,
    description={
        word2vec is the most famous of a family of \emph{neural language models} \cite{Le2014}. Here we'll use it as \textit{Pars pro Toto} for all of these or rather the respectively most appropriate one. \todoparagraph{Write more about it, rather in a section!}
        }
}

\newglossaryentry{doc2vec}{
    type=defs,
    name=Doc2Vec,
    description={
        \todoparagraph{The term Doc2Vec is not originally used by the authors, only \emph{Paragraph Vectors}, but that's what it's generally called}
        }
}


% =========================================================
% CUSTOM TERMS

\newglossaryentry{quant}{
    type=customs,
    name=quantificiation,
    description={
        \hspace{1.2em} In the scope of this thesis, the term \textbf{quantification} refers to the relative score for an n-gram in a document, depending on it's frequency as well as other frequencies, as calculated by one of the \nameref{sec:word_count_techniques}, also called \textbf{quantification measures}. %TODO: to lower-case see https://tex.stackexchange.com/questions/445404/capitalization-variants-of-nameref 
        }
}

\newglossaryentry{entity}{
    type=customs,
    name=entity,
    plural=entities,
    description={
        An entity is a single sample from the handled corpus. Depending on the context, this term may also refer to its associated text (which may, depending on the considered dataset, be the course-description, picture-tags, concatenated-reviews, \dots).
        }
}
