\usepackage{amsfonts}
\usepackage{siunitx}
\newcommand*{\Z}{\mathbb{Z}}
\newcommand*{\Q}{\mathbb{Q}}


% =========================================================
% UNITS

\newglossaryentry{m}{
    name=\ensuremath{m},
    description={Meter},
    unit={\si{m}},
    type=units
}

\newglossaryentry{energyconsump}{
    name=\ensuremath{P},
    description={Power},
    unit={\si{kW}},
    type=units
}

% =========================================================
% SYMBOLS

% \newglossaryentry{symb:Pi}{
%     name=\ensuremath{\pi},
%     description={geometric value},
%     type=symbols,
%     sort=P
% }

% \newglossaryentry{integers}{
%     name=\ensuremath{\Z},
%     type=symbols,
%     description={the ring of integers}, 
%     sort=Z
% }

% \newglossaryentry{rationals}{
%     name=\ensuremath{\Q},
%     type=symbols,
%     description={the field of rational numbers}, 
%     sort=Q,
%     % nonumberlist,
% } 

\newglossaryentry{vector-space}{
    name=\ensuremath{V},
    type=symbols,
    description={a vector space}, 
    sort=V
}

% =========================================================
% DEFINITIONS

\newglossaryentry{ngram}{
    type=defs,
    name=n-gram,
    description={
        n-grams are sequences of consecutive words of length $n$. For example, the text "I eat lunch" contains the 1-grams ["I", "eat", "lunch"], the 2-grams ["I eat", "eat lunch"] and the 3-gram ["I eat lunch"]
        }
}


\newglossaryentry{stopword}{
    type=defs,
    name=stopword,
    description={
        TODO: do
        }
}

\newglossaryentry{lemma}{
    type=defs,
    name=lemma,
    description={
        The \textbf{lemma} of a word is the canonical, base form of a set of words belonging to the same lexeme. \textbf{Lemmatizing} a word refers to the process of finding this base form for (possibly inflected) words. For example, the lemma of the words \textit{going, went, gone} is \textit{go}.
        }
}

\newglossaryentry{doctermmat}{
    type=defs,
    name=document-term \mbox{matrix},
    plural=document-term matrices,
    description={
        A document-term matrix encodes the frequency of terms (words, n-grams or other) for a collection of texts in a matrix. The (often very sparse) matrix has a rows represending the documents and columns corresponding to terms, the individual values encoding the pure counts, frequencies or quantifications of all combinations of document and term.
        }
}

% % TODO maybe have ANOTHER glossary section with the nomenclature used in this thesis which are not-really-definitions?
% \newglossaryentry{quant}{
%     type=customs,
%     name=quantificiation,
%     description={
%         A document-term matrix encodes the frequency of terms (words, n-grams or other) for a collection of texts in a matrix. The (often very sparse) matrix has a rows represending the documents and columns corresponding to terms, the individual values encoding the pure counts, frequencies or quantifications of all combinations of document and term.
%         }
% }